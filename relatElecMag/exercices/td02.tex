%TD n°2
\chapter{Voyage interstellaire}
Ce TD, qui regroupe un certain nombre d’applications de la transformation de Lorentz, a
pour but d’étudier quelques aspects du voyage à des vitesses proches de celle de la lumière. La
première partie permet de mettre en évidence le fameux paradoxe des jumeaux tel qu’il a été
formulé par le physicien Paul Langevin (1911). La deuxième partie propose un point de vue qui
permet de faire disparaître le paradoxe. Enfin, la dernière partie présente un modèle de voyage
plus réaliste que celui proposé précédemment.
\section{Paradoxe des Jumeaux} %1 
Deux jumeaux naissent sur Terre, l’un est appelé le sédentaire et noté S et l’autre est appelé
le voyageur et noté V. Dès leur naissance, les deux jumeaux sont séparés : pendant que S reste
sur la Terre, V monte dans une fusée piloté par le pilote P' et part pour une étoile E située à
une distance L $=$ 40 années-lumière de la Terre et effectue son voyage à la vitesse constante \bi{v}
(par rapport à la Terre) telle que $\gamma = 22$. On note \mc{R} le référentiel inertiel lié à la Terre dans
lequel S est au repos, et \mc{R'} le référentiel inertiel lié au pilote P'. Le point de départ est l’origine
O (x $=$ 0 et t $=$ 0) et le mouvement s’effectue selon l’axe O$x$ du référentiel \mc{R}. On notera T la
durée du voyage de la Terre à l’étoile dans le référentiel \mc{R}.
\begin{enumerate}
  \item Paramétrer les lignes d’univers de S, V et E dans chacun des deux référentiels \mc{R} et \mc{R'} .
Représenter les diagrammes d’espace-temps dans \mc{R} et \mc{R'} .
  \item Calculer la durée T du voyage vue par S puis la durée T' du voyage pour V.
\setcounter{numero}{\theenumi}\end{enumerate}
Les jumeaux S et V, connaissant un brin de relativité, savent que les temps de parcours T et
T' sont différents. Chacun des deux veut savoir quel est l’âge de l’autre lorsque le voyage aller
est terminé. L’âge de V pour S est déterminé ainsi : S demande à un observateur S$_2$ , lié à \mc{R} et
fixe sur l’étoile, de lire l’horloge de V à son arrivée sur l’étoile. De même, pour connaître l’âge
de S pour V, V demande à un observateur V$_2$ , lié à \mc{R'} et coïncidant avec S quand V arrive sur
l’étoile, de lire l’horloge de S.
\begin{enumerate}
  \setcounter{enumi}{\thenumero}
  \item Quels sont les âges de V pour S et de S pour V déduits de la procédure ci-dessus ? Quel
est le jumeau le plus jeune à l’arrivée de V sur l’étoile ? Pourquoi l’âge de S pour V mesuré
ainsi n’est pas égal à T ?
\setcounter{numero}{\theenumi}\end{enumerate} 
On s’intéresse maintenant au trajet retour : on suppose qu’à peine arrivé sur l’étoile, V saute
sur une seconde fusée pilotée par P'' , et qui se dirige à la vitesse $−$\bi{v} vers la Terre. On note \mc{R''}
le référentiel inertiel lié à P'' .
\begin{enumerate}
  \setcounter{enumi}{\thenumero}
  \item Quel est la durée du trajet retour pour S et pour V ? En déduire les temps d’aller-retour
T$_{AR}$[S] et T$_{AR}$[V] pour S et V.
  \item Voici le raisonnement de Langevin : « {\it le temps d’aller-retour} T$_{AR}$[S] {\it pour} S {\it est un temps
propre, il doit donc être égal à} $T_{AR}[V]/\gamma$. De même, le temps d’aller-retour T$_{AR}$[V] pour
V est un temps propre, il doit donc être égal à $T_{AR}[S]/\gamma$. {\it Ces deux phrases sont contradictoires si} $\gamma \neq 1$ ». Où est l’erreur dans le raisonnement de Monsieur Langevin ? Quelle
phrase est la bonne ?

{\footnotesize (Paul Langevin avait parfaitement compris la relativité restreinte et n’a jamais proposé le paradoxe des jumeaux
comme une preuve de l’inconsistance de la relativité restreinte. Au contraire, le « paradoxe » qui porte son
nom a en fait été porté devant ses contemporains par Langevin pour leur montrer que la notion de simultanéité
pouvait induire en erreur.)}
  \item Calculer, pour S, P' , P'' et V, la durée du voyage aller et celle du voyage retour. Proposer
deux méthodes pour le calcul du temps aller dans \mc{R''} et du temps retour dans \mc{R'}.
  \item Enfin, tracer le voyage aller retour dans un diagramme d’espace-temps ($ct'$, $x'$) dans \mc{R'}.
\end{enumerate}
\section{Résolution en termes de battements de coeur} %2 
On considère toujours les deux jumeaux S et V. On suppose que S et V ont tous les deux un
coeur qui bat régulièrement avec une période T$_0$ = 1s. En outre, un appareil envoie un flash
lumineux isotrope à chaque battement de coeur.
\begin{enumerate}
  \item Rappeler la loi de composition des vitesses en relativité restreinte (voir cours) ainsi que la
formule de l’effet Doppler relativiste (voir TD n°1).
  \item Déterminer le nombre de battements de coeur de S qu’observe V pendant le voyage.
  \item Déterminer le nombre de battements de coeur de V qu’observe S pendant le voyage.
  \item Enfin, déterminer le nombre de battements de coeur de V et S vus par le pilote P' de la
première fusée. Conclure.
\end{enumerate}
\section{Mouvement uniformément accéléré}%3 
Nous allons traiter dans cette partie le paradoxe des jumeaux en calculant explicitement
comment évolue le temps propre lors d’un demi-tour « réaliste » du jumeau voyageur.
Pour effectuer son voyage, V monte dans un véhicule spatial de masse M, initialement immobile
sur la Terre, et accélère progressivement en se dirigeant vers E. On admettra ici que le vaisseau
étant soumis à une force constante F dans \mc{R'} (force de poussée), l’équation relativiste de la
dynamique permet d’écrire :
\[
\frac{\text{d}}{\text{d}t}(\gamma(t)\text{M}v(t)) = \text{F}
\]
On note toujours T la durée du voyage aller dans \mc{R}. Au milieu du voyage (en $x =$ L/2, à $t =$ T/2),
le vaisseau se retourne et emploie la force de poussée pour ralentir. La position, la vitesse et
l’accélération du vaisseau dans \mc{R} sont $x(t)$, $v(t)$ et $g(t)$. On notera \mc{R'}$(t)$ le référentiel galiléen
tangent au mouvement du vaisseau à l’instant $t$ dans \mc{R}, défini par les paramètres $\beta_{\mathcal{R'}(t)} = \beta(t)$
et $\gamma_{\mathcal{R'}(t)} = \gamma(t)$ habituels. On notera $\tau(t)$ le temps propre du vaisseau, obtenu en accumulant les
temps propres infinitésimaux dans les référentiels tangents successifs. On notera enfin $a =$ F/M,
$t_0 = c/a$, $\mathcal{L} = ct_0$ . Autant que possible les résultats seront exprimés en fonction de L, $\mathcal{L}$, et $t_0$ .
\begin{enumerate}
  \item Quelles significations physique peut-on donner classiquement à $a$, $t_0$ et $\mathcal{L}$ ?
  \item À partir de l’équation du mouvement du vaisseau dans \mc{R} pendant la phase d’accélération,
exprimer $v(t)$, $\gamma(t)$ puis $x(t)$ et $g(t)$.
  \item Représenter qualitativement les variations de $v$, $x$, $\gamma$ et $g$ pendant l’ensemble du voyage
aller (on raisonnera par symétrie pour la phase de freinage).
  \item Représenter sur un diagramme d’espace-temps la ligne d’Univers du vaisseau.
  \item En utilisant la loi de composition des vitesses, exprimer l’accroissement d$v'$ de la vitesse
dans $\mathcal{R}'$ , pendant un intervalle de temps d$t$ dans $\mathcal{R}$, en fonction de d$v = g$d$t$. En déduire
l’accélération A dans $\mathcal{R}'$ et montrer que A $= a$.
  \item Calculer le temps de parcours T dans $\mathcal{R}$ (on utilisera la symétrie par rapport à la manoeuvre
de retournement). Donner le comportement de T pour un voyage très court ou un voyage
très long (L petit ou grand par rapport à $\mathcal{L}$). Calculer enfin la valeur maximale de $\gamma$.
Applications numériques : L $=$ 40 années lumière, a $=$ 10 m.s$^{−2}$.
  \item Calculer le temps propre $\tau$ en fonction du temps $t$ dans $\mathcal{R}$. En déduire la durée du parcours
$\mathcal{T}$ pour V. Donner les comportements asymptotiques de $\mathcal{T}$ pour un trajet très court ou
très long. Interprétation physique et applications numériques.
\end{enumerate}
\section{Formulaire}%4 
On rappelle que :
\[
\int^x\frac{dt}{\sqrt{a^2+t^2}} = \ln(x+\sqrt{a^2+x^2})+\text{C}
\]
