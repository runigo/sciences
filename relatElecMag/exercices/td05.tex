
\section{Champs et masse d'une particule chargée}%TD n°5
\subsection{Champs d'une particule chargée en mouvement}%1 
On considère une particule de charge q en mouvement rectiligne uniforme à la vitesse v par
rapport au référentiel du laboratoire K . Conformément aux notations de la figure 1, on s'intéresse
aux champs créés par la particule q au point M (x, y, z) à l'instant t où la charge q est à la distance
vt de O .
y y’
(K) (K ’ )
y
M( x , y , z )
R( t )
O
q
O’
q
z
z
v
x
x’
x
vt
z’
Figure 1: Charge en mouvement.
1. Calculer les composantes du quadrivecteur potentiel à dans le référentiel K.
2. En déduire l'expression des champs E et B au point M en fonction de q , v , R = O 0 M et
θ , l'angle entre R et l'axe des abscisses.
3. Dessiner les lignes du champ E et préciser la limite non relativiste des expressions trouvées
à la question précédente.
On considère maintenant deux charges identiques, espacées d'une distance a , se déplaçant à la
même vitesse, parallèlement à l'axe Ox (droites d'équations y = 0 et y = a ).
4. On se place dans le référentiel K. À l'aide des expressions précédentes, montrer que la force
de Lorentz exercée par la charge située sur l'axe (Ox) sur l'autre charge s'écrit :
F
q 2
=
4π 0 a 2
r
1 −
v 2
e y .
c 2
5. Dans l'approximation des faibles vitesses, montrer que cette force se compose d'une force
répulsive et d'une force attractive que l'on interprètera.
1Licence de physique
Année 2011-2012
L7
\subsection{Masse électromagnétique}%2 
On modélise une particule chargée par une sphère de rayon a dont la charge q est répartie en
surface.
1. On considère tout d'abord la charge immobile. Calculer la densité d'énergie du champ
électromagnétique puis l'énergie totale U 0 du champ. On pourra poser e 2 = q 2 /4π 0 .
La particule chargée est maintenant animée d'une vitesse v par rapport au référentiel du
laboratoire (voir Fig . 1).
2. Montrer que l'énergie U et la quantité de mouvement P x du champ dans le référentiel du
laboratoire sont données par les expressions suivantes :
U
P x


β 2
= γ 1+
U 0 ,
3
4U 0
=
γv .
3c 2
(1)
(2)
3. À l'aide de l'expression (2), définir une masse électromagnétique de la particule chargée
notée m élec .
4. Peut-on identifier les expressions (1) et (2) au quadrivecteur énergie-impulsion de la particule dans le référentiel du laboratoire ?
TD n o 1
2
S. Leurent, M. Lilley & S. Nascimbène
