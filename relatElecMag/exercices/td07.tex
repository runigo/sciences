
\section{Rayonnement d'une particule chargée}% TD n 7
\subsection{Préliminaires : puissance et impulsion rayonnées par une charge en mouvement}%1 
ds
R
O
M
v
a
Une particule chargée, de vitesse v et d'accélération a , passe par le entre O d'une sphère de
rayon R à l'instant t 0 . Un point M de la sphère est repéré par le ve teur unitaire n = OM/R .
On considère l'ensemble du rayonnement émis par la particule entre les instants t 0 et t 0 + dt 0 .
(a) Déterminer l'intervalle de temps pendant lequel e rayonnement traversera l'élément de
surface dsn autour de M.
(b) En déduire que la partie de l'énergie rayonnée entre t 0 et t 0 + dt 0 qui traversera ds s'écrit :
dE = Π(M, t 0 + R/c) · nds (1 − β · n) dt 0 .
( ) Conclure que la puissance rayonnée par unité d'angle solide s'écrit :
dE
q 2 kn × ((n − β) × β̇)k 2
dP
=
=
dΩ
dt 0 dΩ
16π 2 ε 0 c
(1 − β · n) 5
(1)
(d) De la même manière, montrer que l'impulsion rayonnée par unité de temps et par unité
d'angle solide s'écrit :
dP
=
dt 0 dΩ

dP
dΩ

n
.
c
(2)
(e) ( calculs longs, à faire chez soi) On va intégrer l'expression (1) afin de déterminer la puissance
rayonnée par la particule. Pour cela, montrer tout d'abord que l'on a :
P =
h
i
q 2
2
2
β̇
I
+
2(
β̇
·
β)
β̇
J
−
(1
−
β
)
β̇
β̇
K
i
i
i
j
ij
16π 2 ε 0 c
1
(3)L7
Li en e de physique
Année 2011-2012
ave
4π
dΩ
=
3
(1 − n · β)
(1 − β 2 ) 2
Z
n i dΩ
1 ∂I
16πβ i
=
=
=
(1 − n · β) 4
3 ∂β i
3(1 − β 2 ) 3


Z
n i n j dΩ
6β i β j
1 ∂ 2 I
4π
=
=
=
δ ij +
(1 − n · β) 5
12 ∂β i ∂β j
3(1 − β 2 ) 3
1 − β 2
I =
J i
K ij
Z
(4)
(5)
(6)
Déduire finalement l'expression de la puissance rayonnée par la particule :
q 2
P =
6πε 0 c 3

a 2 − (a × β) 2
(1 − β 2 ) 3

(7)
On peut montrer, en utilisant une méthode analogue, que l'impulsion cédée au champ par la
charge s'écrit quant à elle :
 
dP
P
(8)
=
β.
dt 0
c
(f ) Considérons le mouvement d'une particule de charge q , soumise à l'action d'un champ magnétique. Le champ accélère la charge, celle-ci rayonne et perd de l'énergie. Or, d'après ce que
vous savez de l'électrodynamique, la variation d'énergie de la particule s'écrit :
dE
= q E · v
dt
Elle est donc nulle dans la situation considérée. Où est le problème ?
(g) On considère une particule de charge q animée d'une faible vitesse ( v ≪ c ). Montrer alors
que la puissance rayonnée par unité d'angle solide s'écrit :
q 2
dP
=
a 2 sin 2 θ
dΩ
16π 2 ε 0 c 3
où θ est l'angle entre l'accélération a et la direction n considérée. Commenter cette relation.
(h) Montrer que l'expression de la puissance rayonnée par une particule non relativiste est donnée
par l'expression suivante (formule de Larmor) :
P Larmor =
q 2
dp
2
3
6πε 0 m c
dt
2
(9)
\subsection{Rayonnement des électrons du LEP}%2 
Le LEP est un accélérateur circulaire de 27 km de circonférence dans lequel des électrons
( m = 0.511 MeV) sont portés à une énergie ultra-relativiste de 50 GeV.
Dans cet accélérateur, les électrons émettent un rayonnement électromagnétique au cours de
deux phases bien précises de leur trajectoire qui sont :
 la phase d'accélération. En effet, à chaque tour, les électrons sont accélérés par un champ
électrique.
 leur mouvement le long du tunnel du LEP, dans lequel un champ magnétique les dévie en
permanence.
o
TD n 7
2
S. Leurent, M. Lilley, S. Nas imbèneL7
Li en e de physique
\subsubsection{Rayonnement d'une charge accélérée linéairement}% 2.1 

On considère dans un premier temps une particule de charge q en mouvement d'accélération
linéaire. Les notations sont pré isées sur la figure 1.
z
z
θ
θ
n
n
v
v
a
y
a
φ
x
y
φ
x
Figure 1: Mouvement linéaire
Figure 2: Mouvement
circulaire
(a) Montrer que la puissance perdue par unité d'angle solide s'écrit :
a 2 sin 2 θ
q 2
dP
=
dΩ
16π 2 ε 0 c 3 (1 − β cos θ) 5
(b) Etudier le diagramme de rayonnement en montrant en particulier que l'énergie est essentiellement rayonnée dans une direction définie par θ max ≃ 1/2γ pour des vitesses proches de celle
de la lumière.
( ) En utilisant (7), déterminer la puissance perdue par la particule. On mettra le résultat sous
la forme :
  2
q 2
dp
P lin =
(10)
6πε 0 m 2 c 3 dt
Commenter e résultat et comparer avec la formule de Larmor.
\subsubsection{Rayonnement d'une charge en mouvement circulaire}% 2.2 
On envisage désormais la situation schématisée sur la figure (2). La particule de charge q
est maintenant en mouvement circulaire uniforme. Le rayonnement ara téristique ainsi émis
s'appelle le rayonnement syn hrotron.
(a) Montrer que la puissance perdue par unité d'angle solide s'écrit :


q 2
sin 2 θ cos 2 φ
a 2
dP
=
1 − 2
dΩ
16π 2 ε 0 c 3 (1 − β cos θ) 3
γ (1 − β cos θ) 2
(b) En déduire que l'énergie est principalement rayonnée dans la direction de la vitesse avec une
dispersion angulaire δθ ≃ 1/γ .
o
TD n 7
3
S. Leurent, M. Lilley, S. Nas imbèneL7
Li en e de physique
Année 2011-2012
( ) Montrer alors que la puissance perdue par la particule s'écrit (on utilisera toujours (7)) :
P ir =
q 2
γ 2
6πε 0 m 2 c 3
dp
dt
2
(11)
Comparer ave les résultats (9) et (10) établis précédemment.
\subsubsection{Ordres de grandeur pour les électrons du LEP}% 2.3 
Etablissons désormais quelques ordres de grandeur relatifs au mouvement des électrons du
LEP.
phase d'accélération
Chaque tour, les électrons pénètrent dans une zone de longueur L = 100 m où ils sont soumis
à un champ électrique uniforme E = 5 MV/m.
(a) En négligeant dans un premier temps le rayonnement, déterminer le gain en énergie des
électrons ∆E après passage dans la cavité accélératrice.
(b) Montrer alors que l'énergie rayonnée s'écrit :


2 qEr l
E ray =
∆E
3 mc 2
où r l est le rayon lassique de l'électron défini par :
q 2
= mc 2
4πε 0 r l
Interpréter cette relation.
( ) Conclure alors que le rayonnement est totalement négligeable dans ce cas.
rayonnement synchrotron
(a) Déterminer l'ordre de grandeur du champ magnétique nécéssaire pour courber la trajectoire
des électrons.
(b) En déduire l'énergie perdue par un électron à haque tour. Conclusion ?
Formulaire
On rappelle que le champ électromagnétique rayonné à grande distance en un événement (r, t)
par une particule de charge q dont la trajectoire est fixée s'écrit :

 

n
×
(n
−
β)
×
β̇
q 

E(r, t) =
(12)
4πε 0 c
(1 − β · n) 3 R
ret
n × E(r, t)
B(r, t) =
(13)
c
où toutes les quantités relatives à la particule sont considérées à l'instant retardé t 0 (r, t) .
o
TD n 7
4
S. Leurent, M. Lilley, S. Nas imbèneL7
Li en e de physique
Année 2011-2012
Intégrales utiles :
Z π
4
3 (14)
4
1
3 (1 − a 2 ) 3 (15)
sin 3 θ dθ =
0
Z π
0
sin 3 θ
5
(1 − a cos θ)
+∞
Z
a
+∞
Z
0
o
TD n 7
dx
p
dx
(x 2 +
dθ =
a 2 ) 2
x 7 (x − a)
=
+∞
Z
a
5
x 3
=
√
16
15a 3
dx
π
= 3
4a
x 2 − a 2
(16)
(17)
S. Leurent, M. Lilley, S. Nas imbène
