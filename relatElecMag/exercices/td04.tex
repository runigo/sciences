
\chapter{Collisions relativistes}% TD n°4

\section{Collisions élastiques}%1 
A l’issue d’une collision élastique entre une particule de masse m et d’énergie cinétique T avec
une autre particule de masse m au repos, les deux particules ont des énergies inégales et leurs
vecteurs vitesse sont inégalement inclinés par rapport à la direction de la particule incidente (ils
forment des angles qu’on notera $\theta_1$ et $\theta_2$ et on pose $\alpha=\theta_1+\theta_2$ ). La mécanique newtonienne
prédit que l’angle $\alpha$ compris entre ces deux vecteurs vitesse sera toujours égal à $\pi$/2. Il n’en va
pas de même en mécanique relativiste, celle-ci prédit un angle inférieur à $\pi$/2.
1. Montrer, en utilisant la conservation de l’impulsion et de l’énergie cinétique, que la mécanique newtonienne prédit que $\alpha = \pi$/2.
2. Utiliser maintenant la conservation de l’énergie-impulsion en relativité restreinte pour donner l’expression de cos$\alpha$ en fonction des énergies des particules. Montrer que $\alpha < \pi$/2.
Montrer que dans le cas $\theta_1=\theta_2$ , on a :
\[
\cos \alpha = \frac{\text T}{{\text T} + 4mc^2}
\]
\section{Collisions inélastiques - Énergie de seuil}%2
On notera $\wtt{Q}$ un quadrivecteur et $\vt{Q}$ sa composante spatiale.
\subsection{Préliminaire : notion de centre de masse en relativité}%2.1
Considérons un ensemble de N particules dont l’une au moins est de masse non nulle. On note
$\wti{p_i}$ la 4−impulsion de la particule $i$ définie dans le référentiel du laboratoire \mc{R}.
\begin{enumerate}
  \item Montrer que la 4-impulsion totale du système $\wtt{P}=\sum \wti{p_i}$ est un quadrivecteur de genre temps.
  \item En déduire qu’il existe un Référentiel Galiléen dans lequel $\vt{P}=\ve{0}$ . Montrer que le
référentiel du centre de masse noté \mc{R}$_{\text CM}$ est défini par rapport à R par sa vitesse :
\[
\ve{\beta}=\frac{\sum_{i=1}^{\text N}\ve{p}(i)c}{\sum_{i=1}^{\text N}\mathcal{E}(i)}
\]
où \mc{E}(i) désigne l’énergie de la particule.
  \item Vérifier que l’on retrouve le Référentiel barycentrique \mc{R}$^∗$ à la limite non relativiste.
\end{enumerate}
\subsection{Seuil d’une réaction nucléaire}
2.2 
On cherche à créer des noyaux d’Aluminium en bombardant une cible de Magnésium à l’aide de
particules $\alpha$ (noyaux d’Hélium), selon la réaction suivante :
\[
_{12}^{24}\tx{Mg}+_{2}^{4}\tx{He} \ \ \longrightarrow \ \ \ _{13}^{27}\tx{Al}+_{1}^{1}\tx{H}
\]
\begin{enumerate}
  \item Quelle doit être l’énergie minimale \mc{E}$^∗$ des réactifs dans le référentiel du centre de masse
pour que la réaction ait lieu ?
  \item En déduire, en repassant dans le référentiel R, que l’énergie cinétique minimale des particules $\alpha$ s’écrit :
\[
\tx{T}_{\tx{seuil}}=
\frac{(m_p c^2 + m_\tx{Al} c^2 )^2 − (m_\alpha  c^2 + m_\tx{Mg} c^2 )^2}{2m_\tx{Mg} c^2}
\]


  \item Application numérique. Quelle doit être la vitesse des particules $\alpha$ ?
\end{enumerate}
\subsection{Matérialisation de photons par création de paires $e^− e^+$}%2.3 
A haute énergie (h$\nu$ > 100Mev), le processus prépondérant dans l’absorption des photons par
la matière est la création de paires électron-positron (On rappelle que le positron noté $e^+$ est
l’antiparticule de l’électron $e^−$ , de masse $m_e$ et de charge $+e$).
\begin{enumerate}
  \item Montrer qu’un photon ne peut pas se matérialiser dans le vide d’après la réaction :
\[
\gamma \ \ \longrightarrow \ \ e^++e^−
\]
  \item La matérialisation nécessite donc la présence d’un catalyseur A qui intervient dans le bilan
sous la forme :
\[
\gamma+A \ \ \longrightarrow \ \ e^++e^−+A
\]
Quel est son rôle ?
Déterminer l’énergie de seuil dans les deux cas suivants :

\ \ \ • A est un électron au repos.

\ \ \ • A est un noyau atomique au repos.
\end{enumerate}

{\bf Données numériques}

\begin{minipage}[c]{.45\linewidth}
\[
\begin{array}{rcl}
\tx{Elément} &   & \tx{masse} \\
 _{12}^{24}\tx{Mg} &   & 23, 985042\ u \\
 _{13}^{27}\tx{Al} &   & 26, 981539\ u \\
 _{1}^{1}\tx{H} &   & 1, 007268\ u \\
 _{2}^{4}\tx{He} &   & 4, 002603\ u \\
e − ,e + &   & 0, 5\ \tx{Mev }
\end{array}
\]
\end{minipage}
\hfill
\begin{minipage}[c]{.45\linewidth}
avec 1 $u =$ 931.48 Mev
\end{minipage}
\subsection{Réaction de désintégration}%2.4 
Un méson  $\pi^0$ de masse $m$ se déplace selon l’axe des x du laboratoire à la vitesse $\beta c$. Il se
désintègre en deux photons. Dans le système dans lequel le méson est au repos, ces photons
sont émis selon une direction faisant un angle $\theta'$ avec l’axe $x'$.
\begin{enumerate}
  \item Déterminer l’énergie des photons dans le référentiel du méson.
  \item En déduire les énergies et les directions de propagation des 2 photons dans le référentiel
du laboratoire en fonction de l’angle $\theta'$.
\end{enumerate}
\section{Mouvement d’une particule chargée dans des champs uniformes
constants}
\subsection{Champ E}%3.1 
On considère le mouvement d’une particule de masse $m$ et de charge $q$ dans un champ électrique
constant {\bf E} porté par l’axe (Ox). A l’instant initial, la particule possède une impulsion {\bf p} $= p_0{\bf u}_y$ .
\begin{enumerate}
  \item Rappeler les résultats de la mécanique classique concernant le mouvement de la particule.
  \item On se place désormais dans le cadre de la relativité restreinte. En intégrant l’équation
différentielle pour l’impulsion, montrer que l’énergie de la particule peut s’écrire sous la
forme suivante :
\[
\mathcal{E} = \sqrt{\mathcal{E}_0^2 + (qc\tx{E}t)^2}
\]
q
On donnera l’expression de \mc{E}$_0$.
  \item Déduisez-en une relation entre {\bf v}, {\bf p} et \mc{E} que l’on intégrera pour trouver les équations du
mouvement.
  \item Trouvez l’équation de la trajectoire de la particule et montrez que dans le cas $v \ll c$, on
retrouve la trajectoire classique.
\end{enumerate}
\subsection{Champ E et B}%3.2 
\begin{enumerate}
  \item Déterminer les caractéristiques de la trajectoire d’une particule relativiste dans un champ
\mc{B} uniforme dans le cas où sa vitesse initiale est dans un plan perpendiculaire au champ.
  \item On considère maintenant la situation où une particule immobile à l’instant initial est
soumise à un champ électrique perpendiculaire à un champ magnétique. Les champs
vérifient l’inégalité suivante : c$^2$ B$^2$ − E$^2$ > 0. Montrer qu’il existe un référentiel galiléen
dans lequel le champ électrique est nul. En déduire la trajectoire du mouvement dans le
référentiel initial.
\end{enumerate}
