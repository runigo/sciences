
\section{Rayonnement synchrotron}%TD n°8
\subsection{Généralités}%1 
On s'intéresse dans cette partie au rayonnement émis par les électrons du synchrotron de
l'université de Cornell. Cette machine est un accélérateur circulaire de rayon D = 200 m dans
lequel les électrons ( m e = 0.511 MeV) sont portés à une énergie de 10 GeV.
z
n
θ
v
y
a
φ
x
On rappelle que la puissance perdue par unité d'angle solide par une particule de charge q en
mouvement circulaire s'écrit, avec les notations de la figure ci-contre :
q 2
a 2
dP
=
dΩ
16π 2  0 c 3 (1 − β cos θ) 3

1 −
sin 2 θ cos 2 φ
γ 2 (1 − β cos θ) 2

On a vu dans le TD précédent que l'essentiel du rayonnement est émis dans un cône d'angle
au sommet 1/γ dans la direction définie par la vitesse instantanée de la particule.
1. Un physicien place un détecteur en un point situé sur l'axe (Oz) à grande distance du
synchrotron. Montrer qualitativement qu'il observe une série périodique d'impulsions lumineuses. Quelle est la période du signal ?
2. Montrer que la durée d'une impulsion s'écrit approximativement :

δt '
3.
D
2c

1
γ 3
Application numérique : quelles sont les longueurs d'onde extrêmales du spectre de rayonnement émis par les électrons ?
1Licence de physique
Année 2011-2012
L7
\subsection{Onduleur}%2 
Dans les synchrotrons de génération récente, on a cherché à aner le spectre émis par les
électrons. Ces machines sont conçues comme des anneaux de stockage circulaires à partir desquels
les électrons sont injectés dans des onduleurs qui permettent d'obtenir un rayonnement quasi-monochromatique dans le domaine des rayons X.
N
S N
S N
S
x
z
y
S
N S
N S
N
Le fonctionnement d'un onduleur est schématisé sur la figure ci-dessus. Un électron relativiste
se propage selon l'axe Oz. Un champ magnétique oscillant obtenu par un arrangement périodique
d'aimants permanents fait osciller l'électron dans le plan (xOz) , ce qui le fait rayonner.
Étude du mouvement de l'électron
Le champ créé par l'onduleur s'écrit B = B 0 cos(k 0 z) u y avec k 0 = 2π/λ 0 où λ 0 est la périodicité spatiale du champ. À t = 0 , l'électron relativiste pénètre dans l'onduleur avec une vitesse
β 0 c ( β 0 ' 1 ).
1. Ecrire l'équation du mouvement dans le référentiel du laboratoire R , en négligeant la réaction de rayonnement. On montrera que l'on peut résoudre cette équation en négligeant
également l'oscillation suivant l'axe (Oz) et on exprimera le résultat en utilisant le paramètre de l'onduleur défini par :
K =
eB 0
k 0 mc
Donner la condition sur K pour que les hypothèses soient valides.
Étude du champ rayonné
On s'intéresse au champ rayonné par l'électron en un point M situé à grande distance R dans
la direction n paramétrée par les angles θ et φ des coordonnées sphériques usuelles. On se limite
au rayonnement reçu à des instants voisins de T = R/c , instants pour lesquels le rayonnement a
été produit par l'électron au voisinage de l'origine O.
2. Montrer que l'instant retardé t 0 s'écrit, dans la limite où l'on peut négliger le retard dû au
mouvement transverse :
t 0 =
t − R/c
1 − β cos θ
3. En négligeant toujours la vitesse transversale de l'électron, déterminer l'expression explicite
du champ électrique rayonné en M . On s'intéresse par la suite à la zone de rayonnement
intense. Déduire de l'expression du champ électrique la pulsation ω(n) du rayonnement
dans la direction n .
4. Estimer la largeur du lobe d'émission. En déduire la structure du champ électromagnétique
dans la zone utile (étude de la polarisation).
TD n o 1
2
S. Leurent, M. Lilley & S. NascimbèneLicence de physique
Année 2011-2012
L7
5. Applications numériques pour des électrons de 5 GeV dans un onduleur de période 1 cm.
6. Quelle est la condition sur le paramètre K pour avoir une source continue de rayonnement
et non pas des impulsions ?
Étude dans le référentiel propre de l'électron
On cherche à retrouver les résultats précédents en se plaçant dans le référentiel R 0 se déplaçant
à la vitesse β 0 cu z par rapport à R .
7. Que devient le champ de l'onduleur dans ce référentiel ? À quelle fréquence excite-t-il
l'électron ?
8. Retrouver alors la fréquence du rayonnement dans le référentiel R .
Formulaire
On rappelle que le champ électromagnétique rayonné à grande distance en un événement (r, t)
par une particule de charge q dont la trajectoire est fixée s'écrit :

 
(n
−
β)
×
β̇
n
×
q 

E(r, t) =
4π 0 c
(1 − β · n) 3 R

ret
B(r, t) =
n × E(r, t)
c
où toutes les quantités relatives à la particule sont considérées à l'instant retardé t 0 (r, t) .
TD n o 1
3
S. Leurent, M. Lilley & S. Nascimbène
