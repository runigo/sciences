
\chapter{Constructions géométriques}% TD n°3
En relativité, l’espace et le temps n’étant pas absolus, il faut raisonner directement dans
l’espace-temps à 4 dimensions. Ce TD propose plus modestement d’interpréter géométriquement
la transformation de Lorentz dans le cadre d’un espace-temps à 2 dimensions (une dimension
temporelle et une dimension spatiale).
On identifie le plan de la feuille à notre espace-temps. Chaque point du plan représente donc
un événement. Pour repérer l’ensemble des événements un observateur \mc{O} appartenant à un
référentiel inertiel \mc{R} a besoin de trois événements :
– O, un événement origine quelconque ;
– un événement B simultané à O (dans \mc{R}), et séparé par une distance qui sera considérée
comme l’échelle de longueur de \mc{R}. On note \bi{e}$_1$ le vecteur {\bf OB}. Cet évènement B est l’émission
d’un flash de lumière.
– l’événement A qui correspond à l’arrivée du flash précédent au point O. Les évènements O
et A, colocalisés, sont séparé par un intervalle de temps qui sera considéré comme l’étalon
temporel de \mc{R} (on rappelle que le temps est mesuré en mètres) ; on note e 0 le vecteur OA.
Grâce à ce repère (O, e 0 , e 1 ), l’observateur peut définir les coordonnées contravariantes dans R
de chaque événement E par :
\[
{\bf OE} \equiv x^\mu {\bi e}_\mu \equiv x^0 {\bi e}_0 + x^1 {\bi e}_1
\]
Dans l’espace-temps on définit le pseudo-produit scalaire de deux vecteurs a et b par
\[
\langle {\bi a}, {\bi b} \rangle \equiv 
\eta_{\mu \nu}{\bi a}^ \mu {\bi b}^ \nu = a^0b^0-a^1b^1
\text{ \ \ \ \ avec \ \ \ \ }
\eta_{\mu \nu} \equiv \langle {\bi e}_\mu, {\bi e}_\nu \rangle =
\left( \begin{array}{ c c }
 1 & 0 \\ 0 & -1
\end{array} \right)
\]
où $\eta_{\mu \nu}$ est appelé métrique de Minkowski. On définit également la pseudo-norme associée (en
relativité on parle d’intervalle) par |{\bi a}|$^2$ = $\langle {\bi a}, {\bi a} \rangle$. On a donc $\langle {\bi e}_0, {\bi e}_1 \rangle = 0$ et $|{\bi e}_0|^2=-|{\bi e}_1|^2=1$.
Dans la suite il faudra prendre garde à ne pas confondre ce produit scalaire avec le produit
scalaire utilisé en géométrie euclidienne que l’on notera ${\bi a}.{\bi b} = a^0 b^0 + a^1 b^1$ (norme associée :
$||{\bi a}||^2 = {\bi a}.{\bi a}$). La façon dont on a choisi les évènements A et B implique que $||{\bi a}||^2 = ||{\bi b}||^2$. Cette
définition du produit scalaire implique que ces vecteurs sont orthonormés au sens euclidien.
\section{Généralités sur les diagrammes d’espace-temps}%1 
\begin{enumerate}
  \item Représenter la ligne d’univers de l’observateur. Représenter les lignes d’univers des extrémités d’une règle. Comment représente-t-on le cône de lumière d’un événement E ?
Quelle propriété vérifie la pente de la ligne d’univers d’une particule ?
  \item On considère désormais un deuxième observateur \mc{O'} appartenant à un référentiel inertiel
\mc{R'} en mouvement à la vitesse v par rapport à \mc{R}. L’événement O est choisi comme origine
commune des deux repères. On note $\bi{e}_0'$ et $\bi{e}_1'$ les vecteurs de base du repère de \mc{R'}. On
rappelle que les coordonnées dans \mc{R'} et dans \mc{R} sont reliées par les transformations de
Lorentz :

\[
x'^\mu = \Lambda^\mu_\nu x^\nu
\text{ \ \ \ \ avec \ \ \ \ }
\Lambda^\mu_\nu =
\left( \begin{array}{ c c }
 \gamma & -\gamma \beta \\ -\gamma \beta & \gamma
\end{array} \right)
\]
Montrer que le passage de
$\bi{e}_\mu$ à $\bi{e}_\mu'$
se fait alors par la matrice $\Lambda^{−1}$ . En déduire
$||\bi{e}_0'||^2$ , $||\bi{e}_1'||^2$,
$\bi{e}_0'.\bi{e}_1'$ , $|\bi{e}_0'|^2$ , $|\bi{e}_1'|^2$ et $\langle \bi{e}_0', \bi{e}_1' \rangle$ Dessiner la ligne d’univers de \mc{O'} et les nouveaux
vecteurs de base.
  \item Montrer simplement que la notion de simultanéité dépend du référentiel.
  \item Trois événements, E$_1$, E$_2$ et E$_3$ se déroulent dans le référentiel \mc{R} dans l’ordre
  E$_1$E$_2$E$_3$ . Supposons que ces mêmes événements se déroulent dans l’ordre E$_3$E$_2$E$_1$ dans un autre référentiel {\mc R'} . Existe-t-il un troisième référentiel {\mc R''} pour lequel ces événements se déroulent dans
l’ordre E$_1$E$_3$E$_2$ ? Justifier graphiquement votre réponse.
  \item Montrer qu’une règle de longueur L dans un des deux référentiels a une longueur L/$\gamma$ dans
l’autre.
\end{enumerate}
\section{Retour sur le paradoxe des jumeaux}%2 
\begin{enumerate}
  \item En reprenant les notations de l’exercice 1 du TD2, représenter sur un schéma les lignes
d’univers des différents protagonistes. Répondre graphiquement aux questions 2. et 3.
  \item Faites un schéma du voyage complet en faisant intervenir les pilotes de l’exercice 2 du
TD2. Montrer alors qu’il n’y a plus de paradoxe : bien que V observe que S vieillit moins
vite que lui pendant chacune des phases aller et retour, il s’attend bien a trouver S plus
vieux que lui à l’arrivée.
\end{enumerate}
