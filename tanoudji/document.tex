\documentclass[12pt, a4paper]{report}
%\documentclass[11pt, a4paper]{article}

%====================== PACKAGES ======================
\usepackage[french]{babel}

\frenchbsetup{StandardLists=true}
\usepackage{enumitem}
\usepackage{pifont}

\usepackage[utf8x]{inputenc}
%\usepackage[latin1]{inputenc}

%pour gérer les positionnement d'images
\usepackage{float}
\usepackage{amsmath}
\usepackage{amssymb}
\DeclareMathOperator{\dt}{dt}
\usepackage{graphicx}
%\usepackage{tabularx}
\usepackage[colorinlistoftodos]{todonotes}
\usepackage{url}

%pour les informations sur un document compilé en PDF et les liens externes / internes
\usepackage[pdfborder=0]{hyperref}
\hypersetup{
	colorlinks = true
	}

%pour la mise en page des tableaux
\usepackage{array}
\usepackage{tabularx}
\usepackage{multirow}
\usepackage{multicol}
\setlength{\columnsep}{50pt}

%pour utiliser \floatbarrier
%\usepackage{placeins}
%\usepackage{floatrow}

%espacement entre les lignes
\usepackage{setspace}

%modifier la mise en page de l'abstract
\usepackage{abstract}

%police et mise en page (marges) du document
\usepackage[T1]{fontenc}
\usepackage[top=2cm, bottom=2cm, left=2cm, right=2cm]{geometry}

%Pour les galerie d'images
\usepackage{subfig}

\usepackage{pdfpages}

\usepackage{tikz}
\usetikzlibrary{trees}
\usetikzlibrary{decorations.pathmorphing}
\usetikzlibrary{decorations.markings}
\usetikzlibrary{decorations.pathreplacing,calligraphy}
%\usetikzlibrary{decorations}
\usetikzlibrary{angles, quotes}
\usepackage{verbatim}

\usepackage{appendix}

\usepackage{comment}

\usepackage{xcolor}

%\PreviewEnvironment{tikzpicture}
%\setlength\PreviewBorder{0pt}%

%====================== INFORMATION ET REGLES ======================

%rajouter les numérotation pour les \paragraphe et \subparagraphe
\setcounter{secnumdepth}{4}
\setcounter{tocdepth}{4}

\hypersetup{							% Information sur le document
pdfauthor = {Stephan Runigo},			% Auteurs
pdftitle = {Documentation},			% Titre du document
pdfsubject = {Documentation},		% Sujet
pdfkeywords = {Document},	% Mots-clefs
pdfstartview={FitH}}	% ajuste la page à la largeur de l'écran
%pdfcreator = {MikTeX},% Logiciel qui a crée le document
%pdfproducer = {} % Société avec produit le logiciel

%======================== DEFINITION COMMANDES ========================
\newcommand{\mt}[1]{\text{#1}}
\newcommand{\ul}[1]{\underline{#1}}
\newcommand{\mc}[1]{\mathcal{#1}}
\newcommand{\pt}[1]{\dot{\text{#1}}}
%======================== DEBUT DU DOCUMENT ========================
%
\begin{document}
%
%régler l'espacement entre les lignes
\newcommand{\HRule}{\rule{\linewidth}{0.5mm}}
%
% Titre, résumé, ... %
%%
%
\begin{titlepage}
%
~\\[1cm]

\begin{center}
%\includegraphics[scale=0.5]{./presentation/chambreABulle}
\end{center}

\textsc{\Large }\\[0.5cm]

% Title \\[0.4cm]
\HRule

\begin{center}
{\huge \bfseries Relativité restreinte et\\
électromagnétisme\\[0.4cm] }
\end{center}

\HRule \\[1.5cm]

\begin{center}
%\includegraphics[scale=0.3]{./presentation/ptoleme}
\end{center}

% Author and supervisor
\begin{minipage}{0.4\textwidth}
\begin{flushleft} \large
%\emph{Auteur:}\\
%Stephan \textsc{Runigo}
\end{flushleft}
\end{minipage}
\begin{minipage}{0.4\textwidth}
\begin{flushright} \large
\emph{Latex-iation:}\\
Stephan \textsc{Runigo}
\end{flushright}
\end{minipage}

\vfill
\begin{minipage}{0.4\textwidth}
\begin{flushleft} \large
Exercices de l'ENS extraits de https://www.phys.ens.fr/
\end{flushleft}
\end{minipage}
\begin{minipage}{0.4\textwidth}
\begin{flushright} \large
\end{flushright}
\end{minipage}

\vfill

% Bottom of the page
{\large \today}

\end{titlepage}

\newpage
\begin{center}
\Large
Résumé
\normalsize
\end{center}
\vspace{3cm}
\begin{itemize}[leftmargin=1cm, label=\ding{32}, itemsep=21pt]
\item {\bf Objet : } Mécanique Quantique.
\item {\bf Contenu : } Notes de cours.
\item {\bf Niveau requis : } Maitrise de sciences physiques.
\end{itemize}

\vspace{3cm} \large

Notes de cours rédigées en 1966 par Serge Haroche
\vspace{3cm}

http://enseignement.phys.ens.fr/spip.php?article110
\vspace{3cm}


%

%
% Table des matières
\tableofcontents
\thispagestyle{empty}
\setcounter{page}{0}
%
%espacement entre les lignes des tableaux
\renewcommand{\arraystretch}{1.5}
%
%====================== INCLUSION DES CHAPITRES ======================
%
~
\thispagestyle{empty}
%recommencer la numérotation des pages à "1"
\setcounter{page}{0}
\newpage
%
\part{Forme lagrangienne de la mécanique quantique}%FORME LAGRANGIENNE DE LA MECANIQUE QUANTIQUE

 \chapter{Introduction}
\section{Les Postulats de la mécanique Quantique}

Les postulats de la mécanique quantique sont de deux
natures essentiellement différentes :
\subsection{Les postulats généraux}
Ce sont les postulats fondamentaux, qui
sont communs à tous les exposés de la mécanique quantique :

\subsubsection{Principe de superposition}
Les états d'un système
physique sont linéairement superposables : si  $|\;\psi_1>$ et $|\;\psi_2$ > sont
deux états du système, $\lambda_1|\;\psi_1>+\lambda_2|\;\psi_2>$
représente également un état du système ($\lambda_1, \lambda_2$ complexes).

Rappelons en fait que les états physiques sont représentés
par des vecteurs d'un espace de Hilbert. Les grandeurs physiques sont
alors représentées par des opérateurs hermitiques à spectre complet
(observables) de cet espace. Les $|$ U$_\mt{n}>$ étant les vecteurs propres
(de valeur propre a$_\mt{n}$) de l'observable A :
\begin{center}
A $|$ U$_\mt{n}>=$ a$_\mt{n}\;|$ U$_\mt{n}>$
\end{center}
il existe entre les $|$ U$_\mt{n}>$ les relations d'orthogonalité et de
fermeture :
\begin{center}
$<$ U$_\mt{n}\;|$ U$_\mt{n'}>=\delta_\mt{nn'}$

$\sum$ $|$ U$_\mt{n}><$ U$_\mt{n}\;|\;=1$
\end{center}

% - 2
\subsubsection{Caractère probabiliste}
Le système physique étant dans
l'état | $\psi$ >, le résultat de la mesure d'une grandeur physique représentée
par l'observable A ne peut être que l'une des valeurs propres
a$_\mt{n}$ de A et la probabilité pour que ce résultat soit a$_\mt{n}$ est
| < U$_\mt{n}$ | $\psi$ > |$^2$, carré du module de l'{\it amplitude de probabilité}
< U$_\mt{n}$ | $\psi$ >.
\subsection{Les postulats de quantification}
Pour trouver les observables associées aux grandeurs physiques,
les relations de commutation entre
ces observables et l'équation d'évolution des vecteurs d'état et des
observables, on fait appel aux postulats de quantification. Ceux-ci
peuvent prendre une {\it forme différente} d'un exposé à l'autre de la
mécanique quantique. On leur impose toutefois une condition : si on
quantifie un systère classique, on doit retrouver les propriétés classiques
à la limite où l'on fait tendre $\hbar$ vers zéro. Les postulats de
quantification doivent ainsi traduire une certaine analogie entre les
mécaniques classique et quantique.

Avant de passer en revue différents formalismes possibles de
quantification, revoyons rapidement les formes que peut prendre la
mécanique classique.
\section{Les différentes formes de la mécanique classique}

\subsection{La forme Newtonienne}
Elle est basée sur l'équation fondamentale de la dynamique, $\vec{F}$ $=$ m$\vec{\gamma}$

\subsection{La forme Lagrangienne}
On définit pour chaque système un lagrangien,
fonction des "coordonnées généralisées" q et de leur dérivée par rapport
au temps. Dans le cas d'une particule définie par ces coordonnées q $=$ x,
dans un potentiel V(x), le lagrangien s'écrit

\begin{center}
$\mt{L}$ $[$ x(t), $\pt{x}$(t) $]$ $= \frac{1}{2}$ m $\pt{x}^2$ $-$ V(x) 
\end{center}
% - 3
A partir du lagrangien, on construit l'action
\[
\mt{S} = \int_\mt{t'}^\mt{t''} \mt{L} [ \mt{x(t)}, \pt{\mt{x}}\mt{(t)} ] \mt{dt}
\]
Parmi toutes les trajectoires x(t) possibles entre un point
x' à l'instant t' et un point x'' à l'instant t'', la trajectoire {\it réellement}
suivie sera celle qui rend l'action S {\it stationnaire}. C'est le principe
variationnel de Hamilton ou de moindre action. On déduit de ce
principe les équations de Lagrange, système d'équations différentielles
du second ordre, et on montre l'équivalence avec le principe fondamental
de la dynamique newtonienne.

\begin{center} \begin{tikzpicture}
\draw [->] (0,0) --++ (5,0) node [below] {t};
\draw [->] (0,0) --++ (0,4) node [left] {x};
\node at (0,0) [left]{O};
\draw [densely dashed] (0,3) node [left] {x'} --++ (1,0);
\draw [densely dashed] (1,0) node [below] {t'} --++ (0,3);
\node at (1,3) [above]{M'};
\draw [densely dashed] (0,1) node [left] {x''} --++ (4,0);
\draw [densely dashed] (4,0) node [below] {t''} --++ (0,1);
\node at (4,1) [right]{M''};
\draw [line width=1.5pt] (1,3) .. controls (1.9,3) and (3.9,1.9) .. (4,1);
\end{tikzpicture} \end{center}

\subsection{La forme Hamiltonienne}
L'évolution du système est décrite à l'aide
de la fonction de Hamilton des coordonnées q et des "moments conjugués"
p $=$ $\frac{\partial \mt{L}}{\partial \pt{q}}$, H(p, q).

La trajectoire se détermine à l'aide des équations canoniques

\[
\frac{\mt{dq}_\mt{i}}{\mt{dt}} = \frac{\partial \mt{H}}{\partial \mt{p}_\mt{i}}
\ \ \ ;\ \ \ \ 
\frac{\mt{dp}_\mt{i}}{\mt{dt}} = \frac{\partial \mt{H}}{\partial \mt{q}_\mt{i}}
\]
qui sont du premier degré.

C'est en général à partir de cette dernière forme que l'on bâti
la mécanique quantique.
% - 4
\section{Forme habituelle des règles de quantification}

On part de la forme hamiltonienne de la mécanique classique
et on postule (principe de correspondance) la relation entre commutateurs
et crochets de Poisson :

\begin{center}
$[$ A, B $]$ $=$ i$\hbar$ \{ A, B \}
\end{center}

Rappelons la définition des crochets de poisson :
\[
\{\mt{ A, B }\} = \sum_\mt{i}
(\frac{\partial \mt{A}}{\partial \mt{q}_\mt{i}}\frac{\partial \mt{B}}{\partial \mt{p}_\mt{i}}
- \frac{\partial \mt{A}}{\partial \mt{p}_\mt{i}}\frac{\partial \mt{B}}{\partial \mt{q}_\mt{i}}).
\]
On en déduit notamment les commutateurs fondamentaux $[$ q, p $]$ $=$ i$\hbar$,
l'équation de Schrödinger, etc.

Ce formalisme hamiltonien a l'avantage de se prêter aisément
aux calculs. Cependant, la forme lagrangienne de la mécanique classique
est plus fondamentale que la forme hamiltonienne : elle découle en effet
d'un principe variationnel et peut ainsi se généraliser à tout système
physique régi par un tel principe. Enfin, elle fait jouer au temps un
rôle plus symétrique que la forme hamiltonienne, où le temps est très
particularisé. Elle pourra donc prendre plus facilement une forme invariante relativiste.
Il est donc intéressant de donner une règle de quantification de la mécanique lagrangienne.
Pour bien comprendre le sens
physique de cette règle, analysons d'abord le passage entre l'optique
géométrique et l'optique ondulatoire :
\section{Passage de l'optique géométrique à l'optique ondulatoire}
L'optique géométrique est régie par le Principe de Fermat :

\begin{center} \begin{tikzpicture}
\draw [->] (0,0) --++ (5,0) node [below] {y};
\draw [->] (0,0) --++ (0,4) node [left] {x};
\node at (0,0) [left]{O};
\draw [densely dashed] (0,3) node [left] {x'} --++ (1,0);
\draw [densely dashed] (1,0) node [below] {y'} --++ (0,3);
\node at (1,3) [above]{M'};
\draw [densely dashed] (0,1) node [left] {x''} --++ (4,0);
\draw [densely dashed] (4,0) node [below] {y''} --++ (0,1);
\node at (4,1) [right]{M''};
\draw [line width=1.5pt] (1,3) .. controls (1.9,3) and (3.9,1.9) .. (4,1);
\end{tikzpicture} \end{center}
 
Le chemin suivi par la lumière de M' à M'' correspond au
temps de parcours {\it minimum} ; de façon plus précise, il correspond à un
temps stationnaire, c'est-à-dire qu'une variation au premier ordre du
chemin autour du chemin suivi entraîne une variation du temps au second
ordre. Notons que les chemins en question ici sont des chemins d'espace
ordinaire (décrits en xy) et non des chemins d'espace temps (décrits en
xt) comme dans le principe de Hamilton.

Un exemple du principe de Fermat est fourni par les lois de
la réflexion, le chemin M'PM'' suivi par la lumière entre M' et M'' étant
extremum parmi tous les chemins M'QM'' possibles.

\begin{center} \begin{tikzpicture}
%miroir
\draw[thick] (0,0)--(8.5,0);
\foreach \z in {0.2,0.5, ...,8.5}
{\draw (\z,0)--++(-0.1,-0.1);}
%rayons
\draw [thick] (0.5,3) node [left]{M'} -- (5,0) node [below]{P} -- (8,2) node [right]{M''};
\draw [thick] (0.5,3) -- (3,0) node [below]{Q} -- (8,2);
%symétrique
\draw [dashed] (0.5,3) -- (0.5,-3) node [left]{N'};
%virtuels
\draw [dashed] (3,0) -- (0.5,-3) -- (5,0);
\end{tikzpicture} \end{center}

Une question importante se pose alors : comment le rayon
lumineux trouve-t-il le chemin de temps stationnaire ? La réponse vient
du caractère ondulatoire de la lumière qui lui permet de “sentir” tous
les chemins possibles.

L'{\it optique ondulatoire} peut se résumer dans le {\bf Principe
d'Huyghens} : Soit $\phi$ (M'', M') l'amplitude de l'onde lumineuse issue de
M' et arrivant en M'' (l'intensité de l'onde est alors | $\phi$ (M'', M') |$^2$).
Le principe d'Huyghens dit que $\phi$ (M'', M') est une somme de contributions,
\begin{center} \begin{tikzpicture}
% axes
\draw [->] (0,0) --++ (5,0) node [below] {y};
\draw [->] (0,0) --++ (0,4) node [left] {x};
\node at (0,0) [left]{O};
% coordonnées et points
\draw [densely dashed] (0,3) node [left] {x'} --++ (1,0);
\draw [densely dashed] (1,0) node [below] {y'} --++ (0,3);
\node at (1,3) [above]{M'};
\draw [densely dashed] (0,1) node [left] {x''} --++ (4,0);
\draw [densely dashed] (4,0) node [below] {y''} --++ (0,1);
\node at (4,1) [right]{M''};
% Chemins
\draw [line width=1.5pt] (1,3) .. controls (1.9,3) and (3.9,1.9) .. (4,1);
\draw [line width=1.5pt] (1,3) .. controls (1.9,3.5) and (3.9,2.4) .. (4,1);
\draw [line width=1.5pt] (1,3) .. controls (1.9,2.5) and (3.9,1.4) .. (4,1);
\draw [line width=1.5pt] (1,3) .. controls (1.9,1.5) and (3.9,0.9) .. (4,1);
\end{tikzpicture} \end{center}
%-6
une pour chaque chemin partant de M' et arrivant en M''. La contribution
de chaque chemin est un nombre complexe dont le module peut être considéré
en première approximation comme constant et dont la phase est
2i$\pi \frac{\mt{t}_\mt{c}}{\mt{T}}$, t$_\mt{c}$ étant le temps mis pour parcourir le chemin en question
et T la période de vibration. Par exemple, dans le cas de la réflexion, le chemin M'QM'' contribue à
$\phi$ (M'', M') par le terme exp(2i$\pi \frac{\mt{M'Q + QM''}}{\mt{cT}}$). (cT $=$ $\lambda$ : longueur d'onde).

Dans le cas limite où toutes les dimensions sont grandes devant
la longueur d'onde $\lambda$ (mathématiquement T $\to$ 0 ou encore $\lambda$ = cT $\to$ 0), d'un
chemin à l'autre la phase varie très vite, Les seuls chemins dont les
contributions ne se détruisent pas par interférence, sont ceux pour lesquels la phase est stationnaire
(c'est-à-dire ceux pour lesquels une variation du chemin au premier ordre entraîne une variation de phase
au deuxième ordre). Tout se passe comme si on pouvait ignorer tous les chemins autres que les chemins de
phase stationnaire, c'est-à-dire de temps
stationnaire. Physiquement, cela veut dire que rien n'est changé à l'amplitude  $\phi$ (M'', M')
si on interpose
un écran qui cache les chemins non
stationnaires, à condition que le diaphragme ait des {\it dimensions grandes}
devant $\lambda$. Tout se passe donc alors comme si la lumière suivait les chemins du {\it principe de Fermat}.

Le Principe de Huyghens contient donc le principe de Fermat à
la limite où $\lambda$ $\to$ 0.

Dans le cas où $\lambda$ ne peut être considéré comme négligeable,
l'optique géométrique ne {\it s'applique plus} et les phénomènes doivent être
décrits à l'aide du principe d'Huyghens : c'est le cas des expériences
de diffraction.

Le passage de la mécanique classique à la forme lagrangienne
de la mécanique quantique repose sur la même idée de base.
% - 7
\section{Idée de base de la formulation lagrangienne de la mécanique quantique}
La mécanique classique est basée sur le {\bf principe de Hamilton} :
le chemin x(t) suivi par la particule correspond à l'action stationnaire
(remarquons que nous raisonnons à nouveau dans l'espace temps (x, t) et
non plus dans l'espace ordinaire (x y)).

\begin{center} \begin{tikzpicture}
% axes
\draw [->] (0,0) --++ (5,0) node [below] {t};
\draw [->] (0,0) --++ (0,4) node [left] {x};
\node at (0,0) [left]{O};
% coordonnées et points
\draw [densely dashed] (0,3) node [left] {x'} --++ (1,0);
\draw [densely dashed] (1,0) node [below] {t'} --++ (0,3);
\node at (1,3) [above]{M'};
\draw [densely dashed] (0,1) node [left] {x''} --++ (4,0);
\draw [densely dashed] (4,0) node [below] {t''} --++ (0,1);
\node at (4,1) [right]{M''};
% Chemins
\draw [line width=1.5pt] (1,3) .. controls (1.9,3) and (3.9,1.9) .. (4,1);
\draw [line width=1.5pt] (1,3) .. controls (1.9,3.5) and (3.9,2.4) .. (4,1);
\draw [line width=1.5pt] (1,3) .. controls (1.9,2.5) and (3.9,1.4) .. (4,1);
\draw [line width=1.5pt] (1,3) .. controls (1.9,1.5) and (3.9,0.9) .. (4,1);
\end{tikzpicture} \end{center}
Comment la particule trouve-t-elle le chemin d'action minimum ?
La réponse est fournie par la théorie quantique : elle “sent” les autres
chemins grâce à son aspect ondulatoire.

Appelons < x''t'' | x't' > l'{\it amplitude de probabilité} pour que la
particule, soit en x'' à l'instant t'' après avoir été en x' à l'instant t'.
< x''t'' | x't' > est une {\it somme} de contributions, une pour chaque chemin
d'espace temps partant de x't' et arrivant en x''t'', La contribution d'un
chemin donné s'écrit : N exp(2i$\pi \frac{\mt{S}}{\mt{h}}$). N est un coefficient de normalisation
indépendant du chemin; S est l'action classique $\int_\mt{t'}^\mt{t''}$ L (x,$\pt{x}$) dt calculée
le long du chemin considéré, h la constante de Planck, qui a bien les
dimensions d'une action (S/h est donc sans dimensions). C'est le {\bf principe
de quantification de Feynman}.

A la limite où $\hbar$ $\to$ 0, les seuls chemins qui contribuent à
< x''t'' | x't' > sans se détruire par interférence correspondent à une
phase stationnaire, donc à une action stationnaire.

% -8
Le Principe de Feynman contient donc le principe de Hamilton
à la limite où $\hbar$ $\to$ 0.

Il existe donc une analogie précise entre l'optique géométrique
et la mécanique classique d'une part, l'optique ondulatoire et la mécanique quantique d'autre part,
l'action remplaçant le temps et la constante
de Planck la période de l'onde lunineuse,

On peut donc dire qu'une expérience de mécanique quantique est
une expérience de diffraction dans l'espace-temps; toutes les propriétés
ondulatoires de la matière apparaissent clairement dans cette formulation
de la mécanique quantique, dont il reste encore à montrer qu'elle est
bien équivalente aux autres.

Résumons, pour terminer, les avantages de ce nouveau point de vue :

1°) Il donne un sens physique plus clair à la correspondance
entre la mécanique classique et la mécanique quantique.

2°) On raisonne dans l'espace-temps, ce qui permet un passage
à la relativité très aisé : il suffit de remplacer dt par d$\tau$, $\tau$ étant le
temps propre de la particule, et prendre pour Lagrangien L une fonction
scalaire d'espace-temps. L'action S = $\int$ L d$\tau$ est alors un scalaire et la
théorie acquiert l'invariance relativiste.

Dans la formulation hamiltonienne au contraire, le temps
est très privilégié et la covariance relativiste des équations n'est pas
apparente.

3°) Le point de vue est plus global : au lieu de considérer des
amplitudes de probabilité pour un état à un instant donné, on associe une
amplitude de probabilité à une {\it histoire entière}, ou chemin, du système.
Ce point de vue se révèle souvent plus fructueux et plus intéressant.

4°) Ce procédé est applicable à des systèmes autres que mécaniques,
à la seule condition que leurs équations classiques découlent d'un
{\it principe variationnel} : c'est le cas du champ électromagnétique dont les
% - 9
équations de Maxwell peuvent se déduire d'un Lagrangien et d'un principe
d'action stationnaire.

La quantification se fait alors en associant à chaque histoire
du champ une amplitude de probabilité proportionnelle à exp(2i$\pi \frac{\mt{S}}{\mt{h}}$).

On voit ainsi l'importance de ce formalisme en théorie quantique
des champs.

%-II - 10
\chapter{Postulats de quantification de Feynman}

\hfill (R.P. Feynman, Fev. Mod. Phys. 20, 2, 1948, p. 367)

\vspace{.5cm}
Dans ce chapitre II, on énonce de façon précise les postulats de Feynman
(\S A). On les applique ensuite, d'abord au cas simple d'une particule libre
(\S B); puis, après avoir défini la fonction d'onde (\S ch au cas d'une particule dens un potentiel V(x), ce qui permet d'obtenir l'équation de Schrödinger .
(\S D) et de prouver ainsi l'équivalence entre la nouvelle formulation de la
mécanique quantique et les précédentes.

\section{Enoncé des postulats}

Les postulats de Feynman permettent de calculer la quantité
< x''t'' | x't' > qui représente l'amplitude de probabilité pour que la
particule, au point x' à l'instant t', se trouve au point x'' à l'instant
t''.

Avant d'énoncer les postulats, voyons comment cette amplitude
se présente dans la formulation usuelle de la mécanique quantique :

\subsection{Etude de < x''t'' | x't' >}

\subsubsection{dans le point de vue de Schrödinger}

Rappelons que dans le point de vue de Schrödinger, les vecteurs
d'état qui représentent le système physique, | $\psi$(t) > dépendent du temps.
Les observables, A, par contre, sont indépendantes du temps, ainsi, bien
entendu, que leurs états propres. Par exemple, X étant l'observable représentant
la position de la particule, on a l'équation aux valeurs propres,
indépendante du temps X | x > $=$ x | x >.

Dans la représentation X, le vecteur | $\psi$(t) > est représenté
par sa fonction d'onde

< x | $\psi$(t) > $=$ $\psi$(x,t).

L'évolution du vecteur d'état | $\psi$(t) > peut être décrite à l'aide
de l'opérateur d'évolution U(t'',t') qui est unitaire (conséquence du fait
que La translation dans le temps est une opération de symétrie) :

| $\psi$(t'') > $=$ U(t'',t') | $\psi$(t') >

Dans ce formalisme, l'amplitude de probabilité s'écrit :

< x''t'' | x't' > $=$ < x'' | U(t'', t') | x' >

% 11
Remarque : < x'' | U(t'',t') | x' > représente la fonction de Green
de l'équation de Schrödinger : c'est la valeur au point x'' à l'instant t''
de la solution de l'équation de Schrödinger qui se réduit à l'instant
t $=$ t' à une fonction de Dirac au point x', $\delta$(x - x'). On appelle
< x''t'' | x't' > également le “propagateur'' de l'équation de Schrödinger.

\subsubsection{dans le point de vue de Heisenberg}

Rappelons que dans le point de vue de Heisenberg, les vecteurs
d'état | $\psi$ >, sont indépendants du temps. Les observables, et en conséquence,
les états propres de ces observables dépendent au contraire du
temps : l'équation aux valeurs propres de l'observable X(t) s'écrit :

X(t) | x,t > $=$ x | x,t >.

A l'instant t, le vecteur | $\psi$ > peut être représenté par sa projection
sur la base complète | x,t > : on obtient la fonction d'onde
< x,t | $\psi$ > $=$ $\psi$(x,t).

Il est évident que ce point de vue est équivalent à celui de
Schrödinger : seul le mouvement relatif du vecteur d'état et des vecteurs
de base des observables a une signification physique et ce mouvement
relatif est le même dans les deux points de vue.

Quelle est, dans ce formalisme, l'amplitude de probabilité pour
que la particule, en x' à l'instant t', se trouve en x'' à l'instant t'' ?
Le vecteur d'état | $\psi$ >, indépendant du temps, doit être vecteur propre
de X(t') correspondant à la valeur propre x'. | $\psi$ > est donc confondu
avec | x',t' >. L'amplitude de probabilité cherchée est égale à la projection
de | $\psi$ > sur l'état propre de l'opérateur X(t''), de valeur rropre
x'', que nous avons désigné par | x'',t'' >. Elle est donc égale à
| x'',t'' | $\psi$ > $=$ < x'',t'' | x',t' >.

Nous voyons donc que la notation que nous avons adoptée a
priori pour cette amplitude est celle du point de vue de Heisenberg.

 
%-12
\subsection{Les postulats de Feynman}

{\bf Postulat I} : < x''t'' | x't' > est constitué par une somme de contributions,
une pour chaque chemin d'espace-temps reliant (x't') à (x'' t'').

La notion de sommation sur chaque chemin d'espace-temps est
ambigüe et pose des problèmes d'analyse fonctionnelle délicats,
Feynman précise cette notion de la façon suivante.

Au lieu de se définir un chemin par la donnée d'une fonction
continue x(t) entre (x't') et (x''t''), on divise l'intervalle t't'' en
n intervalles égaux $\epsilon$ par (n-1) temps intermédiaires t$_\mt{i}$,
t$_\mt{i+1}$, ... et
on définit un chemin par la donnée de la suite double

\begin{center}
t' t$_1$ t$_2$ ... t$_\mt{i}$ t$_\mt{i+1}$ ... t$_\mt{n-1}$ t''

x' x$_1$ x$_2$ ... x$_\mt{i}$ x$_\mt{i+1}$ ... x$_\mt{n-1}$ x''
\end{center}

que l'on note encore
\begin{center}
M' M$_1$ M$_2$ ... M$_\mt{i}$ M$_\mt{i+1}$ ... M$_\mt{n-1}$ M''
\end{center}

On fait ensuite tendre $\epsilon$ vers zéro (ou encore n $=$ (t''-t')/$\epsilon$
vers l'infini).

Cependant, aussi petit que soit $\epsilon$, la donnée d'une suite des M
ne suffit pas à déterminer le chemin continu x(t) :

\begin{center} \begin{tikzpicture}
% axes
\draw [->] (0,0) --++ (8,0) node [below] {t};
\draw [->] (0,0) --++ (0,4) node [left] {x};
\node at (0,0) [left]{O};
% coordonnées et points
\draw [densely dashed] (0,3) node [left] {x'} --++ (1,0);
\draw [densely dashed] (1,0) node [below] {t'} --++ (0,3);
\node at (1.2,3) [above left]{M'};
\draw [densely dashed] (0,2.5) node [left] {x''} --++ (7,0);
\draw [densely dashed] (7,0) node [below] {t''} --++ (0,2.5);
\node at (7,2.8) [right]{M''};
% intermédiaires
\draw (1.5,-0.05) node [below] {t$_1$} --++ (0,0.1);
\draw [line width=1.5pt] (1.5,3.3) --++ (0,0.2) node [above]{M$_1$};
\draw (2,-0.05) node [below] {t$_2$} --++ (0,0.1);
\draw [line width=1.5pt] (2,3.5) --++ (0,0.2) node [above]{M$_2$};
% intermédiaires fin
\draw (6.5,-0.05) node [below] {t$_{\mt{n}-1}$} --++ (0,0.1);
\draw [line width=1.5pt] (6.5,2) --++ (0,0.2) node [left]{M$_{\mt{n}-1}$};
% intermédiaires milieu
\draw (3.5,-0.05) node [below] {t$_\mt{i}$} --++ (0,0.1);
\draw [line width=1.5pt] (3.5,3.05) --++ (0,0.2) node [right]{M$_\mt{i}$};
\draw (4,-0.05) node [below] {t$_{\mt{i}+1}$} --++ (0,0.1);
\draw [line width=1.5pt] (4,2.6) --++ (0,0.2) node [right]{M$_{\mt{i}+1}$};
% Chemin
\draw [line width=1.5pt] (1,3) .. controls (2,4) and (3,3.7) .. (4,2.7);
\draw [line width=1.5pt] (4,2.7) .. controls (5,1.7) and (6,1.5) .. (7,2.5);
\end{tikzpicture} \end{center}
%-13—

Feynman définit le chemin de façon complète en postulant qu'il
faut joindre deux points consécutifs M$_\mt{i}$ et M$_\mt{i+1}$ par le
chemin classique
qui passe par ces deux points (si le Lagrangien du mouvement ne dépend
que de x et $\dot{\mt{x}}$ et non des dérivées supérieures, les équations de
Lagrange
sont du second ordre, et la donnée des deux points M$_\mt{i}$ et
M$_\mt{i+1}$ suffit à définir le chemin classique).

Grâce à cette définition complète du chemin, nous avons pu,
pour chaque valeur de $\epsilon$ (donc de n), paramétrer, à l'aide des
x$_\mt{i}$, l'ensemble des chemins. La sommation sur les chemins revient
alors à intégrer
sur les paramètres x$_\mt{i}$ et on peut écrire :

\[
\tag{1} \mt{< x''t'' | x't' >} = \lim_{\,\epsilon \to \,0} \int .. \int
\mt{dx}_1 .. \mt{dx}_\mt{i} .. \mt{dx}_\mt{n-1}
\phi (\mt{x'},\mt{x}_1 .. \mt{x}_\mt{i} .. \mt{x}_\mt{n-1},\mt{x''})
\]
$\phi$(x',x$_1$ .. x$_\mt{i}$ .. x$_\mt{n-1}$,x'') représente la contribution d'un chemin tel
qu'il a été défini plus haut. le second postulat nous fournira la valeur
de $\phi$.

Remarque : Le fait que < x''t'' | x't' > est une somme de contributions est
en fait une conséquence du principe général de superposition.

X(t) étant une observable, nous avons en effet la relation de
fermeture

\begin{center}
$\int$ | x t > < x t | dx $=$ 1
\end{center}

En injectant les relations de fermeture relatives aux instants
t$_1$, .. t$_\mt{i}$, .. t$_\mt{n-1}$, nous obtenons :

\begin{flushleft}
< x''t'' | x't' > $=$ $\int$ .. $\int$ dx$_1$ .. dx$_\mt{i}$ .. dx$_\mt{n-1}$
\end{flushleft}
\begin{flushright}
< x''t'' | x$_1$t$_1$ > < x$_1$t$_1$ | ... | x$_\mt{n-1}$t$_\mt{n-1}$ > < x$_\mt{n-1}$t$_\mt{n-1}$ | x''t'' >
\end{flushright}

%  14

{\bf Postulat II} : La contribution de chacun des chemins définis plus haut est
(à un facteur de normalisation près, N, qui est le même pour tous les
chemins) égale à exp $\frac{\mt{i}}{\hbar}$ S, S étant l'action classique calculée le long
du chemin.

Remarquons tout d'abord que $\hbar$ a les dimensions d'une action
et que $\frac{\mt{S}}{\hbar}$ est bien, de ce fait, sans dimensions.

D'autre part, d'après la définition du chemin que nous avons
donnée dans le postulat I

\[
\mt{S} = \sum_{\mt{i}=0}^{\mt{n}-1} \mt{S} (\mt{x}_\mt{i}, \mt{x}_\mt{i+1})
 \ \ \ \ \ \ \ \ \ (\mt{en posant }\mt{x}_0 = \mt{x'})
\]

S (x$_\mt{i}$, x$_\mt{i+1}$) représentant l'action prise le long du chemin suivi par
une particule classique entre M$_\mt{i}$ et M$_\mt{i+1}$ c'est-à-dire encore

\[
\mt{S} (\mt{x}_\mt{i}, \mt{x}_\mt{i+1}) = \mt{min } \int_{\mt{t}_\mt{i}}^{\mt{t}_\mt{i+1}}
\mc{L} [ \mt{x(t)}, \dot{\mt{x}}\mt{(t)} ] \mt{dt}
\]
Nous pouvons alors écrire :
\[
\tag{2} \mt{< x''t'' | x't' > } = \lim_{\,\epsilon \to \,0} \mt{N} \int...\int
\mt{dx}_1..\mt{dx}_\mt{i}..\mt{dx}_\mt{n-1} \exp\frac{\mt{i}}{\hbar}
\sum_{\mt{i}=0}^{\mt{n}-1} \mt{S} (\mt{x}_\mt{i}, \mt{x}_\mt{i+1})
\]

C'est cette relation fondamentale qui nous servira par la suite.

{\bf Remarques :}

a) Quelle est la dimension de N ?

< x''t'' | x't' > a pour dimension L$^{-1}$. En effet, pour t'' $=$ t'',
l'amplitude se réduit à < x'' | x' > $= \delta$  (x'' - x'). Or la fonction
$\delta$ a pour dimension l'inverse d'une longueur (d'après la relation
fondamentale $\int \delta$(x)dx $=1$). L'intégrale de l'expression de < x''t'' | x't' > 
a pour dimension L$^{n-1}$. N a donc la dimension L$^{-n}$ et nous écrivons
\[
\mt{N}=\frac{1}{\mt{A}^\mt{n}}
\ \ \ \ \ \ \ \ \ \ \ \mt{(A ayant les dimensions d'une longueur).}
\]
 
%-15
b) Que se passe-t-il à la limite classique ($\hbar \to 0$)?

Les seuls chemins dont les contributions ne se détruisent pas par
interférence sont ceux qui correspondent à l'action stationnaire
par rapport à la variation des x$_\mt{i}$ c'est-à-dire les chemins classiques
du principe de Hamilton (cf Introduction).

Nous allons maintenant appliquer les postulats de Feynman au
calcul du propagateur dans un cas simple : celui de la particule libre.

\section{Calcul effectif de < x''t'' | x't' > pour la particule libre}

\subsection{Expression de S(x$_\mt{i}$, x$_\mt{i+1}$)}

Pour la particule libre, le chemin classique entre M$_\mt{i}$ et M$_\mt{i+1}$
est la ligne droite. La vitesse v est constante et égale à
(x$_\mt{i+1}$ - x$_\mt{i}$)/$\epsilon$

\[
\mt{S}(\mt{x}_\mt{i}, \mt{x}_\mt{i+1}) = \int_{\mt{t}_\mt{i}}^{\mt{t}_\mt{i+1}}
\frac{1}{2}\mt{mv}^2 = \frac{1}{2}\mt{mv}^2 \epsilon
 = \frac{\mt{m}}{2\epsilon}(\mt{x}_\mt{i+1}-\mt{x}_\mt{i})^2
\]
On a donc :
\[
\tag{3} \mt{< x''t'' | x't' > } = \lim_{\,\epsilon \to \,0} \mt{A}^{-\mt{n}}
\int...\int
\exp\frac{\mt{im}}{2\hbar\epsilon}
[(\mt{x}_1-\mt{x'})^2 + (\mt{x}_2-\mt{x}_1)^2 +..+ (\mt{x'' }-\mt{x}_\mt{n-1})^2]
\mt{dx}_1...\mt{dx}_\mt{n-1} 
\]

Il y a dans l'exposant de l'exponentielle autant de carrés que d'intervalles

entre M' et M'', c'est-à-dire n.

\subsection{Changement de variables}

Effectuons le changement de variables

\hspace{2cm} x$_1$ $-$ x' $=$ u$_1$

\hspace{2cm} x$_2$ $-$ x$_1$ $=$ u$_2$

\hspace{2.6cm} :

\hspace{2cm} x'' $-$ x$_{\mt{n}-1}$ $=$ u$_\mt{n}$

%16

Les variables u$_1$, u$_2$,... u$_{\mt{n}-1}$ sont indépendantes. Mais les
n variables u$_1$, u$_2$,... u$_\mt{n}$ sont liées par la relation
u$_1$ $+$ u$_2$ $+$ ... $+$ u$_\mt{n}$ $=$ x'' - x'. Si l'on veut intégrer sur les n variables u$_\mt{n}$ il faut tenir compte
de cette relation en introduisant la fonction
\begin{center}
$\delta$ (x'' $-$ x' $-$ u$_1$ $-$ u$_2$ ... $-$ u$_\mt{n}$)
\end{center}
On a alors
\[
\int...\int \mt{dx}_1...\mt{dx}_\mt{n-1} \to \int...\int
 \delta (\mt{x''} - \mt{x'} - \mt{u}_1 - \mt{u}_2 \ ... - \mt{u}_\mt{n})
\mt{du}_1...\mt{du}_\mt{n}
\]
Prenons pour $\delta$ la représentation intégrale
$\delta$ (x) $=\frac{1}{2\pi\hbar}\int$e$^{\mt{kx/}\hbar}$   dk
et posons x'' $-$ x' $=\xi$ ($\epsilon=\theta$/n) et
t'' $-$ t' $=\theta$.
Il vient alors
\begin{center}
<x''t''|x't'> $=\frac{\mt{A}^{-\mt{n}}}{2\pi\hbar} \int..\int$ du$_1$..du$_\mt{n}$
dk exp $\frac{\mt{i}}{\hbar}[(\mt{u}_1^2+\mt{u}_2^2+..+\mt{u}_\mt{n}^2)\frac{\mt{m}}{2\epsilon}
+\mt{k}(\xi-\mt{u}_1-\mt{u}_2-..-\mt{u}_\mt{n})]$
\end{center}
Nous allons réaliser l'intégration par étapes :
\subsection{Intégration sur u1, u2, ... un}
Nous pouvons écrire :
\begin{center}
exp $\frac{\mt{i}}{\hbar}[\frac{\mt{m}}{2\epsilon}\mt{u}_1^2-\mt{ku}_1]$ $=$
exp $\frac{\mt{im}}{2\epsilon\hbar}[\mt{u}_1-\frac{\epsilon\mt{k}}{m}]^2$
exp $\frac{-\mt{ik}^2\epsilon}{2\mt{m}\hbar}$
\end{center}
D'où
\begin{center}
<x''t''|x't'> $=\frac{\mt{A}^{-\mt{n}}}{2\pi\hbar} [ \int$
exp $\frac{\mt{im}}{2\epsilon\hbar}(\mt{u}-\frac{\epsilon\mt{k}}{\mt{m}})^2 $du$]^\mt{n}
\times \int$  exp$\frac{\mt{i}}{\hbar}[\mt{k}\xi-\frac{\mt{k}^2\theta}{2\mt{m}}]$dk 
\end{center}
(nous avons remplacé n$\epsilon$ par $\theta$)

%-17
or l'intégrale en u se déduit par changement de variable de
l'intégrale $\int$ exp (iu$^2$)du $=$ $\sqrt{\mt{i}\pi}$ qui se calcule grâce à une intégration
de la fonction de variable complexe exp($-$z$^2$) sur le contour du plan complexe
indiqué ci-dessous (méthode des résidus).
\begin{center}
\begin{tikzpicture}
\draw (0,0) node[above] {$O$};
\draw [very thick] (0,0) -- (3,0) ;
\draw [very thick] (2.3,-0.2) -- (2,0) -- (2.3,0.2) ;
\draw (1,0) arc (0:-45:1) ;
\draw [very thick] (3,0) arc (0:-45:3) -- (0,0);
\draw [very thick] (1.25,-1.55) -- (1.6,-1.6) -- (1.55,-1.25);
\draw [very thick] (2.4,-1.2) -- (2.82,-1) -- (3.05,-1.35);
\draw (-22:1.3) node {$\dfrac{\pi}{4}$};
\end{tikzpicture}
\end{center}
On trouve alors
\[
\tag{4}\mt{<x''t''|x't'>} = \frac{1}{2\pi\hbar} \left[ \frac{1}{\mt{A}}
\sqrt{\frac{2\pi\hbar\epsilon\mt{i}}{\mt{m}}} \right]^\mt{n}
\times \int \mt{exp} \frac{\mt{i}}{\hbar}\left[\mt{k}\xi-\frac{\mt{k}^2\theta}{2\mt{m}}\right]dk 
\]

\subsection{Intégration sur k}%4°)

Faisons à nouveau apparaître un carré parfait :
\begin{center}
exp$\frac{\mt{i}}{\hbar}[\mt{k}\xi-\frac{\mt{k}^2\theta}{2\mt{m}}]$ $=$
exp$\frac{-\mt{i}}{\hbar}[\mt{k}\sqrt{\frac{1}{2\mt{m}}}-\sqrt{\frac{\mt{k}^2\theta}{2\mt{m}}}]^2$
\end{center}
L'intégration sur k est analogue ä l'intérration sur u du 3°) et on
trouve enfin :
\[
\tag{5}\mt{<x''t''|x't'>} =\left[ \frac{1}{\mt{A}}
\sqrt{\frac{2\pi\hbar\epsilon\mt{i}}{\mt{m}}} \right]^\mt{n}
e^{-\mt{i}\pi/4}\sqrt{\frac{\mt{m}}{2\pi\hbar\theta}}
\ \mt{exp}\ \frac{\mt{i}\mt{m}\xi^2}{2\hbar\theta}
\]
{\bf Détermination de A} : Pour que le résultat à la limite où  $\epsilon$
$\to$ O (ou n $\to$ $\infty$)
soit indépendant de $\epsilon$ (ou de n), il faut prendre
\[
\tag{6} A = \sqrt{\frac{2\pi\hbar\epsilon\mt{i}}{\mt{m}}}
\]
Nous constatons que A a bien les dimensions d'une longueur.

 
%-18
{\bf Remarques :}

{\bf a)} Si dans la formule (k), on fait brutalement t'' $-$ t' $=$ $\theta$ $=$ 0, on doit
retrouver < x''t'' | x't' > $=$ $\delta$ (x'' - x'), ce qui est bien le cas si A est
donné par (6) (on trouve alors la forme intégrale de la relation $\delta$). C'est
une autre façon de déterminer A.

{\bf b)} Une fois A choisi, la formule (5) ne dépend pas de $\epsilon$. On obtient ainsi,
quel que soit $\epsilon$, la vraie valeur de < xt'' | x't' >. Ceci n'est vrai que
dans le cas simple de la particule libre. Dans les autres cas,
< x''t'' | x't' > sera donné par une limite pour $\epsilon$ $\to$ 0.

Donnons enfin l'expression définitive du propagateur :

\[
\tag{7} \mt{<x''t''|x't'>} =  e^{-\mt{i}\pi/4}
\left[\frac{\mt{m}}{2\pi\hbar(\mt{t''}-\mt{t'})}\right]^{1/2}
\mt{exp}\frac{\mt{im}}{2\hbar}\frac{(\mt{x''}-\mt{x'})^2}{\mt{t''}-\mt{t'}}
\]
\subsection{Calcul de < x''t'' | x't' > dans la formulation habituelle} % 5°)

 

Nous avons < x''t''|x't' > $=$ < x''|U(t'',t')|x'' > $=$
< x''| exp[-i$\frac{\mc H}{\hbar}$(t''-\,t')$]$|x'' >
où $\mc H$ $=$ P$^2$/2m représente le hamiltonien de la particule
libre  $=$ P2/2m (P opérateur impulsion d'états propres | p >)

\[
\mt{< x''| }\mt{e}^{-\mt{i}\frac{\mc H}{\hbar}\mt{(t''-\,t')}}\mt{ |x'' >} = \iint
\mt{< x''|p><p| }\mt{e}^{-\mt{i}\frac{\mc H}{\hbar}\mt{(t''-\,t')}}\mt{ |p><p|x''> dp dp'}
\]

(en injectant deux fois la relation de fermeture $\int$| p > < p | dp $=$ 1).

Or nous avons les relations classiques
\begin{center}
< x'' | p > $= \frac{1}{\sqrt{2\pi\hbar}}\mt{e}^{\frac{\mt{ipx''}}{\hbar}}$
\end{center}
\begin{center}
< p | x' > $= \frac{1}{\sqrt{2\pi\hbar}}\mt{e}^{-\frac{\mt{ipx'}}{\hbar}}$
\end{center}

%
\begin{center}
D'autre part < p | $\mt{e}^{-\mt{i}\frac{\mt{p}^2}{2\mt{m}\hbar}\mt{(t''-t')}}$
| p' > $= \mt{e}^{-\frac{\mt{ip}^2\mt{(t''-t')}}{2\mt{m}\hbar}} \delta$(p$-$p')
\end{center}
\begin{center}
Finalement < x''t'' | x't' >
$=\frac{1}{2\pi\hbar}
\int\mt{exp}\frac{1}{\hbar}[$
p(x''-x')
$-\frac{\mt{p}^2}{2\mt{m}}$(t''-t')$]$dp
\end{center}

ce qui n'est autre que la formule (4) (avec le choix convenable de A).
{\it Les deux méthodes conduisent donc bien au mêne résultat.}

{\bf Remarque} : Dans l'équation de Schrödinger de la particule libre :
\begin{center}
i$\hbar\frac{\partial\psi}{\partial\mt{t}}-\frac{\hbar^2}{2\mt{m}}\frac{\partial^2\psi}{\partial\mt{x}^2} = 0$
\end{center}
remplaçons formellement it par $\tau$ et $\frac{\hbar}{2\mt{m}}$ par D, on obtient l'équation de la diffusion
\begin{center}
$\frac{\partial\psi}{\partial\tau}-\mt{D}\frac{\partial^2\psi}{\partial\mt{x}^2} = 0$.
\end{center}

En effectuant le même changement dans l'expression (7) du propagateur,
on obtient 1e propagateur bien connu de l'équation de 1a diffusion
\begin{center}
$\frac{1}{\sqrt{4\pi\mt{D}\theta}}$ exp $-\frac{\xi^2}{4\mt{D}\theta}$
\end{center}

\section{La fonction d'onde}
Plaçons-nous dans le point de vue de Heisenberg et écrivons la
fonction d'onde :
\[
\tag{8} \psi(\mt{x'',t''}) = \mt{< x''t'' | }\psi > =
\int\mt{< x''t'' | x't' > < x't' | }\psi \mt{ > dx'}
\]
On a simplement introduit La relation de fermeture relative à l'opérateur
X(t').
(8) peut encore s'écrire :
\[
\tag{9} \psi(\mt{x'',t''}) = \int\psi\mt{(x',t') < x''t'' | x't' > dx'}
\]
Cette équation intégrale permet, connaissant $\psi$(x',t') d'en déduire $\psi$(x'',t'')
à tout instant t'' > t'. < x''t'' | x't' > apparaît ainsi comme le noyau de
l'équation intégrale (9).

%
{\bf Remarque} : Il est clair d'après (9) que < x''t''|x't' > a pour dimension L$^{-1}$.
Si l'on remplace maintenant < x''t'' | x't' > par l'expression donnée par le
postulat II, on obtient l'expression de la fonction d'onde dans le formalisme de Feynman de la mécanique quantique :
\[
\tag{10}\psi(\mt{x'',t''}) = \lim_{\,\epsilon \to \,0} \mt{A}^{-\mt{n}}
\int...\int\psi(\mt{x',t'})
\exp\frac{\mt{i}}{\hbar}\sum_\mt{i=0}^\mt{n-1} \mt{S (x}_\mt{i},\mt{x}_\mt{i+1})
\mt{ dx'}\mt{dx}_1...\mt{dx}_\mt{n-1}
\]
La formule (10) est intéressante par son {\it interprétation physique} :
\begin{center}
\begin{tikzpicture}
% Plan d'onde
\draw  [very thick] (0,0) node[below] {t'} -- (0,5);
\draw  [very thick] (5,0) node[below] {t''} -- (5,5);
% Chemin haut
\draw [line width=1.5pt] (0,4) node[left] {x'} .. controls (1,5) and (4.5,3.5) .. (5,3);
\draw [line width=1.5pt] (0,4) .. controls (1,4) and (4.5,2.5) .. (5,3) node[right] {x''};
\draw [line width=1.5pt] (0,4) .. controls (1,3) and (4.5,1.5) .. (5,3);
% Chemin bas
\draw [line width=1.5pt] (0,1) .. controls (1,2.5) and (4.5,3) .. (5,3);
\draw [line width=1.5pt] (0,1) .. controls (1,1.5) and (4.5,2) .. (5,3);
\draw [line width=1.5pt] (0,1) .. controls (1,0.5) and (4.5,1) .. (5,3);
\end{tikzpicture}
\end{center}
$\psi$(x',t') est une onde définie sur la surface t $=$ t'. Chaque point de cette
surface se comporte comme une source d'amlitude $\psi$(x',t') et rayonne vers
le futur t > t'.

L'onde au point x'' de la surface t $=$ t'' s'obtient en sommant les
contributions, exp$\frac{\mt{iS}}{\hbar}$, de {\it tous} les chemins possibles,
issus de {\it toutes} les sources possibles $\psi$(x', t').

Nous retrouvons ainsi l'analogie déjà signalée dans l'Introduction
avec le principe d'Huyphens de l'optique,

Nous pouvons dire que la fonction d'onde $\psi$(x'', t'') résume toutes les
propriétés du système résultant de son histoire passée $[$ état initial (x',t')
{\it et} évolution sous l'effet des diverses interactions entre t' et t''$]$.

Pour montrer l'équivalence entre le formalisme de Feynman et le
formalisme habituel, il nous reste à prouver que la fonction $\psi$(x'',t'') ainsi
définie satisfait bien à l'équation de Schrödinger.
%
\section{Equation d'onde}

Pour établir l'équation d'évolution de (x,t), nous allons
partir d'une forme un peu différente de l'équation (10) en ne considérant
entre t' et t'' qu'un seul intervalle infiniment petit e : on a alors :
\[
\tag{11}\psi(\mt{x}_\mt{k+1},\mt{t}+\epsilon) = \int \frac{\mt{dx}_\mt{k}}{\mt{A}}
\mt{ exp }[\frac{\mt{i}}{\hbar} \mt{ S ( x}_\mt{k}, \mt{x}_\mt{k+1} ) ]
\psi (\mt{x}_\mt{k}, \mt{t} )
\]
L'expression (11) n'est vraie en toute rigueur qu'à la limite où $\epsilon$ tend
vers zéro. Elle permet alors de déterminer de proche en proche la fonction
d'onde $\psi$(x,t+T). Il suffit d'appliquer (11) n=$\frac{\mt{T}}{\epsilon}$ fois consécutives
On retrouve alors l'expression (10).

Supposons que dans (11), l'expression de S(x$_\mt{k}$, x$_\mt{k+1}$) ne soit
pas l'action pour la particule classique entre x$_\mt{k}$ et x$_\mt{k+1}$ mais n'en diffère
que par un terme ordre $\alpha$ > 1 , en $\epsilon^\alpha$. Au bout de n $=$ T/$\epsilon$ applications
de la formule (11), nous aurons un terme qui diffère du terme exact prévu
par le postulat de Feynman en  $\epsilon^\alpha \frac{\mt{T}}{\epsilon}$  $=$  T$\epsilon^{\alpha-1}$ qui tend vers zére avec $\epsilon$. Il
suffit donc, pour obtenir l'équation d'évolution exacte de la fonction
d'onde, de prendre dans la relation (11) une expression approchée au premier
ordre en $\epsilon$ de S (x$_\mt{k}$, x$_\mt{k+1}$) que nous allons maintenant calculer.

Expression approchée de S (x$_\mt{k}$, x$_\mt{k+1}$)

\begin{center}
\begin{tikzpicture}
% Plan d'onde
\draw  [very thick] (0,0) node[below] {t} -- (0,3);
\draw  [very thick] (2,0) node[below] {t+$\epsilon$} -- (2,3);
% Chemin
\draw  [line width=1.5pt] (0,2) node[left] {x$_\mt{k}$} -- (2,1) node[right] {x$_\mt{k+1}$};
\end{tikzpicture}
\end{center}
L'équation (11) nous montre qu'en théorie, il faut pour obtenir
$\psi$(x$_\mt{k+1}$,t+$\epsilon$)
sommer sur {\it toutes} les contributions x$_\mt{k}$ à l'instant t. Nous
allons voir qu'en fait seules celles qui correspondent à
x$_\mt{k+1} - \mt{x}_\mt{k} \leq \epsilon^{1/2}$
%\precsim\lesssim
sont non négligeables. Admettons pour l'instant ce résultat qui signifie
que les variations de $\psi$(x,t) ne sont déterninées que par la valeur de
$\psi$(x,t) au voisinage du point x et qui nous permettra de transformer
l'équation intégrale (11) en équation différentielle, Les valeurs importantes
de S (x$_\mt{k}$, x$_\mt{k+1}$) sont donc celles qui correspondent à un chemin
classique infinitésimal entre t et t$+\epsilon$ avec x$_\mt{k+1} - $x$_\mt{k}$ < $\epsilon^{1/2}$

On montre alors qu'on obtient une expression approchant S (x$_\mt{k}$, x$_\mt{k+1}$)
au premier ordre en $\epsilon$ en considérant le potentiel V(x)
comme constant et égal à V(x$_\mt{k+1}$) entre x$_\mt{k}$ et x$_\mt{k+1}$ et en prenant pour
chemin classique d'espace-temps le segment de droite entre (x$_\mt{k}$, t) et
(x$_\mt{k+1}$, t$+\epsilon$), c'est-à-dire encore le chemin d'une particule classique se
déplaçant entre x$_\mt{k}$ et x$_\mt{k+1}$ à la vitesse uniforme
v $= \frac{\mt{x}_\mt{k+1} - \mt{x}_\mt{k}}{\epsilon}$ . On remplace alors
S (x$_\mt{k}$, x$_\mt{k+1}$)  par
$\frac{\mt{m(x}_\mt{k+1} - \mt{x}_\mt{k})^2}{2\epsilon} - \epsilon\mt{V(x}_\mt{k+1})$

L'expression rigoureuse de S (x$_\mt{k}$, x$_\mt{k+1}$) est
$\int_\mt{t}^{\mt{t}+\epsilon}[\frac{1}{2}\mt{mv}^2 - \mt{V(x)}]\mt{d}\tau$, v et x
étant les vitesses et positions à l'instant $\tau$ de la particule classique
partie de x$_\mt{k}$ à l'instant t et arrivant en x$_\mt{k+1}$ à l'instant t$+\epsilon$. Si 
x$_\mt{k+1}-$x$_\mt{k}$
est au maximum de l'ordre de 2, nous voyons en développant v(x) au voisi
nage de V(x, ..), que V(x) est égal à V(x,,.) à un infiniment petit en 

près. La vitesse moyenne de la particule classique est V $=$  Elle

peut donc tendre vers l'infini en  lorsque  tend vers zéro. Cependant

l'équation fondamentale de la dynamique qui s'écrit m  $=$

tre que l'écart à la vitesse moyenne,  , est toujours infiniment

nous mon

petit de l'ordre de . La variation de l'énergie cinétique, , est donc,

comme la variation de l'énergie potentielle un infiniment petit en.
Après intégration sur le temps, (c'est-à-dire multiplication par on voit

done, que ne diffère de  que par un terme
en et constitue donc bien une approximation valable au premier ordre

inclus en  de 
%

Etablissement de l'équation de Schrôdinrer

Posons maintenant

L'équation (11), compte tenu des approximations précédentes, s'écrit

Or exp  devient une fonction très rapidement oscillante dès que

 alors que  varie très lentement. Donc les seules
valeurs de  qui contribuent de façon importante à l'intégrale (12) sont
comprises entre , ce qui justifie a posteriori la remarque que

nous avions faite pour établir une expression approchée de  et

nous permet de calculer (12) en développant 

Le développement de (12) fait alors intervenir trois intérrales
dont le calcul ne présente aucune difficulté et dont nous donnons ici

l'expression
%
En développant au premier ordre , l'équation
(12) devient : 

En identifiant les termes d'ordre et  on obtient

ce qui n'est autre que l'expression (6) que nous avons déjà obtenue par
un autre moyen.

ce qui n'est autre que l'équation de Schrödinger d'une particule dans un

votentiel V(x).

La formulation de Feynman est donc bien équivalente aux autres

formulations de la mécanique quantique.

Nous voyons que nour obtenir une expression valable jusqu'au rremier ordre

en , il a fallu déveloprer jusqu'au deuxière ordre en 
%


\chapter{Les opérateurs dans le formalisme de Feynman} % III
% 25

L'idée centrale du formalisme de Feynman est d'associer une
amplitude de probabilité à chaque chemin, ou histoire, du système entre
un état initial et un état final donnés. La même idée permet de définir
de façon très simple l'opérateur {\it quantique} G qui correspond à une grandeur
{\it classique} g. (Nous réservons de façon systématique les minuscules aux
grandeurs classiques et les majuscules aux grandeurs quantiques.).

Après avoir défini et étudié les opérateurs dans le formalisme
de Feynman et établi le lien avec le formalisme habituel de la mécanique
quantique pour des cas de plus en plus compliqués (\S A, B, C), nous
appliquons les résultats de cette étude à un problème important, celui de la
théorie des perturbations dépendant du temps (\S D).

\section{Opérateurs relatifs à un instant donné t$_\mt{k}$}
\subsection{Définition de Feynman} % 1°)

Nous allons tout d'abord adopter pour l'amplitude de probabilité < x" t'' | x' t' > la notation plus commode

\[
\mt{<x''t''|x't'>} = \sum_\mt{H} \mt{N exp} \frac{\mt{i}}{\hbar} \mt{S}_\mt{H} =
\lim_{\,\epsilon \to \,0} \frac{1}{\mt{A}}
\int \frac{\mt{dx}_1}{\mt{A}}..\frac{\mt{dx}_{\mt{n-1}}}{\mt{A}}
\mt{exp} \frac{\mt{i}}{\hbar} \sum_\mt{i=0}^\mt{n-1}
\mt{S (x}_\mt{i}, \mt{x}_\mt{i+1})
\]
qui n'est autre qu'une forme condensée de la formule (2) au chapitre II, dans
laquelle $\sum_{H}$ représente la sommation sur tous les chemins, ou histoires 
possibles, telle qu'elle a été définie dans les postulats, S$_\mt{H}$ l'action 
pour le chemin H.
%26
\subsubsection{Opérateur position à l'instant t$_\mt{k}$, x(t$_\mt{k}$)}%a) 

\begin{center} \begin{tikzpicture}
% axes
\draw [->] (0,0) --++ (8,0) node [below] {t};
\draw [->] (0,0) --++ (0,4) node [left] {x};
\node at (0,0) [left]{O};
% coordonnées et points
\draw [densely dashed] (0,2.5) node [left] {x'} --++ (1,0);
\draw [densely dashed] (1,0) node [below] {t'} --++ (0,2.5);
\node at (1.2,2.5) [above left]{M'};
\draw [densely dashed] (0,1) node [left] {x''} --++ (7,0);
\draw [densely dashed] (7,0) node [below] {t''} --++ (0,1);
\node at (7,1.2) [right]{M''};
%  milieu
\draw [densely dashed] (4,-0) node [below] {t$_\mt{k}$} -- (4,2.4) -- (4, 3.15);
\draw [<-, >=latex] (4.05, 2.42) -- (6, 3) node [right]{x$_\mt{C}$(t$_\mt{k}$)};
\draw [<-, >=latex] (4.05, 3.2) -- (5, 3.7) node [right]{x$_\mt{H}$(t$_\mt{k}$)};
% Chemins
\draw [line width=1.5pt] (1,2.5) .. controls (2.5,3) and (6,2) .. (7,1);
\draw [line width=1.5pt, dashed] (1,2.5) .. controls (3,4) and (6,3) .. (7,1);
\end{tikzpicture} \end{center}
Entre deux points M' et M'' de l'espace-temps, nous savons
qu'une {\it particule classique} suit un chemin {\it bien défini}, celui de l'action
{\it stationnaire} (en trait plein sur la figure). Sa position à l'instant t$_\mt{k}$,
x$_\mt{C}$(t$_\mt{k}$), est donc elle aussi bien {\it définie}.

En mécanique quantique, au contraire, la situation est totalement différente. L'amplitude de probabilité < x''t'' | x't' > est la somme
des amplitudes associées à {\it tous} les chemins ou histoires possibles reliant
M à M" et on peut donc dire que la particule est passée à l'instant t$_\mt{k}$, par
{\it toutes} les positions x$_\mt{H}$(t$_\mt{k}$), relatives à {\it toutes} les histoires H.

Il est alors naturel d'envisager la quantité
N$\sum_\mt{H} \mt{x}_\mt{H} (\mt{t}_\mt{k}) \mt{exp} \frac{\mt{i}}{\hbar} \mt{S}_\mt{H}$
obtenue en pondérant la valeur x$_\mt{H}$(t$_\mt{k}$) par l'amplitude de probabilité associée
au chemin correspondant, N exp$\frac{\mt{i}}{\hbar} \mt{S}_\mt{H}$ et en sommant sur toutes les histoires
possibles entre M' et M''.

{\it }Par définition, Feynman appelle la quantité précédente 1'{\it élément
de matrice entre les états} < x''t'' | et | x't' > {\it de l'opérateur} X(t$_\mt{k}$) {\it associé
à la grandeur physique, position à l'instant} t$_\mt{k}$, x(t$_\mt{k}$)
On adopte la notation :
\[
\tag{1}\subset \mt{x''t''} | \mt{X(t}_\mt{k}) | \mt{x't'}\supset =
\mt{N}\sum_\mt{H} \mt{x}_\mt{H} (\mt{t}_\mt{k}) \mt{exp} \frac{\mt{i}}{\hbar} \mt{S}_\mt{H}
=\lim_{\epsilon \to \,0} \frac{1}{\mt{A}}
\int \frac{\mt{dx}_1}{\mt{A}}..\frac{\mt{dx}_{\mt{n-1}}}{\mt{A}} \mt{x}_\mt{k}
\mt{exp} \frac{\mt{i}}{\hbar} \sum_\mt{i=0}^\mt{n-1}
\mt{S (x}_\mt{i}, \mt{x}_\mt{i+1})
\]
%27

{\bf Remarques}

$\alpha$) Pour marquer la différence entre les éléments de matrice au sens de
Feynman et les éléments de matrice au sens habituel de la mécanique
quantique, nous avons adopté la notation $\subset |\ | \supset$ différente de celle de
Dirac $< |\ | >$ . Nous constaterons que ces deux notations ne sont pas
toujours équivalentes.

$\beta$) Quelle est la limite de la quantité
$\frac{\subset \mt{x''t''} | \mt{X(t}_\mt{k}) | \mt{x't'}\supset}{\mt{<x''t''|x't'>}}$
lorsque  $\hbar \to 0$ (limite classique) ?

Il est clair d'après la relation (1) que les seules contributions à
$\subset \mt{x''t''} | \mt{X(t}_\mt{k}) | \mt{x't'}\supset$
qui ne se détruisent pas par interférence
sont celles qui correspondent au voisinage du chemin classique de l'action
stationnaire et qu'alors la quantité
$\frac{\subset \mt{x''t''} | \mt{X(t}_\mt{k}) | \mt{x't'}\supset}{\mt{<x''t''|x't'>}}$
qui est bien homogène à une longueur, tend vers x$_\mt{C}$(t$_\mt{k}$).

\subsubsection{Autres opérateurs} % b)
Les résultats précédents ont une généralisation immédiate :
Soit g(t) une grandeur physique que l'on peut définir à l'instant t$_\mt{k}$, pour
toute histoire H de la particule.  g$_\mt{C}$(t$_\mt{k}$) représente la grandeur relative
à la trajectoire classique et g$_\mt{H}$(t$_\mt{k}$) la grandeur relative à l'histoire H.

Par définition, on appelle élément de matrice au sens de
Feynman entre < x't' | et | x't' > de l'opérateur G(t$_\mt{k}$), correspondant à
la grandeur physique g, la quantité
\[
\subset \mt{x''t''} | \mt{G(t}_\mt{k}) | \mt{x't'}\supset =
\sum_\mt{H}\mt{N} \mt{g}_\mt{H} (\mt{t}_\mt{k}) \mt{exp} \frac{\mt{i}}{\hbar} \mt{S}_\mt{H}
\]
{\bf Remarque} : Comment peut-on définir la vitesse à l'instant t$_\mt{k}$ ?

\begin{center} \begin{tikzpicture}
\coordinate (A) at (1,4) ;
\coordinate (B) at (5,2) ;
\coordinate (C) at (9,4) ;
% axe
\draw [->] (0,0) --++ (10,0) node [below] {t};
% ordonnées et points
\draw [densely dashed] (1,0) node [below] {t$_\mt{k-1}$} -- (A) node [above left]{M$_\mt{k-1}$};
\draw [densely dashed] (5,0) node [below] {t$_\mt{k}$} -- (B) node [below left]{M$_\mt{k}$};
\draw [densely dashed] (9,0) node [below] {t$_\mt{k+1}$} -- (C) node [above right]{M$_\mt{k+1}$};
\draw [densely dashed] (7,0) node [below] {$\frac{\mt{t}_\mt{k}+\mt{t}_\mt{k+1}}{2}$} -- (7,3);
% segments
\draw (A) -- (B) -- (C);
% Chemins
\draw [line width=1.5pt] (A) .. controls (2,3) and (4,2) .. (B);
\draw [line width=1.5pt] (B) .. controls (6,2) and (8,2) .. (C);
\end{tikzpicture} \end{center}
% 28

Rappelons que les chemins H sont en fait définis par une
suite de points M', ... M$_\mt{k-1}$, M$_\mt{k}$, M$_\mt{k+1}$, ... M", reliés les uns aux autres
par des chemins classiques infinitésimaux. Il résulte de cette définition
que la grandeur classique "vitesse au point t$_\mt{k}$" n'est pas définie, puisque
la dérivée $\dot{\mt{x}}$(t) est discontinue en ce point. On convient alors de définir
\[
\mt{v(t}_\mt{k}) =
\lim_{\epsilon \to \,0} \frac{\mt{x(t}_{\mt{k+1}}) - \mt{x(t}_\mt{k})}{\epsilon}
\]
De même, pour la grandeur physique "produit de la position par la vitesse
à l'instent t$_\mt{k}$", on prend, au lieu de x (t$_\mt{k}$) v (t$_\mt{k}$), qui n'est pas défini,
la quantité
\[
\lim_{\epsilon \to \,0} \frac{\mt{x(t}_\mt{k}) + \mt{x(t}_\mt{k+1})}{2}
\ \frac{\mt{x(t}_{\mt{k+1}}) - \mt{x(t}_\mt{k})}{\epsilon}
\]

Ces définitions reviennent à envisager les grandeurs physiques
corresrondantes à un {\it même} instant, par exemple
$\frac{\mt{t}_\mt{k} + \mt{t}_\mt{k+1}}{2}$, où elles sont
définies et à prendre pour chemin classique entre M$_\mt{k}$ et M$_\mt{k+1}$ une droite
d'espace-temps. Ces deux approximations sont justifiées par le fait qu'on
prend les limites des expressions précédentes pour $\epsilon \to \,0$.

(Nous savons (cf ch. II, note p. 22) qu'on peut approcher le chemin classique
M$_\mt{k}$ M$_\mt{k+1}$ par une droite, ce qui justifie la définition ci-dessus
de x(t$_\mt{k}$) v(t$_\mt{k}$) qui correspond à la limite pour $\epsilon \to \,0$
de x($\frac{\mt{t}_\mt{k} + \mt{t}_\mt{k+1}}{2}$)v($\frac{\mt{t}_\mt{k} + \mt{t}_\mt{k+1}}{2}$).
On pourrait envisager d'adopter la définition plus simple
x(t$_\mt{k}$)v(t$_\mt{k}) =
\lim_{\epsilon \to \,0}$ x(t$_\mt{k}$) $\epsilon^{-1}[\mt{x(t}_\mt{k+1}) - \mt{x(t}_\mt{k})]$.
On commet alors sur x(t) une erreur qui peut être en $\epsilon^{1/2}$ et, comme v
peut croître en $\epsilon^{-1/2}$ (cf note p. 22) une erreur sur x(t$_\mt{k}$)v(t$_\mt{k}$) qui peut
rester finie lorsque $\epsilon \to \,0$. Il est donc bien indispensable d'envisager les
grandeurs physiques x(t) et v(t) à un {\it même} instant.)

\subsection{Lien avec le formalisme habituel de la mécanique quantique}% 2°)

\subsubsection{Point de vue de Heisenberg}% a)
L'élément de matrice $\subset \mt{x''t''} | \mt{X(t}_\mt{k}) | \mt{x't'}\supset$ peut s'écrire
en divisant en trois parties l'intégration sur les chemins :
\[
\tag{2}\subset \mt{x''t''} | \mt{X(t}_\mt{k}) | \mt{x't'}\supset =
\lim_{\epsilon \to \,0} \int \mt{x}_\mt{k}\mt{dx}_\mt{k} \left[\frac{1}{\mt{A}}
\int...\int \frac{\mt{dx}_1}{\mt{A}}..\frac{\mt{dx}_\mt{k-1}}{\mt{A}}
\mt{exp} \frac{\mt{i}}{\hbar} \sum_\mt{i=0}^\mt{k-1}
\mt{S (x}_\mt{i}, \mt{x}_\mt{i+1})\right]
\]
\[
\times \left[\frac{1}{\mt{A}}
\int...\int \frac{\mt{dx}_\mt{k+1}}{\mt{A}}..\frac{\mt{dx}_\mt{n-1}}{\mt{A}}
\mt{exp} \frac{\mt{i}}{\hbar} \sum_\mt{i=k}^\mt{n-1}
\mt{S (x}_\mt{i}, \mt{x}_\mt{i+1})\right]
\]
%29
Or $\lim_{\epsilon \to \,0} \frac{1}{\mt{A}}
\int...\int \frac{\mt{dx}_1}{\mt{A}}..\frac{\mt{dx}_\mt{k-1}}{\mt{A}}
\mt{exp} \frac{\mt{i}}{\hbar} \sum_\mt{i=0}^\mt{k-1}
\mt{S (x}_\mt{i}, \mt{x}_\mt{i+1}) = $ < x$_\mt{k}$t$_\mt{k}$ | x't' >

(par définition : cf chapitre II, formule 2)

De même $\lim_{\epsilon \to \,0} \frac{1}{\mt{A}}
\int...\int \frac{\mt{dx}_{k+1}}{\mt{A}}..\frac{\mt{dx}_\mt{n-1}}{\mt{A}}
\mt{exp} \frac{\mt{i}}{\hbar} \sum_\mt{i=k}^\mt{n-1}
\mt{S (x}_\mt{i}, \mt{x}_\mt{i+1}) = $ < x''t'' | x$_\mt{k}$t$_\mt{k}$ >

(2) devient alors
\begin{center}
$\subset \mt{x''t''} | \mt{X(t}_\mt{k}) | \mt{x't'}\supset = \int \mt{dx}_\mt{k}
 < \mt{x''t''} | \mt{x}_\mt{k}\mt{t}_\mt{k} >
 \mt{x}_\mt{k}
 < \mt{x}_\mt{k}\mt{t}_\mt{k} | \mt{x't'} > $
\end{center}
Or on a x$_\mt{k} \delta$(x'$_\mt{k} -$ x$_\mt{k}$) =
< x$_\mt{k}$t$_\mt{k}$ | X(t$_\mt{k}$)| x'$_\mt{k}$t$_\mt{k}$ >
(les éléments de matrice sont ici pris au sens habituel de Heisenberg)

Finalement, on peut donc écrire
\[
\tag{3}\subset \mt{x''t''} | \mt{X(t}_\mt{k}) | \mt{x't'}\supset\ =
\int\int \mt{dx}_\mt{k} \mt{dx'}_\mt{k}
 < \mt{x''t''} | \mt{x}_\mt{k}\mt{t}_\mt{k} >
 < \mt{x}_\mt{k}\mt{t}_\mt{k} | \mt{X(t}_\mt{k})| \mt{x'}_\mt{k}\mt{t}_\mt{k} >
 < \mt{x'}_\mt{k}\mt{t}_\mt{k} | \mt{x't'} >
\]
\begin{center}
$ = < \mt{x''t''} | \mt{X(t}_\mt{k})| \mt{x't'} >$
\end{center}

Il y a donc accord entre la définition de Feynman et la définition habituelle
des éléments de matrice (éans le point de vue de Heisenberg).
\subsubsection{Point de vue de Schrödinger}% b)

Nous appellerons \underline{X} l'opérateur X(t) dans le point de vue de
Schrödinger.

On calcule alors $\subset \mt{x''t''} | \mt{X(t}_\mt{k}) | \mt{x't'}\supset$ à partir de la
formule (3) ci-dessus en remarquant que
\begin{center}
$< \mt{x}_\mt{k}\mt{t}_\mt{k} | \mt{X(t}_\mt{k})| \mt{x'}_\mt{k}\mt{t}_\mt{k} >
= < \mt{x}_\mt{k} | \underline{\mt{X}} | \mt{x'}_\mt{k} >
=\mt{x}_\mt{k}\delta(\mt{x}_\mt{k}-\mt{x'}_\mt{k})$

$< \mt{x''t''} | \mt{x}_\mt{k}\mt{t}_\mt{k} > =
< \mt{x''} | \mt{U(t'',t}_\mt{k})| \mt{x}_\mt{k} >$

$< \mt{x'}_\mt{k}\mt{t}_\mt{k} | \mt{x't'} > =
< \mt{x'}_\mt{k} | \mt{U(t}_\mt{k},\mt{t')}| \mt{x'} >$

(cf chapitre II)
\end{center}

%30
(3) s'écrit alors
\[
\tag{4}\subset \mt{x''t''} | \mt{X(t}_\mt{k}) | \mt{x't'}\supset\ =
\int\int \mt{dx}_\mt{k} \mt{dx'}_\mt{k}
 < \mt{x''} | \mt{U(t'',t}_\mt{k})| \mt{x}_\mt{k} >
 < \mt{x}_\mt{k} | \underline{\mt{X}} | \mt{x'}_\mt{k} >
 < \mt{x'}_\mt{k} | \mt{U(t}_\mt{k},\mt{t')}| \mt{x'} >
\]\[
=< \mt{x''} | \mt{U(t'',t}_\mt{k}) \underline{\mt{X}} \mt{U(t}_\mt{k},\mt{t')}| \mt{x'} >
\]
Remarque :

\begin{minipage}[c]{.15\linewidth}
Posons
\end{minipage}
%\hfill
\begin{minipage}[c]{.35\linewidth}
$|\chi>=\mt{U}^\dagger\mt{(t'',t}_\mt{k})| \mt{x''} >$

$|\psi> = \mt{U(t}_\mt{k},\mt{t')}| \mt{x'} >$
\end{minipage}
\hfill

$|\psi>$ représente l'état à l'instant t$_\mt{k}$ de la particule qui a été localisée
en x' à l'instant t' < t$_\mt{k}$.
$|\chi>$ représente l'état à l'instant t$_\mt{k}$ de la particule qui sera localisée en
x'' à l'instant t'' > t$_\mt{k}$.

On a alors d'après la relation (4) :
\[
\subset \mt{x''t''} | \mt{X(t}_\mt{k}) | \mt{x't'}\supset\ =
<\chi| \underline{\mt{X}} |\psi>
\]
On voit ainsi que l'élément de matrice au sens de Feynman n'est
autre que l'élément de matrice ordinaire à l'instant t$_\mt{k}$ de l'opérateur position \underline{X}, entre l'état physique qui a évolué à partir de la particule localisée
en x' à l'instant t' < t$_\mt{k}$ et l'état qui évoluera à l'instant t'' > t$_\mt{k}$ vers
la particule localisée au point x''. L'élément de matrice au sens de Feynman
ne peut donc en aucun cas être considéré comme une valeur moyenne, résultat
d'une mesure de la grandeur x(t$_\mt{k}$) puisqu'il correspond à un élément de
matrice pris entre {\it deux états différents} $<\chi|$ et $|\psi>$.

\section{Produits de deux orérateurs relatifs à des instants différents t$_\mt{k}$, t$_\mt{l}$}% B. 
\subsection{Définition de Feynman}% 1°) 

De même que nous avons, au paragraphe .1, défini des opérateurs
relatifs à un instant donné t$_\mt{k}$, nous allons étendre cette définition à deux
instants différents t$_\mt{l}$ et t$_\mt{k}$ en posant par exemple t$_\mt{l}$ > t$_\mt{k}$.

%-31
Prenons d'abord pour exemple la grandeur physique x(t$_\mt{k}$) x(t$_\mt{l}$)
produit de la position de la particule à deux instants différents.

Pour une particule classique décrivant M' M'', nous savons que
cette grandeur prend la valeur bien définie  x$_\mt{C}$(t$_\mt{k}$) x$_\mt{C}$(t$_\mt{l}$).

A chaque histoire H contribuant à l'amplitude de probabilité
quantique < x''t'' | x't' > on associe le nombre x$_\mt{H}$(t$_\mt{k}$) x$_\mt{H}$(t$_\mt{l}$).

On définit alors l'élément de matrice au sens de Feynman entre
les états < x''t'' | et | x't' > du produit d'opérateurs X(t$_\mt{k}$) X(t$_\mt{l}$) aux
deux instants t$_\mt{k}$ et t$_\mt{l}$ par la relation
\[
\tag{5}\subset \mt{x''t''} | \mt{X(t}_\mt{k}) \mt{ X(t}_\mt{l}) | \mt{x't'}\supset\ =
\sum_H \mt{N x}_\mt{H}(\mt{t}_\mt{k}) \mt{ x}_\mt{H}(\mt{t}_\mt{l})
\mt{ exp}\frac{\mt{i}}{\hbar}\mt{S}_\mt{H}
\]
Remarque :

À la limite classique où $\hbar \to 0$, on montre, comme au paragraphe 1 :
$$ \lim_{\hbar \to 0} \frac{\subset \mt{x''t''} | \mt{X(t}_\mt{k}) \mt{ X(t}_\mt{l}) | \mt{x't'}\supset}
{< \mt{x''t''} | \mt{x't'}>} = \mt{ x}_\mt{C}(\mt{t}_\mt{k}) \mt{ x}_\mt{C}(\mt{t}_\mt{l})$$
Les définitions précédentes se généralisent immédiatement :
Soient f(t$_\mt{k}$) et g(t$_\mt{l}$) deux grandeurs rhysiques aux instants t$_\mt{k}$ et t$_\mt{l}$,
f$_\mt{H}$(t$_\mt{k}$) et g$_\mt{H}$(t$_\mt{l}$) les valeurs de ces grandeurs pour une histoire H; alors
la quantité :
\[
\tag{6}\subset \mt{x''t''} | \mt{F(t}_\mt{k}) \mt{ G(t}_\mt{l}) | \mt{x't'}\supset\ =
\sum_\mt{H} \mt{N f}_\mt{H}(\mt{t}_\mt{k}) \mt{ g}_\mt{H}(\mt{t}_\mt{l})
\mt{ exp}\frac{\mt{i}}{\hbar}\mt{S}_\mt{H}
\]
représente l'élément de matrice au sens de Feynman entre les états < x't' |
et | x''t'' > du produit des opérateurs $\mt{F(t}_\mt{k}) \mt{ G(t}_\mt{l})$ aux deux instants t$_\mt{k}$
et t$_\mt{l}$.

{\bf Remarque importante :} D'après les formules (5) et (6), il résulte de la commutation des nombres
x$_\mt{H}$(t$_\mt{k}$) x$_\mt{H}$(t$_\mt{l}$), que les opérateurs au sens de Feynman à deux instants
différents commutent :
\[
\subset \mt{x''t''} | \mt{X(t}_\mt{k}) \mt{ X(t}_\mt{l}) | \mt{x't'}\supset\ =
\subset \mt{x''t''} | \mt{X(t}_\mt{l}) \mt{ X(t}_\mt{k}) | \mt{x't'}\supset\
\]
\[
\subset \mt{x''t''} | \mt{F(t}_\mt{k}) \mt{ G(t}_\mt{l}) | \mt{x't'}\supset\ =
\subset \mt{x''t''} | \mt{G(t}_\mt{l}) \mt{ F(t}_\mt{k}) | \mt{x't'}\supset\
\]
L'ordre dans lequel sont rangés les instants t$_\mt{k}$ et t$_\mt{l}$ n'importe pas.
%32

\subsection{Lien avec le formalisme habituel}% 2°)
\subsubsection{Point de vue de Heisenberg}% a)

La formule (5) s'écrit :
\[
\tag{7}\subset \mt{x''t''} | \mt{X(t}_\mt{k}) \mt{ X(t}_\mt{l}) | \mt{x't'}\supset\ =
\lim_{\epsilon \to \,0} \frac{1}{\mt{A}}
\int...\int \frac{\mt{dx}_1}{\mt{A}}..\frac{\mt{dx}_\mt{n-1}}{\mt{A}}
\mt{x}_\mt{k}\mt{x}_\mt{l}
\mt{exp} \frac{\mt{i}}{\hbar} \sum_\mt{i=0}^\mt{n-1}
\mt{S (x}_\mt{i}, \mt{x}_\mt{i+1})
\]
Par un raisonnement analogue à celui déjà fait au \S 1, (7) devient
\[
\subset \mt{x''t''} | \mt{X(t}_\mt{k}) \mt{ X(t}_\mt{l}) | \mt{x't'}\supset\ =
\lim_{\epsilon \to \,0}\int\int\mt{x}_\mt{k}\mt{x}_\mt{l}\mt{dx}_\mt{k}\mt{dx}_\mt{l}
\left[\frac{1}{\mt{A}}\int...\int\frac{\mt{dx}_1}{\mt{A}}..\frac{\mt{dx}_\mt{k-1}}{\mt{A}}
\mt{exp} \frac{\mt{i}}{\hbar} \sum_\mt{i=0}^\mt{k-1}\mt{S (x}_\mt{i}, \mt{x}_\mt{i+1})\right.
\]
\[
\left.\times\frac{1}{\mt{A}}\int...\int\frac{\mt{dx}_\mt{k+1}}{\mt{A}}..\frac{\mt{dx}_\mt{l-1}}{\mt{A}}
\mt{exp} \frac{\mt{i}}{\hbar} \sum_\mt{i=k}^\mt{l-1}\mt{S (x}_\mt{i}, \mt{x}_\mt{i+1})
\times\frac{1}{\mt{A}}\int...\int\frac{\mt{dx}_\mt{l+1}}{\mt{A}}..\frac{\mt{dx}_\mt{n-1}}{\mt{A}}
\mt{exp} \frac{\mt{i}}{\hbar} \sum_\mt{i=l}^\mt{n-1}\mt{S (x}_\mt{i}, \mt{x}_\mt{i+1})\right]
\]
\[
= \int\int \mt{dx}_\mt{k} \mt{dx}_\mt{l}
< \mt{x''t''} | \mt{x}_\mt{l}\mt{t}_\mt{l} > \mt{x}_\mt{l}
< \mt{x}_\mt{l}\mt{t}_\mt{l} | \mt{x}_\mt{k}\mt{t}_\mt{k} >
\mt{x}_\mt{k}< \mt{x}_\mt{k}\mt{t}_\mt{k} | \mt{x't'} >
\]
\[
= \int\int\int\int \mt{dx}_\mt{k} \mt{dx'}_\mt{k} \mt{dx}_\mt{l} \mt{dx'}_\mt{l}
 < \mt{x''t''} | \mt{x}_\mt{l}\mt{t}_\mt{l} >
 < \mt{x}_\mt{l}\mt{t}_\mt{l} | \mt{X(t}_\mt{l})| \mt{x'}_\mt{l}\mt{t}_\mt{l} >
 < \mt{x}_\mt{l}\mt{t}_\mt{l} | \mt{x}_\mt{k}\mt{t}_\mt{k} >
\]
\[
 < \mt{x}_\mt{k}\mt{t}_\mt{k} | \mt{X(t}_\mt{k})| \mt{x'}_\mt{k}\mt{t}_\mt{k} >
 < \mt{x'}_\mt{k}\mt{t}_\mt{k} | \mt{x't'} >
\]
\[
= < \mt{x''t''} | \mt{ X(t}_\mt{l}) \mt{X(t}_\mt{k})| \mt{x't'} >
\]
On a donc finalement
\[
\subset \mt{x''t''} | \mt{X(t}_\mt{k}) \mt{ X(t}_\mt{l}) | \mt{x't'}\supset\ =
< \mt{x''t''} | \mt{ X(t}_\mt{l}) \mt{X(t}_\mt{k}) | \mt{x't'} >
\ \ \ \ \ \ \mt{avec } \mt{t}_\mt{k} < \mt{t}_\mt{l}
\]
{\bf Remarque très importante :}
I1 résulte de la démonstration que nous venons de faire que dans l'élément
de matrice du point de vue de Heisenberg, les temps sont rangés dans l'ordre
croissant de la droite vers la gauche :
\[
\mt{t'} < \mt{t}_\mt{k} < \mt{t}_\mt{l} < \mt{t''}.
\]
%-33
Les opérateurs X(t$_\mt{l}$) et X(t$_\mt{k}$) relatifs à deux instants t$_\mt{k}$ et t$_\mt{l}$ différents
ne commutent pas (nar exemple, X(t) ne peut commuter avec X(t + $\epsilon$), car
on ne peut mesurer simultanément position et vitesse d'une particule)
\begin{center}
X (t$_\mt{l}$) X (t$_\mt{k}$) $\neq$ X (t$_\mt{k}$) X (t$_\mt{l}$)
\end{center}
Introduisons l'opérateur d'ordre P qui, au produit X (t$_\mt{l}$) X (t$_\mt{k}$), associe
X (t$_\mt{l}$) X (t$_\mt{k}$) si t$_\mt{k}$ < t$_\mt{l}$
et X (t$_\mt{k}$) X (t$_\mt{l}$) t$_\mt{k}$ > t$_\mt{l}$:

\[
\mt{P}[\mt{ X(t}_\mt{l}) \mt{X(t}_\mt{k})] = 
\left\{
  \begin{array}{rcr}
    \mt{X(t}_\mt{k}) \mt{ X(t}_\mt{l}) & \mt{si} & \mt{t}_\mt{k} < \mt{t}_\mt{l} \\
    \mt{X(t}_\mt{l}) \mt{ X(t}_\mt{k}) & \mt{si} & \mt{t}_\mt{k} > \mt{t}_\mt{l} \\
  \end{array}
\right.\]
Alors on a la relation fondamentale :

\[
\tag{8}\subset \mt{x''t''} | \mt{X(t}_\mt{k}) \mt{ X(t}_\mt{l}) | \mt{x't'}\supset\ =
\subset \mt{x''t''} | \mt{ X(t}_\mt{l}) \mt{X(t}_\mt{k}) | \mt{x't'}\supset\
\]
\[
 = < \mt{x''t''} | \mt{P}[\mt{ X(t}_\mt{l}) \mt{X(t}_\mt{k})] | \mt{x't'} >
\]
{\bf En résumé} : alors que les opérateurs associés à des temps différents commutent au sens de Feynman, il n'en est pas de même au point de vue de Heisenberg.
L'ordre qu'il faut alors adopter est celui des temps croissants de la droite
vers la gauche.
\subsubsection{Point de vue de Schrödinger}% b)
On déduit immédiatement du point de vue de Heisenberg :

\[
\tag{9}\subset \mt{x''t''} | \mt{ X(t}_k) \mt{ X(t}_l)\ | \mt{x't'}\supset\ =
< \mt{x''t''}|\mt{ U(t'', t}_l)
\ \ul{\mt{X}}\ \mt{ U(t}_l,\mt{ t}_k)
\ \ul{\mt{X}}\ \mt{ U(t}_k,\mt{ t') }|\mt{x't'} >
\]
De façon plus générale, si nous avons affaire à des grandeurs physiques
relatives à des instants t$_k$, t$_l$, ... t$_p$ différents, nous introduisons
l'opérateur d'ordre P qui range les temps dans l'ordre croissant de la
droite vers la gauche et nous avons
\[
\tag{10}\subset \mt{x''t''} | \mt{ X(t}_k) \mt{ X(t}_l)... \mt{ X(t}_p)\ | \mt{x't'}\supset\ =
< \mt{x''t''}|\mt{ P}[\mt{ X(t}_k) \mt{ X(t}_l)... \mt{ X(t}_p)]
\ |\mt{x't'} >
\]
\[=
< \mt{x''t''}|\mt{ U(t'', t}_p)
\ \ul{\mt{X}}\ \mt{ U(t}_p,\mt{ t}_(p-1))
\ \ul{\mt{X}}...
\ \ul{\mt{X}}\ \mt{ U(t}_l,\mt{ t}_k)
\ \ul{\mt{X}}\ \mt{ U(t}_k,\mt{ t') }|\mt{x't'} >
\]
\begin{center}
avec t'' > t$_p$ > t$_{p-1}$ > ... > t$_l$ > t$_k$ > t'
\end{center}
%34

\ul{Remarques} :
a) Nous comprenons maintenant la raison pour laquelle nous avons adopté,
pour désigner les éléments de matrice au sens de Feynman, une notation différente de celle de Heisenberg. La relation (10) nous montre en effet que
les deux notions ne sont pas identiques et qu'il faut, pour passer de l'une
à l'autre, ranger de façon convenable les opérateurs.
b) Que devient maintenant, dans le formalisme de Heisenberg, l'opérateur
associé à la grandeur physique ?

Nous avons vu au paragraphe 1 qu'il faut, pour que cette
quantité soit définie, la remplacer par
\[
\lim_{\epsilon\to\, 0}\frac{\mt{x(t}_k)+\mt{x(t}_{k+1})}{2}\ \frac{\mt{x(t}_{k+1})-\mt{x(t}_k)}{\epsilon}
\]
Il lui correspond, dans le formalisme de Heisenberg, l'opérateur
\[
\lim_{\epsilon\to\, 0}\mt{P}\left[\frac{\mt{x(t}_k)+\mt{x(t}_{k+1})}{2}\ \frac{\mt{x(t}_{k+1})-\mt{x(t}_k)}{\epsilon}\right]
\]
\[
=\lim_{\epsilon\to\, 0}\left\{\frac{1}{2}\mt{X(t}_{k+1})\frac{\mt{X(t}_{k+1})-\mt{X(t}_k)}{\epsilon}+
\frac{1}{2}\ \frac{\mt{X(t}_{k+1})-\mt{X(t}_k)}{\epsilon}\mt{X(t}_k)\right\}
\]
\[
=\frac{1}{2}\left[\mt{X(t}_k)\mt{V(t}_k)+
\mt{V(t}_k)\mt{X(t}_k)\right]
\]


Si l'on adopte donc la définition de Feynman pour les opérateurs
associés à une grandeur physique, on obtient très naturellement la règle de
symétrisation : l'opérateur quantique correspondant à un produit de grandeurs
physiques et le produit symétrisé. Ce produit symétrisé est différent du
produit ordinaire si les opérateurs ne commutent pas.

\subsection{Cas général} % C.

Nous avons défini jusqu'à présent des opérateurs correspondant à
un ou plusieurs instants en nombre fini. Mous avons pu le faire, à la condition
de pouvoir associer à chaque histoire H de la particule définie par la
%35
suite (M', M$_1$, M$_2$, ... M$_i$, ... M$_{n-1}$, M'') une grandeur classique f$_\mt{H}$

Considérons maintenant que f$_\mt{H}$ ne dépend plus seulement d'une
suite finie de positions x$_\mt{H}$(t$_k$) mais, globalement, de tout le chemin x$_\mt{H}$(t).
Nous définissons ainsi comme grandeur physique une fonctionnelle du
chemin f$_\mt{H}=$ f$[$x$_\mt{H}$(t)$]$ qui dépend de \ul{toute} l'histoire H,

Dans les paragraphes 1 et 2, nous avons envisagé des fonctionnelles particulièrement simples. Par exemple :
\[
\tag{\S 1}\mt{f}[\mt{x}_\mt{H}(\mt{t})]=\mt{x}_\mt{H}(\mt{t}_k)
\]
\[
\tag{\S 2}\mt{f}[\mt{x}_\mt{H}(\mt{t})]=\mt{x}_\mt{H}(\mt{t}_k)\mt{x}_\mt{H}(\mt{t}_l)
\]

Nous pouvons maintenant envisager des fonctionnelles beaucoup
plus générales. Par exemple :
\[
\mt{f}[\mt{x}_\mt{H}(\mt{t})]=\int_{\mt{t'}}^{\mt{t''}}\mt{x}_\mt{H}\mt{(t)dt}
\]
\[
\tag{11}\mt{f}[\mt{x}_\mt{H}(\mt{t})]=\int_{\mt{t'}}^{\mt{t''}}\left\{\frac{1}{2}\mt{m}[\dot{\mt{x}}_\mt{H}\mt{(t)}]^2-\mt{v}[\mt{x}_\mt{H}(\mt{t})]\right\}\mt{dt}
\]
\[
=\mt{S}_\mt{H}
\]

Cette dernière fonctionnelle représente simplement l'action
pour l'histoire H. Comme nous l'avons déjà fait en 1 et 2, nous définissons
l'élément de matrice au sens de Feynman entre les états $<$ x''t'' $|$ et $|$ x't' $>$
de l'opérateur fonctionnel F\{X(t)\} correspondant à la grandeur classique f$[$x$_\mt{H}$(t)$]$
par la relation
\[
\tag{12}\subset \mt{ x''t'' }|\mt{ F\{X(t)\} }|\mt{ x't'}\supset\ =
\sum_\mt{H} \mt{N f \{x}_\mt{H}(\mt{t})\}\mt{ exp}\frac{\mt{i}}{\hbar}\mt{S}_\mt{H}
\]
par exemple on définit l'élément de matrice de l'opérateur action quantique
\[
\mc{S}=\int\left[\frac{1}{2}\mt{m}\left\{\frac{\mt{dX(t)}}{\mt{dt}}\right\}^2-\mc{V}[\mt{X(t), t}]\right]\mt{dt}
\]
\[
\tag{13}\subset \mt{ x''t'' }|\ \mc{S}\ |\mt{ x't'}\supset\ =
\sum_\mt{H} \mt{N S}_\mt{H}\mt{ exp }\frac{\mt{i}}{\hbar}\mt{ S}_\mt{H}
\]

%-36
\ul{En résumé} : le formalisme de Feynman nous a permis de définir les opérateurs
quantiques, non plus par leur action sur un vecteur d'état $|\ \phi\ >$ du système
envisagé à un instant donné t, mais plutôt par leurs éléments de matrice
entre deux états à des instants différents, ce qui les associe étroitement
à l'évolution du système entre t' et t''.

Ceci nous a permis de retrouver tout d'abord les opérateurs de
la mécanique quantique habituelle, en précisant de facon plus naturelle la
correspondance entre mécanique classique et quantique et en justifiant la
règle de symétrisation. Enfin, nous pouvons définir (formule 12) une classe
d'opérateurs beaucoup plus vaste, agissant sur tout un intervalle d'espace-temps, celle des opérateurs fonctionnels.

\ul{Remarque importante} : 
Tant que la grandeur classique f$[$x$_\mt{H}$(t)$]$ ne dépendait que d'une suite finie
d'instants t$_\mt{k}$ (\S 1, 2) il a été possible de trouver l'équivalent de l'élément
de matrice de Feynman dans le point de vue de Heisenberg (formule 10). Dans
le cas où f$[$x$_\mt{H}$(t)$]$ est la fonctionnelle la plus générale du chemin H, la
formule (10) peut se généraliser formellement :
\[
\tag{14}\subset \mt{ x''t'' }|\ \mt{F}\mt{\{X(t)\}}\ |\mt{ x't'}\supset\ =\ 
<\mt{ x''t'' }|\ \mt{P}[\mt{F}\mt{\{X(t)\}}]\ |\mt{ x't'}>
\]

La formule (14) signifie que s'il est possible, d'une façon ou d'une autre,
de développer la fonctionnelle F en une série de produits dépendant d'une
suite d'instants t$_1$, t$t_2$ ..., alors il faudra, pour passer au point de vue
de Heisenberg, ranger dans chacun des termes de la série les temps croissants
de la droite vers la gauche. Nous préciserons cette notion de façon naturelle
au paragraphe 4.

\section{Application : Théorie des perturbations dépendant du temps} % D

Le formalisme de Feynman est particulièrement bien adapté à
l'étude des variations apportées à l'amplitude de probabilité < x''t'' | x't' >
par une perturbation du hamiltonien ou du lagrangien du système. Nous allons
voir que l'on peut ainsi retrouver de façon très simple les résultats de la
théorie des perturbations dépendant du temps.
% 37

\subsection{Position du problème}%1°) 

Considérons une particule dans un potentiel v$_0$ (x, t).
Son lagrangien L$_0$ s'écrit
\[
\mt{L}_0=\frac{1}{2}\mt{m}\dot{\mt{x}}^2-\mt{v}_0\mt{(x, t)}
\]
et l'action S$_\mt{OH}$ associée à une histoire H :
\[
\mt{S}_\mt{OH} = \int_\mt{t'}^\mt{t''}\left[\frac{1}{2}\mt{m}\Big(\dot{\mt{x}}_\mt{H}(\mt{t})\Big)^2-\mt{v}_0\Big(\mt{x}_\mt{H}\mt{(t)}\Big)\right]\mt{dt}
\]

L'amplitude de probabilité relative au lagrangien L$_0$, que nous écrirons
$<$ x''t'' $|$ x't' $>_{\mt{S}_0}$ est alors
\[
\tag{15} <\mt{ x''t'' }|\mt{ x't' }>_{\mt{S}_\mt{0}}=\sum_\mt{H}\mt{ N exp }\frac{\mt{i}}{\hbar}\mt{ S}_\mt{0H}
\]
ce qui s'écrit dans le point de vue de Schrödinger
\[
<\mt{ x''t'' }|\mt{ x't' }>_{\mt{S}_0}=<\mt{ x'' }|\mt{ U}_0\mt{(t'',t') }|\mt{ x' }>
\]

U$_0$ (t'', t') étant l'opérateur d'évolution de la particule dans le potentiel v$_0$.

On change maintenant le potentiel qui devient v$_0$(x,t)+w(x,t).
Le lagrangien L s'écrit alors :
\[
\mt{L}=\frac{1}{2}\mt{m}\dot{\mt{x}}^2-\mt{v}_0(\mt{x,t})-\mt{w}(\mt{x,t})=\mt{L}_0-\mt{w}(\mt{x,t})
\]
l'action associée à une histoire H devient
\[
\tag{16}\mt{S}_\mt{H}=\mt{S}_\mt{0H}+\delta\mt{S}_\mt{H}
\]
avec
\[
\tag{17}\delta\mt{S}_\mt{H}=-\int_\mt{t'}^\mt{t''}\mt{w}\big[\mt{x}_\mt{H}(\mt{t})\mt{,t}\big]\mt{dt}
\]
Enfin l'amplitude de probabilité relative au système perturbé, $<$ x''t'' $|$ x't' $>_\mt{S}$ s'écrit
\[
\tag{18}<\mt{ x''t'' }|\mt{ x't' }>_\mt{S}=<\mt{ x'' }|\mt{ U(t'',t') }|\mt{ x' }>=\sum_\mt{H}\mt{ N exp }\frac{\mt{i}}{\hbar}\mt{ S}_\mt{H}
\]
(U(t'',t') étant l'opérateur d'évolution de système perturbé).

Le problème que l'on se pose est, connaissant l'amplitude de
$<$ x''t'' $|$ x't' $>_{\mt{S}_0}$
et la perturbation w (x,t), de déterminer $<$ x''t'' $|$ x't' $>_\mt{S}$, où
ce qui revient au même, dans un point de vue plus familier, connaissant
U$_0$
et w(x,t), déterminer U.

% 38
Pour l'instant, nous ne faisons aucune hypothèse sur la
grandeur de la perturbation w(x,t).

Un cas particulier important est celui où v$_0$ (x,t) = 0.
Le système non perturbé est alors celui d'une particule libre dont on sait
déterminer exactement le propagateur $<$ x''t'' $|$ x't' $>_{\mt{S}_0}$ (cf chapitre II).

Le problème revient alors à déterminer, à partir de l'opérateur d'évolution
de la particule libre, celui de la particule se déplaçant dans un potentiel
quelconque dépendant du temps w(x,t).

\subsection{Expression de $<$ x''t'' $|$ x't' $>_\mt{S}$}%2°)

De (16), (17), (18) on déduit immédiatement
\[
\tag{19} <\mt{ x''t'' }|\mt{ x't' }>_\mt{S}=\sum_\mt{H}\mt{ N exp }-\frac{\mt{i}}{\hbar}\int_\mt{t'}^\mt{t''}\mt{w}\big[\mt{x}_\mt{H}(\mt{t})\mt{,t}\big]\mt{dt exp }\frac{\mt{i}}{\hbar} \mt{ S}_\mt{0H}
\]
ce qui n'est autre que l'élément de matrice au sens de Feynman
\[
\subset \mt{ x''t'' }|\mt{ exp }-\frac{\mt{i}}{\hbar}\int_\mt{t'}^\mt{t''}\mt{W}\big[\mt{X}(\mt{t})\mt{,t}\big]\mt{ dt }|\mt{ x't'}\supset_{\mt{ S}_0}
\]
W$\big[$X(t),t$\big]$ étant l'opérateur quantique associé à w et l'indice S$_0$ étant là
pour rappeler que l'élément de matrice est relatif au système non perturbé.

Nous pouvons donner de l'expression précédente une interprétation physique imagée très simple :

La perturbation w multiplie la contribution de chaque chemin H à l'amplitude
de probabilité par un facteur de déphasage exp $-\frac{\mt{i}}{\hbar}\int_\mt{t'}^\mt{t''}\mt{w}\big[\mt{x}_\mt{H}(\mt{t})\mt{,t}\big]$ dt, La
modification globale de l'amplitude de probabilité s'obtient en sommant
tous ces déphasages sur les chemins. On peut dire, par analogie avec l'optique,
que la perturbation modifie "l'indice de réfraction" dans l'espace-temps et
que le calcul de perturbation se ramène à un calcul de variation d'indice.

% 39
Afin d'alléger les notations, posons :
\[
\begin{array}{rcr}
\mt{w}\big[x_\mt{H}(t),t\big] & = & \mt{w}_\mt{H}\mt{(t)} \\
\mt{W}\big[\mt{X(t),t}\big] & = & \mt{W(t)}
\end{array}
\]
W (t) dépend du temps explicitement et implicitement \big[ par l'intermédiaire
de X (t) \big] . L'opérateur correspondant dans le point de vue de Schrödinger
n'a plus qu'une dépendance explicite dans le temps. Nous l'écrirons \ul{W} (t).
Avec nos nouvelles notations, (19) s'écrit :
\[
\tag{20} <\mt{ x''t'' }|\mt{ x't' }>_\mt{S}\ =\ \subset \mt{ x''t'' }|\mt{ exp }-\frac{\mt{i}}{\hbar}\int_\mt{t'}^\mt{t''}\mt{W (t) dt }|\mt{ x't'}\supset_{\mt{ S}_0}
\]
\[
\tag{21} \ \ \ \ \ \ \ \ \ \ \ \ \ \ \ \ \ \ \ \ \ \ \ \ \ \ \ \ \ \ \ =\ <\mt{ x''t'' }|\mt{ P }\left[\mt{ exp }-\frac{\mt{i}}{\hbar}\int_\mt{t'}^\mt{t''}\mt{W (t) dt }\right]|\mt{ x't' }>_0
\]

L'opérateur d'ordre P a été défini dans le paragraphe 2.3. Nous ne pourrons lui
donner une signification précise qu'en développant l'exponentielle en série,
L'indice 0 de la formule (21) rappelle que les états | x't' > et | x''t'' >
sont les états propres des opérateurs de Heisenberg X(t') et X(t'') de la
\ul{particule non perturbée}.

\subsection{Développerent en série de la perturbation}%3°)
Développons en série :
\[
\tag{22} \mt{ exp }-\frac{\mt{i}}{\hbar}\int_\mt{t'}^\mt{t''}\mt{W (t) dt }\ =\ 1 +\frac{1}{\mt{i}\hbar}\ \frac{1}{1!}\ \int_\mt{t'}^\mt{t''}\mt{d}\tau_1\mt{ W }(\tau_1)
\]
\[
+\ (\frac{1}{\mt{i}\hbar})^2\ \frac{1}{2!}\ \int_\mt{t'}^\mt{t''}\int_\mt{t'}^\mt{t''}\mt{d}\tau_1\ \mt{d}\tau_2\mt{ W }(\tau_1)\mt{ W }(\tau_2)\ +\ ...
\]

Reportons ce développement dans la formule (20) : on obtient
\[
\tag{23} <\mt{ x''t'' }|\mt{ x't' }>_\mt{S}\ =\ \subset \mt{ x''t'' }|\ 1\ |\mt{ x't'}\supset_{\mt{S}_0}
\]
\[
 +\frac{1}{\mt{i}\hbar}\ \frac{1}{1!}\  \subset \mt{ x''t'' }|\int_\mt{t'}^\mt{t''}\mt{ W }(\tau_1)\ \mt{d}\tau_1\ |\mt{ x't'}\supset_{\mt{S}_0}
\]
\[
 +(\frac{1}{\mt{i}\hbar})^2\ \frac{1}{2!}\  \subset \mt{ x''t'' }|\int_\mt{t'}^\mt{t''}\int_\mt{t'}^\mt{t''}\mt{ W }(\tau_1)\ \mt{ W }(\tau_2)\ \mt{d}\tau_1\ \mt{d}\tau_2\ |\mt{ x't'}\supset_{\mt{S}_0}\ +\ ...
\]
%

Pour passer au point de vue de Heisenberg, nous devons faire
agir sur chaque terme de la série (22) l'opérateur d'ordre P.

(23) conduit alors à :
\[
\tag{24} <\mt{ x''t'' }|\mt{ x't' }>_\mt{S}\ =\ <\mt{ x''t'' }|\mt{ x't'}>_{\mt{S}_0}
 +\ \frac{1}{\mt{i}\hbar}\ <\mt{ x''t'' }|\int_\mt{t'}^\mt{t''}\mt{d}\tau_1\mt{ W}(\tau_1)\ |\mt{ x't'}>_0
\]
\[
 +(\frac{1}{\mt{i}\hbar})^2\ \frac{1}{2!}\ <\mt{ x''t'' }|\mt{P}\left[\int_\mt{t'}^\mt{t''}\int_\mt{t'}^\mt{t''}\mt{d}\tau_1\ \mt{d}\tau_2\mt{ W}(\tau_1)\mt{ W}(\tau_2)\right]\ |\mt{ x't'}>_0
\]
\[
 +(\frac{1}{\mt{i}\hbar})^3\ \frac{1}{3!}\ <\mt{ x''t'' }|\mt{P}\left[\int_\mt{t'}^\mt{t''}\int_\mt{t'}^\mt{t''}\int_\mt{t'}^\mt{t''}\mt{d}\tau_1\ \mt{d}\tau_2\ \mt{d}\tau_3\mt{ W}(\tau_1)\mt{ W}(\tau_2)\mt{ W}(\tau_3)\right]\ |\mt{ x't'}>_0\ +\ ...
\]

L'opérateur P va ranger, dans chaque intégrale les opérateurs W qui ne commutent pas dans le point de vue de Heisenberg par ordre de temps croissants
de la droite vers la gauche.

On montre la relation générale :
\[
\tag{25}\mt{P}\left[\int_\mt{t'}^\mt{t''}\int_\mt{t'}^\mt{t''}...\int_\mt{t'}^\mt{t''}\mt{d}\tau_1\ \mt{d}\tau_2\ ...\mt{d}\tau_\mt{n}\mt{ W}(\tau_1)\mt{ W}(\tau_2)...\mt{ W}(\tau_\mt{n})\right]
\]
\[
=\mt{ n!}\int\int ... \int_{\mt{t''}>\tau_\mt{n}>\tau_\mt{n-1}>...>\tau_1> t'}\mt{ W}(\tau_\mt{n})\mt{ W}(\tau_\mt{n-1})...\mt{ W}(\tau_2)\mt{ W}(\tau_1)\mt{d}\tau_1\ \mt{d}\tau_2\ ...\mt{d}\tau_\mt{n}
\]

{\it Preuve} : Il existe n! façons de ranger les n variables $\tau_1$, $\tau_2$, .. $\tau_\mt{n}$ dans un ordre
donné. Le région d'intégration de l'espace R$^n$ : t'' > $\tau_1$ > t'; t'' > $\tau_2$ > t';...  t'' > $\tau_\mt{n}$ > t'...
est constituée par la réunion des n! régions correspondant à
chacun de ces arrangements. L'intégrale I = $\int_\mt{t'}^\mt{t''}..\int_\mt{t'}^\mt{t''}\mt{d}\tau_1..\ \mt{d}\tau_\mt{n}\mt{ W}(\tau_1)..\mt{ W}(\tau_\mt{n})$
est la somme de n! intégrales prises sur chacune de ces régions. A chacune de
ces intégrales, l'opérateur d'ordre P, qui range les opérateurs par ordre de
temps croissants de la droite vers la gauche, fait correspondre par définition
la \ul{même} intégrale J = $\int_{\tau_\mt{n}\ >}..\int_{>\ \tau_\mt{1}}\mt{d}\tau_1..\ \mt{d}\tau_\mt{n}\mt{ W}(\tau_\mt{n})..\mt{ W}(\tau_1)$.
On a donc P\big[\;I\;\big] = n! J,  ce qui n'est autre que la relation (25)

(24) devient alors
\[
 <\mt{ x''t'' }|\mt{ x't' }>_\mt{S}\ =\ <\mt{ x''t'' }|\mt{ x't'}>_{\mt{S}_0}
 +\ \frac{1}{\mt{i}\hbar}\ <\mt{ x''t'' }|\int_\mt{t'}^\mt{t''}\mt{d}\tau_1\mt{ W}(\tau_1)\ |\mt{ x't'}>_0
\]
\[
 +(\frac{1}{\mt{i}\hbar})^2\ <\mt{ x''t'' }|\iint_{\tau_2>\tau_1}\mt{W}(\tau_1)\mt{ W}(\tau_2)\ \mt{d}\tau_1\ \mt{d}\tau_2\ |\mt{ x't'}>_0
\]
\[
\tag{26} +(\frac{1}{\mt{i}\hbar})^3\ <\mt{ x''t'' }|\iiint_{\tau_3>\tau_2>\tau_1}\mt{W}(\tau_3)\mt{ W}(\tau_2)\mt{ W}(\tau_1)\ \mt{d}\tau_1\mt{d}\tau_2\mt{d}\tau_3\ |\mt{ x't'}>_0\ +\ ...
\]
Les factorielles de la relation (24) ont disparu.

Passons maintenant au point de vue de Schrödinger :
(26) devient
\[
\tag{27} <\mt{ x'' }|\mt{ U(t'', t') }|\mt{ x' }>\ =\ <\mt{ x'' }|\mt{ U}_0\mt{(t'', t') }|\mt{ x'}>
\]
\[
+\ \frac{1}{\mt{i}\hbar}\ <\mt{ x'' }|\int_\mt{t'}^\mt{t''}\mt{ U}_0\mt{(t''}, \tau_1)\ \overline{\mt{W}}(\tau_1)
\mt{ U}_0(\tau_1,\mt{ t') }\mt{d}\tau_1\ |\mt{ x'}>
\]
\[
 +(\frac{1}{\mt{i}\hbar})^2\ <\mt{ x'' }|\iint_{\tau_2>\tau_1}\mt{U}_0\mt{(t''}, \tau_2)\ \overline{\mt{W}}(\tau_2)
\mt{ U}_0(\tau_2, \tau_1)\ \overline{\mt{W}}(\tau_1)\ \mt{ U}_0(\tau_1,\mt{ t'})\ \mt{d}\tau_1\ \mt{d}\tau_2\ |\mt{ x'}>\ +\ ...
\]
ce qui nous donne le développement de U en puissance de la perturbation :
\[
\mt{U}\mt{(t'',t'})=\mt{U}_0\mt{(t'',t'})+\mt{U}_1\mt{(t'',t'}) + ... +\mt{U}_\mt{n}\mt{(t'',t'})
\]

% 42
avec
\[
\tag{28}\left\{ \begin{array}{rcl}
\mt{U}_1\mt{(t'',t'}) & = & \frac{1}{\mt{i}\hbar}\ <\mt{ x'' }|\int_\mt{t'}^\mt{t''}\mt{ U}_0\mt{(t''}, \tau_1)
\ \overline{\mt{W}}(\tau_1)\mt{ U}_0(\tau_1,\mt{ t') }\mt{d}\tau_1\ |\mt{ x'}> \\
 & : & \\
 & : & \\
\mt{U}_\mt{n}\mt{(t'',t'}) & = & (\frac{1}{\mt{i}\hbar})^\mt{n}\int...\int\mt{U}_0\mt{(t''},\tau_\mt{n})
\overline{\mt{W}}(\tau_\mt{n})\mt{U}_0(\tau_\mt{n},\tau_{\mt{n}-1})...\overline{\mt{W}}(\tau_1)
\mt{U}_0(\tau_1,\mt{t'})\mt{d}\tau_\mt{n}...\mt{d}\tau_1 \\
 &  & \mt{t''}>\tau_\mt{n}>\tau_{\mt{n}-1}>...>\tau_1>\mt{t'} \end{array}
\right. \]

Nous retrouvons ainsi le développement classique à tous les
ordres de la théorie des perturbations dépendant du temps (cf Messiah, tome II,
p. 620).

\ul{Remarque} :
Le formalisme de Feynman nous a permis d'obtenir, \ul{sans aucun} \ul{calcul}, une
forme symbolique et intégrée de ce développement à tous les ordres (formule 21)
grâce à l'introduction de l'opérateur d'ordre P.

En mécanique quantique habituelle, on procède de façon inverse.
On étebiit d'abord le développement (28) dans lequel les temps $\tau_1,\,\tau_2\,...\,\tau_\mt{n}$
sont ordonnés. On s'affranchit ensuite de cet ordre en faisant apparaître artificiellement les factorielles
qui conduisent à la relation (24) et à la forme symbolique (21).

Les calculs sont ainsi plus longs et plus compliqués que dans
le formalisme de Feynman dans lequel les opérateurs correspondant à des temps
différents commutent. Ce formalisme présente en outre le grand avantage de
donner une signification physique simple (en terme de modification de
l'indice de réfraction dans l'espace-temps) à la théorie des perturbations
dépendant du temps.


\chapter{Principe d'action de Schwinger}%IV
%43
\section{Introduction}%A
Dans les chapitres précédents, nous avons donné un énoncé
des postulats de Feynman de la mécanique quantique et montré comment
les notions de fonction d'onde, d'équation d'onde et d'opérateurs s'en
déduisaient simplement. Rappelons seulement que nous avons défini l'amplitude de probabilité $<$ x''t'' $|$ x't' $>$ par la relation
\[
\tag{1}<\mt{ x''t'' }|\mt{ x't' }>\ =\sum_\mt{H}\mt{ N exp }\frac{\mt{i}}{\hbar}\mt{ S}_\mt{H}
\]
et l'élément de matrice d'un opérateur G associé à la grandeur classique
g par
\[
\tag{2}\subset \mt{x''t'' }|\mt{ G }|\mt{ x't'}\supset\ =
\sum_\mt{H}\mt{ N g}_\mt{H}\mt{ exp }\frac{\mt{i}}{\hbar}\mt{ S}_\mt{H}
\]
Une telle formulation correspond à une interprétation physique très claire,
en termes d'expériences d'interférence et de diffraction dans l'espace-tenps,
mais conduit en pénéral à des calculs très compliqués.

Nous abordons ici le problème sous un angle nouveau en calculant
la variation $\delta<$ x''t'' $|$ x't' $>$ de l'amplitude de probabilité $<$ x''t'' $|$ x't' $>$
pour des variations infinitésimales de x', t' et x'', t''. Nous donnons tout
d'abord le principe du calcul (\S 2), ce qui nous permet de le rattacher
au problème de la variation de l'action classique que nous traitons ensuite
(\S 3). Nous trouvons alors une expression de $\delta<$ x''t'' $|$ x't' $>$ qui nous
permet d'établir des équations de Lagrange entre opérateurs (\S 4), de définir
l'opérateur d'impulsion P (\S 5) et l'opérateur hamiltonien $\mc{H}$(\S 6).

Si le calcul de $<$ x''t'' $|$ x't' $>$ est assez compliqué, nous allons
voir que celui de $\delta<$ x''t'' $|$ x't' $>$ est beaucoup plus simple et conduit à
une introduction élégante et naturelle des relations de commutations, de
l'équation de Schrödinger, etc.

%44

\section{Princine du calcul de $\delta<$ x''t'' $|$ x't' $>$}%B

\begin{center} \begin{tikzpicture}
% axes
\draw [->, very thick] (-0.3,0) --++ (8,0) node [below] {t};
\draw [->, very thick] (0,-0.3) --++ (0,4) node [left] {x};
% Chemins
\draw [line width=1.5pt] (1,3) node [above]{M'} .. controls +(1,0) and +(-1,0.3) .. (4,2.7);
\draw [line width=1.5pt] (4,2.7) node [above]{M} .. controls +(1,-0.3) and +(-1,0.3) .. (6.5,2) node [right]{M''};
\draw [line width=1.5pt, dashed] (0.5,2.5) node [below]{N'} .. controls +(1,0) and +(-1,0.5) .. (3.5,1.9);
\draw [line width=1.5pt, dashed] (3.5,1.9) node [below]{N} .. controls +(1,-0.5) and +(-1,0.5) .. (7,1) node [right]{N''};
% abscisses
\draw [very thin] (1,0) node [below] {t'} -- (1,3);
\draw [very thin] (4,0) node [below] {t} -- (4,2.7);
\draw [very thin] (6.5,0) node [below] {t''} -- (6.5,2);
% déplacements
\draw [very thin, dashed] (1,3) -- (0.5,2.5);
\draw [very thin, dashed] (4,2.7) -- (3.5,1.9);
\draw [very thin, dashed] (6.5,2) -- (7,1);
% points
\draw (1.1,2.58)  --(0.9,2.38) node [above right]{P'};
\draw (1.1,2.38) -- (0.9,2.58);
\draw (4.1,1.8) -- (3.9,1.6) node [above right]{P};
\draw (4.1,1.6) -- (3.9,1.8);
\draw (6.6,1.3) -- (6.4,1.1);
\draw (6.4,1.3) -- (6.6,1.1) node [above left]{P''};
\end{tikzpicture} \end{center}
Il s'agit de calculer la quantité
\begin{center}
$\delta<$ x''t'' $|$ x't' $>=<$ x''+$\delta$x'',t''+$\delta$t'' $|$ x'+$\delta$x',t'+$\delta$t' $>-<$ x''t'' $|$ x't' $>$
\end{center}
$\delta$x'',$\delta$t'' et $\delta$x',$\delta$t' représentant la variation infinitésimale la plus
générale des points d'espace-temps M' et M'' qui se trouvent déplacés
en N' et N''.

Le principe du calcul est d'associer à chaque chemin H joignant
M' à M'' (en trait plein sur la figure) un chemin \ul{H} joignent N' à N'' (en
trait pointillé). Pour cela on se donne deux fonctions continues infinitésimales du temps $\delta$x(t) et $\delta$t(t), assujetties aux conditions aux limites.
\[
\tag{3}\left\{ \begin{array}{c}
 \delta\mt{x(t')}=\delta\mt{x'} \\
 \delta\mt{t(t')}=\delta\mt{t'} \\ \end{array} \right.
\left\{ \begin{array}{c}
 \delta\mt{x(t'')}=\delta\mt{x''} \\
 \delta\mt{t(t'')}=\delta\mt{t''} \\ \end{array} \right.
\]
A chaque point M(x, t) d'un chemin M'M'', on associe ainsi un point
N$[$ x+$\delta$x(t), t+$\delta$t(t) $]$. A M' correspond N' et à M'', N''. Ainsi à
tout chemin reliant M' à M'' correspond un chemin relient N' à N'' et réciproquement. Insistons sur le fait que les fonctions $\delta$x(t) et $\delta$t(t) sont
des fonctions de t ne dépendant pas de l'histoire H.

Appelons maintenant S$_\mt{H}$, l'action pour l'histoire H, S$_{\ul{\mt{H}}}$=S$_\mt{H}+\delta$S$_\mt{H}$
l'action pour l'histoire \ul{H}.

%45
Nous avons, d'après la relation (1)
\[
<\mt{ x''t'' }|\mt{ x't' }>\ =\sum_\mt{H}\mt{ N exp }\frac{\mt{i}}{\hbar}\mt{ S}_\mt{H}
\]
\[
<\mt{x''}+\delta\mt{x'',t''}+\delta\mt{t''}|\mt{x'}+\delta\mt{x',t'}+\delta\mt{t'}>=
\sum_{\ul{\mt{H}}}\mt{ N exp }\frac{\mt{i}}{\hbar}\mt{ S}_{\ul{\mt{H}}}=
\sum_\mt{H}\mt{ N exp }\frac{\mt{i}}{\hbar}\mt{ S}_\mt{H}\mt{ exp }\frac{\mt{i}}{\hbar}\ \delta\mt{S}_\mt{H}
\]
Nous pouvons déveloprer au premier ordre exp $\frac{\mt{i}}{\hbar}\ \delta$S$_H=1+\frac{\mt{i}}{\hbar}\ \delta$S$_H$

Il vient alors, par simple soustraction
\[
\tag{4}\delta<\mt{ x''t'' }|\mt{ x't' }>\ =
\frac{\mt{i}}{\hbar}\sum_\mt{H}\mt{ N }\delta\mt{S}_\mt{H}\mt{ exp }\frac{\mt{i}}{\hbar}\mt{ S}_\mt{H}
\]
Si $\mc{S}$ est l'opérateur correspondant à l'action classique (formule (13) du

chapitre III), la relation (4) s'écrit
\[
\tag{5}\delta<\mt{x''t'' }|\mt{ x't'}>\ =
\frac{\mt{i}}{\hbar}\subset \mt{x''t'' }|\ \delta\mc{S}\ |\mt{ x't'}\supset
\]
Sous la forme différentielle (5), la formulation de Feynman constitue le
\ul{Principe d'Action de} \ul{Schwinger}. Pour expliciter cette formule (5), nous
sommes amenés à calculer la variation $\delta\mt{S}_\mt{H}$ de l'action classique le long
de l'histoire H, pour des variations infinitésimales du chemin \ul{et} des
extrémités.

\section{Variation de l'action classique : Calcul de $\delta$S$_\mt{H}$ et Principe d'Action de Schwinger}%C
Soit P le point du chemin \ul{H} correspondant au même temps que M.
Les coordonnées de M sont x, t, celles de N sont x + $\delta$x(t) et t + $\delta$t(t),
celles de P sont x + $\Delta$x$_\mt{H}$(t) et t.

%46
Il est évident que $\Delta$x$_\mt{H}$(t) est différent de $\delta$x(t) et qu'au premier
ordre, on a :
\[
\tag{6}\Delta\mt{x}_\mt{H}=\delta\mt{x(t)}-\dot{\mt{x}}_\mt{H}\mt{(t)}\ \delta\mt{t(t)}
\]
(On voit immédiatement sur la relation (6) qu'alors que $\delta$x(t) est indépendant
de H, $\Delta$x$_\mt{H}$(t) en dépend ( par l'intermédiaire de $\dot{\mt{x}}_\mt{H}$(t) ). C'est ce qui justifie l'indice H dans la notation $\Delta$x$_\mt{H}$(t).

On comprend d'autre part qu'il a été nécessaire d'introduire les deux variations indépendantes $\delta$x(t) et $\delta$t(t), au lieu d'une seule variation $\Delta$x$_\mt{H}$(t),
de façon à pouvoir varier dans le temps les extrémités du chemin H.)

Soient de même P' et P'' les points du chemin \ul{H} aux instants t' et t''.
S$_{\ul{\mt{H}}}$ s'écrit alors sous forme d'intégrale curviligne :
\[
\tag{7}\mt{S}_{\ul{\mt{H}}}=\oint^\mt{N''}_\mt{N'}\mt{L}_{\ul{\mt{H}}}\mt{(t) dt} =
=\oint^\mt{P'}_\mt{N'}\mt{L}_{\ul{\mt{H}}}\mt{ dt}
+\oint^\mt{P''}_\mt{P'}\mt{L}_{\ul{\mt{H}}}\mt{ dt}
+\oint^\mt{N''}_\mt{P''}\mt{L}_{\ul{\mt{H}}}\mt{ dt}
\]
L$_{\ul{\mt{H}}}$(t) étant le lagrangien classique à l'instant t pour l'histoire \ul{H}.
Nous allons évaluer successivement les différents termes de la formule (7) :
\[
\tag{8}\oint^\mt{P'}_\mt{N'}\mt{L}_{\ul{\mt{H}}}\mt{ dt}\simeq
-\delta\mt{t(t') L}_{\ul{\mt{H}}}\mt{(P')}=-\delta\mt{t' L}_{\ul{\mt{H}}}\mt{(P')}\simeq
-\delta\mt{t' L}_{\ul{\mt{H}}}\mt{(M')}=-\delta\mt{t' L}_{\ul{\mt{H}}}\mt{(t')}
\]
De même
\[
\tag{9}\int^\mt{N''}_\mt{P''}\mt{L}_{\ul{\mt{H}}}\mt{ dt}\simeq
\delta\mt{t'' L}_\mt{H}\mt{(t'')}
\]
Les approximations que nous avons faites, qui reviennent à remplacer L$_{\ul{\mt{H}}}$(t)
par la constante L$_\mt{H}$(t'), lagrangien relatif au chemin H à l'instant t', sont
justifiées car nous cherchons une expression au premier ordre.

%47
Enfin
\[
\tag{10}\oint^\mt{P''}_\mt{P'}\mt{L}_{\ul{\mt{H}}}\mt{ dt}=
\int^\mt{t''}_\mt{t'}\mt{L}[\mt{ x}_\mt{H}\mt{(t)}+\Delta\mt{x}_\mt{H}\mt{(t), }
\dot{\mt{x}}_\mt{H}\mt{(t)}+\Delta\dot{\mt{x}}_\mt{H}\mt{(t), t }]\mt{ dt}
\]
Notons que $\dot{\mt{x}}_\mt{H}$(t)+$\Delta\dot{\mt{x}}_\mt{H}$(t) représente la vitesse au point P, égale à
$\frac{\mt{d}}{\mt{dt}}[\mt{x}_\mt{H}$(t)+$\Delta\mt{x}_\mt{H}$(t)] et qu'il en résulte la relation évidente
$\Delta\dot{\mt{x}}_\mt{H}$(t) = $\frac{\mt{d}}{\mt{dt}}\Delta\mt{x}_\mt{H}$(t).

En regroupant les relations (8), (9) et (10) et en soustrayant
\[
\mt{S}_\mt{H}=
\int^\mt{t''}_\mt{t'}\mt{L }[\mt{ x}_\mt{H}\mt{(t)},
\dot{\mt{x}}_\mt{H}\mt{(t) }]\mt{ dt,}
\]
il vient
\[
\tag{11}\delta\mt{S}_\mt{H}=-\delta\mt{t' }\mt{L}_\mt{H}\mt{(t')}+\delta\mt{t'' }\mt{L}_\mt{H}\mt{(t'')}+
\int^\mt{t''}_\mt{t'}\mt{L}[\mt{ x}_\mt{H}+\Delta\mt{x}_\mt{H}\mt{, }
\dot{\mt{x}}_\mt{H}+\Delta\dot{\mt{x}}_\mt{H}\mt{, t }]\mt{ dt}
\]
\[-
\int^\mt{t''}_\mt{t'}\mt{L }[\mt{ x}_\mt{H},
\dot{\mt{x}}_\mt{H}\mt{, t }]\mt{ dt}
\]
En développant les deux derniers termes jusqu'au premier ordre en $\Delta$x$_\mt{H}$ et $\Delta\dot{\mt{x}}_\mt{H}$
il vient
\[
\int^\mt{t''}_\mt{t'}\mt{L}[\mt{ x}_\mt{H}+\Delta\mt{x}_\mt{H}\mt{, }
\dot{\mt{x}}_\mt{H}+\Delta\dot{\mt{x}}_\mt{H}\mt{, t }]\mt{ dt}
-\int^\mt{t''}_\mt{t'}\mt{L }[\mt{ x}_\mt{H},
\dot{\mt{x}}_\mt{H}\mt{, t }]\mt{ dt}
\]
\[
=\int^\mt{t''}_\mt{t'}(\Delta\mt{x}_\mt{H}\frac{\partial\mt{L}}{\partial\mt{x}_\mt{H}}
+\Delta\dot{\mt{x}}_\mt{H}\frac{\partial\mt{L}}{\partial\dot{\mt{x}}_\mt{H}})\mt{ dt}
\]
\[
=\int^\mt{t''}_\mt{t'}\left[\Delta\mt{x}_\mt{H}\frac{\partial\mt{L}}{\partial\mt{x}_\mt{H}}
+(\frac{\mt{d}}{\mt{dt}}\Delta\mt{x}_\mt{H})\frac{\partial\mt{L}}{\partial\dot{\mt{x}}_\mt{H}}\right]\mt{ dt}
\]
%48
et après une intégration par parties élémentaires :
\[
\tag{12}\int^\mt{t''}_\mt{t'}\mt{L }[\mt{ x}_\mt{H}+\Delta\mt{x}_\mt{H}\mt{, }
\dot{\mt{x}}_\mt{H}+\Delta\dot{\mt{x}}_\mt{H}\mt{, t }]\mt{ dt}
-\int^\mt{t''}_\mt{t'}\mt{L }[\mt{ x}_\mt{H},
\dot{\mt{x}}_\mt{H}\mt{, t }]\mt{ dt}
\]
\[
=\Delta\mt{x}_\mt{H}\mt{(t'')}\frac{\partial\mt{L}}{\partial\dot{\mt{x}}_\mt{H}}\mt{(t'')}
-\Delta\mt{x}_\mt{H}\mt{(t')}\frac{\partial\mt{L}}{\partial\dot{\mt{x}}_\mt{H}}\mt{(t')}
+\int^\mt{t''}_\mt{t'}\Delta\mt{x}_\mt{H}\left[\frac{\partial\mt{L}}{\partial\mt{x}_\mt{H}}
-\frac{\mt{d}}{\mt{dt}}\frac{\partial\mt{L}}{\partial\dot{\mt{x}}_\mt{H}}\right]\mt{ dt}
\]
\[
=\mt{p}_\mt{H}\mt{(t'')}\Delta\mt{x}_\mt{H}\mt{(t'')} 
-\mt{p}_\mt{H}\mt{(t')}\Delta\mt{x}_\mt{H}\mt{(t')}
+\int^\mt{t''}_\mt{t'}\Delta\mt{x}_\mt{H}\left[\frac{\partial\mt{L}}{\partial\mt{x}_\mt{H}}
-\frac{\mt{d}}{\mt{dt}}\frac{\partial\mt{L}}{\partial\dot{\mt{x}}_\mt{H}}\right]\mt{ dt}
\]
p$_\mt{H}$(t)$=\frac{\partial\mt{L}}{\partial\dot{\mt{x}}_\mt{H}}$(t) est le moment conjugué de x.

Finalement, compte tenu de (12), (11) devient
\[
\tag{13}\delta\mt{S}_\mt{H}=\int^\mt{t''}_\mt{t'}\Delta\mt{x}_\mt{H}(\frac{\partial\mt{L}}{\partial\mt{x}_\mt{H}}
-\frac{\mt{d}}{\mt{dt}}\frac{\partial\mt{L}}{\partial\dot{\mt{x}}_\mt{H}})\mt{ dt}
+\mt{L}_\mt{H}\mt{(t'') }\delta\mt{t''}+\mt{p}_\mt{H}\mt{(t'') }\Delta\mt{x}_\mt{H}\mt{(t'')}
\]
\[
-\mt{L}_\mt{H}\mt{(t') }\delta\mt{t'}-\mt{p}_\mt{H}\mt{(t') }\Delta\mt{x}_\mt{H}\mt{(t')}
\]
On peut maintenant éliminer $\Delta$x$_\mt{H}$ en utilisant la relation (6).
On introduit alors la fonction hamiltonienne H$_\mt{H}$(t) = P$_\mt{H}$(t) $\dot{\mt{x}}_\mt{H}$(t) - L$_\mt{H}$(t)
et (13) devient
\[
\tag{14}\delta\mt{S}_\mt{H}=\int^\mt{t''}_\mt{t'}(\delta\mt{x(t)}-\dot{\mt{x}}_\mt{H}\mt{(t) }\delta\mt{t(t)})
\left(\frac{\partial\mt{L}}{\partial\mt{x}_\mt{H}}
-\frac{\mt{d}}{\mt{dt}}\frac{\partial\mt{L}}{\partial\dot{\mt{x}}_\mt{H}}\right)\mt{ dt}
\]
\[
-\mt{H}_\mt{H}\mt{(t'') }\delta\mt{t''}+\mt{p}_\mt{H}\mt{(t'') }\delta\mt{x''}
+\mt{H}_\mt{H}\mt{(t') }\delta\mt{t'}-\mt{p}_\mt{H}\mt{(t') }\delta\mt{x'}
\]
Remarques :

— Le calcul précédent est un calcul de base de mécanique analytique \ul{classique}.
Il montre notamment comment s'introduisent de façon naturelle les notions de
moment conjugué et de hamiltonien.

— Si on pose $\delta$x' = $\delta$x'' = $\delta$t' = $\delta$t'' = 0, la variation de l'action $\delta$S$_\mt{H}$ se
réduit à $\int^\mt{t''}_\mt{t'}\Delta\mt{x}_\mt{H}(\frac{\partial\mt{L}}{\partial\mt{x}_\mt{H}}-\frac{\mt{d}}{\mt{dt}}\frac{\partial\mt{L}}{\partial\dot{\mt{x}}_\mt{H}})$dt. Si on postule que le chemin classique
%49
entre M' et M'' est le chemin d'action stationnaire, la trajectoire
classique x$_\mt{C}$(t) doit être telle que $\delta$S$_\mt{H}$ est nul $\forall\Delta$x(t), et on retrouve alors les équations de Lagrange $\frac{\partial\mt{L}}{\partial\mt{x}_\mt{C}}-\frac{\mt{d}}{\mt{dt}}\frac{\partial\mt{L}}{\partial\dot{\mt{x}}_\mt{C}}=0$

Cependant, pour une histoire H \ul{quelconque}, l'intégrale $\int^\mt{t''}_\mt{t'}(\frac{\partial\mt{L}}{\partial\mt{x}_\mt{H}}-\frac{\mt{d}}{\mt{dt}}\frac{\partial\mt{L}}{\partial\dot{\mt{x}}_\mt{H}})\Delta$x$_\mt{H}$
 dt n'est pas nulle et doit, en conséquence,
figurer dens l'expression (14).

La relation (14) est une relation de la mécanique classique,
entre grandeurs classiques.

Cependant, nous avons vu (chapitre III) que le formalisme de
Feynman permet d'associer à toute grandeur classique g$_\mt{H}$ un orérateur
quantique G par la relation (2). Ainsi, à x$_\mt{H}$(t) on associe l'opérateur
X(t), à $\dot{\mt{x}}_\mt{H}$(t) l'opérateur $\dot{\mt{X}}$(t), à L(x$_\mt{H}$, $\dot{\mt{x}}_\mt{H}$) l'opérateur $\mc{L}$(X, $\dot{\mt{X}}$),
à p = $\frac{\partial\mt{L}}{\partial\dot{\mt{x}}}$ l'opérateur P = $\frac{\partial\mc{L}}{\partial\dot{\mt{X}}}$ et enfin au hamiltonien H $=\mt{p}_\mt{H}\dot{\mt{x}}_\mt{H}-\mt{L}_\mt{H}$,
l'opérateur $\mc{H}=\mt{P}\dot{\mt{X}}-\mc{L}(\mt{X},\dot{\mt{X}})$.

La relation (14) se transcrit immédiatement aux opérateurs en
raison de la linéarité de la relation (2) et compte tenu de (5), il vient
\[
\tag{15}\delta<\mt{ x''t'' }|\mt{ x't' }>\ =\ 
<\mt{x''+}\delta \mt{x'', t''}+\delta \mt{t'' }|\mt{ x'}+\delta \mt{x', t'}+\delta \mt{t'}> - <\mt{ x''t'' }|\mt{ x't' }>
\]
\[
=\frac{\mt{i}}{\hbar}\subset \mt{x''t'' }|\ \delta\mt{S}\ |\mt{ x't'}\supset
\]
\[
=\frac{\mt{i}}{\hbar}\subset \mt{x''t'' }|\mt{ P(t'')}\delta\mt{x''}-\mt{ P(t')}\delta\mt{x'}
-\mc{H}\mt{(t'')}\delta\mt{t''}+\mc{H}\mt{(t')}\delta\mt{t'}\ |\mt{ x't'}\supset
\]
\[
+\frac{\mt{i}}{\hbar}\subset \mt{x''t'' }
\int^\mt{t''}_\mt{t'}
\left(\frac{\partial\mc{L}}{\partial\mt{X}}
-\frac{\mt{d}}{\mt{dt}}\frac{\partial\mc{L}}{\partial\dot{\mt{X}}_\mt{H}}\right)
\left[\delta\mt{x(t) I}-\delta\mt{t(t)}\dot{\mt{X}}\mt{(t)}\right]\mt{dt}\ |\mt{ x't'}\supset
\]

La formule (15) qui traduit le \ul{Principe d'Action de Schwinger}
va constituer le point de départ de notre étude. Elle appelle quelques
remarques :

%50
$\alpha$) $\delta$x(t) ne dépendant pas de H, l'élément de matrice
$\sum_\mt{H}(\mt{exp}\frac{\mt{i}}{\hbar}$ S$_\mt{H})\delta$x(t)
se factorise et s'écrit $\delta\mt{x(t)}\subset \mt{x''t'' }|\mt{ I }|\mt{ x't'}\supset$, I étant l'opérateur identité,
$\delta$x(t) doit donc être considéré comme un nombre, ou ce qui revient au mêre,
comme un multiple de la matrice unité. C'est ce qui explique l'introduction
de la matrice unité I dans le deuxième terme de la relation (15).

$\beta$) La notion d'intégrale d'opérateur introduite dans le deuxième terme de
(15) peut se comprendre comme une intégrale au sens de Riemam, limite d'une
somme d'opérateurs pris à des instants infiniment voisins entre t' et t''.


\section{Equations de Lagrange en mécanique quantique}%D
Faisons tout d'abord l'hypothèse que la variation aux bornes est
nulle : $\delta$x' = $\delta$x'' = $\delta$t' = $\delta$t'' = 0. N' est confondu avec M', N'' avec M''.
H et \ul{H} représentent alors deux chemins infiniment voisins passant tous
deux par les points M' et M''. Pour passer de l'un à l'autre, on se donne
les deux fonctions infinitésimales $\delta$x(t) et $\delta$t(t). Mais, dans ce cas particulier, il est inutile d'introduire deux variations indépendantes et on peut
poser $\delta$t(t) $\equiv$ 0, La correspondance est donc assurée à l'aide de la fonction
infinitésimale $\delta$x(t), identique pour tous les chemins H, obéissant aux relations $\delta$x(t') = $\delta$x(t'') = 0 et à part cela arbitraire.

Comme à tout chemin H correspond un chemin \ul{H} et réciproquement
et que les points de départ et d'arrivée sont les mêmes, on a évidemment
\[
\tag{16}\sum_\mt{H}\mt{N exp}\frac{\mt{i}}{\hbar}\mt{S}_\mt{H}=
\sum_\mt{\ul{H}}\mt{N exp}\frac{\mt{i}}{\hbar}\mt{S}_\mt{\ul{H}}
\]
le passage de l'un à l'autre de ces expressions revenant à effectuer un
changement d'indice "muet" H.

On déduit de (16) que $\delta$ < x''t''|x't' > = 0, ce qui, compte tenu
de (15), conduit à
\[
\tag{17}\subset \mt{x''t'' }|
\int^\mt{t''}_\mt{t'}
\left[\frac{\partial\mc{L}}{\partial\mt{X}}
-\frac{\mt{d}}{\mt{dt}}\frac{\partial\mc{L}}{\partial\dot{\mt{X}}}\right]
\delta\mt{x(t) }\mt{dt}\ |\mt{ x't'}\supset=0
\]

%51
La relation (17) est vraie quelle que soit la fonction infinitésimale $\delta$x(t)
satisfaisant aux conditions $\delta$x(t') $= \delta$x(t'') $=$ 0,
Choisissons pour $\delta$x(t) une fonction nulle partout, sauf dans un intervalle
très petit autour de t. Pour que (17) soit satisfaite, il faut que l'on ait :
\[
\subset \mt{x''t'' }|
\left[\frac{\partial\mc{L}[\mt{X(t), }\dot{\mt{X}}\mt{(t), t}]}{\partial\mt{[X(t)}]}
-\frac{\mt{d}}{\mt{dt}}\frac{\partial\mc{L}[\mt{X(t), }\dot{\mt{X}}\mt{(t), t}]}{\partial\dot{\mt{X}}\mt{(t)}}\right]
|\mt{ x't'}\supset=0
\]
ce qui nous conduit à la relation entre opérateurs :
\[
\tag{18}\frac{\partial\mc{L}}{\partial\mt{X}}\mt{(t)}
-\frac{\mt{d}}{\mt{dt}}\frac{\partial\mc{L}}{\partial\dot{\mt{X}}}\mt{(t)}=0\ \ \ \ \ \ (\forall\mt{t})
\]

La relation (18) est la généralisation aux opérateurs quantiques
des équations de Lagrange classiques, Elle a été déduite directement des
postulats de Feynman, ou ce qui revient au même, du Principe d'Action de
Schwinger. Elle est valable quel que soit $\hbar$ et non pas seulement à la
limite classique.

Les opérateurs qui interviennent dans cette équation de Lagrange
étant des opérateurs à un seul instant t, nous savons (cf chapitre III)
qu'il y a équivalence entre les définitions de Feynmen et de Heisenberg
des opérateurs (à condition d'appliquer la règle de symétrisation).

On retrouve ainsi le fait que dans la représentation de Heisenberg,
les opérateurs satisfont aux équations de le mécanique classique. Prenons en
effet pour exemple le cas d'une particule dans un potentiel V(x). Le lagrangien quantique s'écrit  - V(X) et l'équation (18) conduit à la relation entre onérateurs, au sens de Feynman et de Heisenberg :
\[
\tag{19}\mt{m}\frac{\mt{d}\dot{\mt{X}}}{\mt{dt}}=
-\frac{\partial\mt{V}}{\partial\mt{X}}
\]
qui n'est autre que l'équation fondamentale de la dynamique newtonienne.
Dans le formalisme habituel de Heisenberg, (19) s'établit à partir de
l'équation d'évolution de l'opérateur P :

%52
\[
\tag{20}\mt{i}\hbar=[\mt{P, }\mc{H}]
\]
Or
\[
\tag{21}[\mt{P, }\mc{H}]=-\mt{i}\hbar\frac{\partial\mc{H}}{\partial\mt{X}}=
-\mt{i}\hbar\frac{\partial\mt{V}}{\partial\mt{X}}
\]
(20) et (21) redonnent (19).

Si on prend la moyenne des deux membres de (19) dans un état
| $\psi$ > quelconque, on retrouve le fait que les moyennes quantiques obéissent
aux lois d'évolution des grandeurs classiques correspondantes : c'est le
\ul{théorème d'Ehrenfest}.

Reprenons maintenant la relation fondamentale (15), en tenant
compte du fait que $\frac{\partial\mc{L}}{\partial\mt{X}}
-\frac{\mt{d}}{\mt{dt}}\frac{\partial\mc{L}}{\partial\dot{\mt{X}}}$ est identiquement nul. Il vient :
\[
\tag{22}\delta<\mt{ x''t'' }|\mt{ x't' }>\ =\frac{\mt{i}}{\hbar}\subset \mt{x''t'' }|\ \delta\mt{S}\ |\mt{ x't'}\supset
\]
\[
=\frac{\mt{i}}{\hbar}\subset \mt{x''t'' }|\ \mt{ P(t'')}\delta\mt{x''}-\mt{ P(t')}\delta\mt{x'}
-\mc{H}\mt{(t'')}\delta\mt{t''}+\mc{H}\mt{(t')}\delta\mt{t'}\ |\mt{ x't'}\supset
\]
Chacun des opérateurs de la relation (22) est maintenant un opérateur à
un seul temps et on peut écrire dans le point de vue de Heisenberg :
\[
\tag{23}<\mt{x''+}\delta \mt{x'', t''}+\delta \mt{t'' }|\mt{ x'}+\delta \mt{x', t'}+\delta \mt{t'}> - <\mt{ x''t'' }|\mt{ x't' }>
\]
\[
=\frac{\mt{i}}{\hbar}< \mt{x''t'' }|\ \mt{ P(t'')}\delta\mt{x''}-\mt{ P(t')}\delta\mt{x'}
-\mt{H(t'')}\delta\mt{t''}+\mt{H(t')}\delta\mt{t'}\ |\mt{ x't'}>
\]

\section{Opérateur impulsion : P(t)}%E
Faisons dans la relation (23) $\delta$x''$= \delta$t'$= \delta$t''$= 0;\ \delta$x' $\neq 0$.
On obtient la relation :
\[
\tag{24}<\mt{x'', t'' }|\mt{ x'}+\delta \mt{x', t'}> - <\mt{ x''t'' }|\mt{ x't' }>=-\frac{\mt{i}}{\hbar}\delta\mt{x'}< \mt{x''t'' }|\ \mt{ P(t')}\ |\mt{ x't'}>
\]
La relation (24),étant vraie quel que soit < x''t'' | , entraîne
\[
|\mt{ x'}+\delta \mt{x', t'}>\ =|\mt{ x't' }>-\frac{\mt{i}\delta\mt{x'}}{\hbar}\ \mt{ P(t')}\ |\mt{ x't'}>
\]
%53
Soit
\[
\tag{25}|\mt{ x'}+\delta \mt{x', t'}>\ =\left[1-\frac{\mt{i}\delta\mt{x'}}{\hbar}\ \mt{ P(t')}\right]\ |\mt{ x't'}>
\]
L'opérateur P (t') étant hermitique, 1 - $\frac{\mt{i}\delta\mt{x'}}{\hbar}$ P(t') est un
opérateur unitaire infinitésimal qui translate | x', t' > en | x' + x'', t' >.
P(t), opérateur impulsion, est donc le "générateur" du groupe des translations dans l'espace des états de la mécanique quantique. C'est la propriété
fondamentale de P. A partir de (25), nous pouvons retrouver les différentes
propriétés de l'opérateur P :


\subsubsection{Commutateur [x (t'), P (t')]}%a
D'après (25) :
\[
\mt{P(t')}\ |\mt{ x', t'}>\ =\frac{\mt{i}\hbar}{\delta\mt{x'}}\ 
[\ |\mt{ x'}+\delta \mt{x', t'}>-|\mt{x', t'}>]
\]
et
\[
\mt{X(t') P(t')}\ |\mt{ x', t'}>\ =
\frac{\mt{i}\hbar}{\delta\mt{x'}}\ 
[\ (\mt{ x'}+\delta \mt{x'})|\mt{ x'}+\delta \mt{x', t'}>-
\mt{ x'}|\mt{x', t'}>]
\]
D'autre part
\begin{center}
X(t') $|$ x't' $> =$ x' $|$ x't' $>$

et \ \ \ \ P(t') X(t') $|$ x't' $> = \frac{\mt{i}\hbar}{\delta\mt{x'}}$
$[$ x' $|$ x' $+ \delta$x', t' $> -$ x' $|$ x't'$>]$
\end{center}
Finalement
\begin{center}
$[$ X(t') P(t') - P(t') X(t') $]\ |$ x't' $> = [$ X(t'), P(t') $]\ |$ x't' $> = \mt{i}\hbar |$ x' + $\delta$x', t' >
\end{center}
Cette relation est vérifiée à le limite où $\delta$x' tend vers zéro.
On a donc :
\begin{center}
$[$ X(t'), P(t') $]\ |$ x't' $> = \mt{i}\hbar\ |$ x't' $>$
\end{center}
Les vecteurs | x't' > à t' fixé constituent un ensemble complet.
On en déduit donc :
\[
\tag{26}[\mt{ X(t), P(t) }]=\mt{i}\hbar\ \ \ \ \ \ (\forall\mt{t})
\]
et dans le formalisme de Schrôdinger :
\[
\tag{27}[\ \mt{\ul{X}},\ \mt{\ul{P}}\ ]=\mt{i}\hbar
\]
ce qui constitue la relation de commutation fondamentale des opérateurs
position et impulsion.
%54

\subsection{Opérateur P dans la représentation x}%b

La relation conjuguée de (25) s'écrit :
\[
\tag{28}<\mt{ x}+\delta \mt{x, t }|\ =<\mt{ x, t }|+\frac{\mt{i}\delta\mt{x}}{\hbar}<\mt{ x, t }|\ \mt{ P(t) }
\]
Nous cherchons à déterminer l'action de P(t) sur un état $|\psi>$ défini
par sa fonction d'onde $<$ x, t $|\psi> = \psi$ (x, t), c'est-à-dire qu'à partir
de $\psi$(x, t) nous cherchons la fonction d'onde $\phi$(x, t) $= <$ x, t $|$ P(t) $|\psi>$.
Multiplions (28) membre à membre par $|\psi>$.

Il vient  $\psi$(x + $\delta$x, t) $=\psi$(x, t) $+ \frac{\mt{i}\delta\mt{x}}{\hbar}\phi$(x, t).
Soit  $\phi$(x, t) $=\frac{\hbar}{\mt{i}}\frac{\psi\mt{(x+}\delta\mt{x, t)}-\psi\mt{(x, t)}}{\delta\mt{x}}$

Cette relation n'est vérifiée qu'à la limite où $\delta$x $\to$ 0. On a donc en fait :
\[
\tag{29}\phi\mt{(x,t)}=\frac{\hbar}{\mt{i}}\frac{\partial}{\partial\mt{x}}\psi\mt{(x,t)}
\]
et \ul{en représentation x, l'opérateur P est donc} $\frac{\hbar}{\mt{i}}\frac{\partial}{\partial\mt{x}}$.

 

Le formalisme de Feynman permet ainsi de retrouver de façon
simple les propriétés fondamentales de l'opérateur impulsion P en le rattachant au groupe des opérateurs de translation dans l'espace des états.
\section{Opérateur hamiltonien H (t)}%F
Reprenons maintenant la relation (23) en y faisant
$\delta$x'' $=\delta$x' $=\delta$t'' $= 0$; $\delta$t' $\neq 0$.

On obtient la relation
\[
\tag{30}<\mt{x'' t'' }| \mt{x', t'}+\delta\mt{t'}> - <\mt{ x''t'' }|\mt{ x't' }>=
\frac{\mt{i}\delta\mt{t'}}{\hbar}<\mt{ x''t'' }|\ \mc{H}\mt{(t') }|\mt{ x't' }>
\]
La relation (30) étant vraie, quel que soit < x''t'' | , entraîne
\begin{center}
$|$ x', t' $+\delta$t'$>=|$ x't' $>+
\frac{\mt{i}\delta\mt{t'}}{\hbar}\ \mc{H}$(t'$|$ x't' $>$
\end{center}
%55
Soit
\[
\tag{31}|\mt{ x', t'}+\delta\mt{t'}>=\left[1+
\frac{\mt{i}\delta\mt{t'}}{\hbar}\ \mc{H}\mt{(t') }\right]|\mt{ x't' }>
\]
L'opérateur  étant hermitique, 1+
$\frac{\mt{i}\delta\mt{t'}}{\hbar}$ $\mc{H}$(t') est l'opérateur
infinitésimal unitaire qui translate | x' t' > dans le temps en
| x, t' + $\delta$t' >.

Le hamiltonien $\mc{H}$(t) est le générateur du groupe des translations dans le
temps.

A partir de l'équation (31), nous pouvons retrouver l'équation
de Schrödinger.

La relation conjuguée de (31) s'écrit :
\begin{center}
$<$ x, t' $+\delta$t $|=<$ x, t $|-
\frac{\mt{i}\delta\mt{t}}{\hbar}\ <$ x, t $|\mc{H}$(t)
\end{center}

En multipliant membre à membre par un état $| \psi >$ , on obtient
\begin{center}
$\psi$ (x, t$+\delta$t) $-\ \psi$ (x, t) $=
-\frac{\mt{i}\delta\mt{t}}{\hbar}\ <$ x, t $|\mc{H}$(t)$| \psi >$
\end{center}
où, à la limite où $\delta$t $\to$ 0 :
\[
\tag{32}\mt{i}\hbar\frac{\partial}{\partial\mt{t}}\psi( \mt{x, t} )=
\ <\mt{ x, t }|\ \mc{H}\mt{(t) }|\psi>
\]
(32) n'est rien d'autre que l'équation de Schrödinger en représentation x.
En effet
\[
\mc{H}=\frac{\mt{P}^2}{2\mt{m}}+\mt{V(x)}=
-\frac{\hbar^2}{2\mt{m}}\frac{\partial^2}{\partial\mt{x}^2}+\mt{V(x)}
\]


et (32) s'écrit :
\[
\tag{33}\mt{i}\hbar\frac{\partial}{\partial\mt{t}}\psi\mt{(x,t)}=
-\frac{\hbar^2}{2\mt{m}}\frac{\partial^2\psi\mt{(x)}}{\partial\mt{x}^2}+\mt{V(x)}\psi\mt{(x)}
\]
Nous retrouvons ainsi, de façon beaucoup plus élégante, l'équation de
Schrôdinger que nous avons déjà déduite directement des postulats de
Feynman au chapitre II.


\chapter{Fonction de Green. Propagateurs}
%V-FONCTION DE GREEN. PROPAGATEURS
% 56
\section{Introduction}% A
— Pour décrire l'évolution d'un système quantique au cours
du temps, on peut adopter deux points de vue : celui de l'équation de
Schrödinger :
\[
\tag{1}\mt{i}\hbar\frac{\partial}{\partial\mt{t}}|\psi>=\mc{H}|\psi>
\]
ou celui de l'équation intégrale
\[
\tag{2}\psi(\mt{x}_2,\mt{t}_2)=\int\mt{dx}_1<\mt{x}_2\mt{t}_2|\mt{x}_1\mt{t}_1>\psi(\mt{x}_1,\mt{t}_1)
\ \ \ \ \ \ \ \ \mt{t}_2\geqslant\mt{t}_1
\]

Toute la première partie de cette étude nous a montré l'équivalence entre ces deux points de vue.

— Rappelons que le second point de vue est beaucoup plus physique
et qu'il traduit en quelque sorte un principe d'Huyghens dans l'espace-temps.
Il montre bien l'analogie existant entre les mécaniques classique et quantique . Enfin il est particulièrement bien adapté au cas des champs relativistes et au problème des perturbations.

— Mathématiquement, cependant, le calcul direct de $<$ x$_2$t$_2\ |$ x$_1$t$_1>$
à partir du postulat de Feynman est assez compliqué. Sur l'exemple de la
particule libre, nous avons constaté que le calcul de cette amplitude de
probabilité est plus simple à partir de l'équation de Schrödinger (1).

Nous sommes donc amenés à adopter le compromis suivant :

a) Nous décrirons l'évolution du système dans le temps par l'équation intégrale (2).

b) Nous calculerons $<$ x$_2$t$_2\ |$ x$_1$t$_1>$ à partir de l'équation aux dérivées partielles (1).

% 57
 
— Le problème b) n'est autre que celui du calcul des fonctions
de Green de l'équation de Schrödinger. Ce problème est extrêmement général
en physique et on le retrouve dès qu'on traite une équation aux dérivées
partielles avec des conditions aux limites : c'est le cas de l'équation
de Poisson, des équations de Maxwell, de l'équation de la diffusion, des
équations de Schrödinger, Klein-Gordon et Dirac, etc.

Etant donnée son importance, il est intéressant d'étudier ce
problème de façon systématique et de dégager ainsi les diverses méthodes
possibles de calcul de l'amplitude de probabilité $<$ x$_2$t$_2\ |$ x$_1$t$_1>$.
\section{Définition des fonctions de Green}% B

Nous allons nous placer dans un espace-temps à quatre dimensions
et adopter la notation $<\vec{\mt{r}}_2$t$_2\ |\ \vec{\mt{r}}_1$t$_1>$ au lieu de $<$ x$_2$t$_2\ |$ x$_1$t$_1>$.
D'autre part, on posera souvent
\begin{center}
$<\vec{\mt{r}}_2$t$_2\ |\ \vec{\mt{r}}_1$t$_1>$ = K($\vec{\mt{r}}_2$t$_2,\ \vec{\mt{r}}_1$t$_1$) = K(2,1)
\end{center}

\subsection{Calcul de K(2,1) à partir des états propres de $\mc{H}$ (supposé
indépendant du temps)}% 1°)

Soit $\mc{H}$ le hamiltonien indépendant du temps, de valeurs propres
E$_n$ et de vecteurs propres $|$ u$_n>$ : on a les relations
\begin{center}
$\mc{H}|$ u$_n>\ =\ $E$_n\ |$ u$_n>$

$<$ u$_n\ |$ u$_{n'}>\ =\ \delta_{nn'}$ 

$\underset{n}{\sum}\ |$ u$_n><$ u$_n|=$ 1
\end{center}
Plaçons-nous dans la représentation $\vec{\mt{r}}$ et posons  $<\vec{\mt{r}}\ |$ u$_n>\ =\ $u$_n(\vec{\mt{r}})$

La relation de fermeture conduit à :
\begin{center}
$\underset{n}{\sum}\ <\vec{\mt{r}}\ |$ u$_n><$ u$_n\ |\ \vec{\mt{r'}}>\ =
\underset{n}{\sum}\ $ u$_n(\vec{\mt{r}})$ u$^*_n(\vec{\mt{r'}})=
\delta(\vec{\mt{r}}-\vec{\mt{r'}})$ 
\end{center}

% 58
Décomposons la fonction d'onde $|\ \psi$(t) $>$ sur les états  $|$ u$_n>$ :
\[
\tag{3}|\ \psi(\mt{t})>\ = \underset{n}{\sum}\ | \mt{u}_n>< \mt{u}_n\ |\ \psi\mt{(t)}>
\]
\[
=\underset{n}{\sum}\ | \mt{u}_n>\mt{c}_n\mt{(t)}
\]
avec
\[
\tag{4}\mt{c}_n\mt{(t)}=< \mt{u}_n\ |\ \psi\mt{(t)}>
\]
c$_n$(t) vérifie l'équation
\begin{center}
i$\hbar\dot{\mt{c}}_n = $E$_n$c$_n$
\end{center}
soit
\[
\tag{5}\mt{c}_n(\mt{t})=\mt{c}_n\mt{ e}^{-\mt{i}\frac{\mt{E}_n\mt{t}}{\hbar}}
\]
On déduit de (3) et de (5) :
\begin{center}
$\psi(\vec{\mt{r}}_1,\mt{t}_1)=<\vec{\mt{r}}_1|\psi(\mt{t}_1)>=
\underset{n}{\sum}\ \mt{u}_n(\vec{\mt{r}}_1)\ \mt{c}_n\mt{ e}^{-\mt{i}\frac{\mt{E}_n\mt{t}_1}{\hbar}}$
\end{center}
et de (4) et (5) :
\[
\tag{6}\mt{c}_n=\int\mt{u}_n^*(\vec{\mt{r}}_1)\ \mt{ e}^{\mt{i}\frac{\mt{E}_n\mt{t}_1}{\hbar}}
\psi(\vec{\mt{r}}_1,\mt{t}_1)\mt{d}^3\vec{\mt{r}}_1
\]
D'autre part
\begin{center}
$\psi(\vec{\mt{r}}_2,\mt{t}_2)=
\underset{n}{\sum}\ \mt{c}_n\ \mt{u}_n(\vec{\mt{r}}_2)\ \mt{ e}^{-\mt{i}\frac{\mt{E}_n\mt{t}_2}{\hbar}}$
\end{center}
et compte tenu de (6) : 
\[
\psi(\vec{\mt{r}}_2,\mt{t}_2)=\int
\underset{n}{\sum}\ \mt{u}_n(\vec{\mt{r}}_2)\ \mt{u}_n^*(\vec{\mt{r}}_1)\ \mt{ e}^{-\mt{i}
\frac{\mt{E}_n}{\hbar}(\mt{t}_2-\mt{t}_1)}\ \psi(\vec{\mt{r}}_1,\mt{t}_1)\ \mt{d}^3\vec{\mt{r}}_1
\]
On a donc
\[
\tag{7}\mt{K(2,1)}=\underset{n}{\sum}\ \mt{u}_n(\vec{\mt{r}}_2)\ \mt{u}_n^*(\vec{\mt{r}}_1)\ \mt{ e}^{-\mt{i}
\frac{\mt{E}_n}{\hbar}(\mt{t}_2-\mt{t}_1)}
\]
Nous avons ainsi défini K(2,1) en fonction des vecteurs propres et
des valeurs propres du hamiltonien $\mc{H}$. K(2,1) est donc une quantité
extrêmement riche en information, puisque sa connaissance exige la diagonalisation complète du hamiltonien $\mc{H}$.
% 59

Si on fait dans (7) t$_2=$ t$_1=$ t, on trouve
\begin{center}
K($\vec{\mt{r}}_2$, t; $\vec{\mt{r}}_1$, t)=$\underset{n}{\sum}$ u$_n(\vec{\mt{r}}_2)$ u$_n^*(\vec{\mt{r}}_1)=
\delta(\vec{\mt{r}}_2-\vec{\mt{r}}_1)$
\end{center}
Nous avons déjà établi ce résultat au chapitre II sur la quantité
$<$ x$_2$ t$_2\ |$ x$_1$ t$_1$ $>$, On voit ainsi que K(2,1) est une \ul{distribution} et non
une fonction.

Notons que l'expression (7) de K(2,1) ne fait pas d'hypothèse
sur l'ordre des temps t$_1$ et t$_2$. Cependant, comme l'état du système à l'instant t$_2$ ne peut dépendre que de son état à des instants antérieurs, on complètera la définition de K(2,1) par la condition K(2,1) = 0 si t$_2<$t$_1$.
Finalement
\[
\tag{8}\mt{K}(\vec{\mt{r}}_2\mt{ t}_2, \vec{\mt{r}}_1\mt{ t}_1)=\underset{n}{\sum} \mt{u}_n(\vec{\mt{r}}_2)\mt{u}_n^*(\vec{\mt{r}}_1)\mt{ e}^{-\mt{iE}_n\frac{(\mt{t}_2-\mt{t}_1)}{\hbar}}\theta(\mt{t}_2-\mt{t}_1)
\]
$\theta$(t$_2-$t$_1$)étant la fonction échelon unité définie par
\[\left \{
 \begin{array}{r c l}
  \theta(\mt{t}_2-\mt{t}_1) & = & 1 \mt{ si t}_2 \geqslant\mt{ t}_1\\
  \theta(\mt{t}_2-\mt{t}_1) & = & 0 \mt{ si t}_1 <\mt{ t}2
 \end{array}
\right . \]
\subsection{Equation satisfaite par K(2,1)}% 

Par définition, nous avons
\begin{center}
$[$ i$\hbar\frac{\partial}{\partial\mt{t}_2}-$
H$(\vec{\mt{r}}_2)\ ]$ u$_n(\vec{\mt{r}}_2)$
e$^{-\mt{i}\frac{\mt{E}_n\mt{t}_2}{\hbar}}=0$
\end{center}
D'autre part
\begin{center}
$\frac{\mt{d}}{\mt{dt}_2}\ \theta$(t$_2-$t$_1)=\delta($t$_2-$t$_1)$
\end{center}
% 60

Il en résulte
\begin{center}
$[$ i$\hbar\frac{\partial}{\partial\mt{t}_2}-$ H$(\vec{\mt{r}}_2)\ ]$K$(\vec{\mt{r}}_2\mt{t}_2,\vec{\mt{r}}_1\mt{t}_1)=$
i$\hbar\underset{n}{\sum}$u$_n(\vec{\mt{r}}_2)$u$^*_n(\vec{\mt{r}}_1)$e$^{-i\mt{E}_n(\frac{\mt{t}_2-\mt{t}_1}{\hbar})}
\delta(\mt{t}_2-\mt{t}_1)$

$=$ i$\hbar\underset{n}{\sum}$u$_n(\vec{\mt{r}}_2)$u$^*_n(\vec{\mt{r}}_1)\delta(\mt{t}_2-\mt{t}_1)$

$=$ i$\hbar\delta(\vec{\mt{r}}_2-\vec{\mt{r}}_1)\delta(\mt{t}_2-\mt{t}_1)=$ i$\hbar\delta(2,1)$

\end{center}

En conclusion, K(2,1) est définie par les relations :
\[
\tag{9-a}[\ i\hbar\frac{\partial}{\partial\mt{t}_2}-\ \mt{H}(\vec{\mt{r}}_2)\ ]
\ \mt{K}(2,1)=\ \mt{i}\hbar\ \delta(2,1)
\]
\[
\tag{9-b}\mt{K}=0\ \ \ \ si\ \ \ \mt{t}_2<\mt{t}_1
\]

Les équations (9-a) et (9-b) définissent la \ul{fonction de Green retardée
K(2,1)} de l'équation de Schrödinger (avec un hamiltonien indépendant
du temps).

\ul{Remarques} :
— L'équation (9-b) est indispensable à la définition complète de K. En
effet (9-a) définit K à l'addition près de n'importe quelle solution de
l'équation homogène
\begin{center}
$[$ i$\hbar\frac{\partial}{\partial\mt{t}_2}-$H($\vec{\mt{r}}_2)\ ]$ K(2,1) $=$ 0
\end{center}

(9-b) constitue une \ul{condition aux limites} qui achève de définir K.

— Il résulte clairement des équations (9) que K(2,1) est \ul{une distribution}.

\subsubsection{Cas général (où $\mc{H}$ dépend du temps)}%3°) 
\ul{Définition} : On appelle fonction de Green retardée de l'équation de
Schrödinger
\begin{center}
$[$ i$\hbar\frac{\partial}{\partial\mt{t}_2}-$H($\vec{\mt{r}}_2)\ ]\ \psi\ =$ 0,
\end{center}
la distribution K(2,1) définie par les
conditions :
\[
\tag{10-a}[\mt{ i}\hbar\frac{\partial}{\partial\mt{t}_2}-H(\vec{\mt{r}}_2,\mt{ t}_2)\ ]\mt{ K(2,1) }=
\mt{i}\hbar\delta(2,1)
\]
\[
\tag{10-b}\mt{K}=0\ \ \ \ \ \mt{si t}_2<\mt{t}_1
\]

% 61
Les équations (10-a) et (10-b) sont une simple généralisation
au cas d'un hamiltonien dérendant du temps des équations (9-a) et (9-b).

Montrons que la solution K(2,1) de ce système satisfait à la
propriété fondamentale (2) :
\[
\psi(\vec{\mt{r}}_2,\mt{t}_2)=\int\mt{d}^3\vec{\mt{r}}_1\mt{ K(2,1) }\psi(\vec{\mt{r}}_1,\mt{t}_1)
\ \ \ \ \ \ \ \ \mt{t}_2\geqslant\mt{t}_1
\]

Multiplions pour cela membre à membre (10-a) par $\psi(\vec{\mt{r}}_1,\mt{t}_1)$ et intégrons
sur $\vec{\mt{r}}_1$.

Il vient :
\[
\tag{11}\mt{i}\hbar\frac{\partial}{\partial\mt{t}_2}\int\mt{K(2,1)}\psi(\vec{\mt{r}}_1,\mt{t}_1)\mt{d}^3\vec{\mt{r}}_1-
\mt{H}(\vec{\mt{r}}_2,\mt{t}_2)\int\mt{K(2,1)}\psi(\vec{\mt{r}}_1,\mt{t}_1)\mt{d}^3\vec{\mt{r}}_1
\]
\[
=\mt{i}\hbar\ \delta(\mt{t}_2-\mt{t}_1)\ \psi(\vec{\mt{r}}_2,\mt{t}_1)
\]
Si t$_2>$ t$_1$, i$\hbar\ \delta(\mt{t}_2-\mt{t}_1)\ \psi(\vec{\mt{r}}_2,\mt{t}_1)=0$ et $\int\mt{K(2,1)}\psi(\vec{\mt{r}}_1,\mt{t}_1)\mt{d}^3\vec{\mt{r}}_1$ satisfait bien à l'équation de Schrödinger.
Intégrons maintenant (11) sur la variable t$_2$ entre t$_1-\epsilon$ et t$_1+\epsilon$ :
On a
\[
\tag{12}\mt{i}\hbar\int_{\mt{t}_1-\epsilon}^{\mt{t}_1+\epsilon}\mt{dt}_2
\frac{\partial}{\partial\mt{t}_2}\int\mt{K(2,1)}\psi(\vec{\mt{r}}_1,\mt{t}_1)\mt{d}^3\vec{\mt{r}}_1-
\int_{\mt{t}_1-\epsilon}^{\mt{t}_1+\epsilon}\mt{dt}_2\mc{H}(\vec{\mt{r}}_2,\mt{t}_2)\int\mt{K(2,1)}\psi(\vec{\mt{r}}_1,\mt{t}_1)\mt{d}^3\vec{\mt{r}}_1
\]
\[
=\mt{i}\hbar\ \int_{\mt{t}_1-\epsilon}^{\mt{t}_1+\epsilon}\mt{dt}_2\ \delta(\mt{t}_2-\mt{t}_1)\ \psi(\vec{\mt{r}}_2,\mt{t}_1)
\]
Lorsque $\epsilon \to 0$, la deuxième intégrale du premier membre, qui est l'intégrale
d'une quantité bornée, tend vers zéro.

La première intégrale est égale à
\[
\mt{i}\hbar\left[\int\mt{K}(\vec{\mt{r}}_2\mt{t}_2,\vec{\mt{r}}_1\mt{t}_1)
\ \psi(\vec{\mt{r}}_1\mt{t}_1)\ \mt{d}^3\vec{\mt{r}}_1\right]_{\mt{t}_1-\epsilon}^{\mt{t}_1+\epsilon}
\]

D'après (10-b) K est nul pour t$_1-\epsilon$.

% 62
Finalement la première intégrale se réduit à
\[
\mt{i}\hbar\lim_{\epsilon\to\,0}\int\mt{K}(\vec{\mt{r}}_2\ \mt{t}_1+\epsilon,\vec{\mt{r}}_1\ \mt{t}_1)
\ \psi(\vec{\mt{r}}_1\mt{t}_1)\ \mt{d}^3\vec{\mt{r}}_1
\]
Quant au second membre, il est égal à $\mt{i}\hbar\ \psi(\vec{\mt{r}}_2,$ t$_1)$.
Lorsque t$_2\to$t$_1$, on a donc
\[
\int\mt{K}(\vec{\mt{r}}_2\ \mt{t}_2,\vec{\mt{r}}_1\ \mt{t}_1)
\ \psi(\vec{\mt{r}}_1\ \mt{t}_1)\ \mt{d}^3\vec{\mt{r}}_1\ \to\ \psi(\vec{\mt{r}}_2,\ \mt{t}_1)
\]

En conclusion : La fonction de Green retardée K(2,1) est telle que
 représente la solution de l'équation de
Schrödinger égale à  à l'instant t.

Nous avons ainsi rattaché de façon rigoureuse le problème de
l'évolution du systère aux instants t$_2\geqslant$ t$_1$, à la recherche de la fonction
de Green retardée de l'équation de Schrödinger.

Nous allons voir que la fonction de Green est également très
utile pour la résolution d'un prand nombre de problèmes en physique.

\section{Utilité des Fonctions de Green}% C. .

Les fonctions de Green interviennent essentiellement dans la
recherche des solutions d'une équation aux dérivées partielles satisfaisant
à certaines conditions aux limites. Nous pouvons envisager plusieurs types
de problèmes :

\subsection{Soit D (x$_1$, x$_2$, ... x$_n$) un opérateur différentiel linéaire}% 1°)
\begin{flushright}
(que nous noterons symboliquement D$_{\vec{\mt{x}}}$).
\end{flushright}

Soit à chercher les solutions  de $\phi(\vec{\mt{x}})=\phi(\mt{x}_1,\mt{ x}_2,\ ...\mt{ x}_n)$
l'équation D$_{\vec{\mt{x}}}\ \phi(\vec{\mt{x}})=\rho(\mt{x})$, satisfaisant à des conditions aux limites
linéaires (ex. : annulation de $\phi$ sur certaines surfaces).
$\rho(\vec{\mt{x}})$ est une fonction donnée, appelée fonction \ul{source}.

% 63
De telles équations sont extrêmement fréquentes :

\ul{Exemples} : Équation de Poisson \hspace{2cm} $\Delta$V $=-\frac{\rho}{\epsilon_0}$

Propagation du potentiel vecteur \hspace{2cm}
$(\Delta-\frac{1}{\mt{c}^2}\frac{\partial^2}{\partial\mt{t}^2})\vec{\mt{A}}=-\mu_0\vec{\mt{j}}$

On appelle alors fonction de Green du problème la distribution
K(x$_1$, y$_1$, x$_2$, y$_2$, ... x$_n$, y$_n$) qui vérifie

a) l'équation \hspace{2cm} D$_{\vec{\mt{x}}}$ K$(\vec{\mt{x}},\vec{\mt{y}})=\delta(\vec{\mt{x}}-\vec{\mt{y}})$.

b) les conditions aux limites imposées.

Ecrivons alors $\rho(\vec{\mt{x}})=\int\rho(\vec{\mt{y}})\ \delta(\vec{\mt{x}}-\vec{\mt{y}})$ d$\vec{\mt{y}}$.
D'après la linéarité de l'équation et des conditions aux
limites, la solution $\phi(\vec{\mt{x}})$ cherchée est alors
\begin{center}
$\phi(\vec{\mt{x}})=\int\rho(\vec{\mt{y}})$ K$(\vec{\mt{x}}-\vec{\mt{y}})$ d$\vec{\mt{y}}$
\end{center}

\ul{Exemple} : Equation de Poisson $\Delta$V $=-\frac{1}{\epsilon_0}\rho$(x, y, z)
avec la condition V $=$ 0 à l'infini

On sait que la solution de $\Delta$K $=\delta$ (r - r') est K $=-\frac{1}{4\pi|\vec{\mt{r}}-\vec{\mt{r}}\,'|}$.
On a donc le résultat classique :
\[
\mt{V}(\vec{\mt{r}})=\frac{1}{4\pi\epsilon_0}
\int\frac{\rho(\vec{\mt{r}}\,')}{|\vec{\mt{r}}-\vec{\mt{r}}\,'|}\mt{d}^3\vec{\mt{r}}\,'
\]

\subsection{Autres types de problèmes}% 2°)  :

— Problème des fonctions propres et valeurs propres d'une
équation aux dérivées partielles satisfaisant à des conditions aux limites
données :
% 64

\ul{Exemple} : — Problème de la diffusion : recherche des états stationnaires de
collision de l'équation de Schrödinger ayant un certain comportement
asymptotique. La méthode des fonctions de Green conduit alors à l'équation
intégrale de la diffusion, ou équation de Lippmann-Schwinger.

— Problème de deux équations aux dérivées partielles voisines :
On montre alors que les fonctions de Green des deux équations sont reliées
par une équation intégrale. La résolution de cette équation par itération
et la représentation diagrammatique de la solution sont très commodes
(problèmes de perturbation, de collision, etc...).

Nous étudierons ces différents problèmes par la suite.

\section{Calcul pratique des fonctions de Green}% D. .

Lorsque l'équation aux dérivées partielles est une équation à
coefficients constants, la méthode de choix consiste à passer par 1a
transformée de Fourier.

Nous allons l'illustrer sur deux exemples :

\subsection{Equation de Schrödinger d'une particule libre}% 1°)  :

Nous avons déjà obtenu la quantité $<$ x$_2$t$_2|$x$_1$t$_1> =$ K(2,1)
dans le cas d'une particule libre de deux façons différentes : directement
à partir du postulat de Feynman d'une part, à partir de l'opérateur d'évolution
U(t, t$_0$) de l'équation de Schrödinger d'autre part (cf chapitre II,
\S 2). Nous allons déterminer ici K(2,1) en tant que fonction
de Green de la particule libre, ce qui nous donnera un troisième procédé
de calcul.

K(2,1) satisfait aux équations (10-a) et (10-b) qui, transcrites au hamiltonien de la particule libre, s'écrivent :
\[
\tag{13-a}\mt{i}\hbar\frac{\partial}{\partial\mt{t}_2}\mt{K}(\vec{\mt{r}}_2\mt{t}_2,\vec{\mt{r}}_1\mt{t}_1)
+\frac{\hbar^2}{2\mt{m}}\Delta_2\mt{K}(\vec{\mt{r}}_2\mt{t}_2,\vec{\mt{r}}_1\mt{t}_1)
=\mt{i}\hbar\ \delta(\vec{\mt{r}}_2-\vec{\mt{r}}_1)\ \delta(\mt{t}_2-\mt{t}_1))
\]
\[
\tag{13-b}\mt{K}(\vec{\mt{r}}_2\mt{t}_2,\vec{\mt{r}}_1\mt{t}_1)=0\ \ \ \ \mt{si }\mt{t}_2<\mt{t}_1
\]

% 65
Il est évident, en raison de l'invarience du problème par
translation, que K n'est une fonction que de $\vec{\mt{r}}_2-\vec{\mt{r}}_1$ et t$_2-$t$_1$ que
nous notons K($\vec{\mt{r}}_2-\vec{\mt{r}}_1$, t$_2-$t$_1$). Faisons l'hypothèse que K($\vec{\mt{r}}_2-\vec{\mt{r}}_1$, t$_2-$t$_1$)
a une transformée de Fourier G$(\vec{\mt{k}},\omega)$ définie par :
\[
\tag{14}\mt{K}(\vec{\mt{r}}_2-\vec{\mt{r}}_1,\mt{t}_2-\mt{t}_1)=(\frac{1}{2\pi})^4
\int..\int\mt{G}(\vec{\mt{k}},\omega)\mt{e}^{\mt{i}[\vec{\mt{k}}(\vec{\mt{r}}_2-\vec{\mt{r}}_1)-\omega(\mt{t}_2-\mt{t}_1)]}\mt{d}^3\mt{k d}\omega
\]
La transformée de Fourier de (13-a) nous conduit à :
\[
\tag{15}\left[\hbar\omega-\frac{\hbar^2\mt{k}^2}{2\mt{m}}\right]\mt{G}(\vec{\mt{k}},\omega)=\mt{i}\hbar
\]

On peut affirmer que si K(2,1) a une transformée de Fourier,
cette transformée de Fourier G$(\vec{\mt{k}},\omega)$ satisfait à l'équation (15).

On a ainsi remplacé l'équation aux dérivées partielles (13-a)
par une équation algébrique pour laquelle l'inconnue est la \ul{distribution}
G$(\vec{\mt{k}},\omega)$.

Une solution particulière de l'équation (15) est
\[
\tag{16}\mt{G}(\vec{\mt{k}},\omega)=\mt{i}\hbar\ \mc{P}\left[\frac{1}{\hbar\omega-\frac{\hbar^2\mt{k}^2}{2\mt{m}}}\right]
\]
Rappelons que $\mc{P}(\frac{1}{\mt{x}})$ à est la distribution définie par la relation :
\[
<\mc{P}(\frac{1}{\mt{x}}),\phi(\mt{x})>=
\lim_{\epsilon\to\,0}\left[\int^{-\epsilon}_{-\infty}\frac{\phi(\mt{x})}{\mt{x}}\mt{dx}+
\int^{+\infty}_{+\epsilon}\frac{\phi(\mt{x})}{\mt{x}}\mt{dx}\right]
\]
$\mc{P}$ s'appelle aussi "partie principale au sens de Cauchy".
Il est alors très facile de vérifier que (16) est bien une solution de
l'équation (15). A cette solution particulière, il faut ajouter la solution
générale de l'équation homogène $[\hbar\omega-\frac{\hbar^2\mt{k}^2}{2\mt{m}}]\mt{G}(\vec{\mt{k}},\omega)=0$
qui est $\mt{i}\hbar\mt{C}\ \delta(\hbar\omega-\frac{\hbar^2\mt{k}^2}{2\mt{m}})$ où C est une constante arbitraire.

% 66
Finalement la solution générale de (15) est :
\[
\tag{17}\mt{G}(\vec{\mt{k}},\omega)=\mt{i}\hbar\ \left[\mc{P}\frac{1}{\hbar\omega-\frac{\hbar^2\mt{k}^2}{2\mt{m}}}+\mt{C}\ \delta(\hbar\omega-\frac{\hbar^2\mt{k}^2}{2\mt{m}})\right]
\]

Il est facile de vérifier que cette distribution G$(\vec{\mt{k}},\omega)$ admet une
transformée de Fourier (c'est en effet une distribution "tempérée").
Cette transformée de Fourier, K$(\vec{\mt{r}}_2-\vec{\mt{r}}_1,\mt{t}_2-\mt{t}_1)$ vérifie alors l'équation
(13-a) et admet elle-même une transformée de Fourier (qui est évidemment
G$(\vec{\mt{k}},\omega)$ . Nous justifions donc ainsi a posteriori l'hypothèse de départ
qui était l'existence d'une transformée de Fourier pour une solution de
l'équation (13-a).

Mais n'oublions pas que cette solution ne sera la fonction de
Green cherchée que si elle vérifie également la condition aux limites supplémentaire (13-b) que nous n'avons pas encore satisfaite.

Montrons que cette condition aux limites sur K se traduit par
un choix particulier de la constante C qui intervient dans G$(\vec{\mt{k}},\omega)$
(formule 17).

Envisageons a priori les deux valeurs de C, C$=\pm$i$\pi$.
Nous définissons alors, à partir de (17), deux distributions que nous notons
\[
\tag{18}\mt{G}_\pm(\vec{\mt{k}},\omega)=\mt{i}\hbar\ \left[\mc{P}\frac{1}{\hbar\omega-\frac{\hbar^2\mt{k}^2}{2\mt{m}}}
\mp\mt{i}\pi\ \delta(\hbar\omega-\frac{\hbar^2\mt{k}^2}{2\mt{m}})\right]
\]
ce qui nous conduit à envisager les deux distributions de x
\[
\mc{P}(\frac{1}{\mt{x}})\mp\mt{i}\pi\ \delta(\mt{x})
\]
Or nous avons l'égalité
\[
\tag{19}\lim_{\epsilon\to\,0_+}\frac{1}{\mt{x}\pm\mt{i}\epsilon}
=\lim_{\epsilon\to\,0_+}\left[\frac{\mt{x}\mp\mt{i}\epsilon}{\mt{x}^2+\epsilon^2}\right]
=\lim_{\epsilon\to\,0_+}\left[\frac{\mt{x}}{\mt{x}^2+\epsilon^2}\mp\mt{i}\pi\frac{1}{\pi}\frac{\epsilon}{\mt{x}^2+\epsilon^2}\right]
\]
\[
=\mc{P}(\frac{1}{\mt{x}})\mp\mt{i}\pi\ \delta(\mt{x})
\]
(l'égalité (19) se démontre aisément en faisant agir les distributions des
deux membres sur une même fonction $\phi$(x) ).

% 67
(19) nous permet donc d'écrire (18) sous la forme
\[
\tag{20}\mt{G}_\pm(\vec{\mt{k}},\omega)=\mt{i}\hbar\lim_{\epsilon\to\,0_+}
\frac{1}{\hbar\omega-\frac{\hbar^2\mt{k}^2}{2\mt{m}}\pm\mt{i}\epsilon}
\]
Les deux distributions G$_\pm$ sont deux solutions particulières de l'équation
(17). Nous allons voir que leurs transformées de Fourier K$_{(\pm)}(\vec{\mt{r}}_2-\vec{\mt{r}}_1,\mt{t}_2-\mt{t}_1)$,
qui vérifient (13-a) satisfont respectivement aux conditions
\[
\tag{21-a}\mt{K}_{(+)}=0\ \ \ \ \ \mt{si t}_2<\mt{t}_1\ \ \ \ \ \mt{et}
\]
\[
\tag{21-b}\mt{K}_{(-)}=0\ \ \ \ \ \mt{si t}_2>\mt{t}_1\ \ \ \ \ \ \ \,
\]

K$_{(+)}$ constituera ainsi la \ul{fonction de Green retardée} de la
particule libre, ce qui résout le problème que nous nous étions posé.
Quant à K$_{(-)}$ nous l'appellerons la \ul{fonction de Green avancée} de la
particule libre.

Pour démontrer les relations (21-a) et (21-b), il suffit
d'exprimer K$_{(\pm)}$ à partir de (14)
\[
\tag{22}\mt{K}_{(\pm)}(\vec{\mt{r}}_2-\vec{\mt{r}}_1,\mt{t}_2-\mt{t}_1)=(\frac{1}{2\pi})^4
\lim_{\epsilon\to\,0_+}\iiint\overrightarrow{\mt{d}^3\mt{k}}\mt{ e}^{\mt{i}\vec{\mt{k}}(\vec{\mt{r}}_2-\vec{\mt{r}}_1)}
\int\mt{d}\omega\frac{\mt{i}\hbar\mt{ e}^{-\mt{i}\omega(\mt{t}_2-\mt{t}_1)}}{\hbar\omega-\frac{\hbar^2\mt{k}^2}{2\mt{m}}\pm\mt{i}\epsilon}
\]
L'introduction de $\pm\mt{i}\epsilon$ dans le dénominateur de l'expression (20) de G$_\pm$
revient à déplacer les pôles de $\frac{1}{\hbar\omega-\frac{\hbar^2\mt{k}^2}{2\mt{m}}}$ dans le plan complexe hors
de l'axe réel de façon à donner un sens à l'intégration sur  dans le
deuxième membre de (22).

\ul{Par exemple pour} G$_+$, le pôle est repoussé dans le demi-plan
inférieur (fig. a) et l'intégration sur l'axe réel devient possible.

\vspace{0.3cm}
\begin{minipage}[c]{.45\linewidth}
\begin{center} \begin{tikzpicture}
\draw [->] (-2.25,0) -- (2.25,0) node [below]{$\omega$};
\draw (0,0.07) -- (0,-0.07) node [above right]{$\hbar\frac{\mt{k}^2}{2\mt{m}}$};
\draw [dotted] (0,0) --(0,-0.7);
\draw (0,-0.6) node [above right]{$\epsilon$};
%\draw (2.95,-0.7) ;
\draw (-0.1,-0.8) -- (0.1,-0.6);
\draw (-0.1,-0.6) -- (0.1,-0.8);
\end{tikzpicture}

fig. a
\end{center}
\end{minipage}
\hfill
\begin{minipage}[c]{.45\linewidth}
\begin{center} \begin{tikzpicture}
\draw (-2.25,0) -- (-0.35,0);
\draw [->] (0.35,0) -- (2.25,0) node [below]{$\omega$};
\draw (0,0) node [below]{$\hbar\frac{\mt{k}^2}{2\mt{m}}$};
\draw [dotted] (-0.25,0) --(0.25,0);
\draw (-0.1,-0.1) -- (0.1,0.1);
\draw (-0.1,0.1) -- (0.1,-0.1);
\draw (-0.35,0)  arc (180:0:0.35);
\end{tikzpicture}

fig. b
\end{center}
\end{minipage}
\vspace{0.3cm}

% 68
On peut encore dire que sans déplacer le pôle, on l'a contourmné par
un demi-cercle de rayon infiniment petit $\epsilon$ dans le demi-plan supérieur
(fig. b) (c'est en effet ainsi que l'on interprète l'intégration si on
remplace dans (14) non pas la forme (20) mais la forme (18) de G$_{(+)}$)

De même pour G$_-$, on a les deux intégrations équivalentes schématisées par les fig. a' et b' :

\vspace{0.3cm}
\begin{minipage}[c]{.45\linewidth}
\begin{center} \begin{tikzpicture}
\draw [->] (-2.25,0) -- (2.25,0) node [below]{$\omega$};
\draw (0,0.07) -- (0,-0.07) node [below]{$\hbar\frac{\mt{k}^2}{2\mt{m}}$};
\draw [dotted] (0,0) --(0,0.7);
\draw (0,0.6) node [below right]{$\epsilon$};
%\draw (2.95,-0.7) ;
\draw (-0.1,0.8) -- (0.1,0.6);
\draw (-0.1,0.6) -- (0.1,0.8);
\end{tikzpicture}

fig. a'
\end{center}
\end{minipage}
\hfill
\begin{minipage}[c]{.45\linewidth}
\begin{center} \begin{tikzpicture}
\draw (-2.25,0) -- (-0.35,0);
\draw [->] (0.35,0) -- (2.25,0) node [below]{$\omega$};
\draw (0,0) node [above right]{$\hbar\frac{\mt{k}^2}{2\mt{m}}$};
\draw [dotted] (-0.25,0) --(0.25,0);
\draw (-0.1,-0.1) -- (0.1,0.1);
\draw (-0.1,0.1) -- (0.1,-0.1);
\draw (-0.35,0)  arc (-180:0:0.35);
\end{tikzpicture}

fig. b'
\end{center}
\end{minipage}
\vspace{0.3cm}

Pour calculer l'intégrale en $\omega$ de (22), nous allons utiliser la méthode
des résidus qui consiste à fermer le contour d'intégration de l'axe réel
par un demi-cercle de rayon tendant vers l'infini et sur lequel l'intégrale
à calculer tend vers zéro. Deux cas se présentent alors :

$\alpha$) t$_2-$t$_1<0$: Pour que $|\mt{e}^{-\mt{i}\omega(\mt{t}_2-\mt{t}_1)}|$ tende vers zéro, sur le grand
cercle, la partie imaginaire de $\omega$ doit être positive et le demi-grand
cercle doit se trouver dans le demi-plan supérieur. Alors G$_+$ n'a pas de
pôles à l'intérieur du contour d'intégration et la méthode des résidus
nous donne K$_{(+)}=0$.

$\beta$) t$_2-$t$_1>0$: il faut au contraire fermer le contour par un demi-cercle
dans le demi-plan inférieur et on a alors K$_{(-)}=0$.

% 69
Nous avons ainsi établi les relations annoncées (21-a) et (21-b) et
K$_+$ est bien la fonction de Green retardée de la particule libre. Nous
allons maintenant achever de la calculer.

\ul{Calcul de K$_+$(2,1)} :

Pour t$_2$ > t$_1$ nous avons vu qu'il faut fermer le contour
d'intégration par un demi-grand cercle dans le demi-plan inférieur.
Il est facile de s'assurer que l'intégrale sur ce demi-grand cercle
tend vers zéro et on a :
\[
\lim_{\epsilon\to\,0_\pm}\int\frac{\mt{d}\omega\mt{ e}^{-\mt{i}\omega(\mt{t}_2-\mt{t}_1)}}{\hbar\omega-\frac{\hbar^2\mt{k}^2}{2\mt{m}}+\mt{i}\epsilon}=
-2\mt{i}\pi\lim_{\epsilon\to\,0_+}\mt{Résidu}\left[\frac{\mt{ e}^{-\mt{i}\omega(\mt{t}_2-\mt{t}_1)}}{\hbar\omega-\frac{\hbar^2\mt{k}^2}{2\mt{m}}+\mt{i}\epsilon}\right]=
-\frac{2\mt{i}\pi}{\hbar}\mt{ e}^{-\mt{i}\frac{\hbar\mt{k}^2}{2\mt{m}}(\mt{t}_2-\mt{t}_1)}
\]

La relation (22) devient alors :
\[
\tag{23}\mt{K}_+(2,1)=\frac{1}{(2\pi)^3}
\iiint\mt{d}^3\mt{k}\mt{ e}^{\mt{i}\vec{\mt{k}}(\vec{\mt{r}}_2-\vec{\mt{r}}_1)}
\mt{ e}^{-\mt{i}\frac{\hbar\mt{k}^2}{2\mt{m}}(\mt{t}_2-\mt{t}_1)}
\]
\[
=\left[\frac{1}{2\pi}
\int\mt{d}\mt{k}_\mt{x}\mt{ e}^{\mt{i}\mt{k}_x(\mt{x}_2-\mt{x}_1)}
\mt{ e}^{-\mt{i}\frac{\hbar\mt{k}_\mt{x}^2}{2\mt{m}}(\mt{t}_2-\mt{t}_1)}\right]\times
\mt{(intégrales analogues en y et z)}
\]


Faisons le changement de variable p $=\hbar$k$_\mt{x}$ , L'intégration sur k devient
\[
\frac{1}{2\pi\hbar}
\int\mt{d}\mt{p}\mt{ e}^{\frac{\mt{i}}{\hbar}\left[\mt{p}(\mt{x}_2-\mt{x}_1)
-\frac{\mt{p}^2}{2\mt{m}}(\mt{t}_2-\mt{t}_1)\right]}
\]

intégrale que nous avons déjà calculée au chapitre II, \S 2  et qui
représente précisément le propagateur $<$ x$_2$t$_2\,|\,$x$_1$t$_1>$ à une dimension
(formule (7) du chapitre II).

% 70
On trouve finalement, en tenant compte de la condition K$_{(+)}=0$ si t$_2<$ t$_1$

\[
\tag{24}\mt{K}_{(+)}(\vec{\mt{r}}_2-\vec{\mt{r}}_1,\mt{t}_2-\mt{t}_1)=
\theta(\mt{t}_2-\mt{t}_1)\mt{e}^{-\frac{3\mt{i}\pi}{4}}
\left[\frac{\mt{m}}{2\pi\hbar(\mt{t}_2-\mt{t}_1)}\right]^\frac{3}{2}
\mt{ exp }\frac{\mt{im}}{2\hbar}\frac{(\vec{\mt{r}}_2-\vec{\mt{r}}_1)}{\mt{t}_2-\mt{t}_1}
\]
Le formule (24) n'est autre que l'extension au problème à trois dimensions
de la formule (7) du chapitre II.

\subsection{Equation des potentiels}% 2°) 

L'équation satisfaite en électromagnétisme par le potentiel
scalaire V est
\[
(\Delta-\frac{1}{\mt{c}^2}\frac{\partial^2}{\partial\mt{t}^2})\mt{K}_+(2,1)=-\frac{\rho}{\epsilon_0}
\]
Cherchons la fonction de Green retardée de cette équation, c'est-à-dire
la distribution K, (2, 1) vérifiant les relations :
\[
\tag{25-a}\left[\Delta_2-\frac{1}{\mt{c}^2}\frac{\partial^2}{\partial\mt{t}_2^2}\right]\mt{K}_+(2,1)=
\delta(\vec{\mt{r}}_2-\vec{\mt{r}}_1)\ \delta(\mt{t}_2-\mt{t}_1)
\]
\[
\tag{25-b}\mt{K}_+(2,1)=0\ \ \ \ \ \mt{si}\ \ \ \ \ \mt{t}_2<\mt{t}_1
\]
Il est évident, en raison de l'invariance du problème par translation, que
K$_+$(2,1) n'est une fonction que de $\vec{\mt{r}}_2-\vec{\mt{r}}_1$ et $\mt{t}_2-\mt{t}_1$. Faisons, comme pour
la fonction de Creen de la particule libre, l'hypothèse que K$_+$(2,1) a une
transformée de Fourier G$_+(\vec{\mt{k}},\omega)$ définie par la relation
\[
\tag{26}\mt{K}_+(2,1)=\frac{1}{(2\pi)^4}\int\!\!...\!\!\int \mt{G}_+(\vec{\mt{k}},\omega)
\ \mt{e}^{\mt{i}\left[\vec{\mt{k}}(\vec{\mt{r}}_2-\vec{\mt{r}}_1)-\omega(\mt{t}_2-\mt{t}_1)\right]}
\mt{d}^3\vec{\mt{k}}\ \mt{d}\omega
\]
La transformée de Fourier de (25-a) nous conduit à : 
\[
\tag{27}\left[-\mt{k}^2+\frac{1}{c^2}\omega^2\right]\mt{G}_+(\vec{\mt{k}},\omega)=1
\]
 
% 71
Nous avons ainsi, comme pour la particule libre, remplacé
l'équation aux dérivées partielles (25-a) par une équation algébrique
dont l'inconnue est la distribution G$_+(\vec{\mt{k}},\omega)$. Le problème est cependant
plus compliqué, car la quantité $\frac{\omega^2}{\mt{c}^2}-\mt{k}^2$ possède deux zéros $\omega=\pm$ck

Il est très facile de vérifier qu'une solution particulière
de (27-a) est
\[
\mt{G}_+(\vec{\mt{k}},\omega)=
\frac{\mt{c}}{2\mt{k}}\left\{\mc{P}\left[\frac{1}{\omega-\mt{ck}}\right]-\mc{P}\left[\frac{1}{\omega+\mt{ck}}\right]\right\}
\]

A cette solution particulière, il faut ajouter la solution générale de
l'équation homogène $[\frac{\omega^2}{\mt{c}^2}-\mt{k}^2]$G$\vec{\mt{k},\omega}=0$, qui est $-\frac{\mt{c}}{2\mt{k}}[\lambda\delta(\omega-\mt{ck})-\lambda'\delta(\omega+\mt{ck})]$ ($\lambda$ et $\lambda'$
 nombres complexes quelconques)

Finalement, la solution générale de (27) est
\[
\mt{G}_+(\vec{\mt{k}},\omega)=
\frac{\mt{c}}{2\mt{k}}\left\{\mc{P}\left[\frac{1}{\omega-\mt{ck}}\right]-\lambda\delta(\omega-\mt{ck})-\mc{P}\left[\frac{1}{\omega+\mt{ck}}\right]+\lambda'\delta(\omega+\mt{ck})\right\}
\]

Elle dépend des deux constantes complexes $\lambda$ et $\lambda'$. Comme pour la particule
libre, envisageons a priori les quatre solutions correspondant à
\begin{center}
$\begin{array}{r c l}
\lambda & = & \alpha\mt{i}\pi \ \ \ \ \ (\mt{ avec }\alpha\mt{ et }\alpha'=\pm1)\\
\lambda' & = & \alpha'\mt{i}\pi
\end{array}$
\end{center}
Elles peuvent s'écrire
\[
\mt{G}_+(\vec{\mt{k}},\omega)=\lim_{\epsilon\to\,0_+}
\frac{\mt{c}}{2\mt{k}}\left[\frac{1}{\omega-\mt{ck}+\alpha\mt{i}\epsilon}-\frac{1}{\omega+\mt{ck}+\alpha'\mt{i}\epsilon}\right]
\]

Nous avons ainsi obtenu quatre distributions G$(\vec{\mt{k}},\omega)$ pour lesquelles les
pôles $\omega=\pm\mt{ck}$ sont déplacés d'une quantité infiniment petite dans le
plan complexe au-dessus ou au-dessous de l'axe réel, On peut encore dire
que dans les intégrations où intervient G$(\vec{\mt{k}},\omega)$, il existe quatre facons
de contourner par un demi-cercle de rayon infiniment petit $\epsilon$ les pôles $\omega=\pm\mt{ck}$

% 72
De façon précise,
\ul{à $\alpha=\alpha'=+1$}, il correspond %%%%%%%%%%%%%%%%%%%%%%%  1

\vspace{0.3cm}
\begin{minipage}[c]{.45\linewidth} \begin{center}
la configuration

\begin{tikzpicture}
\draw [->] (-2.25,0) -- (2.25,0) node [below]{$\omega$};
\draw (-0.95,-0.07) -- (-0.95,0.07) node [above]{-ck};
\draw [dotted] (-0.95,0) -- (-0.95,-0.7); \draw (-1.05,-0.8) -- (-0.85,-0.6); \draw (-1.05,-0.6) -- (-0.85,-0.8);
\draw (0.95,-0.07) -- (0.95,0.07) node [above]{+ck};
\draw [dotted] (0.95,0) -- (0.95,-0.7); \draw (0.85,-0.8) -- (1.05,-0.6); \draw (0.85,-0.6) -- (1.05,-0.8);
\end{tikzpicture} \end{center} \end{minipage}
\hfill
\begin{minipage}[c]{.45\linewidth} \begin{center}
ou le contour d'intégration \vspace{0.3cm}

\begin{tikzpicture}
\draw (-2.25,0) -- (-1.45,0); \draw (-0.75,0) -- (0.75,0); \draw [->] (1.45,0) -- (2.25,0) node [below]{$\omega$};
\draw (-1,-0.05) node [below]{-ck}; \draw (1.2,-0.05) node [below]{ck};
\draw (-1.2,-0.1) -- (-1,0.1); \draw (-1.2,0.1) -- (-1,-0.1); \draw (-1.45,0)  arc (180:0:0.35);
\draw (1,-0.1) -- (1.2,0.1); \draw (1,0.1) -- (1.2,-0.1); \draw (0.75,0)  arc (180:0:0.35);
\end{tikzpicture} \end{center} \end{minipage}
\vspace{0.3cm}

\ul{à $\alpha=\alpha'=-1$} %%%%%%%%%%%%%%%%%%%%%%%%  2

\vspace{0.3cm}
\begin{minipage}[c]{.45\linewidth} \begin{center}
\begin{tikzpicture}
\draw [->] (-2.25,0) -- (2.25,0) node [below]{$\omega$};
\draw (-0.95,-0.07) -- (-0.95,0.07) node [below]{-ck};
\draw [dotted] (-0.95,0) -- (-0.95,0.7); \draw (-1.05,0.8) -- (-0.85,0.6); \draw (-1.05,0.6) -- (-0.85,0.8);
\draw (0.95,-0.07) -- (0.95,0.07) node [below]{+ck};
\draw [dotted] (0.95,0) -- (0.95,0.7); \draw (0.85,0.8) -- (1.05,0.6); \draw (0.85,0.6) -- (1.05,0.8);
\end{tikzpicture} \end{center} \end{minipage}
\hfill
\begin{minipage}[c]{.45\linewidth} \begin{center}
\vspace{0.3cm}
ou \hspace {0.3cm}
\begin{tikzpicture}
\draw (-2.25,0) -- (-1.45,0); \draw (-0.75,0) -- (0.75,0); \draw [->] (1.45,0) -- (2.25,0) node [above]{$\omega$};
\draw (-1,0.05) node [above]{-ck}; \draw (1.2,0.05) node [above]{ck};
\draw (-1.2,-0.1) -- (-1,0.1); \draw (-1.2,0.1) -- (-1,-0.1); \draw (-1.45,0)  arc (-180:0:0.35);
\draw (1,-0.1) -- (1.2,0.1); \draw (1,0.1) -- (1.2,-0.1); \draw (0.75,0)  arc (-180:0:0.35);
\end{tikzpicture} \end{center} \end{minipage}
\vspace{0.3cm}

\ul{à $\alpha=1;\ \alpha'=-1$} %%%%%%%%%%%%%%%%%%%%%%%%  3

\vspace{0.3cm}
\begin{minipage}[c]{.45\linewidth} \begin{center}
\begin{tikzpicture}
\draw [->] (-2.25,0) -- (2.25,0) node [below]{$\omega$};
\draw (-0.95,-0.07) -- (-0.95,0.07) node [below]{-ck};
\draw [dotted] (-0.95,0) -- (-0.95,0.7); \draw (-1.05,0.8) -- (-0.85,0.6); \draw (-1.05,0.6) -- (-0.85,0.8);
\draw (0.95,-0.07) -- (0.95,0.07) node [above]{+ck};
\draw [dotted] (0.95,0) -- (0.95,-0.7); \draw (0.85,-0.8) -- (1.05,-0.6); \draw (0.85,-0.6) -- (1.05,-0.8);
\end{tikzpicture} \end{center} \end{minipage}
\hfill
\begin{minipage}[c]{.45\linewidth} \begin{center}
\vspace{0.3cm}
ou \hspace {0.3cm}
\begin{tikzpicture}
\draw (-2.25,0) -- (-1.45,0); \draw (-0.75,0) -- (0.75,0); \draw [->] (1.45,0) -- (2.25,0) node [above]{$\omega$};
\draw (-1,0.05) node [above]{-ck}; \draw (1.2,-0.05) node [below]{ck};
\draw (-1.2,-0.1) -- (-1,0.1); \draw (-1.2,0.1) -- (-1,-0.1); \draw (-1.45,0)  arc (-180:0:0.35);
\draw (1,-0.1) -- (1.2,0.1); \draw (1,0.1) -- (1.2,-0.1); \draw (0.75,0)  arc (180:0:0.35);
\end{tikzpicture} \end{center} \end{minipage}
\vspace{0.3cm}

\ul{à $\alpha=-1;\ \alpha'=1$} %%%%%%%%%%%%%%%%%%%%%%%%  4

\vspace{0.3cm}
\begin{minipage}[c]{.45\linewidth} \begin{center}
\begin{tikzpicture}
\draw [->] (-2.25,0) -- (2.25,0) node [below]{$\omega$};
\draw (-0.95,-0.07) -- (-0.95,0.07) node [above]{-ck};
\draw [dotted] (-0.95,0) -- (-0.95,-0.7); \draw (-1.05,-0.8) -- (-0.85,-0.6); \draw (-1.05,-0.6) -- (-0.85,-0.8);
\draw (0.95,-0.07) -- (0.95,0.07) node [below]{+ck};
\draw [dotted] (0.95,0) -- (0.95,0.7); \draw (0.85,0.8) -- (1.05,0.6); \draw (0.85,0.6) -- (1.05,0.8);
\end{tikzpicture} \end{center} \end{minipage}
\hfill
\begin{minipage}[c]{.45\linewidth} \begin{center}
\vspace{0.3cm}
ou \hspace {0.3cm}
\begin{tikzpicture}
\draw (-2.25,0) -- (-1.45,0); \draw (-0.75,0) -- (0.75,0); \draw [->] (1.45,0) -- (2.25,0) node [above]{$\omega$};
\draw (-1,-0.05) node [below]{-ck}; \draw (1.2,0.05) node [above]{ck};
\draw (-1.2,-0.1) -- (-1,0.1); \draw (-1.2,0.1) -- (-1,-0.1); \draw (-1.45,0)  arc (180:0:0.35);
\draw (1,-0.1) -- (1.2,0.1); \draw (1,0.1) -- (1.2,-0.1); \draw (0.75,0)  arc (-180:0:0.35);
\end{tikzpicture} \end{center} \end{minipage}
\vspace{0.3cm}

Chacune de ces quatre distributions G$(\vec{\mt{k}},\omega)$ correspond à une fonction
K(2,1) vérifiant des conditions aux limites particulières.

Pour calculer l'intégrale en $\omega$ de (26) par la méthode des résidus, dans
le cas où t$_2-$t$_1<$ 0, il faut fermer le contour d'intégration par un
demi-cercle dans le demi-plan supérieur et c'est donc la première configuration $\alpha=\alpha'=+1$ qui convient pour donner un résultat nul.

La transformée de la fonction de Green cherchée est donc
\[
\mt{G}_+(\vec{\mt{k}},\omega)=\lim_{\epsilon\to\,0_+}
\frac{\mt{c}}{2\mt{k}}\left\{\frac{1}{\omega+\mt{i}\epsilon-\mt{ck}}-\frac{1}{\omega+\mt{i}\epsilon+\mt{ck}}\right\}
\]
\[
\tag{28}\mt{G}_+(\vec{\mt{k}},\omega)=\lim_{\epsilon\to\,0_+}\frac{1}{\frac{(\omega+\mt{i}\epsilon)^2}{\mt{c}^2}-\mt{k}^2}
\]

% 73
L'intégration (26) nous conduit alors à la fonction de Green
retardée K(2,1). On intègre sur $\omega$ par la méthode des résidus :
\[
\lim_{\epsilon\to\,0_+}\int_{-\infty}^{\infty}\frac{\mt{e}^{-\mt{i}\omega(\mt{t}_2-\mt{t}_1)}}{\frac{(\omega+\mt{i}\epsilon)^2}{\mt{c}^2}-\mt{k}^2}\mt{d}\omega=
-2\mt{i}\pi\ \theta(\mt{t}_2-\mt{t}_1)\lim_{\epsilon\to\,0_+}\sum\mt{Résidu}\left[\frac{\mt{e}^{-\mt{i}\omega(\mt{t}_2-\mt{t}_1)}}{\frac{(\omega+\mt{i}\epsilon)^2}{\mt{c}^2}-\mt{k}^2}\right]
\]
\[
=2\mt{i}\pi\mt{ c }\theta(\mt{t}_2-\mt{t}_1)\left[\frac{\mt{e}^{\mt{ikc}(\mt{t}_2-\mt{t}_1)}-\mt{e}^{-\mt{ikc}(\mt{t}_2-\mt{t}_1)}}{2\mt{k}}\right]
\]
(26) donne alors :
\[
\mt{K}_+(2,1)=\frac{\mt{ic}}{(2\pi)^3}\iiint \mt{e}^{\mt{i}\vec{\mt{k}}(\vec{\mt{r}}_2-\vec{\mt{r}}_1)}\frac{\mt{e}^{\mt{ikc}(\mt{t}_2-\mt{t}_1)}-\mt{e}^{-\mt{ikc}(\mt{t}_2-\mt{t}_1)}}{2\mt{k}}\theta(\mt{t}_2-\mt{t}_1)\vec{\mt{d}^3\mt{k}}
\]

En passant en coordonnées sphériques, l'axe Oz étant dirigé suivant $\vec{\mt{r}}_2-\vec{\mt{r}}_1$, $\int\vec{\mt{d}^3\mt{k}}$
 devient $\int2\pi\mt{k}^2\mt{ dk}\mt{ sin }\psi\mt{ d}\psi$ et
\[
\mt{K}_+(2,1)=\frac{\mt{ic }\theta(\mt{t}_2-\mt{t}_1)}{(2\pi)^2}\int_0^\pi\int_0^\infty \mt{e}^{\mt{ik}|\vec{\mt{r}}_2-\vec{\mt{r}}_1|\mt{cos}\psi}\left(\frac{\mt{e}^{\mt{ikc}(\mt{t}_2-\mt{t}_1)}-\mt{e}^{-\mt{ikc}(\mt{t}_2-\mt{t}_1)}}{2}\right)\mt{k sin}\psi\mt{ dk d}\psi
\]
Posons t$_2-$t$_1=\tau$, $|\vec{\mt{r}}_2-\vec{\mt{r}}_1|=\,$r et effectuons l'intégration sur $\psi$ : il
vient :
\[
\mt{K}_+(2,1)=\frac{1}{(2\pi)^2}\ \frac{\mt{c}}{2\mt{r}}\ \theta(\tau) \int_0^\infty (\mt{e}^{\mt{ikr}}-\mt{e}^{-\mt{ikr}}) (\mt{e}^{\mt{ikc}\tau}-\mt{e}^{-\mt{ikc}\tau})\mt{ dk}
\]
Soit
\[
\mt{K}_+(2,1)=\frac{1}{(2\pi)^2}\ \frac{\mt{c}}{2\mt{r}}\ \theta(\tau) \int_0^\infty \left[\mt{e}^{\mt{ik(r+c}\tau)}+\mt{e}^{-\mt{ik(r+c}\tau)}-\mt{e}^{\mt{ik(r-c}\tau)}-\mt{e}^{-\mt{ik(r-c}\tau)}\right]\mt{ dk}
\]
\[
=\frac{1}{(2\pi)^2}\ \frac{\mt{c}}{2\mt{r}}\ \theta(\tau) \int_{-\infty}^\infty \left[\mt{e}^{\mt{ik(r+c}\tau)}-\mt{e}^{\mt{ik(r-c}\tau)}\right]\mt{ dk}
\]
\[
=\frac{\mt{c}}{4\pi\mt{r}}\ \theta(\tau) \big[\delta(\mt{r+c}\tau)-\delta(\mt{r-c}\tau)\big]
\]

% 74
r $=|\vec{\mt{r}}_2-\vec{\mt{r}}_1|$ est positif. Le terme $\theta(\tau)$ permet donc de supprimer la
distribution $\delta$(r + c$\tau$) et finalement :
\[
\tag{29}\mt{K}_+(2,1)=-\frac{\mt{1}}{4\pi\mt{r}}\ \delta(\mt{r-c}\tau)\ \theta(\tau)=-\frac{\mt{c}}{4\pi\mt{r}}\ \delta(\tau-\frac{\mt{r}}{\mt{c}})\ \theta(\tau)
\]

\[
=-\frac{1}{4\pi|\vec{\mt{r}}_2-\vec{\mt{r}}_1|}\ \delta\left[\mt{t}_2-\mt{t}_1-\frac{|\vec{\mt{r}}_2-\vec{\mt{r}}_1|}{\mt{c}}\right]\ \theta(\mt{t}_2-\mt{t}_1)
\]
On en déduit immédiatement le potentiel au point $\vec{\mt{r}}_2$, t$_2$  en présence d'une
distribution de charges $\rho(\vec{\mt{r}}_1$, t$_1)$ :
\[
\tag{30}\mt{V}(\vec{\mt{r}}_2, t_2)=\frac{1}{4\pi\epsilon_0}\ \int \rho(\vec{\mt{r}}_1, \mt{t}_1)\ \frac{1}{|\vec{\mt{r}}_2-\vec{\mt{r}}_1|}\ 
\delta\left[\mt{t}_2-\mt{t}_1-\frac{|\vec{\mt{r}}_2-\vec{\mt{r}}_1|}{\mt{c}}\right]\mt{d}^3\vec{\mt{r}_1}\mt{ dt}_1
\]
\[
=\frac{1}{4\pi\epsilon_0}\ \int \frac{\rho(\vec{\mt{r}}_1, \mt{t}_2-\frac{|\vec{\mt{r}}_2-\vec{\mt{r}}_1|}{\mt{c}})}{|\vec{\mt{r}}_2-\vec{\mt{r}}_1|}
\mt{d}^3\vec{\mt{r}_1}
\]
(30) n'est autre que l'équation bien connue des "potentiels retardés" que
l'on retrouve ainsi à partir de la fonction de Green du d'Alembertien.

\ul{Remarque} : La condition aux limites imposée à la fonction de Green K$_+$(2,1)
fait jouer au temps un rôle privilégié. Dans la transformée de Fourier
G$_+(\vec{\mt{k}}, \omega)$, c'est $\omega$ qui est particularisé (cf relation (28) ). On voit ainsi
que cette condition ne satisfait pas à l'invariance relativiste. Nous verrons qu'en relativité, il faut imposer à la fonction de Green d'autres conditions, satisfaisant à l'invariance relativiste et qui correspondent à des
contours d'intégration pour lesquels les pôles $\pm$ck sont repoussés de part
et d'autre de l'axe réel.
% 75

\ul{En conclusion} : Nous avons montré que la technique de la transformation
de Fourier nous permet de calculer les fonctions de Green des équations
aux dérivées partielles à coefficients constants. Les conditions aux
limites se traduisent simploment par un déplacement dans le plan complexe,
dans un sens bien déterminé, des pôles de cette transformée de Fourier,
Cette propriété nous permet de tenir compte de \ul{façon analytiquement} simple
du \ul{principe physique de causalité} (K = O si t < t).

Nous allons maintenent appliquer les techniques des fonctions
de Green à la théorie des perturbations :

\section{Développement en série de perturbations}% E.

Soit un système quantique dont l'évolution est décrite par un
hamiltonien H$_0$. Supposons connue sa fonction de Green retardée K$_0$(2,1).
Elle vérifie les relations :
\[
\tag{31-a}\left[\mt{i}\hbar\frac{\partial}{\partial \mt{t}_2}-\mt{H}_0(2)\right]\mt{K}_0(2,1)=\mt{i}\hbar\delta_4(2,1)
\]
\[
\tag{31-b}\mt{K}_{0+}(2,1)=0\ \ \ \ \ \mt{si} \ \ \ \ \ \mt{t}_2<\mt{t}_1
\]
Appliquons maintenant une perturbation V (dépendante au indépendante du
temps). Le hamiltonien devient H = H$_0$ + V. Cherchons à déterminer, à partir
de K$_{0+}$ et de V, la fonction de Green retardée du hamiltonien H, c'est-à-dire
la distribution K$_+$(2,1) vérifiant :
\[
\tag{32-a}\left[\mt{i}\hbar\frac{\partial}{\partial \mt{t}_2}-\mt{H}(2)\right]\mt{K}(2,1)=\mt{i}\hbar\delta_4(2,1)
\]
\[
\tag{32-b}\mt{K}_+(2,1)=0\ \ \ \ \ \mt{si} \ \ \ \ \ \mt{t}_2<\mt{t}_1
\]
Pour cela, écrivons l'équation (32-a) sous la forme inhomogène :
\[
\left[\mt{i}\hbar\frac{\partial}{\partial \mt{t}_2}-\mt{H}_0(2)\right]\mt{K}_+(2,1)=\mt{i}\hbar\delta_4(2,1)+\mt{V}(2)\mt{K}(2,1)
\]
et considérons formellement le second membre comme une source $\rho$(2).

% 76

Nous avons vu alors (\S 3) que si K'(2,1) vérifie
\[
\tag{33}\left[\mt{i}\hbar\frac{\partial}{\partial \mt{t}_2}-\mt{H}_0(2)\right]\mt{K'}(2,1)=\delta_4(2,1)
\]
alors
\[
\tag{34}\mt{K}_+(2,1)=\int\rho(3)\mt{ K'}(2,3)\mt{ d}3
\]
Or, d'après (31-a), on voit que K'(2, 1) = $\frac{\mt{K}_{0+}(2,1)}{\mt{i}\hbar}$.
(34) s'écrit alors
\[
\mt{K}_+(2,1)=\frac{1}{\mt{i}\hbar}\int\mt{K}_{0+}(2,3)\big[\mt{i}\hbar\ \delta_4(3,1)+\mt{V}(3)\mt{ K}_+(3,1)\big]\mt{ d}3
\]
Soit :
\[
\tag{35}\mt{K}_+(2,1)=\mt{K}_{0+}(2,1)+\frac{1}{\mt{i}\hbar}\int\mt{K}_{0+}(2,3)\mt{ V}(3)\mt{ K}_+(3,1)\mt{ d}3
\]

Il est facile de montrer que K$_+$(2,1), défini par l'équation (35),
satisfait bien à l'équation (32-a). Nous montrons dans la suite qu'elle
satisfait à la relation (32-b). les deux fonctions de Green retardées
du hamiltonien non perturbé et du hamiltonien perturbé sont donc reliées
entre elles par \ul{l'équation intégrale (35)}.

Nous pouvons, dans le second membre de (35), remplacer K$_+$(3, 1)
par son expression intégrale et on obtient ainsi par itération le \ul{développement
en série de Neumann-Liouville} de la perturbation :
\[
\tag{36}\mt{K}_+(2,1)=\mt{K}_{0+}(2,1)+\frac{1}{\mt{i}\hbar}\int\mt{K}_{0+}(2,3)\mt{ V}(3)\mt{ K}_{0+}(3,1)\mt{ d}3
\]
\[
+(\frac{1}{\mt{i}\hbar})^2\int\mt{K}_{0+}(2,3)\mt{ V}(3)\mt{ K}_{0+}(3,4)\mt{ V}(4)\mt{ K}_{0+}(4,1)\mt{ d}3\mt{ d}4 + ...
\]

Dans chaque terme de la série (36), la présence des fonctions de
Green retardées K$-{0+}$ fait que les temps sont automatiquement rangés par
ordre croissant entre t$_1$ et t$_2$ : par exenple, le troisième terme est nul
% 77
à moins que l'on ait simultanément  t$_2\geqslant$ t$_3$; t$_3\geqslant$ t$_4$; t$_4\geqslant$ t$_1$. Il est donc
nul si t$_2<$ t$_1$. Il en est de même de tous les termes de ce développement.
La solution de l'équation (35) est donc bien telle que K$_+$(2, 1) = 0 si
t$_2<$ t$_1$. C'est donc bien la \ul{fonction de Green retardée} du système perturbé.

\ul{Remarques} :
Dans le développement (36), les intégrations sur les variables 3, 4, etc.
(c'est-à-dire sur $\vec{\mt{r}}_3$,$\mt{t}_3$; $\vec{\mt{r}}_4$,$\mt{t}_4$, etc.) sont libres et s'étendent à l'espace-temps tout entier.
Les conditions aux limites sur les fonctions de Green
sont \ul{implicitement} contenues dans les intégrales et annulent toutes les contributions
autres que t$_2\geqslant$ ... $\geqslant$ t$_3\geqslant$ t$_1$. Le développement (36) correspond
à une simplification notable de l'écriture par rapport au développerent
classique de le page 41 (formule (27) ). Cette simplification est \ul{effective}
car nous avons vu qu'on peut tenir compte des conditions aux limites de façon
mathématiquement simple.
\section{Représentation diagrammatique du développement de Neumann-Liouville}% F.

Nous allons définir une convention qui va nous permettre d'associer un diagramme à chaque terme du développement (36).

\ul{Conventions} :

a) Les temps sont représentés en ordonnée, croissant de bas en
haut, et en abscisse, on portera une "variable d'espace".

b) A chaque terme K$_{0+}$(i, j) sera associée une ligne \ul{pointillée}
joignant i et j

\vspace{0.3cm}
\begin{minipage}[c]{.45\linewidth} \begin{center}
\begin{tikzpicture}
\draw [dashed] (0,-1) node [left]{j} -- (0,1) node [left]{i};
\draw (-0.1,-1.1) -- (0.1,-0.9); \draw (-0.1,-0.9) -- (0.1,-1.1);
\draw (-0.1,1.1) -- (0.1,0.9); \draw (-0.1,0.9) -- (0.1,1.1);
\draw (-0.2,-0.1) -- (0,0.1); \draw (0,0.1) -- (0.2,-0.1) node [right]{\ \ \ K$_{0+}$(i, j)};
\end{tikzpicture} \end{center} \end{minipage}
\hfill
\begin{minipage}[c]{.45\linewidth} \begin{center}
\begin{tikzpicture}
\draw (0,-1) node [left]{j} -- (0,1) node [left]{i};
\draw (-0.1,-1.1) -- (0.1,-0.9); \draw (-0.1,-0.9) -- (0.1,-1.1);
\draw (-0.1,1.1) -- (0.1,0.9); \draw (-0.1,0.9) -- (0.1,1.1);
\draw (-0.2,-0.1) -- (0,0.1); \draw (0,0.1) -- (0.2,-0.1) node [right]{\ \ \ K$_+$(i, j)};
\end{tikzpicture} \end{center} \end{minipage}
\vspace{0.3cm}

c) A chaque terme K$_+$(i, j) sera associée une ligne \ul{pleine} joignant i et j.

d) A chaque terme V (i) un cercle autour du point i.

% 78
On associe ainsi à chaque intégrale du déveloprement (36) un
diagramme, Chaque diagramme est affecté d'un facteur de poids $(\frac{1}{\mt{i}\hbar})^n$, où n
représente le nombre de cercles du diagramme, et pour obtenir le terme du
développement (36) associé, il faut intégrer librement sur les variables
entourées d'un cercle. On obtient alors le développement diagrammatique :
\begin{flushright}
\begin{tikzpicture}
\def\espace{2.3}
\draw (-0.25,-1.5) node {$\bullet$} node [left]{1} -- (0.25,1.5) node {$\bullet$} node [left]{2};
\draw (\espace/2,0) node{\bf{=}};

\draw [dashed] (\espace-0.25,-1.5) node {$\bullet$} node [left]{1} -- (\espace+0.25,1.5) node {$\bullet$} node [left]{2};
\draw (3*\espace/2,0) node{\bf{+}};

\draw [dashed] (2*\espace-0.25,-1.5) node {$\bullet$} node [left]{1} -- (2*\espace+0.9,0) node {$\bullet$} ;
\draw (2*\espace+0.9,0) circle (0.25);
\draw (2*\espace+1.2,0.4) node {3};
\draw [dashed] (2*\espace+0.9,0) -- (2*\espace+0.25,1.5) node {$\bullet$} node [left]{2};

\draw (6.5*\espace/2,0) node{\bf{+}};

\draw [dashed] (4*\espace-0.5,-1.5) node {$\bullet$} node [left]{1} -- (4*\espace,-0.5) node {$\bullet$};
\draw (4*\espace,-0.5) circle (0.25);
\draw (4*\espace+0.4,-0.7) node {4};
\draw [dashed] (4*\espace,-0.5) -- (4*\espace+1,0.5) node {$\bullet$};
\draw (4*\espace+1,0.5) circle (0.25);
\draw (4*\espace+1.5,0.5) node {3};
\draw [dashed] (4*\espace+1,0.5) -- (4*\espace,1.5) node {$\bullet$} node [left]{2};
\draw (11*\espace/2,0) node{\bf{+ ....}};
\draw (13*\espace/2,0) node{(37)};

\end{tikzpicture} \end{flushright}

Il est évident qu'avec nos conventions, il y a correspondance
biunivoque entre chaque terme du développement diagrammatique et chaque
terme de la série (36).

Les diagrammes conduisent à une représentation physique simple de
la perturbation : \ul{l'amplitude de probabilité} pour que la particule en $\vec{\mt{r}}_1$
à l'instant t$_1$ se trouve en $\vec{\mt{r}}_2$ à l'instant t$_2$, en présence du potentiel
perturbateur V, est la somme de plusieurs contributions :

— une correspondant aux chemins où elle ne subit aucune interaction avec V

— une correspondant aux chemins où elle subit une interaction avec V

— une correspondent aux chemins où elle subit deux interactions avec V

etc...

% 79
Prenons pour exemple le \ul{diagramme de diffusion double} :
\begin{center}
\begin{tikzpicture}
\draw [dashed] (0.5,-1.5) node {$\bullet$} node [left]{1} -- (0,-0.5) node {$\bullet$};
\draw (0,-0.5) circle (0.25);
\draw (0.4,-0.7) node {4};
\draw [dashed] (0,-0.5) -- (1,0.5) node {$\bullet$};
\draw (1,0.5) circle (0.25);
\draw (1.5,0.5) node {3};
\draw [dashed] (1,0.5) -- (-1,2.5) node {$\bullet$} node [left]{2};
\end{tikzpicture} \end{center}
Ce diagramne représente une particule qui va librement (sous l'effet de
H$_0$ seul) de 1 à 4, qui est diffusée en 4 par V, qui va ensuite librement
à 3 où elle est de nouveau diffusée par V pour aller ensuite librement
jusqu'en 2. A ce chemin correspond une certaine \ul{amplitude} de probabilité.
Pour avoir la contribution des processus de diffusion double, il faut
sommer sur \ul{toutes} les positions des points intermédiaires 4 et 3. Pour
avoir la contribution de \ul{tous} les processus de diffusion, il faut ensuite
sommer tous les diarramres correspondant aux différents ordres de diffusion.
\ul{En résumé} : Pour joindre 1 à 2, plusieurs chemins sont possibles. À chaque
chemin est associée une amplitude de probabilité. Il faut sommer sur toutes
ces amplitudes pour avoir l'amplitude de probabilité globale K(2,1).
On retrouve ainsi l'idée de base de l'approche de \ul{Feynman} de la mécanique
quantique.

% 80
Remarques : $\alpha$) Les conditions aux limites sur K font que les diagrammes du type
\begin{center}
\begin{tikzpicture}
\draw [dashed] (2,-1.5) node {$\bullet$} node [right]{1} -- (1,0.5) node {$\bullet$};
\draw (1,0.5) circle (0.25);
\draw [very thick] (1.4,-0.8) -- (1.5,-0.5);\draw [very thick] (1.5,-0.5) -- (1.75,-0.65);

\draw (1.4,0.5) node {4};
\draw [dashed] (1,0.5) -- (0,-0.5) node {$\bullet$};
\draw (0,-0.5) circle (0.25);
\draw [very thick] (0.5,0.25) -- (0.5,0);\draw [very thick] (0.5,0) -- (0.75,0.05);

\draw (0.4,-0.5) node {3};
\draw [dashed] (0,-0.5) -- (-1.2,1.5) node {$\bullet$} node [left]{2};
\draw [very thick] (-0.6,0.2) -- (-0.6,0.5);\draw [very thick] (-0.6,0.5) -- (-0.3,0.4);

\end{tikzpicture} \end{center}
que nous n'avons pas exclu \ul{a priori} ont une contribution nulle.

En \ul{mécanique quantique non relativiste}, de tels diagrammes n'ont
pas de signification physique. Ils correspondent en effet à un retour du
chemin dans le passé où encore à la présence de trois particules entre les
instants t$_3$ et t$_4$.

En \ul{mécanique quantique relativiste}, ils ont au contraire une
signification, car il existe alors une possibilité de création et d'annihilation
de particules (le nombre de particules n'est plus une constante
du mouvement).

Par exemple, dans le cas où la particule est un électron, nous
verrons que le diagramme représente un électron partant de 1 à l'instant t$_1$,
voyant apparaître une paire électron-positron sous l'effet du potentiel V
à l'instant t$_3$, annihilé par le positron de la paire à l'instant t$_4$ tandis
que l'électron de la paire arrive en 2 à l'instant t$_2$. Le positron s'interprète
alors naturellement comme un électron d'énergie négative se propageant
dans le sens inverse du temps (théorie de Feynman du positron).
% 81

$\beta$) Avec les conventions adoptées, l'équation intégrale (35) peut se
représenter par le diagramme :
\begin{flushright}
\begin{tikzpicture}
\def\espace{2.3}
\draw (-0.25,-1.5) node {$\bullet$} node [left]{1} -- (0.25,1.5) node {$\bullet$} node [left]{2};
\draw (\espace/2,0) node{\bf{=}};

\draw [dashed] (\espace-0.25,-1.5) node {$\bullet$} node [left]{1} -- (\espace+0.25,1.5) node {$\bullet$} node [left]{2};
\draw (3*\espace/2,0) node{\bf{+}};

\draw (2*\espace-0.25,-1.5) node {$\bullet$} node [left]{1} -- (2*\espace+0.9,0) node {$\bullet$} ;
\draw (2*\espace+0.9,0) circle (0.25);
\draw (2*\espace+1.35,0.1) node {3};
\draw [dashed] (2*\espace+0.9,0) -- (2*\espace+0.25,1.5) node {$\bullet$} node [left]{2};

\draw (10*\espace/2,0) node{(38)};

\end{tikzpicture} \end{flushright}
Ce diapramme (38) peut s'interpréter aussi comme le résultat obtenu en
sommant toutes les parties de diagramme comprises entre 1 et 3 dans le
développement diagrammatique (37) (1a particule va de 1 à 3 en régime
"forcé" sous l'effet de H$_0$ + V, puis librement (sous l'effet de H$_0$) de
3 en 2). Nous définissons ainsi sur les diagrammes une opération très
importante, \ul{la ressommation}.

\section{Opérateur fonction de Green. Propagateur}% G.

\subsection{Cas général}% 1°) :

Nous nous sommes jusqu'à présent placés dans la représentation $\vec{\mt{r}}$
dans laquelle la fonction de Green retardée s'écrit :
\[
\mt{K}_+(2,1)=\theta(\mt{t}_2-\mt{t}_1)<\vec{\mt{r}}_2\mt{ t}_2\ |\ \vec{\mt{r}}_1\ \mt{t}_1>
=\theta(\mt{t}_2-\mt{t}_1)<\vec{\mt{r}}_2\ |\mt{ U(t}_2\mt{,t}_1)\ |\ \vec{\mt{r}}_1>
\]

K$_+$(2, 1) est donc un élément de matrice de l'opérateur $\theta(\mt{t}_2-\mt{t}_1)\mt{ U(t}_2\mt{,t}_1)$.

% 82

Ceci nous conduit à nous affranchir de la représentation $\vec{\mt{r}}$ en définissant
l'opérateur fonction de Green retardé par la relation
\[
\tag{39}\boxed{\mc{K}_+(\mt{t}_2,\mt{t}_1)=\mt{ U(t}_2\mt{,t}_1)\ \theta(\mt{t}_2-\mt{t}_1)}
\]
L'opérateur d'évolution U(t$_2$,t$_1$) satisfait à l'équation de Schrödinger :
\[
\mt{i}\hbar\ \frac{\mt{d}}{\mt{dt}_2}\ \mt{ U(t}_2\mt{,t}_1)-\mt{H}(\mt{t}_2)\mt{ U(t}_2\mt{,t}_1)=0
\]
On en déduit aisément que l'opérateur $\mc{K}_+(\mt{t}_2,\mt{t}_1)$ est \ul{défini} par les
\ul{deux relations} :
\[
\tag{40-a}\left[\mt{i}\hbar\ \frac{\mt{d}}{\mt{dt}_2}-\mt{H}(\mt{t}_2)\right]\mc{K}_+(\mt{t}_2,\mt{t}_1)=\mt{i}\hbar\ \delta(\mt{t}_2-\mt{t}_1)
\]
\[
\tag{40-b}\mc{K}_+(\mt{t}_2,\mt{t}_1)=0 \ \ \ \ \ \ \mt{si}\ \ \ \ \ \mt{t}_2<\mt{t}_1
\]

Les relations (40-a) et (40-b) sont en effet équivalentes à (39). Elles
sont l'équivalent pour les orérateurs fonction de Green des relations
(10-a) et (10-b) de définition des fonctions de Green. De même que pour
les fonctions de Green, la condition (40-b) est indispensable à la définition
complète de K. Notons cependant qu'à la différence de (10-a) et
(10-b), la distribution  de (40-a) est à une dimension : seul le temps
reste une variable.

\ul{Définition} : On appelle \ul{propagateur} $\mc{G}_+(\omega)$ l'opérateur transformée de
Fourier de $\mc{K}_+$ par rapport à la variable t$_2$ : $\mc{G}_+(\omega)$ est défini par la
relation :
\[
\mc{K}_+(\mt{t}_2,\mt{t}_1)=\frac{1}{2\pi} \int\mt{e}^{\mt{i}\omega\mt{t}_2}\ \mc{G}_+(\omega,\mt{t}_1)\ \mt{d}\omega
\]

1l est facile de montrer que tous les résultats établis au \S 5 sur le
développement en série de perturbation s'adapte sans changement aux opérateurs
fonction de Green. Notamment les équations (35) et (36) deviennent :
\[
\tag{42}\mc{K}_+(\mt{t}_2,\mt{t}_1)=\mc{K}_{0+}(\mt{t}_2,\mt{t}_1) + \frac{1}{\mt{i}\hbar}\int
\mc{K}_{0+}(\mt{t}_2,\mt{t}_3)\ \mt{V}(\mt{t}_3)\ \mc{K}_+(\mt{t}_3,\mt{t}_2)\ \mt{dt}_3
\]
\[
\tag{43}\mc{K}_+(\mt{t}_2,\mt{t}_1)=\mc{K}_{0+}(\mt{t}_2,\mt{t}_1) + \frac{1}{\mt{i}\hbar}\int
\mc{K}_{0+}(\mt{t}_2,\mt{t}_3)\ \mt{V}(\mt{t}_3)\ \mc{K}_{0+}(\mt{t}_3,\mt{t}_1)\ \mt{dt}_3
\]
\[
+\ (\frac{1}{\mt{i}\hbar})^2\int\mc{K}_{0+}(\mt{t}_2,\mt{t}_3)\ \mt{V}(\mt{t}_3)\ \mc{K}_{0+}(\mt{t}_3,\mt{t}_4)\ \mt{V}(\mt{t}_4)\ \mc{K}_{0+}(\mt{t}_4,\mt{t}_1)\ \mt{dt}_3\ \mt{dt}_4\ +\ ...
\]
%83
Le développement (43), compte tenu de (39) redonne immédieterent la formule
classique (27) du chapitre 3.

\subsection{Cas où le hamiltonien H ne dépend pas du temps}% 2°) :

Un cas particulier très important et très fréquent est celui où
H est indépendant du temps.
On connaît alors U(t$_2$-t$_1$) :

\[
U(\mt{t}_2-\mt{t}_1)=\mt{e}^{-\frac{\mt{iH}(\mt{t}_2-\mt{t}_1)}{\hbar}}
\]

L'opérateur fonction de Green ne dépend plus alors que de la différence
t$_2$ - t$_1$ (il y a invariance par translation dans le temps) et (39) devient :
\[
\tag{44}\mc{K}_+(\mt{t}_2,\mt{t}_1)=\mc{K}_+(\mt{t}_2-\mt{t}_1)=\mt{e}^{-\frac{\mt{iH}(\mt{t}_2-\mt{t}_1)}{\hbar}}\ \theta(\mt{t}_2-\mt{t}_1)
=\mt{e}^{-\frac{\mt{iH}\tau}{\hbar}}\ \theta(\tau)
\]
en posant $\tau=$ t$_2-$t$_1$.

Posons $\omega=$ E/$\hbar$; \ul{le propagateur} $\mc{G}_+(\omega)$ devient $\mc{G}_+(\mt{E})$ et il est défini
par :
\[
\tag{45}\mc{K}_+(\tau)= \frac{1}{2\pi\hbar}\ \int_{-\infty}^{+\infty} \mt{e}^{-\frac{\mt{iE}\tau}{\hbar}}\ \mc{G}_+(\mt{E})\ \mt{dE}
\]

% 84

ou encore
\[
\mc{G}_+(\mt{E})=\int_{-\infty}^{+\infty} \mt{e}^{\frac{\mt{iE}\tau}{\hbar}}\ \mc{K}_+(\tau)\ \mt{d}\tau
\]

soit :
\[
\tag{46}\mc{G}_+(\mt{E})=\int_0^{+\infty} \mt{e}^{\frac{\mt{i(E-H)}\tau}{\hbar}}\ \mt{d}\tau
\]

La relation (46) définit $\mc{G}_+(\mt{E})$ comme une intégrale d'opérateur. Dans
une représentation déterminée $\mc{G}_+(\mt{E})$ définit une distribution (par
exemple la fonction de Green en représentation $\vec{\mt{r}}$ qui est du type $\int_0^\infty \mt{e}^{\mt{ikx}}\ \mt{dk}$.
Une telle quantité n'a pas de sens en tant que \ul{fonction}. En tant que distribution, on peut la considérer comme
\[
\lim_{\epsilon\to 0_+}\int_0^\infty \mt{e}^{\mt{ikx}}\ \mt{e}^{-\epsilon\mt{k}}\ \mt{dk} = \lim_{\epsilon\to 0_+}\int_0^\infty \mt{e}^{\mt{ik(x+i}\epsilon)}\ \mt{dk}
\]
ce qui revient à introduire dans l'intégrale un facteur de convergence aussi
faible qu'on veut, qui donne un sens à l'intégrale, mais qui ne modifie pas
l'action de la distribution sur une fonction à support borné (ou suffisamment
décroissante à l'infini).
Or
\[
\lim_{\epsilon\to 0_+}\int_0^\infty \mt{e}^{\mt{ik(x+i}\epsilon)}\ \mt{dk} = \lim_{\epsilon\to 0_+}\ \frac{1}{\mt{x+i}\epsilon}
\]
L'équation (46) conduit donc à la relation
\[
\tag{47}\mc{G}_+(\mt{E})=\lim_{\epsilon\to 0_+}\ \frac{\mt{i}\hbar}{\mt{E-H+i}\epsilon}
\]
L'opérateur H ayant ses valeurs propres réelles, E - H n'a pas d'inverse
lorsque E est égal à l'une de ces valeurs propres. La formule (47) nous
montre donc que de façon analogue à la méthode employée plus haut, le
% 85
\ul{propagateur} s'obtient en déplaçant d'une façon infiniment petite les valeurs
propres du hamiltonien H dans le plan complexe. z étant la variable complexe,
nous appelons \ul{résolvante} l'opérateur $\mt{G(z)}=\frac{1}{\mt{z-H}}$.
(47) s'écrit alors $\mc{G}_+(\mt{E})=\mt{i}\hbar\lim_{\epsilon\to 0_+}\mt{G(E+i}\epsilon)$ et
(45) devient :
\[
\tag{48}\mc{K}_+(\tau)=\frac{\mt{i}}{2\pi}\ \lim_{\epsilon\to 0_+}\int\mt{e}^{-\frac{\mt{iE}\tau}{\hbar}}\ \mt{G(E+i}\epsilon)\ \mt{dE}
\]
\[
\ \ \ =\frac{\mt{i}}{2\pi}\ \int_{\mt{C}_+}\mt{e}^{-\frac{\mt{iz}\tau}{\hbar}}\ \frac{1}{\mt{z - H}}\ \mt{dz}
\]
C$+$ représentant la droite parallèle à l'axe réel déplacée d'une quantité
infiniment petite dans le demi-plan supérieur :\begin{center}
\begin{tikzpicture}
\draw [->] (-3.5,0.3) -- (4,0.3) node [above]{$C_+$};
\draw [dashed] (-3.5,0) -- (0.1,0); \draw [->] (0.1,0) -- (4,0);
\foreach \s in {1,2,...,6}
{
\draw (-\s*\s/12-0.1,0.1) -- (-\s*\s/12+0.1,-0.1);
\draw (-\s*\s/12+0.1,0.1) -- (-\s*\s/12-0.1,-0.1);
}
\end{tikzpicture} \end{center}

On peut définir de même l'opérateur fonction de Green retardé $\mc{K}_-(\tau)$ par
l'équation (4O-a) et la relation : $\mc{K}_-(\tau)=0$ \ \ si \ \ $(\tau)>0$ . On montre alors aisément
\[
\tag{49}\mc{K}_-(\tau)=-\mt{U}(\tau)\ \theta(-\tau)
\]

% 86
Le propagateur retardé est alors
\[
\mc{G}_-(\mt{E})=-\int_{-\infty}^0 \mt{e}^{\frac{\mt{i(E-H)}\tau}{\hbar}}\ \mt{d}\tau
\]
\[
=\lim_{\epsilon\to 0_+}\ \frac{\mt{i}\hbar}{\mt{E-H-i}\epsilon}
\]

et :
\[
\tag{50}\mc{K}_-(\tau)=\frac{\mt{i}}{2\pi}\ \int_{\mt{C}_-}\mt{e}^{-\frac{\mt{iz}\tau}{\hbar}}\ \frac{1}{\mt{z - H}}\ \mt{dz}
\]
C$_-$ représentant la droite parallèle à l'axe réel déplacée d'une quantité
infiniment petite dans le demi-plan inférieur :\begin{center}
\begin{tikzpicture}
\draw [->] (-3.5,-0.3) -- (4,-0.3) node [below]{$C_-$};
\draw [dashed] (-3.5,0) -- (0.1,0); \draw [->] (0.1,0) -- (4,0);
\foreach \s in {1,2,...,6}
{
\draw (-\s*\s/12-0.1,0.1) -- (-\s*\s/12+0.1,-0.1);
\draw (-\s*\s/12+0.1,0.1) -- (-\s*\s/12-0.1,-0.1);
}
\end{tikzpicture} \end{center}

De (39) et (49) on déduit U$(\tau)=\mt{K}_+(\tau)-\mt{K}_-(\tau)$
et de (48) et (50) :
\[
\tag{51}\mt{U}(\tau)=\frac{\mt{i}}{2\pi}\ \int_\mt{C}\mt{e}^{-\frac{\mt{iz}\tau}{\hbar}}\ \frac{1}{\mt{z - H}}\ \mt{dz}
\]
avec C = C$_+-$C$_-$ :\begin{center}
\begin{tikzpicture}
\draw (-3.5,0.3) -- (4,0.3);
\draw [thick] (1,0.45) -- (1.2,0.3);\draw [thick] (1,0.15) -- (1.2,0.3);
\draw (-3.5,-0.3) -- (4,-0.3);
\draw [thick] (1.2,-0.45) -- (1,-0.3);\draw [thick] (1.2,-0.15) -- (1,-0.3);
\draw [dashed] (-3.5,0) -- (0.1,0); \draw [->] (0.1,0) -- (4,0);
\foreach \s in {1,2,...,6}
{
\draw (-\s*\s/12-0.1,0.1) -- (-\s*\s/12+0.1,-0.1);
\draw (-\s*\s/12+0.1,0.1) -- (-\s*\s/12-0.1,-0.1);
}
\end{tikzpicture} \end{center}

% 87
Reprenons enfin la formule (42) qui s'écrit, dans le cas d'un
hamiltonien H et une perturbation V \ul{indépendants du temps} :
\[
\tag{52}\mc{K}_+(\mt{t}_2-\mt{t}_1)=\mc{K}_{0+}(\mt{t}_2-\mt{t}_1) + \frac{1}{\mt{i}\hbar}\int
\mc{K}_{0+}(\mt{t}_2-\mt{t}_3)\ \mt{V}\ \mc{K}_+(\mt{t}_3-\mt{t}_1)\ \mt{dt}_3
\]

L'intégrale de (52) est un produit de convolution qui devient un produit
ordinaire par transformation de Fourier :
\[
\tag{53}\mc{G}_+(\mt{E})=\mc{G}_{0+}(\mt{E})+\mc{G}_{0+}(\mt{E})\ \mt{V}\ \mc{G}_+(\mt{E})
\]

(53), compte tenu de (47), s'écrit encore :
\[
\tag{54}\frac{1}{\mt{E-H+i}\epsilon}=\frac{1}{\mt{E-H}_0\mt{+i}\epsilon}+\frac{1}{\mt{E-H}_0\mt{+i}\epsilon}\ \mt{V}\ \frac{1}{\mt{E-H+i}\epsilon}
\]

La relation (54) apparaît ainsi comme \ul{une conséquence de (52)}. On peut
également la démontrer beaucoup plus simplement à partir de la relation
générale entre opérateurs :
\[
\frac{1}{\mt{A}}-\frac{1}{\mt{B}}=\frac{1}{\mt{B}}\ (\mt{B}-\mt{A})\ \frac{1}{\mt{A}}
\]

\part{Application des fonctions de Green à l'étude du spectre du hamiltonien}%APPLICATION DES FONCTIONS DE GREEN A L'ETUDE DU SPECTRE DU HAMILTONTEN
%
\section{Introduction}%
% 88 


Nous avons, dans les chapitres précédents, développé le formalisme de Feynman.
Ceci nous a amené tout naturellement à l'étude de
la fonction de Green de l'équation de Schrödinger et à celle de sa transformée
de Fourier, le propagateur.

Nous avons ainsi fait connaissance avec certains objets mathématiques, d'une
utilisation très courante en physique théorique moderne et qui se révèlent très
importants pour l'étude d'un grand nombre
de problèmes.

Nous allons, dans cette partie du cours, passer en revue un
certain nombre de ces problèmes afin de nous familiariser un peu plus
avec ces objets mathématiques; ces problèmes permettent par ailleurs
d'introduire des notions très importantes qui nous serviront constamment
par la suite.

Le thème général de notre étude sera la recherche des valeurs
propres et des états propres d'un hamiltonien $\mc{H}$ constitué d'une partie
, dont le spectre est supposé connu, et d'une perturbation V. Cela pourra être
le cas d'un système libre placé dans un potentiel extérieur ou de 
deux systèmes couplés par une interaction. Nous nous attacherons, en outre,
à donner une image physique des états propres de $\mc{H}$ en étudiant comment
ils peuvent s'obtenir, dans une approche \ul{temporelle}, à partir de ceux de
$\mc{H}_0$.

Le spectre de $\mc{H}$ présentera en général deux parties : un spectre continu et un
spectre discret. Cette division sera respectée dans le
plan de cette étude qui sera le suivant :

% 89 
1°) \ul{Etude des états stationnaires de collisions} (états propres
du spectre continu) :

Nous étudierons tout d'abord, de façon mathématique, les états
propres du spectre continu (non normés) satisfaisant à certaines conditions
aux limites. Ceci nous conduira à l'équation intégrale de la diffusion, ou
équation de Lippmann-Schwinger.

Nous donnerons ensuite une image physique et temporelle de ces
états, ce qui nous conduira à la matrice S et à la théorie des collisions.
Nous terminerons par quelques méthodes d'approximation.

2°) \ul{Etude des états liés} (états propres du spectre discret) :

Nous verrons l'utilité de la résolvante pour l'étude des états
liés du hamiltonien $\mc{H}$, Ce qui nous donnera une nouvelle théorie des perturbations stationnaires.

3°) \ul{Etude des états instables} :

Nous aborderons enfin le problème des états instables qui, dans
une certaine mesure, est intermédiaire entre celui des états liés et celui
des états de collision. Nous donnerons, à partir de la résolvante une théorie de la durée de vie.

% 90
\chapter{ETATS STATIONNAIRES DE COLLISION}% 1
\section{Introduction}% A
\subsection{Description du système}% 1°) 

- Le système que nous étudierons sera celui d'une particule
dans un potentiel V($\vec{\mt{r}}$). Son hamiltonien s'écrira donc
\begin{center}
H=T+v($\vec{\mt{r}}$)
\end{center}
T représente le hamiltonien d'énergie cinétique égal à p$^2$/2m ;
V($\vec{\mt{r}}$) un potentiel décroissant avec la distance plus vite que $\frac{1}{|\vec{\mt{r}}\,|}$,
non nécessairement central.

- Nous pouvons ramener au problème précédent, celui de deux
particules A et B en interaction. Le hamiltonien s'écrit alors
\[
\mt{H} = \frac{\mt{P}_\mt{A}^2}{2\mt{m}_\mt{A}} + \frac{\mt{P}_\mt{B}^2}{2\mt{m}_\mt{B}} + \mt{V(A,B)}
\]
Le système présente l'invariance de translation et l'impulsion totale du
mouvement est donc constante.

Nous pouvons séparer le mouvement du centre de masse, rectiligne et uniforme,
du mouvement relatif dont le hamiltonien s'écrit
\begin{center}
H = T + V \hspace{2cm} avec \hspace{1cm} T = P$^2$/2m
\end{center}

\begin{minipage}[c]{.35\linewidth}
\begin{center}
en posant :
\end{center}
\end{minipage}
\hfill
\begin{minipage}[c]{.55\linewidth}
\[
\begin{array}{rl}
\mt{m} & = \frac{\mt{m}_\mt{A}\mt{m}_\mt{B}}{\mt{m}_\mt{A}+\mt{m}_\mt{B}} \hspace{1cm} \mt{(masse réduite)}\\
\vec{\mt{P}} & = \frac{\mt{m}_\mt{B}\vec{\mt{P}}_\mt{A}-\mt{m}_\mt{A}\vec{\mt{P}}_\mt{B}}{\mt{m}_\mt{A}+\mt{m}_\mt{B}}\\
\vec{\mt{r}} & =\vec{\mt{r}}_\mt{A}-\vec{\mt{r}}_\mt{B}
\end{array}
\]
\end{minipage}

Les particules A et B peuvent de plus être douées de degrés
de liberté interne (spin).

- On pourrait envisager le cas plus complexe où A et B sont
eux-mêmes des états liés de plusieurs particules, par exemple des atomes.
Le hamiltonien relatif s'écrirait alors
\[
\mc{H} = \mt{T}_\mt{A,B} + \mt{h}_\mt{A} + \mt{h}_\mt{B} + \mt{V}_\mt{AB}
\]
où T$_\mt{A,B}$ représente l'énergie cinétique relative, h$_\mt{A}$ et h$_\mt{B}$ les énergies internes des
atomes A et B, V$_\mt{AB}$ l'interaction.

Un tel cas pourrait conduire à des collisions inélastiques
(où les atomes A et B voient leur énergie interne modifiée) ou à des collisions
de réarrangement où la composition de A et B peut être elle-même
modifiée).

Nous n'étudierons pas ces cas plus complexes pour ne pas
alourdir trop l'exposé. En fait, les méthodes que nous allons étudier
s'appliquent parfaitement à ces problèmes.

Nous ne soulèverons pas non plus au début les difficultés
liées à l'indiscernabilité des particules et nous considérerons les collisions
entre deux particules différentes.

\subsection{Le spectre de T et de H}% 2°) 

\subsubsection{Spectre de T}% a)  :
C'est un spectre continu de 0 à l'infini que l'on peut écrire
E$_\mt{i}=\frac{\hbar^2\mt{k}_\mt{i}^2}{2\mt{m}}$. A chaque valeur de E$_\mt{i}$
sont associés les états propres $|\ \Phi_{\vec{\mt{k}}_\mt{i}}>$
( avec $|\ \vec{\mt{k}}_\mt{i}\ |=\mt{k}_\mt{i}$ ) qui dans la représentation $\vec{\mt{r}}$
s'écrivent $<\vec{\mt{r}}\ |\ \Phi_{\vec{\mt{k}}_\mt{i}}>=(\frac{1}{2\pi})^{3/2}\mt{ e}^{\vec{\mt{k}}_\mt{i}.\vec{\mt{r}}}$.
( Nous omettrons parfois dans la suite le facteur de normalisation $(\frac{1}{2\pi})^{3/2}$ )

 
% 92 
Le système $|\ \phi_{\vec{\mt{k}}_\mt{i}}>$, lorsqu'on a spécifié éventuellement les
états de spin des deux particules $\mu_\mt{A}$ et $\mu_\mt{B}$ constitue un système complet
de fonctions propres. (De façon rigoureuse, il faudrait écrire
$|\ \vec{\mt{k}}_\mt{i},\,\mu_\mt{A},\,\mu_\mt{B}>$.
Pour simplifier, nous écrirons les états propres $|\ \phi_{\vec{\mt{k}}_\mt{i}}>$
ou même $|\ \phi_\mt{i}>$ tant qu'il n'y aura pas d'ambiguité).

Nous ne soulèverons pas ici les difficultés liées au spectre
continu (les fonctions d'onde représentant $|\ \phi_\mt{i}>$ ne sont pas de carré
sommable et seuls les paquets d'onde ont un sens physique).

\subsubsection{Spectre de H}% b)  :
Nous allons voir qu'à chaque état propre $|\ \phi_\mt{i}>$ de T, de valeur propre
$\mt{E}_\mt{i}=\frac{\hbar^2\mt{k}_\mt{i}^2}{2\mt{m}}$, on peut associer au moins deux états propres
$|\ \psi_\mt{i}^\pm>$ du hamiltonien H, de \ul{même valeur} propre $\mt{E}_\mt{i}$

Le spectre de H comprend donc au moins une partie continue qui
va de zéro à l'infini.

On sait d'autre part que si V est attractif, il y a en plus un
spectre discret de valeurs négatives, les états correspondants étant les
états liés : le spectre de H présente donc en général l'aspect suivant :\begin{center}
\begin{tikzpicture}
\draw [thick] (0.1,-0.13) -- (0.1,0.13) node [above]{$0$}; \draw [thick] [->] (0.1,0) -- (4,0) node [above]{$+\infty$};
\foreach \s in {1,2,...,8}
{
\draw (-\s*\s/12-0.1,0.1) -- (-\s*\s/12+0.1,-0.1);
\draw (-\s*\s/12+0.1,0.1) -- (-\s*\s/12-0.1,-0.1);
}
\end{tikzpicture} \end{center}

Notons que les $|\ \psi_\mt{i}^\pm>$, qui constituent ce qu'on appellera
les "états stationnaires de collision", ne forment donc certainement pas
en général un système complet, car ils ne représentent pas la partie
discrète du spectre.

% 93 
\subsection{Plan suivi}% 3°) 
- Nous allons tout d'abord, de façon purement mathématique, rechercher les états
propres de H ayant un comportement asymptotique donné,
que nous appellerons $|\ \psi_\mt{i}^+>$ et $|\ \psi_\mt{i}^->$ (états stationnaires de collision).
Pour cela, nous utiliserons les fonctions de Green. Nous serons conduits
à l'équation intégrale de la diffusion (\S B).

- Nous verrons ensuite à quoi correspondent physiquement les
états mathématiques ainsi introduits. Nous verrons notamment que si on
établit \ul{adiabatiquement} la perturbation V et si on appelle U l'opérateur
d'évolution correspondant à cet établissement adiabatique, alors
\[
|\ \psi_\mt{i}^+>=\lim_{\mt{t}_0\to-\infty}\mt{ U(0, t}_0\ |\ \phi_\mt{i}>
\]
\[
|\ \psi_\mt{i}^->=\lim_{\mt{t}_0\to+\infty}\mt{ U(0, t}_0\ |\ \phi_\mt{i}>
\]
ce qui justifiera les définitions d'onde entrante et d'onde sortante données à
$|\ \psi_\mt{i}^+>$ et à $|\ \psi_\mt{i}^->$

- Nous serons alors amenés à étudier la quantité 
\[
\mt{S}=\lim_{ \mt{t}_2\to-\infty,\ \mt{t}_1\to+\infty}
\mt{U}(\mt{t}_1,\mt{t}_2) \hspace{3cm} \mt{(matrice S)}
\]

- Nous verrons enfin l'utilité des états stationnaires de collision
pour le calcul des sections efficaces de collision.

\section{Approche mathématique}% B
\subsection{Problème mathématique}% 1°) 
Les états propres du spectre continu de $\mc{H}$ sont donnés par
l'équation de Schrödinger indépendante du temps :
\[
\tag{1}-\frac{\hbar^2}{2\mt{m}}\ \Delta\psi(\vec{\mt{r}}\,) + \mt{V}(\vec{\mt{r}}\,)\ \psi(\vec{\mt{r}}\,)
 = \mt{E}\ \psi(\vec{\mt{r}}\,) \hspace{3cm} (\mt{E}>0)
\]
% 94
Posons \hspace{1.5cm} E $=\frac{\hbar^2\mt{k}^2}{2\mt{m}}$ \hspace{1.5cm}
V$(\vec{\mt{r}}\,)=\frac{\hbar^2}{2\mt{m}}$ U(r) \hspace{1.5cm}
(1) devient alors
\[
\tag{2}\big[-\Delta + \mt{U}(\vec{\mt{r}}\,)\ \big]\ \psi(\vec{\mt{r}}\,)=\mt{k}^2\ \psi(\vec{\mt{r}}\,)
\]
Nous cherchons les solutions de (2) $\psi_{\vec{\mt{k}}_\mt{i}}^\pm(\vec{\mt{r}}\,)$ qui
lorsque | $\vec{\mt{r}}$ | tend vers l'infini ont pour forme asymptotique
\[
\Lambda_\pm(\vec{\mt{r}}\,)=\mt{e}^{i\vec{\mt{k}}_\mt{i}\vec{\mt{r}}}
+ \mt{f}_\pm(\vec{\mt{k}}_\mt{i},\,\theta,\,\phi)\ \frac{\mt{e}^{\pm i \mt{k}_\mt{i}\mt{r}}}{r}
\hspace{2cm} \mt{avec} \hspace{1cm} \mt{k}_\mt{i}=|\,\vec{\mt{k}}_\mt{i}\,| \ \ ;\ \ \mt{r}=|\,\vec{\mt{r}}\ |
\]

ce qui signifie que l'on a, lorsque | $\vec{\mt{r}}$ | tend vers l'infini :

\[
\tag{3}\psi_{\vec{\mt{k}}_\mt{i}}^\pm(\vec{\mt{r}}\,)-\Lambda_\pm(\vec{\mt{r}}\,)=\mt{O}\left(\frac{1}{\mt{r}}\right)
\]
Nous appellerons ces solutions ondes stationnaires de collisions entrantes
et sortantes.

Le fait qu'il existe des solutions asymptotiques de cette forme pour un
potentiel U($\vec{\mt{r}}\,$) quelconque \ul{n'est pas évident}. Nous pouvons ici
établir une condition \ul{nécessaire} de son existence : \ul{il faut que U$(\vec{\mt{r}}\,)$ décroisse plus vite que 1/r}.

En effet, aussi bien e$^{\mt{i}\vec{\mt{k}}_\mt{i}\vec{\mt{r}}}$ que
$\frac{\mt{e}^{\mt{i k}_\mt{i}\mt{r}}}{\mt{r}}$ sont solutions de
l'équation "non perturbée" $(\Delta+\mt{k}_\mt{i}^2)$ f $=$ 0.

% 95

Il en résulte que
\[
\begin{array}{rl}
\big[\Delta+\mt{k}_\mt{i}^2-\mt{U}(\vec{\mt{r}}\,)\big] \big[\mt{A}_\pm(\vec{\mt{r}}\,)-\psi_{\vec{\mt{k}}_\mt{i}}^\pm(\vec{\mt{r}}\,)\big] & = \big[\Delta+\mt{k}_\mt{i}^2-\mt{U}(\mt{r}\,)\big]\mt{A}_\pm(\vec{\mt{r}}\,) \\
 & =-\mt{U}(\vec{\mt{r}}\,)\mt{A}_\pm(\vec{\mt{r}}\,) + \frac{\mt{e}^{\pm\mt{i k}_\mt{i}\mt{r}}}{r}\ \Delta\mt{f}_\pm(\vec{\mt{k}}_\mt{i},\,\theta,\,\phi)
\end{array}
\]

Or $\Delta\mt{f}_\pm(\vec{\mt{k}}_\mt{i},\,\theta,\,\phi)$ décroit en 1/r$^2$ lorsque r $\to\infty$, quel que soit
f$_\pm(\vec{\mt{k}}_\mt{i},\,\theta,\,\phi)$ (d'après la forme du laplacien en coordonnées sphériques).
On en déduit donc (à moins que U$(\vec{\mt{r}}\,)$ décroisse plus vite que 1/r$^3$) que
la partie principale de $\big[\Delta+\mt{k}_\mt{i}^2-\mt{U}(\vec{\mt{r}}\,)\big] \big[\mt{A}_\pm(\vec{\mt{r}}\,)-\psi_{\vec{\mt{k}}_\mt{i}}^\pm(\vec{\mt{r}}\,)\big]$ est $-\mt{U}(\vec{\mt{r}}\,)\mt{e}^{i\vec{\mt{k}}_\mt{i}\vec{\mt{r}}}$.

Or, d'après la définition (3), la partie principale de
$\mt{A}_\pm(\vec{\mt{r}}\,)-\psi_{\vec{\mt{k}}_\mt{i}}^\pm(\vec{\mt{r}}\,)$
 est en 1/r$^\alpha$ avec $\alpha>1$. La partie principale de
$\big[\Delta+\mt{k}_\mt{i}^2-\mt{U}(\vec{\mt{r}}\,)\big] \big[\mt{A}_\pm(\vec{\mt{r}}\,)-\psi_{\vec{\mt{k}}_\mt{i}}^\pm(\vec{\mt{r}}\,)\big]$
doit donc également être en 1/r$^\alpha$
(à cause du terme en $\mt{k}_\mt{i}^2$, les termes en $\Delta$ et U$(\vec{\mt{r}}\,)$ conduisent à des
ordres supérieurs).

\ul{Il est donc nécessaire} que U$(\vec{\mt{r}}\,)$ décroisse en 1/r$^\alpha$ avec
$\ul{\alpha>1}$ . Nous imposerons cette condition à U$(\vec{\mt{r}}\,)$ dans la suite de cette
étude. ({\footnotesize 
Nous excluons donc le cas du potentiel coulombien en 1/r. On sait
dans ce cas traiter le problème rigoureusement (cf. Messiah, Mécanique Quantique, t. I, page 357).})

Nous montrons par le même raisonnement que $\psi_{\vec{\mt{k}}_\mt{i}}^\pm(\vec{\mt{r}}\,)$, s'il
existe, est solution de (2) avec k = k$_i$, et représente donc un état propre H d'énergie ($\hbar^2$k$_i^2$)/2m.

\subsection{Fonction de Green de $\Delta+\mt{k}_\mt{i}^2$}% 2°) 
Pour déterminer, si elles existent, les fonctions $\psi_{\vec{\mt{k}}_\mt{i}}^\pm(\vec{\mt{r}}\,)$,
solutions de (2), admettant $\mt{A}_\pm(\vec{\mt{r}}\,)$ pour forme  asymptotique, 1a méthode
de choix est celle des fonctions de Green.

% 96
Il suffit en effet d'écrire l'équation (2) formellement
sous la forme inhomogène :
\[
\tag{4}(\Delta+\mt{k}_\mt{i}^2)\ \psi(\vec{\mt{r}}\,)=\mt{U}(\vec{\mt{r}}\,)\ \psi(\vec{\mt{r}}\,)
\]
et de considérer U$(\vec{\mt{r}}\,)\ \psi(\vec{\mt{r}}\,)$ comme une "source".

Nous sommes donc amenés à chercher la fonction de Green
G$_{\vec{\mt{k}}_\mt{i}}(\vec{\mt{r}}-\vec{\mt{r}}\,')$
de l'opérateur $\Delta+\mt{k}_\mt{i}^2$ vérifiant les "bonnes conditions"
aux limites pour r$\to\infty$.

L'équation vérifiée par G$_{\vec{\mt{k}}_\mt{i}}(\vec{\mt{r}}-\vec{\mt{r}}\,')$ est
\[
\tag{5}(\Delta+\mt{k}_\mt{i}^2)\mt{G}_{\vec{\mt{k}}_\mt{i}}(\vec{\mt{r}}-\vec{\mt{r}}\,')=\delta(\vec{\mt{r}}-\vec{\mt{r}}\,')
\]

Nous ne préciserons pas à ce niveau la forme des conditions aux limites sur G.
Nous nous contenterons de choisir a priori parmi
les solutions deux fonctions G$_+$ et G$_-$ et nous vérifierons qu'elles conduisent
bien, par résolution de (4), aux fonctions $\psi_{\vec{\mt{k}}_\mt{i}}^\pm(\vec{\mt{r}}\,)$.

Pour résoudre (5), nous allons, selon une méthode qui nous
est maintenant habituelle, utiliser la transformée de Fourier : Posons
\[
\tag{6}\mt{G}_{\vec{\mt{k}}_\mt{i}}(\vec{\mt{r}}-\vec{\mt{r}}\,')=\left(\frac{1}{2\pi}\right)^3
\int \mt{G}_{\vec{\mt{k}}_\mt{i}}(\vec{\chi}\,)\mt{e}^{i\vec{\chi}(\vec{\mt{r}}-\vec{\mt{r}}\,')}\mt{d}^3\chi
\]
La transformée de Fourier de (3) conduit à :
\[
\tag{7}(-\chi^2+\mt{k}_\mt{i}^2)\ \mt{G}_{\vec{\mt{k}}_\mt{i}}(\vec{\chi})=1
\]

L'équation (7) est analogue à l'équation (27) du chapitre 5
. Sa solution générale s'écrit :
\[
\mt{G}_{\vec{\mt{k}}_\mt{i}}(\vec{\chi})=\frac{1}{2\mt{k}_\mt{i}}\left\{\mc{P}\left[\frac{1}{\chi-\mt{k}_\mt{i}}\right]-\lambda\delta(\chi-\mt{k}_\mt{i})
-\mc{P}\left[\frac{1}{\chi+\mt{k}_\mt{i}}\right]+\lambda'\delta(\chi+\mt{k}_\mt{i})\right\}
\]
Envisageons a priori les solutions
\[
\left\{
\begin{array}{rl}
\lambda & =-i\pi \\
\lambda' & =+i\pi
\end{array}
\right.
\hspace{2cm}\mt{et}\hspace{1cm}
\left\{
\begin{array}{rl}
\lambda & =+i\pi \\
\lambda' & =-i\pi
\end{array}
\right.
\]
% 97
Elles donnent
\[
\mt{G}_{\vec{\mt{k}}_\mt{i}}^\pm(\vec{\chi})=\lim_{\epsilon'\to 0_+}\frac{-1}{2\mt{k}_\mt{i}}\left[\frac{1}{\chi-\mt{k}_\mt{i}\mp\mt{i}\epsilon'}-\frac{1}{\chi+\mt{k}_\mt{i}\pm\mt{i}\epsilon'}\right]
\]
\[
=\lim_{\epsilon'\to 0_+}\frac{1}{\mt{k}_\mt{i}^2-\chi^2\pm 2\mt{i}\mt{k}_\mt{i}\epsilon'}
\]
Soit, en faisant le changement de variable $\epsilon$ = $2\mt{k}_\mt{i}\epsilon'$ :
\[
\tag{8}\mt{G}_{\vec{\mt{k}}_\mt{i}}^\pm(\vec{\chi})=\lim_{\epsilon\to 0_+}\frac{1}{\mt{k}_\mt{i}^2-\chi^2\pm \mt{i}\epsilon}
\]
Ce sont les deux fonctions G$_{\vec{\mt{k}}_\mt{i}}^\pm(\vec{\chi})$ que nous allons choisir a priori pour
notre problème : elles correspondent aux contours d'intégration des figures a) et b).

\vspace{0.3cm}
\begin{minipage}[c]{.45\linewidth}
\begin{center} \begin{tikzpicture}
\draw [very thick] (-2.25,0) -- (-1.45,0); \draw [very thick] (-0.75,0) -- (0.75,0); \draw [very thick] (1.45,0) -- (2.25,0);
\draw (-1,-0.05) node [below]{-k$_\mt{i}$}; \draw (1.2,0.05) node [above]{+k$_\mt{i}$};
\draw (-1.2,-0.1) -- (-1,0.1); \draw (-1.2,0.1) -- (-1,-0.1); \draw [very thick] (-1.45,0)  arc (180:0:0.35);
\draw (1,-0.1) -- (1.2,0.1); \draw (1,0.1) -- (1.2,-0.1); \draw [very thick] (0.75,0)  arc (-180:0:0.35);
\end{tikzpicture}

(a) : G$_{\vec{\mt{k}}_\mt{i}}^+(\chi)$
\end{center}
\end{minipage}
\hfill
\begin{minipage}[c]{.45\linewidth}
\begin{center} \begin{tikzpicture}
\draw [very thick] (-2.25,0) -- (-1.45,0); \draw [very thick] (-0.75,0) -- (0.75,0); \draw [very thick] (1.45,0) -- (2.25,0);
\draw (-1,0.05) node [above]{$-$k$_\mt{i}$}; \draw (1.2,-0.05) node [below]{$+$k$_\mt{i}$};
\draw (-1.2,-0.1) -- (-1,0.1); \draw (-1.2,0.1) -- (-1,-0.1); \draw [very thick] (-1.45,0)  arc (-180:0:0.35);
\draw (1,-0.1) -- (1.2,0.1); \draw (1,0.1) -- (1.2,-0.1); \draw [very thick] (0.75,0)  arc (180:0:0.35);
\end{tikzpicture}

(b) : G$_{\vec{\mt{k}}_\mt{i}}^-(\chi)$
\end{center}
\end{minipage}
\vspace{0.3cm}

Nous constatons que G$^\pm$ ne sont fonctions que de $\chi^2$ et
nous allons les écrire G$^\pm(\chi^2)$.

Pour obtenir G$_{\vec{\mt{k}}_\mt{i}}^\pm(\vec{\mt{r}}-\vec{\mt{r}'})$, il suffit d'utiliser la formule (6) : en passant en coordonnées
cylindriques avec $\vec{\mt{r}}-\vec{\mt{r}'}$ pour axe polaire et en posant $|\vec{\mt{r}}-\vec{\mt{r}'}|=\rho$, (6) devient
% 98
\[
\mt{G}_{\vec{\mt{k}}_\mt{i}}^\pm(\vec{\mt{r}}-\vec{\mt{r}}\;')=\frac{1}{(2\pi)^3}\int_0^\infty 2\pi\ \chi^2\mt{ d}\chi\mt{ G}^\pm(\chi^2)\int_0^\pi \mt{e}^{\mt{i}\chi\rho\cos\theta}\sin\theta\mt{ d}\theta
\]
et après intégration sur $\theta$ :
\[
\mt{G}_{\vec{\mt{k}}_\mt{i}}^\pm(\vec{\mt{r}}-\vec{\mt{r}}\;')=\frac{1}{\mt{i}\rho}\frac{1}{(2\pi)^2}\int_0^\infty\chi\mt{d}\chi(\mt{ e}^{\mt{i}\chi\rho}-\mt{e}^{-\mt{i}\chi\rho}) \mt{ G}^\pm(\chi^2)
\]
\[
=\frac{1}{\mt{i}\rho}\frac{1}{(2\pi)^2}\int_{-\infty}^\infty\chi\mt{d}\chi\mt{ e}^{\mt{i}\chi\rho}\mt{ G}^\pm(\chi^2)
\]
\[
=\frac{1}{\mt{i}\rho}\frac{1}{(2\pi)^2}\int_{-\infty}^\infty\chi\mt{d}\chi\mt{ e}^{\mt{i}\chi\rho}\ \frac{1}{\mt{k}_\mt{i}^2-\chi^2\pm \mt{i}\epsilon}
\]

Pour calculer cette intégrale par la méthode des résidus, il faut fermer
le contour d'intégration par un demi-grand cercle dans le plan des $\chi$ tel
que | e$^{\mt{i}\chi\rho}$ | $=$ e$^{-\mc{I}\mt{m}\chi\rho}$ tende vers zéro. Il faut donc fermer le contour
dans le demi-plan supérieur

\vspace{0.3cm}
\begin{minipage}[c]{.45\linewidth}
\begin{center} \begin{tikzpicture}
\draw [very thick] (-2.25,0) -- (-1.45,0); \draw [very thick] (-0.75,0) -- (0.75,0); \draw [very thick] (1.45,0) -- (2.25,0);
\draw (-1,-0.05) node [below]{$-$k$_\mt{i}$}; \draw (1.2,0.05) node [above]{$+$k$_\mt{i}$};
\draw (-1.2,-0.1) -- (-1,0.1); \draw (-1.2,0.1) -- (-1,-0.1); \draw [very thick] (-1.45,0)  arc (180:0:0.35);
\draw (1,-0.1) -- (1.2,0.1); \draw (1,0.1) -- (1.2,-0.1); \draw [very thick] (0.75,0)  arc (-180:0:0.35);
 \draw [very thick] (-2.25,0)  arc (180:0:2.25);
 \draw [very thick] (1.25,1.7) -- (1.1,1.95) -- (1.4,1.95);
\end{tikzpicture}

(a) : G$_{\vec{\mt{k}}_\mt{i}}^+(\chi)$
\end{center}
\end{minipage}
\hfill
\begin{minipage}[c]{.45\linewidth}
\begin{center} \begin{tikzpicture}
\draw [very thick] (-2.25,0) -- (-1.45,0); \draw [very thick] (-0.75,0) -- (0.75,0); \draw [very thick] (1.45,0) -- (2.25,0);
\draw (-1,0.05) node [above]{-k$_\mt{i}$}; \draw (1.2,-0.05) node [below]{+k$_\mt{i}$};
\draw (-1.2,-0.1) -- (-1,0.1); \draw (-1.2,0.1) -- (-1,-0.1); \draw [very thick] (-1.45,0)  arc (-180:0:0.35);
\draw (1,-0.1) -- (1.2,0.1); \draw (1,0.1) -- (1.2,-0.1); \draw [very thick] (0.75,0)  arc (180:0:0.35);
 \draw [very thick] (-2.25,0)  arc (180:0:2.25);
 \draw [very thick] (1.25,1.7) -- (1.1,1.95) -- (1.4,1.95);
\end{tikzpicture}

(b) : G$_{\vec{\mt{k}}_\mt{i}}^-(\chi)$
\end{center}
\end{minipage}
\vspace{0.3cm}

et on a
\[
\mt{G}_{\vec{\mt{k}}_\mt{i}}^\pm(\vec{\mt{r}}-\vec{\mt{r}}\;')=2\mt{i}\pi\times\frac{1}{\mt{i}\rho}\ \frac{1}{(2\pi)^2}\mt{Résidu}\left[\frac{\chi\mt{ e}^{\mt{i}\chi\rho}}{\mt{k}_\mt{i}^2-\chi^2\pm \mt{i}\epsilon}\right]
\]
 
Soit finalement
\[
\tag{9}\mt{G}_{\vec{\mt{k}}_\mt{i}}^\pm(\vec{\mt{r}}-\vec{\mt{r}}\;')=-\frac{1}{4\pi}\ \frac{\mt{ e}^{\pm\mt{i k}_\mt{i}|\vec{\mt{r}}-\vec{\mt{r}}\;'|}}{|\vec{\mt{r}}-\vec{\mt{r}}\;'|}
\]
G$^\pm$ sont donc les solutions à onde sortante (ou entrante)

% 99
\subsection{Equation intégrale de la diffusion}%3°) 
Reprenons maintenant l'équation (4) et traitons le second
membre U$(\vec{\mt{r}})$ $\psi(\vec{\mt{r}})$ comme une source $\rho(\vec{\mt{r}})$.

Nous savons que $\mt{G}_{\vec{\mt{k}}_\mt{i}}^\pm(\vec{\mt{r}}-\vec{\mt{r}}\;')$ permet de construire les solutions
\[
\tag{10}\psi(\vec{\mt{r}})=\int\mt{G}_{\vec{\mt{k}}_\mt{i}}^\pm(\vec{\mt{r}}-\vec{\mt{r}}\;')\ \rho(\vec{\mt{r}}\;')\mt{ d}^3\vec{\mt{r}}\;'
\]
\[
=\int\mt{G}_{\vec{\mt{k}}_\mt{i}}^\pm(\vec{\mt{r}}-\vec{\mt{r}}\;')\mt{ U}(\vec{\mt{r}}\;')\ \psi(\vec{\mt{r}}\;')\ \mt{ d}^3\vec{\mt{r}}\;'
\]

L'équation (10) est une équation intégrale qui remplace
l'équation (4) : $\psi(\vec{\mt{r}})$ satisfaisant (10), satisfait nécessairement (4).
Mais on peut ajouter à $\psi(\vec{\mt{r}})$ une solution de l'équation "sans second
membre" $(\Delta+\mt{k}_\mt{i}^2)\ \phi(\vec{\mt{r}})=0$, par exemple e$^{\mt{i}\vec{\mt{k}}_\mt{i}.\vec{\mt{r}}}$.

L'équation intégrale (10) devient alors
\[
\tag{11}\psi(\vec{\mt{r}})=\mt{e}^{\mt{i}\vec{\mt{k}}_\mt{i}.\vec{\mt{r}}}+\int\mt{G}_{\vec{\mt{k}}_\mt{i}}^\pm(\vec{\mt{r}}-\vec{\mt{r}}\;')\mt{ U}(\vec{\mt{r}}\;')\ \psi(\vec{\mt{r}}\;')\ \mt{ d}^3\vec{\mt{r}}\;'
\]

Il est facile de s'assurer que toute solution de l'équation
intégrale (11) satisfait également à l'équation (4).

Nous admettrons sans discussion que si le potentiel U$(\vec{\mt{r}})$,
qui décroît plus vite que 1/r est suffisamment régulier, l'équation intégrale (11)
admet une solution pour G$_{\vec{\mt{k}}_\mt{i}}^+(\vec{\mt{r}}-\vec{\mt{r}}\;')$ et une solution pour
G$_{\vec{\mt{k}}_\mt{i}}^-(\vec{\mt{r}}-\vec{\mt{r}}\;')$. Nous montrons par la suite que ces solutions admettent
A$_\pm(\vec{\mt{r}})$ pour forme asmptotique. Nous avons donc ramené le problème que
nous nous étions posé à la solution de l'équation intégrale :
\[
\tag{12}\psi_{\vec{\mt{k}}_\mt{i}}^\pm(\vec{\mt{r}})=\mt{e}^{\mt{i}\vec{\mt{k}}_\mt{i}.\vec{\mt{r}}}+\int\mt{G}_{\vec{\mt{k}}_\mt{i}}^\pm(\vec{\mt{r}}-\vec{\mt{r}}\;')\mt{ U}(\vec{\mt{r}}\;')\ \psi_{\vec{\mt{k}}_\mt{i}}^\pm(\vec{\mt{r}}\;')\ \mt{ d}^3\vec{\mt{r}}\;'
\]
qui porte le nom d'\ul{équation intégrale de la diffusion}.

% 100


Montrons en effet que les solutions $\psi_{\vec{\mt{k}}_\mt{i}}^\pm(\vec{\mt{r}})$ (dont nous
admettons l'existence) ont le bon comportement asymptotique.

L'intégrale de (12) convergeant, il existe nécessairement
une valeur de | $\vec{\mt{r}}$ | telle que la contribution à l'intégrale des valeurs
de $\vec{\mt{r}}\,'$ telles que | $\vec{\mt{r}}\,'$ | > | $\vec{\mt{r}}$ | soit négligeable : en d'autres termes,
on peut toujours prendre | $\vec{\mt{r}}$ | suffisamment grand pour que | $\vec{\mt{r}}\,'$ | soit
très petit devant | $\vec{\mt{r}}$ |.

On peut alors développer | $\vec{\mt{r}}-\vec{\mt{r}}\,'$ | = r-$\frac{\vec{\mt{r}}}{|\vec{\mt{r}}|}\vec{\mt{r}}\,'$
 ou encore,
en appelant $\vec{\mt{n}}$ le vecteur unitaire dans la direction $\theta$, $\phi$ de $\vec{\mt{r}}$ :
\[
|\vec{\mt{r}}-\vec{\mt{r}}\,'|=\mt{r}-\vec{\mt{n}}.\vec{\mt{r}}\,'
\]

On développe alors :
\[
\mt{G}_{\vec{\mt{k}}_\mt{i}}^\pm(\vec{\mt{r}}-\vec{\mt{r}}\;')\sim-\frac{1}{4\pi}\frac{\mt{e}^{\pm\mt{i k}_\mt{i}\mt{r}}}{\mt{r}}\times\mt{e}^{\mp\mt{i k}_\mt{i}\vec{\mt{n}}.\vec{\mt{r}}\,'}
\]
et (12) conduit à :
\[
\tag{13}\psi_{\vec{\mt{k}}_\mt{i}}^\pm(\vec{\mt{r}})\sim
\mt{e}^{\mt{i}\vec{\mt{k}}_\mt{i}.\vec{\mt{r}}}-
\frac{1}{4\pi}\frac{\mt{e}^{\mt{i k}_\mt{i}\mt{r}}}{\mt{r}}
\int\mt{e}^{\mp\mt{i k}_\mt{i}\;\vec{\mt{n}}.\vec{\mt{r}}\,'}\mt{ U}(\vec{\mt{r}}\;')
\ \psi_{\vec{\mt{k}}_\mt{i}}^\pm(\vec{\mt{r}}\;')\mt{ d}^3\vec{\mt{r}}\;'
\]
L'intégrale au second membre n'est plus qu'une fonction de k$_\mt{i}$ et de $\vec{\mt{n}}$,
c'est-à-dire de $\theta$ et $\phi$ et on peut l'écrire :
\[
\tag{14}\mt{f}_\pm(\vec{\mt{k}}_\mt{i},\theta, \phi)=
-\frac{1}{4\pi}
\int\mt{e}^{\mp\mt{i k}_\mt{i}\;\vec{\mt{n}}.\vec{\mt{r}}\,'}\mt{ U}(\vec{\mt{r}}\;')
\ \psi_{\vec{\mt{k}}_\mt{i}}^\pm(\vec{\mt{r}}\;')\mt{ d}^3\vec{\mt{r}}\;'
\]
et (13), compte tenu de (14), nous donne bien le comportement asymptotique
cherché.

Les solutions de l'équation intégrale (12) sont donc bien
les ondes stationnaires de diffusion cherchées et les fonctions de Green
choisies a priori étaient bien celles qui correspondaient au problème
étudié. De plus, la formule (14) nous donne le calcul explicite, connaissant
% 101
l'état stationnaire de collision, de la fonction f$_\pm(\vec{\mt{k}}_\mt{i},\theta, \phi)$, qui comme
nous le verrons joue un rôle essentiel dans le calcul des sections efficaces.

{\footnotesize
Nous pouvons montrer que l'intégrale (14) est toujours convergente à
l'infini si U$(\vec{\mt{r}}\,)$ décroît plus vite que 1/r. En effet nous pouvons majorer
$\psi_{\vec{\mt{k}}_\mt{i}}^\pm(\vec{\mt{r}})$ par un nombre M et U$(\vec{\mt{r}}\,')$ par 1/r'$^\alpha$ ($\alpha>1$).

En intégrant d'abord sur es angles polaires de $\vec{\mt{r}}\,'$ par rapport
à $\vec{\mt{n}}$, il vient
$
|\mt{f}|<\frac{\mt{M}}{\mt{k}_\mt{i}}\int\frac{\sin \mt{k}_\mt{i} \mt{r}'}{\mt{r}'^{\alpha+1}}\mt{ r}'^2\mt{ dr}'
$
qui converge puisque \ul{$\alpha+1>2$}.
}

Il nous reste maintenant, comme nous l'avons fait dans les
chapitres précédents, à nous affranchir de la représentation r et à donner une forme
intrinsèque à l'équation intégrale (12).

\subsection{Lien entre les fonctions de Green G$_\pm(\vec{\mt{r}}-\vec{\mt{r}}\;')$ et le
propagateur de T}%4°) 

Rappelons que le propagateur avancé ou retardé de T s'écrit
\[
\mc{G}_\pm(\mt{E}_\mt{i})=\lim_{\epsilon\to 0_+}\frac{1}{\mt{E}_\mt{i}-\mt{T}\pm\mt{i}\epsilon}=\lim_{\epsilon\to 0_+}\frac{1}{\frac{\hbar^2\mt{k}_\mt{i}^2}{2\mt{m}}-\mt{T}\pm\mt{i}\epsilon}
\]
{\footnotesize
La définition que nous donnons ici du propagateur diffère d'un terme
en i$\hbar$ de celle du chapitre V de la première partie.
}

La relation (6) peut s'écrire
\[
\tag{15}\mt{G}^\pm_{\vec{\mt{k}}_\mt{i}}(\vec{\mt{r}}-\vec{\mt{r}}\,')=\lim_{\epsilon\to 0_+}\left(\frac{1}{2\pi}\right)^3
\int
\mt{e}^{i\vec{\chi}.\vec{\mt{r}}}
\frac{1}{\mt{k}_\mt{i}^2-\chi^2\pm \mt{i}\epsilon}
\mt{e}^{-i\vec{\chi}.\vec{\mt{r}}\,'}
\mt{d}^3\vec{\chi}
\]

Appelons $|\;\vec{\chi}>$ les états propres de T de valeur propre $\frac{\hbar^2\chi^2}{2\mt{m}}$
 représentant une onde plane de vecteur d'onde $\vec{\chi}$.

% 102
Nous avons les relations
\[
\left\{
\begin{array}{rcl}
 <\vec{\mt{r}}\;|\;\vec{\chi}> & = & (\frac{1}{2\pi})^{3/2}\mt{e}^{i\vec{\chi}.\vec{\mt{r}}} \\
 <\vec{\chi}\;|\;\vec{\mt{r}}\;'> & = & (\frac{1}{2\pi})^{3/2}\mt{e}^{-i\vec{\chi}.\vec{\mt{r}}\,'}
\end{array} \right.
\]

Compte tenu de ces relations et du fait que l'ensemble des $|\;\vec{\chi}>$ est
complet, (15) peut s'écrire :
\[
\mt{G}^\pm_{\vec{\mt{k}}_\mt{i}}(\vec{\mt{r}}-\vec{\mt{r}}\,')=\frac{\hbar^2}{2\mt{m}}
\lim_{\epsilon\to 0_+}\int
<\vec{\mt{r}}\;|\;\vec{\chi}>
<\vec{\chi}\;|\;
\frac{1}{\mt{E}_\mt{i}-\mt{T}\pm\mt{i}\epsilon}
\;|\;\vec{\chi}\,'><\vec{\chi}\,'\;|\;\vec{\mt{r}}\;'>
\mt{d}^3\vec{\chi}\mt{ d}^3\vec{\chi}\,'
\]
\[=\frac{\hbar^2}{2\mt{m}}
\lim_{\epsilon\to 0_+}
<\vec{\mt{r}}\;|\;
\frac{1}{\mt{E}_\mt{i}-\mt{T}\pm\mt{i}\epsilon}
\;|\;\vec{\mt{r}}\;'>
\]
Soit
\[
\tag{16}\mt{G}^\pm_{\vec{\mt{k}}_\mt{i}}(\vec{\mt{r}}-\vec{\mt{r}}\,')=
\frac{\hbar^2}{2\mt{m}}
<\vec{\mt{r}}\;|\;
\mc{G}_\pm(\mt{E}_\mt{i})
\;|\;\vec{\mt{r}}\;'>
\]

\ul{Remarque} :
Ce résultat peut être obtenu très simplement par ailleurs : l'équation
(5) vérifiée par $\mt{G}^\pm_{\vec{\mt{k}}_\mt{i}}(\vec{\mt{r}}-\vec{\mt{r}}\,')$ est la \ul{transformée de Fourier par rapport
au temps} de l'équation vérifiée par la fonction de Green de la particule
libre (équa. 13-a, p. 6) (à un coefficient en $\hbar$ près). D'autre part les
conditions aux limites adoptées sur les "transformées de Fourier totales"
(par rapport au temps et à l'espace) sont les mêmes pour
$\mt{G}^\pm_{\vec{\mt{k}}_\mt{i}}(\vec{\mt{r}}-\vec{\mt{r}}\,')$ et
pour les fonctions de Green retardées et avancées de la particule libre
(comparer les formules (20) p. 67 et (8) de ce chapitre qui sont identiques à des
changements de variable évidents près : la "variable d'énergie"
$\hbar\omega$ est remplacée par k$_i^2$ et la "variable d'impulsion" k par $\chi$). On en déduit que
$\mt{G}^\pm_{\vec{\mt{k}}_\mt{i}}(\vec{\mt{r}}-\vec{\mt{r}}\,')$ est la \ul{transformée
de Fourier par rapport au temps}
de la fonction de Green retardée ou avancée de la particule libre (à un
coefficient en $\hbar$ près). Or nous savons (cf page 81) que la fonction de
%
Green de la particule libre est l'élément de matrice entre <$\vec{\mt{r}}$ | et | $\vec{\mt{r}}$ '>
de l'opérateur fonction de Green. $\mt{G}^\pm_{\vec{\mt{k}}_\mt{i}}(\vec{\mt{r}}-\vec{\mt{r}}\,')$ est donc l'élément de matrice entre <$\vec{\mt{r}}$ | et | $\vec{\mt{r}}$ '> de la
transfornée de Fourier (par rapport au
temps évidemment) de 1‘opérateur fonction de Green, c'est-à-dire du propagateur
avancé ou retardé (au coefficient $\hbar^2$/2m près). C'est ce qu'exprime
la formule (16) ci-dessus.

\subsection{Equation de Lippmann-Schwinger}%5°) 

Compte tenu de (16), l'équation intégrale de la diffusion (12)
peut s'écrire en considérant $\psi_{\vec{\mt{k}}_\mt{i}}^\pm(\vec{\mt{r}}\,)$ comme La fonction d'onde du vecteur
| $\psi_{\vec{\mt{k}}_\mt{i}}^\pm>$
 dans la représentation $|\;\vec{\mt{r}}>$ et en notant que
$<\vec{\mt{r}}\;'\;|$ V $\;|\;\vec{\mt{r}}\;''>=\delta$ ($\vec{\mt{r}}\;'-\vec{\mt{r}}\;'')$ V $(\vec{\mt{r}}\;'\;)$
\[
<\vec{\mt{r}}\;|\;\psi_{\vec{\mt{k}}_\mt{i}}^\pm>=
<\vec{\mt{r}}\;|\;\phi_{\vec{\mt{k}}_\mt{i}}>+
\int<\vec{\mt{r}}\;|\;\mc{G}_\pm(\mt{E}_\mt{i})\;|\;\vec{\mt{r}}\;'>
<\vec{\mt{r}}\;|\mt{ V }|\;\vec{\mt{r}}\;''><\vec{\mt{r}}\;''\;|\;\psi_{\vec{\mt{k}}_\mt{i}}^\pm>
\mt{ d}^3\vec{\mt{r}}\;'\mt{ d}^3\vec{\mt{r}}\;''
\]
ce qui représente la projection sur $|\;\vec{\mt{r}}>$ de l'équation entre vecteurs
d'états :
\[
\tag{17}|\;\psi_\mt{i}^\pm>=|\;\phi_\mt{i}>+\frac{1}{\mt{E}_\mt{i}-\mt{T}\pm\mt{i}\epsilon}
\mt{ V }|\;\psi_\mt{i}^\pm>
\]

(17) représente l'équation de Lippmann-Schwinger de la diffusion.
On peut lui associer l'équation conjuguée entre bras :
\[
\tag{18}<\psi_\mt{i}^\pm\;|=<\phi_\mt{i}\;|+
<\psi_\mt{i}^\pm\;|\mt{ V }\frac{1}{\mt{E}_\mt{i}-\mt{T}\pm\mt{i}\epsilon}
\]
dans laquelle il faut remarquer le changement de signe devant i$\epsilon$.

Nous pouvons donner à l'équation (17) une autre forme, non
intégrale. Partons de la relation générale entre opérateurs :
\[
\frac{1}{\mt{A}}-\frac{1}{\mt{B}}=\frac{1}{\mt{B}}(\mt{B}-\mt{A})\frac{1}{\mt{A}}
\]

% 104
\[
\begin{array}{rcl}
\mt{et posons \hspace{3cm} A} &=& \mt{E}_\mt{i}-\mt{T}\pm\mt{i}\epsilon\\
\mt{B} &=&\mt{E}_\mt{i}-\mt{H}\pm\mt{i}\epsilon \\
\end{array}
\]
Nous avons alors \ B$-$A$=-$V \ et
\[
\frac{1}{\mt{E}_\mt{i}-\mt{T}\pm\mt{i}\epsilon}-\frac{1}{\mt{E}_\mt{i}-\mt{H}\pm\mt{i}\epsilon}=
\frac{1}{\mt{E}_\mt{i}-\mt{H}\pm\mt{i}\epsilon}(-\mt{ V })\frac{1}{\mt{E}_\mt{i}-\mt{T}\pm\mt{i}\epsilon}
\]
Soit
\[
\frac{1}{\mt{E}_\mt{i}-\mt{T}\pm\mt{i}\epsilon}=\frac{1}{\mt{E}_\mt{i}-\mt{H}\pm\mt{i}\epsilon}
\left[1-\mt{ V }\frac{1}{\mt{E}_\mt{i}-\mt{T}\pm\mt{i}\epsilon}\right]
\]
d'où
\[
\frac{1}{\mt{E}_\mt{i}-\mt{T}\pm\mt{i}\epsilon}\mt{ V }=\frac{1}{\mt{E}_\mt{i}-\mt{H}\pm\mt{i}\epsilon}
\left[\mt{ V }-\mt{ V }\frac{1}{\mt{E}_\mt{i}-\mt{T}\pm\mt{i}\epsilon}\mt{ V }\right]
\]
\[
=\frac{1}{\mt{E}_\mt{i}-\mt{H}\pm\mt{i}\epsilon}\mt{ V }
\left[1-\frac{1}{\mt{E}_\mt{i}-\mt{T}\pm\mt{i}\epsilon}\mt{ V }\right]
\]
d'où
\[
\tag{19}\frac{1}{\mt{E}_\mt{i}-\mt{T}\pm\mt{i}\epsilon}\mt{ V }|\;\psi_\mt{i}^\pm>=
\frac{1}{\mt{E}_\mt{i}-\mt{H}\pm\mt{i}\epsilon}\mt{ V}
\left[\;|\;\psi_\mt{i}^\pm>-\frac{1}{\mt{E}_\mt{i}-\mt{T}\pm\mt{i}\epsilon}\mt{ V }|\;\psi_\mt{i}^\pm>\right]
\]
Or le terme entre crochets dans le membre de droite n'est autre que
$|\;\phi_\mt{i}>$(d'après (17)).

(17) et (19) entraînent donc
\[
\tag{20}|\;\psi_\mt{i}^\pm>=|\;\phi_\mt{i}>+\frac{1}{\mt{E}_\mt{i}-\mt{H}\pm\mt{i}\epsilon}\mt{ V }|\;\phi_\mt{i}>
\]

\ul{Remarque} :
Nous avons donc remplacé l'équation intégrale (17) par l'équation (20),
qui n'est plus intégrale. Cependant la difficulté est reportée sur le calcul des
éléments de matrice < r | $\frac{1}{\mt{E}_\mt{i}-\mt{H}\pm\mt{i}\epsilon}$ | r' > du propagateur
$\frac{1}{\mt{E}-\mt{H}\pm\mt{i}\epsilon}$ du Hamiltonien H.

% 105

Nous savons d'ailleurs (grâce à la remarque du 4°) qui se transpose sans difficulté)
que cet élément de matrice représente une des fonctions
de Green solution de l'équation :
\[
[\Delta+\mt{k}^2-\mt{U(r)}]\mt{ G}_{\vec{\mt{k}}}(\vec{\mt{r}}-\vec{\mt{r}}\,')
=\delta(\vec{\mt{r}}-\vec{\mt{r}}\,')
\]

Nous pouvons obtenir, à partir de l'équation (17), un développement en série de
l'état stationnaire de collision :
\[
\tag{21}|\;\psi_\mt{i}^\pm>=\left[1+\frac{1}{\mt{E}_\mt{i}-\mt{T}\pm\mt{i}\epsilon}\mt{ V }+\frac{1}{\mt{E}_\mt{i}-\mt{T}\pm\mt{i}\epsilon}\mt{ V }\frac{1}{\mt{E}_\mt{i}-\mt{T}\pm\mt{i}\epsilon}\mt{ V }+...\right]|\;\phi_\mt{i}>
\]

Ce développement constitue le \ul{développement de Born} de l'état stationnaire
de collision. Il nous permettra d'obtenir le déveloprement à différents
ordres de la section efficace différentielle de diffusion.

\subsection{Propriétés mathématiques des états stationnaires de collision}% 6°)
{\bf—} Etats propres de H

Nous savons déjà que les états stationnaires de collision | $\psi_\mt{i}^+>$
et | $\psi_\mt{i}^->$ sont états propres de H avec la valeur propre E$_i=\frac{\hbar^2\mt{k}_\mt{i}^2}{2\mt{m}}$.

{\bf—} Orthonormalisation

Calculons le produit scalaire $<\psi_\mt{j}^+\;|\;\psi_\mt{i}^+>$

En prenant pour $<\psi_\mt{j}^+\;|$ la forme (20) et pour $|\;\psi_\mt{i}^+>$ la forme
(17), il vient :
\[
<\psi_\mt{j}^+\;|\;\psi_\mt{i}^+>=<\phi_\mt{j}\;|\;\phi_\mt{i}>+
<\phi_\mt{j}\;|\;\frac{1}{\mt{E}_\mt{i}-\mt{T}+\mt{i}\epsilon}\mt{ V }|\;\psi_\mt{i}^+>+
<\phi_\mt{j}\;|\mt{ V }\frac{1}{\mt{E}_\mt{j}-\mt{H}-\mt{i}\epsilon}\;|\;\psi_\mt{i}^+>
\]
\[
=<\phi_\mt{j}\;|\;\phi_\mt{i}>+
\frac{1}{\mt{E}_\mt{i}-\mt{E}_\mt{j}+\mt{i}\epsilon}<\phi_\mt{j}\;|\mt{ V }|\;\psi_\mt{i}^+>+
\frac{1}{\mt{E}_\mt{j}-\mt{E}_\mt{i}-\mt{i}\epsilon}<\phi_\mt{j}\;|\mt{ V }|\;\psi_\mt{i}^+>
\]
\[
=<\phi_\mt{j}\;|\;\phi_\mt{i}>=\delta({\vec{\mt{k}}_\mt{j}}-{\vec{\mt{k}}_\mt{i}})
\]
% 106 
On montre une formule identique pour $<\psi_\mt{j}^-\;|\;\psi_\mt{i}^->$. On a donc :
\[
\tag{22}<\psi_\mt{j}^+\;|\;\psi_\mt{i}^+>=\delta({\vec{\mt{k}}_\mt{j}}-{\vec{\mt{k}}_\mt{i}})
\]
\[
\tag{23}<\psi_\mt{j}^-\;|\;\psi_\mt{i}^->=\delta({\vec{\mt{k}}_\mt{j}}-{\vec{\mt{k}}_\mt{i}})
\]
L'ensemble des $|\;\psi_\mt{i}^+>$ et l'ensemble des $|\;\psi_\mt{i}^->$, pris séparément
forment donc \ul{deux ensembles} \ul{orthonormés}.

Nous admettrons que si on ajoute à chacun de ces ensembles
les vecteurs propres $|\;\psi_\beta>$ du spectre discret de H, on obtient un système orthonormé
complet (il est évident que les $|\;\psi_\beta>$ et les $|\;\psi_\mt{i}^+>$, correspondant à des valeurs propres
différentes de H sont orthogonaux).

On a donc les relations
\[
\tag{24}\int|\;\psi_\mt{i}^+><\psi_\mt{i}^+\;|\mt{ di}+\sum_\beta|\;\psi_\beta><\psi_\beta\;|=1
\]
\[
\tag{25}\int|\;\psi_\mt{i}^-><\psi_\mt{i}^-\;|\mt{ di}+\sum_\beta|\;\psi_\beta><\psi_\beta\;|=1
\]
Calculons enfin le produit scalaire $<\psi_\mt{j}^-\;|\;\psi_\mt{i}^+>$.

Nous avons :
\[
<\psi_\mt{j}^-\;|\;\psi_\mt{i}^+>=<\phi_\mt{j}\;|\;\phi_\mt{i}>+
<\phi_\mt{j}\;|\;\frac{1}{\mt{E}_\mt{i}-\mt{T}+\mt{i}\epsilon}\mt{ V }|\;\psi_\mt{i}^+>+
<\phi_\mt{j}\;|\mt{ V }\frac{1}{\mt{E}_\mt{j}-\mt{H}+\mt{i}\epsilon}\;|\;\psi_\mt{i}^+>
\]
\[
=\delta({\vec{\mt{k}}_\mt{i}}-{\vec{\mt{k}}_\mt{j}})+<\phi_\mt{j}\;|\mt{ V }|\;\psi_\mt{i}^+>
\left\{
\frac{1}{\mt{E}_\mt{i}-\mt{E}_\mt{j}+\mt{i}\epsilon}+\frac{1}{\mt{E}_\mt{j}-\mt{E}_\mt{i}+\mt{i}\epsilon}
\right\}
\]

{\bf—} Matrice R. Matrice S.

Posons $<\phi_\mt{j}$ | V | $\psi_\mt{i}^+>=$R$_\mt{ji}$ et considérons R$_\mt{ji}$ comme l'élément de
matrice entre $<\phi_\mt{j}$ | et | $\phi_\mt{i}>$ d'une matrice R,
dite \ul{matrice de réaction}
\[
\tag{26}\mt{R}_\mt{ji}=<\phi_\mt{j}\;|\mt{ R }|\;\phi_\mt{i}>=<\phi_\mt{j}\;|\mt{ V }|\;\psi_\mt{i}^+>
\]
% 107
On a donc :
\[
<\psi_\mt{j}^-\;|\;\psi_\mt{i}^+>=\delta({\vec{\mt{k}}_\mt{i}}-{\vec{\mt{k}}_\mt{j}})+
\mt{R}_\mt{ji}
\frac{-2\mt{i}\epsilon}{\epsilon^2+(\mt{E}_\mt{i}-\mt{E}_\mt{j})^2}
\]
Or
\[
\lim_{\epsilon\to0}\frac{2\mt{i}\epsilon}{\epsilon^2+(\mt{E}_\mt{i}-\mt{E}_\mt{j})^2}=
2\mt{i}\pi\,\delta(\mt{E}_\mt{i}-\mt{E}_\mt{j})
\]
Finalement :
\[
\tag{27}<\psi_\mt{j}^-\;|\;\psi_\mt{i}^+>=\delta({\vec{\mt{k}}_\mt{i}}-{\vec{\mt{k}}_\mt{j}})-
2\mt{i}\pi\,\mt{R}_\mt{ji}\,\delta(\mt{E}_\mt{i}-\mt{E}_\mt{j})
\]
\ul{Remarque} : si E$_\mt{i}\neq\mt{E}_\mt{j}$, $|\;\psi_\mt{j}^->$ et $|\;\psi_\mt{i}^+>$
sont vecteurs propres de H correspondant
à des valeurs propres \ul{différentes}. Il est donc normal que la relation
(27) donne alors
\[
<\psi_\mt{j}^-\;|\;\psi_\mt{i}^+>=0
\]

Par définition, nous appellerons élément de matrice entre $<\phi_\mt{j}$ | et | $\phi_\mt{i}>$
\ul{de la matrice S de} \ul{collision} $<\psi_\mt{j}^-\;|\;\psi_\mt{i}^+>$, et nous
avons donc la relation :
\[
\tag{28}\mt{S}_\mt{ji}=\delta({\vec{\mt{k}}_\mt{i}}-{\vec{\mt{k}}_\mt{j}})-
2\mt{i}\pi\,\mt{R}_\mt{ji}\,\delta(\mt{E}_\mt{i}-\mt{E}_\mt{j})
\]
% 108

\section{Approche physique}% C

Nous avons, dans le \S 2, trouvé des états propres de H
ayant à l'infini un comportement asymptotique en onde plane $+$ onde sortante ou
entrante. Nous avons obtenu pour ces états une équation intégrale et un développement
en série ainsi que certaines propriétés d'orthogonalité et de fermeture.
Il nous reste maintenant à dégager la signification physique des états $|\;\psi_\mt{i}^+>$
et $|\;\psi_\mt{i}^->$.

Nous allons montrer que $|\;\psi_\mt{j}^+>$ est l'état que l'on obtient
à l'instant t $=$ 0 en étant parti de l'état libre $|\;\phi_\mt{j}>$ à l'instant
t $= -\infty$ : le couplage V transforme l'état libre initial $|\;\phi_\mt{j}>$ en $|\;\psi_\mt{j}^+>$.
De même nous allons montrer que $|\;\psi_\mt{j}^->$ est l'état à l'instant t $=$ 0, qui
sous l'effet de V deviendra l'état $|\;\phi_\mt{j}>$ à l'instant t $= +\infty$ (d'ailleurs
$|\;\psi_\mt{j}^->$ et $|\;\psi_\mt{j}^+>$ se déduisent l'un de l'autre par renversement du temps,
si l'on fait abstraction des spins).

Ces propriétés justifieront le nom d'état stationnaire
sortant ou entrant donné à $|\;\psi_\mt{j}^+>$ et à $|\;\psi_\mt{j}^->$.

\ul{Remarque importante} :
$|\;\psi_\mt{j}^\pm>$ sont des \ul{états propres} du hamiltonien H. Ils sont donc \ul{stationnaires} et
\ul{n'évoluent pas} au cours du temps. En toute rigueur, les propriétés
que nous venons d'énoncer sur les liens entre $|\;\psi_\mt{j}^\pm>$ et $|\;\phi_\mt{j}>$ sont donc
inexactes : l'état $|\;\phi_\mt{j}>$ qui n'est pas un état propre de H ne pourra pas
évoluer à l'instant t $=$ 0 vers l'état $|\;\psi_\mt{j}^+>$ et de même l'état propre
$|\;\psi_\mt{j}^->$ ne pourra pas évoluer à l'instant t $= +\infty$ vers l'état $|\;\phi_\mt{j}>$.
Cependant, nous allons voir que les propriétés que nous avons énoncées
sont des \ul{propriétés limites}, valables sous certaines conditions qu'il va
falloir préciser. Nous allons en considérer trois :

1°) Evolution du système sous l'effet d'un branchement, ou
d'une coupure, adiabatique du couplage V.

2°) Evolution d'un état initial introduit progressivement
 
% 109
3°) Evolution d'un paquet d'onde formé avec les états stationnaires de collision.

\subsection{Branchement (ou coupure) adiabatique de la perturbation}% 1°) 

\begin{center} \begin{tikzpicture}
\draw (0,0) -- (3,0);
\draw [->] (0,-3) -- (0,3.1) node [left] {t};
\node at (0,0) [left]{0};
\draw [densely dashed, <->] (0,0.5)  -- (2,0.5);
\node at (1.3,0.5) [above] {V};
\draw [densely dashed, <->] (2,-0.1) -- (2,-2);
\node at (2,-1) [right]{$\hbar/\epsilon$};
\draw [line width=1.5pt] (0.01,-3) .. controls (0.01,-1) and (1.5,-0.5) .. (2,0);
\draw [line width=1.5pt] (0.01,3) .. controls (0.01,1) and (1.5,0.5) .. (2,0);
%\draw [line width=1.5pt] (1,3) .. controls (1.9,3) and (3.9,1.9) .. (4,1);
\end{tikzpicture} \end{center}

Supposons que la perturbation V stationnaire est remplacée
par une perturbation dépendant du temps
(avec e très petit).

cela revient à considérer que la perturbation V a été branchée sur. un intervalle
de temps de l'ordre de  dans le passé et qu'elle est coupée
dans le futur avec la même constante de temps (cf figure). Lorsque $\epsilon$ ,
le branchement ou la coupure deviennent de plus en plus longs et la perturbation
est pratiquement égale à V sur un très grand intervalle de temps autour de t = O :
nous avons un branchement (ou une coupure) adiabatique de
la perturbation,

Envisageons maintenant un instant t très lointain dans le
passé et antérieur à l'établissement de la perturbation (c'est-à-dire tel
). Considérons qu'à cet instant, l'état initial est constitué par

une onde libre de vecteur d'onde

Cherchons à déterminer l'état du système  à l'instant . Le
hamiltonien total du système est  et

nous pouvons développer sa fonction de Green avancée, X  en fonction de

% 110

la fonction de Green avancée de ,
On obtient alors le développement diagrammatique :

qui permet d'écrire :

\[
\tag{29}=
\]
avec :

Evaluons les termes successifs du développement (29) :
1er terme :  étant un état propre de T, on a évidemment
 (avec les conventions de phase choisies).

solos

 
% 111

2e terme : Le 2e terme peut s‘écrire en tenant compte de (30) et de la
relation 

Comme  , nous pouvons remplacer la borne inférieure par
et nous obtenons 

3e terme : Il s'écrit

Soit, en faisant le changement de variable 

Comme , nous pouvons encore remplacer la borne inférieure par ,
et nous obtenons

La loi de formation des termes successifs devient évidente : avec les
changements de variable

%
()
\[
\tag{31}=
\]
\[
\tag{32}=
\]

% 112 
On obtient les termes

Lorsque $\epsilon$ > 0, , chaque terme de la série ainsi obtenue tend vers le terme
correspondant du développement de Born de.

Si l'on suppose que les deux développements sont uniformément convergents,
on en déduit

On montrerait de même que


L'état  est l'état à l'instant t = O qui a évolué à
$\epsilon$ 1

partir d'un état initial d'onde libre établi à un instant antérieur au

branchement adiabatique de la perturbation,
. La relation (31) signifie donc qu'à la limite où le branchement
devient de plus en plus lent (e + 0) et où l'instant initial, tout en étant
antérieur à l'établissement de la perturbation, tend vers = +, l'onde libre
évolue vers l'état stationnaire de collision d'onde sortante. De même la
relation (32) signifie qu'à la limite où la coupure devient de) plus en plus
lente (e + O) et où l'instant final, tout en étant postérieur à la disparition
de la perturbation, tend vers +, l'état stationnaire de collision
. d'onde entrante évolue vers l'onde libre, Nous avons ainsi. donné une première
interprétation physique des états . De 

% 113

\subsection{}%
2°) Introduction progressive de l'état initial

Revenons au couplage stationnaire V.
Introduisons à un instant t < O un état d'onde libre 
. À l'instant t = 0, cet état est devenu

\[
\tag{33}=
\]
 exprime pas simplement car  est pas un état propre
. de H et la relation (33), développée sur les  , fait intervenir un
paquet complexe d'ondes de collision : on dit qu'on 8 un régime transitoire
dû au fait que l'état introduit n'est pas un état propre du hamiltonien.
Essayons d'échapper à cet inconvénient en envisageent un état introduit
de façon progressive entre les instants - t et O : de façon plus précise,
supposons qu'à l'instant t < O, on introduise un état  , à l'instant
 un état à, (, etc. et étudions ce que devient cette superposition linéaire d'états,
 sous l'effet du hamiltonien H à l'instant t = 0.

Nous obtenons un état

(le facteur 1/r est un facteur de normalisation. Supposons en effet qu'il
n'y ait pas de couplage et que H = T. Alors

Envisageons maintenant que l'instant  et introduisons alors un facteur de convergence
$\epsilon$ et/H qui représente un effet d'amortissement des ondes
introduites ns un passé lointain, on obtient un état

%
\[
\tag{35}=
\]
$\epsilon$ 

(le facteur e/H est encore introduit pour des raisons de normalisation :

supposons en effet que H = T. Alors, on montre que .
(35) s'intègre alors immédiatement

et finalement, puisque :

Ce qui n'est autre que la définition (20) de, D'où :

\[
\tag{36}=
\]
limite
Nous avons ainsi fourni une deuxième image physique de  :
 est l'état obtenu à l'instant t = O à la limite où l'on a introduit
de façon progressive des états d'onde libre  depuis l'instant t = - 
avec un amortissement tendant vers zéro.

% 115

\subsection{}%
3°) Evolution d'un paquet d'ondes

Lemme préliminaire.
Soit f(x) une fonction suffisamment régulière de x (continue,
intégrale, différentiable) et soient  les deux fonctions de t définies

Cherchons la limite de  lorsque t ,

Remarque :

I1 est évident que de la façon dmt nous avons posé le problème, la limite
e , doit être prise avant la limite
Posons f(x) = f(0) + f(x) - f(0)

Nous avons

Le première intégrale tend vers zéro lorsque

 n'a pas de singularité (pour x = 0, elle tend vers

En effet
f(x) lorsque $\epsilon$ + 0). L'intégrale sur x du produit de cette fonction régulière
par l'exponentielle oscillante  dont la période en x, F tend vers
zéro lorsque , tend elle-même vers zéro lorsque , si  est

ordre, et on peut dire de façon plus précise que la contribution de la première

une courbe en "cloche" de largeur Ax, la largeur de est du même

 intégrale de (38) devient négligeable dès que la période devient petite

devant, c'est-à-dire dès que 

Quant à la seconde intégrale

nous la calculons, selon une méthode habituelle, par les résidus en fermant
le contour vers le bas si t > O et vers le haut si t < O.

% 116

Finalement, on trouve

et
Ces résultats sont indépendants de | t | et on a donc :

\[
\tag{39-a}=
\]
\[
\tag{39-b}=
\]
\[
\tag{39-c}=
\]
\[
\tag{39-d}=
\]

Rappelons que la limite $\epsilon$ , est prise avant la limite  , cette
dernière signifiant simplement que , x étant la largeur de

\subsubsection{}%
a) Définition et propriétés du paquet d'ondes libres :

Considérons, dans une situation où V est nul, un paquet d'ondes
libres formé par une superposition linéaire d'ondes planes 
(qui seront ici normées à l'unité) :

\[
\tag{40}=
\]
La sommation résume une sommation sur le module  et sur la

direction  du vecteur d'onde de :

 est une fonction régulière de  et de  que nous supposerons
de plus très concentrée autour des valeurs moyennes  et .

% 117

Nous avons ainsi défini un paquet d'ondes libres ,
dont l'évolution au cours du temps, en l'absence de V, est parfaitement
connue (équation ),
Ce paquet d'onde possède une direction moyenne Qi, une vitesse moyenne :  et une
énergie moyenne , Ces résultats sont classiques. Pour trouver la région de l'espace
où se trouve concentré le paquet d'ondes à l'instant t, il faut passer en
représentation r et appliquer
la méthode de La phase stationnaire (cf Messiah, page 3).
Nous supposons que la phase de est telle qu'à
l'instant t = 0, le paquet se trouve autour de F = 0, dans la
région où sera appliqué v (r).
. Nous allons rappeler quelques résultats classiques relatifs
aux dimensions et à l'étalement de ce paquet d'ondes 
 les dimensions longitucinales et transversales du paquet d'ondes sont
d'autant plus grandes que la larreur de la fonction  autour de
 et  est plus petite. Nous ferons l'hypothèse que les dispersions en
direction et en énergie  et  seront suffisamment petites pour que les
dimensions du paquet d'ondes soient très grandes devant la portée r du
potentiel :
% 118

- Etalement du paquet d'ondes : la longueur du paquet d'ondes, , est
de l'ordre de . Le temps de passage du paquet d'ondes en un point

 ( étant la dispersion en énergie T )
ce qui n'est autre que l'expression de la quatrième relation d'incertitude
temps-énergie.

Pendant le temps 1, le paquet d'ondes subit un étalement (dû à
la dispersion des vitesses Av = dk), Cet étalement est égal à

. Cet étalement peut être considéré comme négligeable tant
qu'il est très petit devant la longueur Ax du paquet d'ondes, c'est-à-dire

tant que 

Nous ferons l'hypothèse que pendant le temps mis par le paquet d'ondes

pour passer en un point  il subit un étalement négligeable,
c'est-à-dire que l'on a :

 soit 

En résumé, nous envisageons donc un paquet d'ondes libres  , qui
passe sur la région où règnera l‘interaction, suffisamment bien défini en
énergie et en direction pour que Ses dimensions soient grandes devant la
portée effective de i'interaction V et pour que son étalement soit négligeable
durant son passage dans la région de cette portée effective.

\subsubsection{}%
b) Paquet d'onde formé avec les

Considérons maintenant une situation où l'interaction V ()
existe et envisageons le paquet d'ondes formé avec les états de collision
 (ou les états ,avec les mêmes coefficients  que
ceux que nous avons définis pour le paquet d'ondes libres. A chaque paquet
d'ondeslibres  vérifiant les conditions du \S a), nous associons

ainsi deux "paquets de collision"  définis par

% 119

dont l'évolution aucours du temps est parfaitement connue (et donnée par
(u1)) puisque les | vs > sont des états propres du hamiltonien global
H = T + V avec la valeur propre 

Nous avons ainsi créé deux paquets d'ondes  parti
culièrement bien adaptés au problème de la collision. I1 nous reste à voir
le lien qui existe erire ces paquets d'ondes et le paquet d'ondes libres
 auquel ils correspondent. Raisonnons tout d'abord sur .
 
\subsubsection{}%
c) Lien entre .

Remplaçons dans (41) par l'expression (17) de l'équation de Lippmann-Schwinger. On obtient : Et

Soit, en utilisant la relation de fermeture,

en posant
 (selon la définition 26) .
En détaillant les intégrations, (k2) s'écrit :

Nous avons posé.

%
(012)

\[
\tag{45}=
\]

% 120

Pour simplifier cette relation, nous devons faire appel aux hypothèses
de a) et b)  le coefficient  n'est important qu'au voisinage
immédiat de  (sur une étendue ). Or R  varie peu en
 approximation de Born,  est égal
à , transformée de Fourier au potentiel V (x) à la valeur

 étant fixé,  varie donc notablement dans
l'espace des K; sur des intervalles de l'ordre de 1/r. r étant la portée
effective du potentiel V Or au \S a), nous avons justement choisi 
suffisamment petit pour que l'intervalle de variation correspondant pour
 soit très petit devant . IL en résulte donc que 
varie peu en  sur l'intervalle , largeur de la fonction  On
peut donc écrire la dernière intégration de (h3) :

Posons alors

La relation (3) nous conduit done à calculer

c'est-à-dire, en effectuant le changement de variables 

En supposant que  est une fonction suffisamment régulière
de  on peut calculer  à l'aide du lemme résumé dans les
relations (39-a) et (39-b) et on a donc

% 121

Remarquons également que la limite  signifie simplement que À
est beaucoup plus grand que l'inverse de la largeur en ,
qui d'après la relation de définition (4) n'est autre que la dispersion en
énergie  du coéfficient . On peut donc remplacer la limite

, c'est-à-dire  grand devant

le temps de passage du paquet d'ondes libres correspondant en un point (ou
 par la condition

ce qui revient au même dans la région de portée du potentiel.

En résumé, (3) compte tenu des relations (46) nous conduit aux
relations

\[
\tag{472}=
\]
avec les définitions :

Pour le paquet d'ondes  une démonstration analogue peut être faite.

Elle conduit notamment au résultat

\[
\tag{48}=
\]
% 122

\subsubsection{}%
d) Conclusion et inteprétation physique.

Nous venons donc de montrer que le paquet d'ondes  ,
formé à l'aide des états stationnaires de coliision, se trouve dans le
paquet d'ondes libres  construit à l'aide des mêmes coefficients.

passé lointain (en fait si t ) pratiquement confondu avec le

De même le paquet d'ondes  , dans le futur lointain
(en fait si t  ) sera pratiquement confondu avec le paquet d'ondes

libres  correspondant,

De tels résultats sont extrêmement importants en ce qui concerne
la théorie des collisions : dans une expérience de collision, on prépare
en effet les particules incidentes, dans un passé lointain, loin de la
cible, dans une région où le potentiel V n'existe pas.

L'état initiai correspond donc à un paquet d'ondes libres
 constitué à l'aide des  Mais les  n'étant pas des
états propres de H, l'évolution d'un tel paquet d'ondes pendant et après
la collision n'est pas facile à calculer. L'intérêt des états stationnaires de
collision apparaît ici : nous venons en effet de montrer que dans
le passé lointain (c'est-à-dire longtemps avant la collision), il revient
au même de développer le paquet d'ondes suivant les états libres  ou
les états stationnaires de collision  . On peut donc prendre pour
état initial de la collision un état  , au  Les
 étent, eux, des états propres de H, l'évolution de  est
connue et conduit pour le futur lointain (c'est-à-dire longtemps après la
collision) à la formule (7-b). Cette formule, qui décrit donc l'état,
après collision d'un paquet d'ondes libres incident, est très importante
car c'est elle qui va nous permettre de calculer les sections efficaces
de collision.

Nous pouvons déjà, à ce stade, remarquer qu'elle représente le
paquet d'ondes libres incident auquel s'ajoute un paquet d'ondes libres,
diffusé dans toutes les directions de l'espace, mais dont le module du
vecteur d'onde est centré autour de la valeur k; du paquet incident

% 123

(en raison du terme ) de la relation (7-c) ce qui n'est autre
que l'expression de la conservation de l'énergie au cours de la collision élastique.

Enfin, si on veut avoir une représentation spatio-temporelle
de la collision, il faut passer en représentation r et appliquer la méthode de la
phase stationnaire pour déterminer la répartition à chaque
instant du paquet d'ondes. On obtient alors l'image représentée par les

figures suivantes (cf Messiah, p. 316 et suivantes).

% 12

Nous avons ainsi établi un nouveau lien physique entre les
états stationnaires de collision  et les états libres  , par
l'intermédiaire des paquets d'ondes. C'est cette dernière propriété des
 qui e le plus de signification physique car nous venons de montrer
qu'elle est liée de façon évidente à la description même de l'expérience
physique. Elle va nous permettre également d'éclairer sous un jour nouveau
"l'approche adiabatique" ou celle de l'introduction progressive de l'état
"initial" faites aux \S 1°) et 2°) :

\subsubsection{}%
e) Lien avec les 2 autres approches physiques (\S 3-1 et 3-2)

Plus la fonction de répartition  est étroite autour
des valeurs moyennes  plus le paquet d'ondes  est proche
de l'état stationnaire  et plus le paquet d'ondes  est proche
de l'état stationnaire de diffusion  et  sont les
paquets d'ondes à l'instant t = O à La limite où on a remplacé 
par une fonction de Dirac au point  Mais tant que  possède
une certaine dispersion AE en énergie, on a vu qu'il existe un temps
, tel que le "paquet de collision" qui est très proche de l'état
stationnaire de collision à l'instant t = 0, se réduise à ce temps au "paquet libre",
qui est lui très proche de l'état libre 

Dans les deux autres approches physiques que nous avons données,
le fait d'établir la perturbation sur un intervalle de temps H/e ou d'introduire
progressivement l'état initial sur un intervalle de temps H/e
fait que l'état obtenu à l'instant t = O n'est pas (pour $\epsilon$  O) un état
propre de H, Il existe sur son énergie une incertitude, qui provient de sa
préparation même de l'ordre de $\epsilon$. On peut encore dire, qu'avant de faire
tendre $\epsilon$ + 0, l'état obtenu à l'instant t = O,  qui obéit à l'équation 

% 125

est une superposition d'états stationnaires  avec une dispersion
de l'ordre de $\epsilon$. . représente donc en quelque sorte un "paquet
d'ondes" de largeur en énergie $\epsilon$, auquel nous pouvons appliquer les résultats précédents .

En d'autres termes, le fait d'introduire progressivement la
perturbation (ou l'état initial) sur un temps de l'ordre de H/ est une
façon commode de simuler un paquet d'ondes de dispersion en énergie $\epsilon$ (s)
Il est de plus beaucoup plus aisé de travailler avec des états du type
. qu'avec des paquets d'ondes. C'est 1à tout l'intérêt de la
théorie formelle des collisions qui est basée sur l'emploi systématique
des états 

Il faut cependant prendre une précaution essentielle : le
paramètre $\epsilon$, qui a été introduit dans tous les calculs a, ainsi que nous
venons de le voir, une signification physique très simple et fondamentale :
il représente l'incertitude sur l'énergie dans l'expérience physique réelle :

cette incertitude existe toujours. Tous les calculs mathématiques conduisent
aux grandeurs physiques devront donc être faits avec un $\epsilon$ fini différent de
zéro. Ce n'est qu'à la fin des calculs qu'on pourra faire tendre $\epsilon$ vers zéro.
Deux cas peuvent alors se présenter : ou bien la limite existe et elle
représente la grandeur physique cherchée dans le cas où la définition en

 On peut encore dire qu'il revient au même de "faire passer" le paquet
d'ondes pendant un temps  devant la perturbation (ce qui correspond
à l'expérience physiaue réelle) ou, par exemple, de brancher la perturbation sur
une onde plane libre pendant le même temps (\S 3-1).

 
% 126

énergie de l'expérience était très bonne; ou bien la limite n'existe pas
ou est absurde : cela signifie alors que le résultat physique dépend cru
cialement de la forme du paquet d'ondes et la théorie formelle des collisions
ne peut s'appliquer : il faut raisonner sur les paquets d'ondes.

En d'autres termes, tant que le résultat physique est indépendant de la
forme du paquet d'ondes incident, la théorie formelle des collisions
s'applique et permet de résoudre les problèmes de façon élégante. Nous
allons d'ailleurs en voir un exemple en appliquant toute l'étude précédente à la
théorie des collisions.

\section{}%
D - Application de l'étude précédente : Théorie des collisions
\subsection{}%
1°) Définition de la section efficace différentielle de diffusion

\subsubsection{}%
a) Probabilité de transition :

Soit  la probabilité de trouver à l'instant t le
vecteur d'onde de 1a particule dans  solide .

Soit  la dérivée 

La quantité  représente la probabilité pour

qu'entre les instants t et t  la particule ait été diffusée dans
l'angle solide 

si  est ndérendent du temps, on définit alors 
comme la probebilité  de transition par unité de temps.

Enfin, si  vecteur d'onde de la particule: incidente, il
est évident que 

 représente la probabilité totale pour qu'après
la collision da particule ait été diffusée dans l'angle solide 
et on a la relation

% 127

\subsubsection{}%
b) Flux des particules libres :
Supposons maintenant que V = O et que la particule est
toujours libre.

 Appelons le flux à l'instant t, au point r de la
particule libre et prenons le point F à l'intérieur de la zone, d'extension
où règne le potentiel V. Le flux  ne varie d'ailleure pas
d'un point à l'autre de cette zone si, comme nous en avons fait l'hypothèse,
la largeur du paquet d'ondes est grande devant

Soit  la densité de probabilité de présence de la
particule libre  à l'instant t. On a alors

Soit  une surface située en r et perpendiculaire à la vitesse initisle
de le particule libre : la probabilité totale pour que la particule
libre soit passée à travers  est

\subsubsection{}%
c) Définition de la section efficace différentielle de diffusion :

La section efficace différentielle de diffusion pour
l'angle solide  est une surface  définie de la façon suivante :

La probabilité globale pour que, sous l'effet de V, la particule soit diffusée
dans l'angle solide  est égale à  fois
la probabilité globale pour que la particule passe à travers la surface
 si elle restait toujours libre,  est perpendiculaire à la
vitesse de la particule incidente et située  un point  où règne le
potentiel V),

\[
\tag{49}=
\]
\[
\tag{50}=
\]
% 128

Par définition, on a donc

Vis-à-vis de la diffusion dans la direction  tout se passe donc comme
si le potentiel V était remplacé par la surface , perpendiculaire
à la vitesse de la particule libre incidente : toute particule libre passant à
travers  est diffusée dans la direction  d'où le nom de
section efficace de diffusion donné à .

La définition (h9) présente une très grande importance car elle.
est indépendante de la forme du paquet d'ondes si celui-ci est suffisamment large
(c'est ce que nous démontrons plus loin). La notion de section
efficace présente ainsi un très grand intérêt physique et elle correspond
bien à ce que mesure l'expérimentateur.

Nous n'avons jusqu'ici étudié que la diffusion d'une seule parti
cule par un seul centre diffuseur, Les problèmes réels sont plus compliqués
du fait des interactions possibles entre les particules incidentes et de
la diffusion multiple due au grand nombre de particules de la cible, Mais
nous n'étudierons pas ces problèmes.

Notons enfin que l'on obtient la section efficace totale en sommant sur les angles
solides la section efficace différentielle de diffusion

total
Pour effectuer le calcul de la section efficace différentielle

de diffusion, nous allons utiliser deux méthodes :

% 129

a) Le méthode des paquets d'ondes, la plus physique, qui nous permettra
d'interpréter de façon physique simple les conditions mathématiques qui
rendent la section efficace indépendante de la forme du paquet d'ondes.
b) La théorie formelle des coïlisions, beaucoup plus simple, mais présentant des
points délicats qu'il nous faudra discuter.

d'ondes

A l'instant  , nous envisageons le paquet d'ondes libres , qui est équivalent au paquet .

A l'instant  , le paquet d'ondes est représenté par la
formule (7-b). Nous calculerons tout d'abord la probabilité 
pour que la particule ait alors sa direction comprise dans l'angle solide.

Nous calculerons ensuite le flux du paquet d'ondes libres incident en un point où
règne le potentiel V, à l'instant t, puis nous intègrerons le flux de  dans le temps.

Nous calculerons enfin la section efficace en appliquent la relation (49),

Nous ne ferons le calcul de la section efficace que pour une direction  nettement
différente de fa; de façon que dans le calcul de
, seul le 2e terme de la relation (7-b) intervienne : physiquement,
cela revient à placer le détecteur dens une direction différente de la direction
incidente, de façon à éviter qu'il ne détecte le paquet d'ondes non diffusé,
vecteur D'après (7-b), la probabilité de trouver à  direction du
onde entre  et  et sa longueur entre  et  est

% 130

\[
\tag{51}=
\]
\[
\tag{52}=
\]
\[
\tag{53}=
\]

On obtient P (a, + ©) en sommant sur  :

  est donné par la relation (k7-c).

Dans cette relation, varie notablement sur des intervalles en  de l'ordre
de 1/r,  : portée du potentiel), c'est-à-dire
varie très peu sur l'intervalle de variation de , la longueur
du paquet d'ondes étant très grande devant 

On peut donc remplacer, dans la formule (51) le terme

 qui figure dans l'expression de  par

Compte tenu de cette remarque et de (7-c), (51) devient :

Toujours en raison de la variation rapide de  autour de  on

peut maintenant remplacer dans (52)  par

ou encore par

(52) devient alors

% 131

- Nous devons calculer le flux en un point F où règne le potentiel, du
paquet d'ondes libres. Ce flux  est égal à 

On a donc

- Le paquet d'ondes libres  est composé d'ondes planes  dont
les directions sont lérèrement dispersées autour de la direction  qu'on
prenne pour axe. Cette dispersion des directions est responsables des
dimensions transversales finies du paquet d'ondes. Elle ne se fait sentir
que sur les bords latéraux du paquet d'ondes, Au centre du paquet, notamment
dans la région où règne le potentiel V, la fonction d'onde est la
même que si on remplaçait chaque onde plane  par l'onde. plane du
vecteur d'onde  dirigé suivant Oz. Cette approximation n'est valable que
si la région où règne le potentiel reut être considérée comme "au centre"
du paquet d'ondes, donc si les dimensions transversales du paquet d'ondes
sont grandes devant la portée du potentiel. On peut alors remplacer le paquet d'ondes libres . Est
par le paquet à une dimension est

Posons
On a alors
\[
\tag{55}=
\]
% 132

\[
\tag{56}=
\]


Dans (55), nous pouvons écrire après un calcul élémentaire :

Dans (5), l'intégration de  s'effectuera en pratique sur un
intervalle de temps de l'ordre du passage du paquet d'ondes au point r,

qui est (cf page 118).
Pendant cet intervalle de temps, la phase de la première exponentielle du
second membre de (56) varie d'un terme de l'ordre

Or nous avons fait l'hypothèse que <<1 (étalement négligeable

du paquet d'ondes). La phase de la première exponentielle varie donc très

La troisième exponentielle est un facteur de phase indépendant de

peu autour de zéro et on peut remplacer  par 1.
la variable k; et qui disparaîtra quand on formera .

Finalement, on peut écrire

Posons 
 devient  avec

% 133

L'intégration sur u donne 
et finalement
\[
\tag{57}=
\]
De (49), (53) et (57), on déduit

Nous avons ainsi obtenu, comme annoncé, une section efficace de

diffusion incépendente de la forme du paauet d'ondes, le terme dépendant de

cette forme  s'étant éliminé dans le quotient des relations
(53) et (57);  est à noter que la probabilité  et le flux

 n'étaient pas, quant à eux, indépendants de cette forme

le calcul que nous verons de feire est ripoureux moyennant les
conditions imposées au paquet d'ondes (longueur et largeur suffisantes).
Si ces conditions n'étaient pas satisfaites, il n'aurait pas été possible de
définir une section efficace indépendente de la forme du paquet d'ondes.

La relation (58) nous montre l'importance de la matrice  de
réaction qui, à elle seule, permet de calculer les sections efficaces
Notons enfin que la formule (58) se différencie par un facteur  des formules de
certains auteurs qui utilisent au lieu d'ondes planes "normées" à

l'unité,  $\epsilon$ , comme nous l'avons fait, des ondes planes non

normées e . Le facteur  se retrouve alors dans l'élément de
matrice
\[
\tag{59}=
\]
\[
\tag{60}=
\]
\[
\tag{61}=
\]
% 134

\subsection{}%
3°) Calcul de o par la théorie formelle des collisions

\subsubsection{}%
a) Introduction : Les calculs à l'aide des paquets d'ondes,
dont la signification physique est très claire, sont longs et assez pénibles.

La théorie formelle des collisions consiste à travailler avec
des "quasi états de collision"  qui vérifient l'équation de
Lippnann-Schwinger

avec un paremètre $\epsilon$ petit mais non nul, ce qui permet de simuler un paquet
d'ondes.

Les calculs sont alors beaucoup plus simples, mais comportent de
nombreux pièmes. En règle générale, il faut garder à l'esprit le sens physique
de $\epsilon$ qui représente la dispersion en énergie de la particule incidente
et ne faire tendre e vers zéro qu'à la fin des calculs,

Supposons done qu'à l'instant t = O, la particule ait été préparée
de telle manière qu'elle se trouve dans l'état  défini par la relation (59). .
n'est pas un état propre de H, C'est un état approché
ayant une dispersion $\epsilon$ en énergie. Il évolue, après , sous l'effet du

hamiltonien H = T + V et à un instant t, il devient

La probabilité de trouver à l'instant t la particule dans l'état ,
 s'écrit :


Nous allons tout d'abord calculer la dérivée à l'instant t = O de cette
quantité, puis nous rattacherons cette dérivée à la section efficace.

 
% 135

\subsubsection{}%
b) Calcul de 

En dérivant la formule (60) par rapport au temps, on obtient :

\[
\tag{62}=
\]

Compte tenu de (61) et (62), on a :

Le terme entre crochets est épal à la différence de deux termes complexes
conjugués l'un de l'autre, On a donc

\[
\tag{63}=
\]

Rappelons que H = T + V.

On a notamment

 est en effet un état propre de T de valeur propre .

Or  qui est réel, a sa partie imaginaire nulle.
Finalement, on peut donc, dans (63), remplacer H par V et on obtient

Posons

D'autre part, pour calculer , remplaçons  par son

expression (59); on obtient alors

% 136

\[
\tag{66}=
\]
\[
\tag{67}=
\]
\[
\tag{68}=
\]
Compte tenu de (65) et (66), (64) devient :

 Cependant, on peut montrer, en calculant la dérivée seconde, que

Nous n'avons en toute rigueur calculé la dérivée  qu'au point

 varie très peu sur des temps inférieurs à . On peut donc considérer


et représentant la probabilité de transition par unité de temps vers
l'état | \S; ?e
Comme nous l'avons déjà fait, nous allons supposer que i  j,

Pi comme une constante, Mis donnée par la formule (67)

autrement dit que l'observation se fait dans une direction différente de
la direction incidente, On a alors, d'après (67),

Nous voyons que la probabilité de transition n'est appréciable
que si l'énergie E est égale à E à  près. Ceci rend compte de la conservation
de l'énergie, compte tenu de l'incertitude $\epsilon$ sur l'état initial.

Nous devons calculer en fait la probabilité de transition par
unité de temps vers un état dont le vecteur d'onde  pointe
dans l'angle solide.

% 137

 $\epsilon$ 
Nous avons évidemment (en supposant que Rji varie peu lorsque
 varie autour de 

\[
\tag{69}=
\]
avec le changement de variable
 (69) devient

Si  est suffisamment petit, l'intégrale en E donne tout simplement 
et on obtient

\[
\tag{70}=
\]
si (avec  dans )

\subsubsection{}%
c) Calcul de la section efficace :
Il ne nous reste plus qu'à calculer le flux de l'onde plane
incidente . En représentation  et 

Le flux est également indépendant du temps. En utilisant (49), il vient
alors

Ce n'est que maintenent que nous pouvons faire tendre (e) vers zéro.

D'après (65), on a

% 138

Lorsqu'on fait tendre $\epsilon$ + O0, les seules divergences pourraient provenir
du terme en  Mais cet opérateur est entouré par les opérateurs V et n'agit pas
directerent sur une fonction .

On peut ainsi montrer qu'en général, Bi converge régulièrement vers
 lorsque $\epsilon$ + O, Lorsque $\epsilon$ est suffisamment petit,
 est pratiquement égal à  et varie très peu quand  varie autour
de , ce qui justifie l'hypothèse faite pour établir (69) et permet de
remplacer  par .
Finalement

avec 

Nous retrouvons ainsi l'expression (58),

\subsubsection{}%
d) Remarques diverses :

a) Ressemblance de la formule () avec la règle d'or de Fermi :

La formule (68), donnant la probabilité par unité de temps 

est riroureuse, Cependant, si on remplace  par 
(approximation de Born), on obtient la relation approchée

qui n'est autre que la règle d'or de Fermi,

) Difficultés de la théorie formelle des collisions :

Nous avons longuement insisté sur le fait qu'il fallait raisonner
avec des états  avec $\epsilon$ fini non nul, de façon à simuler un paquet
d'ondes physique de dispersion en énergie $\epsilon$, Si dans l'expression (61) :

on avait fait tendre dès le début  vers zéro, aurait été remplacé à la limite par
l'état stationnaire 

aurait été une constante indérendante du temps. On aurait alors eu 

% 139

De même si dans l'expression (63) de  :

on avait fait tendre $\epsilon$ vers zéro, et remplacé ,

fonction propre de H de valeur propre on aurait eu

On voit ainsi clairement qu‘en théorie formelle des collisions
il ne faut faire tendre  qu'à la fin des calculs: Ceci re résulte
pas d'un simple artifice methématique, mais se trouve lié à la signification
physique même de .

Citons comme dernier exemple le cas où, au lieu de traiter le
spectre continu, on a enfermé ia particule dans une boîte de côté L. On
est alors amené à la fin des calculs à faire tendre  et, , il
intervient dans le calcui des termes du type  qui conduisent à une forme
indéterminée : c'est la nature physique de $\epsilon$ qui permet de lever l'indétermination : 
étant la dispersion en énergie du "pseudo paquet d'ondes", la
longueur de ce dernier est de l'ordre de v. Pour que les conditions aux
limites ne modifient pas le problème, il faut évidemment que l'on ait L 
soit , IL faut dons à la fin des calculs faire tendre

) Expression de la section efficace en fonction de l'amplitude  de diffusion

En posant dans la formule (1)

 

devient une fonction de , et de  qui s'écrit

On peut donc dans (5) remplacer . Soit finalement

\[
\tag{73}=
\]
% 140

\subsubsection{}%
4°) Théorème optique

Elle s'écrit, d'après (63),

(compte tenu de la relation de fermeture .

Or l'opérateur H est hermitique et sa valeur moyenne dans l'état 

est réelle. On a donc

La relation (7) exprime le fait que la somme des probabilités
de transition par unité de temps vers tous les états  est nulle :
c'est une équation de conservation de la probabilité de présence globale
de la particule : en effet, en intégrant (7) on arrive à l'équation

 (indépendante du temps)

qui exprime le fait que la somme des probabilités pour que la particule
soit dans un état  quelconque est constante.

La relation (7h), compte tenu de (67), conduit à 3

Si  est suffisamment petit, 21 se comporte comme une

distribution de Dirac. On a alors

% 141

ou, à la limite où  :

ce qui s'écrit encore, compte tenu de (50) :

ou bien, compte tenu de (72) :

La relation (75) (ou 75) porte 1e nom de relation de
Bohr-Peieris-Piaczek et traduit le "théorème optique" : la section
efficace totale de diffusion est proportionnelle à le partie imeginaire de
l'amplitude de diffusion vers l'avant.

Le théorème optique se présente comme une conséquence
de la conservation de ia norme ou de la probabilité de présence de la
particule : physiquement, la probabilité de présence globale de la particule
doit se conserver au cours de la collision. Pour qu'il apparaisse,
après collision, une probabilité de présence non nulle dans des directions
autres que la direction incidente, il faut donc qu'après la collision, la
probabilité de présence dans la direction incidente soit plus petite
qu'avant la collision. Ceci n'est possible que s'il existe une interférence
destructrice entre l'onde transmise (qui est identique à l'onde
incidente) et l'onde diffusée vers l'avant, décrite à l'aide de
. La quantité dont diminue la probabilité de présence dans la
direction incidente mesure l'"absorption" sur le faisceau incident qui
est égale à la probabilité totaïe de diffusion dans toutes les autres

directions, liée à  IL est donc naturel qu'il existe une relation

entre . le théorère optiaue montre de plus que c'est

la partie imasinaire de l'amplitude de diffusion vers l'avant  qui

\[
\tag{71}=
\]
\[
\tag{78}=
\]

% 1

décrit l'absorption sur le faisceau incident.

\subsubsection{}%
5°) Approximation de Born

Nous avons vu dans les paragraphes précédents que la quantité
essentielle pour le calcul des sections efficaces de collision est l'élément de
matrice R..,

Or on a :

Soit en introduisant dans le dernier membre 1a relation de fermeture

L'équation intégrale (76) est l'analogue pour les éléments de la matrice
de Péaction de l'équation de Lippmann-Schwinger pour les états de collision.
Pour résoudre les différents problèmes de collision, on est amené à appliquer à
(76) les diverses méthodes de résolution et d'approximation des
équations intégrales,

On peut notamment faire le développement en série de NeumannLiouville:

qui porte le nom de Dévelonnement de Born des éléments de la matrice de
réaction. Ce développement permet de calculer la section efficace sous
forme d'un développement limité en V.

Au premier ordre, nous avons l'approximation de Bormm : 

qui conduit à la section efficace :

Nous n'aborderons pas ici la discussion de la validité de cette approximation,
ni de ses applications.

\[
\tag{80}=
\]

%13

\section{}%
E - Matrice S
Nous allons maintenant définir et étudier de façon précise

la matrice de collision S.

\subsection{}%
1°) Rappels sur la représentation d'interaction ou "point
D

- de vue de Dirac")
Dans le point de vue de Schrödinger, l'état  évolue
conformément à l'équation de Schrödinger
ou encore sous l'effet de l'opérateur unitaire U (t", t')

 vérifiant l'équation

Effectuons sur les vecteurs d'états et sur les observables ia trensformation
unitaire définie par l'opérateur  et V deviennent
alors :

Il est alors immédiat de montrer que | y (t) > évolue dans le temps conformément
à l'équation :

% 1

Par définition  est le vecteur d'état du système dans
la "représentation" d'interaction, L'équation (80) montre que

 n'évolue pas si l'interaction V est nulle : autrement
dit, le point de vue d'interaction sépare dans le mouvement ce
qui provient du hamiltonien de la particule libre T et ce qui
provient du potentiel V qui décrit la collision.

On peut montrer aisément que l'équation (80) est équivalente à :

\[
\tag{81}=
\]
où U (t'", t') est l'opérateur unitaire défini par

\[
\tag{82}=
\]
vérifiant l'équation

\[
\tag{83}=
\]

et possédant les mêmes propriétés de groupe que U (t", t') :

\[
\tag{84-a}=
\]
\[
\tag{84-b}=
\]
\[
\tag{84-c}=
\]
Les opérateurs T et V étant indépendants du temps, nous pouvons
de plus écrire :

et l'équation (82) devient alors :

% 
L'intérêt de la représentation d'interaction dans le problème de la collision
réside dans le fait que le vecteur d'état n'évolue
pas si l'interaction est nulle : longtemps avant ou après la collision,
les deux systèmes dont on étudie l'évolution sont éloignés l'un de
l'autre; ils n'interagissent pas et le vecteur d'état qui les représente dans
la représentation d'interaction est immobile : la collision
doit donc faire évoluer Le vecteur d'état du système d'une position
fixe, dans le passé lointain que nous roterons  à une
autre position fixe, dans le futur lointain que nous noterons ,

conformément à la figure ci-dessous

Nous allons étudier, dans ce qui suit, la transformation S qui permet de
passer de :

et nous allons voir dans quelle mesure on peut assimiler 

% 146

\subsection{}%
2°) Définition de la matrice S. Méthode

\subsubsection{}%
a) Position du problème

Soient  et  deux paquets d'ondes libres
en représentation d'interaction. Ces paquets n'évoluent pas au
cours du temps et peuvent s'écrire :

Les états 

norme est finie,
 sont des états physiques puisque leur

Considérons la quantité

qui représente l'amplitude de probabilité pour que le système
étant dans l'état  à l'instant t' se trouve dans l'état
à l'instant t",

Nous allons montrer que cette quantité admet une
limite lorsque  et  et nous appellerons cette

limite élément de matrice de la matrice de collision S entre
 matrice de collision 

%

\subsubsection{}%
b) Définition et propriétés des vecteurs  et 

Dans la représentation de Schrödinger, considérons les deux
paquets de collision associés à  et  que nous avons
étudiés au \S C : Est

D'après (h7-a), (kB) et (79-a), nous savons que

En passant en représentation d'interaction :

On a d'après (88-a) et (88-b) :

\[
\tag{89-a}=
\]
\[
\tag{89-b}=
\]

% 148

\subsubsection{}%
c)Limite de 

Considérons la quantité
D'après (8h-a), elle est égale à

et donc indépendante de  et de ,
D'autre part, d'après (89-a) et (89-b), la différence

tend vers zéro lorsque ,

On en déduit donc :

ce qui constitue la propriété annoncée au \S a),
On a donc, d'après (87) :
\[
\tag{91}=
\]
a Matrice S
A l'instant t = O, la représentation d'interaction

est équivalente à celle de Schrödinger. Nous avons donc

et (91) s'écrit :

Mais, d'autre part, grâce à la relation de fermeture, on a

% 149

(92) et (93) nous conduisent à poser

\[
\tag{94}=
\]

ce qui rejoint la définition formelle (28) de la matrice de collision :

Les éléments de matrice  apparaissent ainsi comme des distributions
qui agissent sur les fonctions (cf équation 93).
Remarque : 
Nous avons montré que  admet une limite lorsque
, Lorsque  et  sont des paquets d'ondes.
Nous avons vu que cette limite se nous axors naturellement à l'aide
des éléments de matrice
Mais nous n'avons, en aucune façon, prouvé que l'opérateur 
admet une limite lorsque 

Nous allons étudier maintenant la quantité
et nous constaterons notamment qu elle n'admet pas de Limite lorsque

\subsection{}%
3°) Calcul direct 

Au lieu de raisonner sur des paquets d'ondes, nous allons revenir
à une théorie formelle des collisions et raisonner sur des états libres
Nous savons qu'il faut alors (cf \S 3 et 4) introduire dans l'interaction un
facteur de convergence dont nous avons suffisamment explicité la signification
physique. Nous allons procéder ici d'une façon quelque peu différente,
d'une part pour montrer que l'opérateur  n'admet pes en toute rigueur de limite
pour , et d'autre part parce que la
méthode que nous allons employer est celle qu'utilisent de nombreux auteurs

pour introduire la matrice S et calculer la section efficace de collision.

% 150

a Nous allons calculer l'amplitude de probabilité
 pour que le système qui était à l'instant -1 dans l'état libre représenté dons
 le point de vue d'interaction par le ket fixe  se trouve à l'instant +T dans
l'état Nous étudierons ce que devient cette quantité lorsque t tend vers
l'infini et dans quelle mesure il est possible alors
de lui donner un sens,

L'équation (83), compte tenu de , conduit pour

%
 à l'équation intégrale :

dont le développement par itérations successives est :

Ce développement (96) a été rencontré à diverses reprises sous des
formes légèrement différentes : en effet, d'après (82), on a
\[
\tag{97}=
\]
D'autre part, pour t > O, U (rt, -1) est identique à l'opérateur
fonction de Green avancée x, (rt, -t) (cf formule 39, page 82),

On a donc alors

\[
\tag{98}=
\]

D'autre part, l'opérateur fonction de Green avancée du système
 s'écrit :

libre  s'écrit

()
\[
\tag{99}=
\]
\[
\tag{100}=
\]

% 151

Compte tenu de (79-b), (98) et (99), le développement (96) est identique au
développement en série de Neumann-Liouville de l'opérateur fonction de Green
avancée (cf formule 3, page 83).

Il faut cependant remarquer que dans le développement (96),
les temps sont rangés  dans l'ordre  et que les

opérateurs  et 

ne commutant pas, leur ordre dans les intégrales doit être conservé.
A l'aide de (96), nous allons étudier le développement en
puissance de V de l'amplitude de probabilité

\subsubsection{}%
a) Terme d'ordre zéro :

\subsubsection{}%
b) Terme d'ordre 

Il correspond à ce terme l'ensemble des diagrammes

qui décrivent une transition à l'instant t' de l'état i à l'état j sous

l'effet de l'interaction V. Le terme du premier ordre représente  interférence
des chemins correspondant aux diverses valeurs de 

est pondéré par son amplitude de probabilité  Visé

Chaque chemin

%
 
Nous allons voir que l'interférence de ces différents
chemins sera destructrice si la phase varie trop vite d'un chemin
à l'autre, c'est-à-dire si

\[
\tag{101}=
\]

Nous interprétons ainsi la conservation de l'énergie
(plus précisément la "quatrième" relation d'incertitude "temps-énergie") en mécanique quantique,

De façon plus précise, nous avons

ce qui, en posant :

nous conduit à écrire le terme d'ordre 1 sous la forme.

\[
\tag{103}=
\]

Le fonction  est la fonction de diffraction classique, normée, représentée ci-dessous

%
Cette fonction a une largeur de l'orére de , ce qui justifie
l'inégalité (101).

Cependant, pour une valeur fixée de cette fonction
n'admet-pas de limite lorsque . L'opérateur  n'a donc
pas de limite lorsque  puisque son élément de matrice
le , au premier ordre, n'en a pas. Physiquement,
cette absence de limite se comprend aisément : l'introduction brutale
à un instont donné  d'un état propre  du hamiltonien libre T
feit apparaître un régime "transitoire" qui subsiste à l'instant rt (quel
que grand que soit T) et qui explique l'oscillation et l'absence de limite
lorsque T +, de  Nous aurions évité ces
régimes transitoires gênants et qui n'ont pes de sens physique en introduisant
dans V un facteur de convergence (branchement adiabatique) ou en
introduisant progressivement l'état initial.

Nous pouvons également remarquer que les états physiques réels
sont normés (paquets d'ondes) et qu'en conséquence, les éléments de matrice
de  doivent être celculés sur des fonctions de cnrré sommable,
les éléments de matrice  agissant sur ces fonctions
comme des distributions. Nous savons alors, qu'au sens des distributicens,
 tend vers la distribution de Dirac  lorsque
, Nous pouvons donc dire que le terme du prenier orêre du développement
de  admet, lorsque  , une limite au sens des
distributions agissant sur les fonctions d'onde de carré sommable, cette

limite étant

% 15

\subsubsection{}%
c) Termes d'ordre 2 :

Au terme (104) correspondent les diagrammes du type :

qui décrivent l'évolution du système de l'état  à l'état
 via l'état  sous l'effet de deux transitions aux
instants t" et t'. Pour obtenir le terme du deuxième ordre (104),
il faut faire interférer tous les chemins de ce type en sommant
sur t' compris entre =T et +t , puis sur t" - t' compris entre
 et O et enfin sur l'état intermédiaire k,. Nous pouvons enfin
faire tendre t vers l'infini pour obtenir, comme au premier ordre,
une distribution.

% 155

\[
\tag{(105)}=
\]
Fixons d'abord l'état intermédiaire  et sa durée
La sommation sur l'exponentielle  fait, comme
pour le premier ordre, que 1'interférence est destructrice si

L'amplitude de probabilité est encore nulle si l'énergie de l'état
initial n'est pas érale à celle de l'état final :

Supposons maintenant que  fixons l'état intermédiaire  de
et sommons sur la durée de la transition intermédiaire  : Deux
cas se présentent alors

a) le facteur de phase est le même quel que soit
(égal à 1) : tous les chemins interfèrent de façon constructive et la durée
de la transition intermédiaire peut prendre toutes les veleurs possibles :
on dit que l'on a une trans sition intermédiaire réelle
d'un chemin à l'autre, la phase varie. Il va y avoir une
interférence partiellement destructive : il faut calculer l'intégrale

en posant u=t"-t'.

Cette intégrale n'admet pas de limite au sens des fonctions, mais nous savons
qu'au sens des distributions, elle tend vers  qui, pour

 est égal à 

% 156

Nous obtenons, à un facteur 2 près, le même résultat si

dans  (105), on prend comme borne inférieure

En effet
C'est dans cette mesure que l'on peut dire que ces transitions
que l'on appelle virtuelles, car elles ne conservent pas l'énergie, durent un temps de l'ordre de.

 

Si on résume les résultats précédents, le terme d'ordre
 du développement de  tend au sens des distributions, lorsque , vers

Si on résume les résultats des \S a), b) et c), on obtient :

(au sens
des distributions)

Soit, en rapprochant ce résultat de la formule (77) de la page 142 :

\[
\tag{106}=
\]

% 157

La formule (106) nous montre donc, qu'au sens des
distributions sur les foncticns d'onde de carré sommable,
 tend vers 8; et on retrouve ainsi les résultats du \S 2,

Remarque  importante : Calculons la probabilité de transition
 à la limite . Si i  j, la relation
(106) nous conduit à l'expression

qui fait intervenir le carré d'une distribution , ce ce qui n'a à pas. e

sense L'erreur à ne pas commettre est en effet de faire tendre T, qui

ion, vers l'infini avant

ique de La coili

représente le temps
d'avoir achevé Le calcul des grandeurs physiques.
Si on se souvient que  provient de la fonction

 on doit envisager en fait la fonction

qui est représentée sur la figure ci-dessous :

% 158

et que l'on peut encore écrire :

tend donc au sens des
Le fonction

distributi ver et est équivalent à .
On en déduit :

d'où l'existence d'une probabilité de transition par unité de
temps :

analogue à celle déjà trouvée par ailleurs (formule 68, page 136),
et permettant de retrouver la section efficace de collision.

Le principe à suivre est donc, comme toujours, de ne faire tendre
le temps de la collision (r ou Y/e) vers l'infini qu'un fois tous
les calculs mathématiques achevés et les grandeurs physiques obtenues,

% 159
\subsection{}%
4°) Collisions entre particules identiques

Nous n'avons envisaré jusqu'à présent que le problème de
collision de particules différentes. Nous allons voir maintenant
que le formalisme de la matrice S permet de traiter aisément, en introduisant le
principe de Pauli, le problème des collisions de particules identiques (bosons ou
fermions). Nous allons au préalable
analyser à nouveau une expérience de collision entre particules différentes :
\subsubsection{}%
a) Particules différentes

Envisageons, dans un problème à symétrie cylindrique, la diffusion de deux
particules différentes A et A dans le référentiel
du centre de masse. Soient O et 5 les angles polaires des directions
incidentes de A, et de À, et plaçons un détecteur dans l'angle polaire 
L'amplitude de probabilité pour que la diffusion de A ait iieu dans l'angle 68
(figure a) peut s'écrire :

\[
\tag{8}=
\]
 est 1ié simplement  à  qui permet de calculer la probabilité de transition et
 la section efficace.
De même l'amplitude de probabilité pour que le diffusion de
ait lieu dens l'angle 6 (figure 8) peut s'écrire :

% 160
Figure

Figure 8

La probabilité de trouver la particule A dans la
direction 6 est , La probabilité de trouver la particule  dans La direction  est
 Si maintenant,
le détecteur ne permet pas de distinguer les particules différentes , et  et
détecte donc indifféremment  ou A, la
probabilité de trouver l'une quelconque des particules dans
direction  est alors .
\subsubsection{}%
b) Particules identiques

Le processus physique est maintenant le suivant

deux particules identiques du type  se dirigent l'une vers
l'autre dans le référentiel du centre de masse. Elles interagissent et on cherche
la probabilité pour qu'un détecteur,
la direction 6, en détecte une, On cherche donc l'amplitude

processus unique représenté par la figure ci-dessous :

la

dans
du

 
% 161

Du point de vue mathématique, humérotons les particules et

envisageons les quatre processus formels représentés par les figures

et les amplitudes de probabilité ci-dessous :

% 162

Du fait de l'indiscernabilité des particules, le
hamiltonien H, et par suite S, sont invariants par permutation
des particules 1 et 2, En désignant par P l'opérateur de permutation, on a

D'où l'on déduit :

ce qui permet d'identifier les amplitudes  et  égales à
 et par un raisonnement analogue les amplitudes  et 
égales à 

Quels sont maintenant les états mathématiques 
et qui représentent l'état physique initial  et

l'état physique final 

D'après le postulat de symétrisation, ce sont les deux
États, symétriques ou antisymétriques, selon qu'il s'agit de bosons
ou de fermions :

+1 bosons.

-1 fermions)

% 163

L'amplitude de probabilité associée à la diffusion s'écrit

alors :

soit, en se reportant aux expressions des amplitudes (1) (2) (3) (k) :

Par suite, la probabilité du processus de diffusion dans l'angle 0 sera

ce qui est différent de  qui était 1a probabilité pour que le détecteur détecte
une particule (sans préciser laquelle),
lorsque les deux particules sont différentes.

Lorsque les particules sont identiques, il y a donc interférence entre les
amplitudes de probabilité associées eux deux schémas
de diffusion :

% 164

Ceci est particulièrement frappant pour la diffusion à angle
droit  :

pour deux particules différentes : probabilité

pour deux bosons : probabilité
pour deux fermions : probabilité

\subsubsection{}%
c) Interprétation dans le formalisme de Feynman

Dans le formalisme de Feynman, l'amplitude de probabilité
associée à un processus de diffusion quantique s'obtient en sommant
les contributions de tous les chemins classiques H partant de l'état
initial et aboutissent à l'état final. On peut dans ces chemins classiques
suivre d'un bout à l'autre la trajectoire de la particule et

il existe alors deux types de chemins aboutissant au même état final :

Il faut sommer les amplitudes de probabilité associées aux chemins du

type.  On obtient .

% 165

De même, pour les chemins du type (b), on obtient ,
11 faut maintenant tenir compte de l'interférence entre les chemins
du type (a) et ceux du type (t). Pour cela, nous devons introduire

dans notre formalisme un postulat supplémentaire de symétrisation qui

indique qu'il faut sommer les chemins du type (a) et (b) avec un facteur de
phase égal à +1 ou -1 suivant qu'il s'agit de bosons ou de
fermions.

Si les particules sont différentes, les deux types de chemins (a) et (b)
ne peuvent interférer, car ils conduisent à des états
finaux différents.

Si les particules sont identiques, les deux tyres de chemins

conduisent au même état final : ie système peut donc prendre

l'autre type de chemins et tant qu'on ne fait pas de mesure pour se
voir quel chemin il emprunte, il y a interférence entre les emplitudes
de probabilité associées aux deux tyres de chemins.
\subsubsection{}%
d) Exemple d'application : diffusion de deux électrons
Supposons l'énergie suffisamment basse pour que l'interaction ne
dépende pas des spins. Deux cas peuvent alors se présenter :

a) cas où les deux spins sont parallèles :

Les deux particules sont alors identiques et la probabilité de

diffusion à angle droit est nulle,

% 166

B) cas où les deux spins, sont antiparallèles :

Les particules sont alors différentes vis-à-vis de la
détection et la probabilité de détecter un électron, sans préciser
lequel, à angle droit est alors .

Ainsi, bien que les forces ne dépendent pas des spins, la section
efficace dépend de façon critique de l'orientation relative des
spins. Il y a une analogie très nette avec les termes d'énergie
d'échange, introduits par le principe d'exclusion de Pauli qui
distinguent, en dehors de toute interaction magnétique, les états
singulets et triplets de l'atome d'hélium.

) Cas de deux électrons non polarisés :

Chacun des processus possibles :

sois

probabilité

s'effectue avec une probabilité 1/h, La probabilité totale de détec
tion à angle droit est donc dans ce cas .

 
% 167

\section{}%
F - Diffusion par un système. de N particules dans l'approximation de Born

\subsection{}%
1°) Introduction

Au lieu d'envisager, comme nous l'avons fait jusqu'ici la diffusion per un seul
centre, nous allons étudier maintenant la diffusion par
un système constitué par N particules (par exemple,les divers électrons d'un
atome).

Cette étude va nous permettre de relier les sections efficaces
de diffusion à des grandeurs physiques importantes, caractéristiques du système de
N particules : la densité simple, la fonction de corrélation spatiale
(densité à deux corps), 1e fonction de corrélation spatio temporelle,

Toutes ces grandeurs physiques ont une interprétation cleire et
simple. Il est intéressant de pouvoir les déduire de la mesure des sections
efficaces et de comparer les résultats aux prévisions de a mécæiique statistique
qui permet de les calculer théoriquement.

Nous ferons le calcul des sections efficaces dans le cadre de
l'approximation de Born (ou ce qui revient au même, comme nous l'avons vu dans
le cadre de la règle d'or de Fermi). Cette approximation est valable pour la
diffusion des photons, des électrons rapides et des neutrons lents,

Nous mettrons ici principalement l'accent sur l'établissement
des relations entre les sections efficaces et les diverses fonctions
caractéristiques ainsi que sur la validité des approximations, Pour
l'étude de l'aspect "problème à N corps" proprement dit, on peut se rapporter à l'article de
Van Hove (Phys. Rev. 95, 249, 1954) ou à l'ouvrage de Nozières ("le problème
à N corps").

% 166

\subsection{}%
2°)Notations

\subsubsection{}%
a) Diverses particules

Soient

système,  la position de leur centre de masse et  la coordonnée relative

les coordonnées des N particules du

de la ième particule. On a évidemment

 est le vecteur d'onde du centre de masse après la collision (il est supposé
nul avant la collision).

 est la position de la particule incidente,  son vecteur d'onde avant la
collision,  son vecteur d'onde après la collision;  le transfert au cours de
la collision :

D'autre part, l'énergie transférée par la particule incidente au cours de la
collision est

\subsubsection{}%
b) Etats du système

L'état des N particules diffusantes dans le système de leur centre
de masse est respectivement avant et après la collision :

| nm > d'énergie E,

| n> d'énergie E,

% 169

si bien que l'on pourra noter l'état du système global en interaction avant

et après la collision par :

d'énergie

énergie
 d'énergie

en appelant m et M respectivement les masses de la particule incidente et de
l'ensemble du système diffuseur.

Dans la représentation r, ces états sont représentés par les
fonctions d'onde

et

\subsubsection{}%
c) Interaction entre le système diffuseur et la particule incidente
Nous supposerons que le potentiel d'interaction ne dépend que
de la position relative de la particule incidente et des particules diffusantes,
supposées identiques et le hamiltonien d'interaction peut s'écrire :

marque :

Il semble que le cas où la particule incidente est un photon échappe à cette
théorie. Nous verrons qu’il n'en est rien et que les calculs que nous. allons

faire s'étendent aisément à ce cas.

ment de matrice  de  entre l'état initial

\subsection{}%
3) Calcul de

et l'état final

Calculons l'élément de matrice 

Il s'écrit :
 
% 170
\[
\tag{112}=
\]
\[
\tag{113}=
\]
Or nous pouvons écrire

L'expression (111) s'écrit alors

Nous allons intégrer cette expression successivement sur les différentes variables :
a) Intégration sur

b) Intégration sur  :
La distribution , introduite par l'intégration précédente conduit à
remplacer dans (113)  par .
L'élément de matrice devient alors
 
% 171
\[
\tag{114}=
\]


ce qui peut encore s'écrire, après une transformation évidente

Soit en prenant comme nouvelles variables Le  et  :

c) Intégrations sur  et 

Introduisons la transfornée de Fourier du potentiel V

valeur YX :

Il vient alors :

L'intégration sur Fest immédiate :

Finalement, l'élément de matrice de H, entre les États final et initial s'écrit :

L'expression (114) est essentielle pour le calcul des sections efficaces  l'approximation
de Born. Elic se présente sous une forme remarquablement simyile

- le premier terme  décrit la conservation de l'impulsion globale au
cours de la diffusion : La quantité d'impulsion x cédée par la perticule incidente
est transmise intégralement au système diffuseur, Ceci étant, l'impulsion 

% 172

 ne joue pas de rôle essentiel dans notre problème et nous pouvons ne
plus en tenir compte : si la messe M est grande devant la masse m de la particule
diffusée, l'énergie cinétique d'ensemble ER est négligeable devant

mouvement

les autres énergies du problème. On peut donc négliger l'ensemble du

système diffuseur (ce qui revient à confondre r; et P;) étant bien entendu
que ce mouvement existe et satisfait à la conservation de l'impulsion globale.

 le deuxième terme , transformée de Fourier à le valeur x du potentiel

V ne dépend que de l'interaction ob du transfert d'impulsion (et pas de l'état

quantique du système difluseur

- le troisième terme  dépend de l'état initial et de
il
l'état final du système diffuseur à N particules (dans le référentiel de son
centre de masse) (mais ne dépend pas du potentiel).

Remarque  cas des photons

\[
\tag{115}=
\]

Montrens que la diffusion des photons (de longueur d'onde suffisamment courte) par
un système de  électrons liés (atome) conduit à une expression analogue à (11).
Le hamiltonien d'interaction s'obtient en développant le terme
 où  représente la masse de l'électron,  sa charge, 

l'impulsion du électron et  (61) la valeur eu point  du potentiel
vecteur.

L'interaction s'écrit alors

Pour des longueurs d'onde suffisamment courtes (diffusion de rayons X), on montre

que le dernier terme est prépondérant et on peut alors écrire

% 173
\[
\tag{116}=
\]


Or le potentiel vecteur du champ électromagnétique quantifié s'écrit sous
la forme
où EL porte sur tous les vecteurs d'onde et  sur tous les états de polarisation,
représente le vecteur de polarisation, à x et , les opérateurs
de création et d'annihilation d'un photon dans 1'état K,  et a un coefficient

dépendant notamment de 

Calculons l'élément de matrice de H, l'atome passant de l'état
l'état | n > et le photon de l'état  à l'état.
Les seuls termes dont la contribution ne sera pas nulle seront ceux qui contiendront
le produit  à et on aura :

ce qui est bien une expression analopue à (11) dans laquelle  est ren

placé par On ne retrouve pas le terme, mais cela est
normal car nous n'avons pas tenu compte ici des degrés de liberté de translation
du système atomique global,

\subsection{}%
) Diffusion élastique ou cohérente

\subsubsection{}%
a) Définition et exemples :
On dit que la diffusion est élastique (ou cohérente) si l'état quantique du système
diffuseur ne change pas au Cours de la diffusion :

C'est par exemple le cas de la diffusion Thomson des photons X par un atome :

Dans ce cas la conservation de l'énergie au cours de la diffusion exige

 En k (on néglige l'énergie cinétique d'ensemble)

et on a donc

%
\[
\tag{117}=
\]
\[
\tag{118}=
\]
\[
\tag{119}=
\]
\[
\tag{120}=
\]

\subsubsection{}%
b) Calcul de la section efficace :

A l'approximation de Born, la section efficace est donnée par
la fornule (78) de la page 142, qui s'applique mème au cas de la diffusion
pur un centre diffuseur complexe puisqu'elle peut s'établir à partir de la
règle d'or de Fermi :

Nous devons calculer la quantité

que l'on peut encore écrire :

ce qui n'est autre que la probabilité pour que la 17m particule soit au point
Fr, In sommant cette expression sur i, dans l'expression (118), on obtient la
probabilité globale pour qu'une particule quelconque soit au point F, c'est-à-dire
encore la densité à un corps Pa (M). Finalement, d'après (118) :

et d'après (11) et (117), on obtient la section efficace de diffusion élastique

qui dans le cas de la diffusion des photons s'écrit

% 175

Remarques :

- Il est clair, d'après la définition de la densité à un corps, que

 Rappelons en effet que

- La transformée de Fourier de la densité à un corps qui intervient dans la
section efficace s'appelle le facteur de forme du système diffuseur.

- On voit que la section efficace de diffusion élastique se met sous la forme
remarquablement simple du produit de deux termes dont l'un dépend de l'interaction,
mais non pas du système diffuseur et l'autre du facteur de forme
du système diffuseur et non pas de l'interaction.

On.conçoit ainsi que les expériences de diffusion élastique sont
très utiles pour l'étude de 1s répartition des particules dans le centre diffuseure

Dans le cas de la diffusion par un système cristallin, la formule
(119) contient notamment la formule de Bragge
\subsubsection{}%
c) Discussion de la diffusion Thomson

On appelle "diffusion Thomson" la diffusion élastique de photons X
par un atome : la diffusion  fait elors sans chengement de fréquence et la

section efficace différentielle de diffusion est donnée par la formule (120) qui

s'écrit, en posant = re (rayon classique de l'électron) :

% 
Vous devons envisarer deux cas limites intéressants :

a) Photons X mous : Le longueur d'onde À des photons est
grande devant les dimensions atomiques (de l'ordre de quelques angstrôms
par exemple). Le facteur oscillant  reste sensiblement égal à 1 dans
le domaine où la densité est importante, si bien que l'on a

 (Z : nombre d'électrons)

La relation (121) se réduit alors à

Le diffusion est isotrope et la section efficace est proportionnelle au carré

de la charge électronique,

8) Photons X "durs" : La longueur d'onde À est petite devant les
dimensions atomiques. Le facteur e oscille un très grand nombre de fois
dans le domaine d'extension de , sauf si | x | est suffisamment voisin
de zéro : l'intégrale  se moyenne alors à zéro, sauf pour
La diffusion est fortement concentrée dans la direction avant.

\subsection{}%
5°) Cas général : section efficace
\subsubsection{}%
a) Définition :
En général la diffusion n'est pas élastique, c'est-à-dire que le
centre diffuseur peut changer d'état quantique. Dans ce cas, l'énergie de la
particule incidente,  est modifiée par la diffusion et devient :

On peut alors définir une section efficace différentielle de diffusion, plus générale
que celle que nous avons envisagée jusqu'à présent, que nous appellerons

% 177
\[
\tag{122}=
\]
\[
\tag{123}=
\]

 et qui est telle que aide dfde est égal au quotient par le flux
de la particule incidente, de la probabilité par unité de temps pour que la
particule ait été diffusée dans l'angle solide df autour de Q avec une énergie comprise
entre $\epsilon$ et $\epsilon$ +de. On voit que cette section efficace est une
quantité plus complexe que la section efficace puisqu'elle exige pour
sa mesure à la fois une analyse en direction et en énergie de la particule
diffusée. On doit s'attendre également à ce que sa mesure fournisse des
renseignements plus riches sur la répartition des particules dans le centre
diffuseur. :

\subsubsection{}%
b) Calcul de 

La probabilité par unité de temps de voir la particule diffusée
dans l'angle solide  avec un nombre d'onde compris entre k et k + dk

s'écrit, en appliquant la règle d'or de Fermi :

L'état initial du système diffuseur est  on somme sur les états finaux
 puisqu'on n fobserve que ia particule diffusée.
En remarquant que  , que le flux incident est 

et que la distribution  peut s'écrire

on obtient la section efficace érentiente :

On peut généraliser cette expression en cons sidérent que l'état initial du centre
diffuseur n'est pas l'état  , mais un état statistique représenté par la
matrice densité

% 178
\[
\tag{127}=
\]
\[
\tag{128}=
\]


On obtient alors

Soit, en posant

La section efficace se présente ainsi comme le produit de deux quantités dont
l'une, , ne dépend que de la nature de l'interaction et l'autre,
ne dépend que du centre diffuseur. Nous allons maintenant étudier cette dernière quantité.
\subsubsection{}%
c) Transformation de 

En utilisant la forme intégrale de la distribution de Dirac :

il vient

Soit en tenant compte de la relation de fermeture i de la
relation qui permet de passer de la représentation" de Schrödinger à celle de
Heisenberg :

% 179
Posons

On peut alors écrire

\[
\tag{129}=
\]

avec

où la notation < > représente la valeur moyenne dans l'état initial représenté
par la matrice densité

Remarque : I1 apparaît dans l'expression  le produit de convolution de deux

distributions 6. Les opérateurs  et  relatifs à deux instants différents
ne commutent pas et il faut donc respecter l'ordre des termes dans l'expression (130).

\subsection{}%
6°) Interprétation physique Approximation quasistatique

 

Afin d'interpréter physiquement l'expression relativement complexe

de , nous allons tout d'abord analyser le cas simple où  = 0. Nous allons
tout d'abord montrer que le quantité G  intervient dans une approximation
très fréquente, l'approximation quasistatique.

% 180
\[
\tag{131}=
\]

\subsubsection{}%
a) Section efficace :

Très souvent, on ne mesure pas à Ja fois l'angle de diffusion

et l'énergie de la particule diffusée, On mesure seulement l'angle de diffusion.
Au lieu de s'intéresser à la section efficace différentielle

double  on s'intéresse alors à la section efficace intégrée sur
l'énergie

Cette quantité n'est en général pas simple à calculer. En effet, nous avons
posé (relation (127))

et la quantité dépend en général de  comme il est facile de s'en assu
rer sur l'expression (125) et sur la figure de la page 168 :  dépend de x
par l'intermédiaire de  et du rapport  et pour un angle de diffusion
 donné, x lui-même dépend du transfert d'énergie .

Il faut donc, dans l'intégration (131), tenir compte d'une dépendance explicite

pendant un cas où le problème est considérablement simplifié, c'est celui de

et implicite en w de » Ce qui rend le calcul difficile. Il existe cel'approximation quasistatique.

\subsubsection{}%
b) Approximation quasistatique :

Faisons l'hypothèse que le transfert d'énergie est très faible
devant l'énergie de la particule incidente, Cela s'exprime sur la figure
(pe 168) par la relation BC << OA,
C'est le cas par exemple de la diffusion des rayons X par un atome (le déplacement
Compton est alors très petit devant l'énergie des photons X), ou celui

de 1a diffusion d'électrons rapides (10 Kev) par un atome,

% 181
\[
\tag{132}=
\]
\[
\tag{133}=
\]

Dans ce cas, le triangle AOB est pratiquement isocèle et on
peut négliger la variation de x avec  : en effet, M reste alors pratiquement égal
à CA et ne dépend plus que de l'angle de diffusion :

Dans l'intégration (131) sur , À reste alors constant et peut être sorti
du signe somme. D'autre part  n'est plus qu'une fonction explicite
de  :

D'autre part, d'après (129)

d'où

Nous voyons alors que la section efficace se factorise une fois de plus

en deux termes dont l'un, , ne dépend que de l'interaction, et l'autre ne
dépend que du centre diffuseur par l'intermédiaire de .

\subsubsection{}%
c) Sens physique de :

 et  qui sont deux opérateurs relatifs au même instant
commutent (à la différence de  et ). On peut done les considérer

comme des nombres, et dans L'expression

effectuer le produit de convolution. I1 vient

% 182
\[
\tag{136}=
\]


Dans la double somme séparons les termes i = j et. Les termes

i = j donnent

On a donc finalement

avec

Afin d'interpréter g (T), fixons i et considérons tout d'abord la quantité

Si l'on réserve la sommation sur j, l'intégration sur les variables autres que

Fi représente la probabilité pour que la particule i soit en A et la particule
j en  .

L'intégration sur Tr; donne la probabilité pour que les particules
i et j soient distantes de re

La sommation sur  donne alors  ou encore la probabilité
pour qu'une particule quelconque (autre que i) soit à une distance r de la
particule i.

% 163

 représente donc la densité au point r du système, vue à partir de
la particule i.

On doit donc avoir , ce que l'on vérifie
aisément :

or

Revenons maintenent à 

qui s'interprète maintenant comme étant la densité moyenne au point r du
système vue à partir d'une particule quelconque du système (la position de
cette particule étent prise pour origine )e

On peut encore dire que  est la probabilité pour qu'une

particule quelconque du système étant en un point, une autre particule

quelconque du système soit, au même instant, à une distance vectorielle
de ce pointe

 mesure donc les corrélations statiques entre les particules et est relie
à la fonction de distribution à deux corps de la mécanique statistique.

% 184
\[
\tag{137}=
\]
\[
\tag{138}=
\]
La figure ci-dessous représente l'allure générale de  dans un fluide

de particules neutres

 attraction

disparition de la corrélation

 répulsion

Nous sommes maintenant en mesure de justifier le terme d'approximation
quasistatique : dans cette approximation, la diffusion ne dépend du système diffuseur
que par l'intermédiaire de la quantité  qui décrit

les corrélations entre les positions de deux particules à un instant donné.

Nous allons maintenant préciser le lien entre et .
) Lien entre  et  :

Nous pouvons d'après (132) écrire

ou encore pour la diffusion des photons :

permet donc d'atteindre la transformée de Fourier de .

La mesure de

% 185

\[
\tag{139}=
\]
\[
\tag{140}=
\]

Exemples d'application de la formule précédente :

Revenons à la diffusion des rayons X par un atome :
Nous avons vu que la section efficace de diffusion élastique
(diffusion Thomson) était donnée par la relation

Lorsque la diffusion s'accompagne d'un changement de fréquence du photon,
on dit qu'il s'agit de la diffusion Compton. La formule (138) donne la
section efficace globale, compte tenu de la diffusion élastique et de la
diffusion avec changement de fréquence. I1 s'agit donc de la section efficace des processus Thomson et Compton :

Distinguons, comme nous l'avons déjà fait, le cas des rayons X mous et des
rayons X durs 3

a) Photons X mous :  est  devant l'extension des

fonctions  et  : on peut remplacer e par 1 dans l'intégravion

des formules (139) et (140),

La diffusion Thomson donne une section efficace

et 1a diffusion "Thomson + Compton" une section efficace 

On en déduit que la diffusion est élastique (uniquement Thomson).

Le recul produit par l'impact du photon X mou, peu énergique, est "encaissé"
l'atome tout entier qui reste dans le même état quantique : le photon ne change
pas de fréquence. Il existe une certeine enalogie avec l'effet Méssbauer âans
lequel le recul est "encaissé" par un réseau tout entier et non par les atomes
individuels qui le constituent.

% 186
\[
\tag{141}=
\]

 Photons X durs : L est petit devant l'extension des

fonctions  et . L'exponentielle 

oscille très rapidement
pour , ce qui rend négligeables les intégrales correspondantes, Si on
exclut la diffusion en avant, on a

tandis que la diffusion "Thomson + Compton" donne

La diffusion, dans les directions autres que la direction avant,
est uniquement du type Compton (avec changement de fréquence).

Si on se souvient que  est la section efficace
différentielle de diffusion classique d'un électron, tout se passe alors
comme si les Z électrons diffusaient indépendamment les uns des autres : Le
recul produit par l'impact du photon X dur, très énergique, est encaissé non
plus par l'atome, mais par les électrons individuels.

La mesure de la section efficace de diffusion des photons X durs
est donc proportionnelle au nombre d'électrons diffuseurs et permet la mesure

du numéro atomique (cf expériences de Bækla).
 
% 187

\subsection{}%
7°) Interprétation physique et propriétés essentielles de
\subsubsection{}%
a) Interprétation approchée :

En première approximation, on peut ne pas tenir compte du
fait que  et ne commutent pas dans l'expression (130), les
considérer comme des nonbres et effectuer le produit de convolution entre
les deux distributions de Dirac. On assimile alors  à la quantité

ce qui constitue l'approximation classique

Par un raisonnement analogue à celui développé à propos de
, on montre que cette quantité, , que l'on appelle la
Fonction de Corrélation spatio-temporelle de Van Hove, représente la probabilité
pour qu'une particule quelconque du système étant en un point donné,
une particule .quélconque du système (la même ou ue autre) soit à une distance
vectorielle ? de ce point, un instant t plus tard
\subsubsection{}%
b) Effets quantiques :

Cependant le fait que  et  ne commutent pas fait que
 n'est pas égal à cette différence est responsable d'effets
purement quantiques,

Par exemple  est manifestement réel alors que 

est en général complexe. En effet, on peut écrire, d'après (130)

 ne commutent pas.

De façon plus précise, en effectuant le changement de variable muette

% 188

L'état initial du système diffuseur

étant stationnaire, la valeur moyenne  n'est pas modifiée par une
translation dans le temps de  si bien que l'on a :
On aboutit donc à la relation

\[
\tag{142}=
\]
qui était évidente a priori puisque d'après (129), elle constitue la condition
nécessaire et suffisante pour que  soit réel.
\subsubsection{}%
c) Valeur asymptotique de 
On doit s'attendre à ce que les corrélations disparaissent
lorsque  où . Or en l'absence de corrélations, la valeur
moyenne d'un produit est égele au produit des valeurs moyennes. On a donc

Remarque : On aurait dû en fait écrire . L'état initial du
système diffuseur étant stationnaire, les densités ne dépendent pas du temps.
On déduit donc d'après (130) :

asymptotique

Pour un système homogène p est constant et égale à N/V.
On a alors

% 189
\[
\tag{144}=
\]
\[
\tag{145}=
\]
La quantité physiquement intéressante, pour la mesure des corrélations
spatio-temporelles est la différence entre  et sa partie asymptotique, si bien que nous poserons :

On a alors

Or la deuxième intégrale s'écrit

et représente le produit de  par la transformée de Fourier d'un produit

de convolution, qui est égale au produit des transformées de Fourier. On a

donc

D'où

On en déduit donc que la contribution à  de la valeur asymptotique de
 permet de décrire la diffusion élastique alors que la contribution de
 permet de décrire la diffusion inélastique.
\subsubsection{}%
à) Portée de la corrélation. Temps de corrélation :

Nous avons vu que c'est finalement la quantité  qui
mesure les corrélations spatio-temporelles entre les particules du centre

diffuseur. Cette quantité tend vers zéro lorsque

% 190
\[
\tag{146}=
\]
\[
\tag{147}=
\]

Physiquement, on peut dire que si une particule passe en un
point donné à un instant donné, elle perturbe la distribution des particules
aux points environnants et aux instants voisins. C'est cette perturbation qui
est représentée par . Elle s'évanouit à une distance suffisante et
au bout d'un temps suffisamment longs qualitativement, on peut définir une
longueur Re telle que  soit négligeable quel que soit t si
 représente la portée de la corrélation.

De même, on peut définir un temps T, tel que ,  soit
négligeable quel que soit r si  représente le temps de corrélation ou de relaxation,
au bout duquel la perturbation a disparu.

Les notions de portée et de temps de corrélation sont très importants. Leur sens
physique est très clair, mais il est difficile en général
de les calculer.

On peut qualitativement dire, la "partie inélastique" de S (X)
étant la transformée de Tourier de , que la diffusion inélastique

n'est importante que si les inégalités suivantes sont satisfaites

\subsubsection{}%
e) Condition de validité de l'approximation quasistatique :
Nous pouvons maintenant revenir aux conditions de validité de

l'approximation quasistatique.

colors
% 191


\[
\tag{148}=
\]

\[
\tag{149}=
\]

Nous avons vu qu'il faut que le triangle AOB soit pratiquement isocèle et
notamment on doit avoir

Or , avec 

D'où  (d'après(147))

en appelant V, la vitesse de la particule incidente.

De même AB = | X | (d'après(1l6))

Finalement l'inégalité (148) s'écrit

soit R,

TD est le temps mis par la particule pour parcourir une distance égale à la
portée de la corrélation.

Si ce temps T est très court devant le temps de relaxation To la particule
incidente va trop vite pour être sensible aux corrélations temporelles : la
diffusion est décrite par la corrélation spatiale . On peut dire

 à un instant

encore que la particule diffusée voit le centre diffuseur fig
donné si elle va suffisamment vite. D'où le nom d'approximation quasistatique
et le signification physique de l'inégalité (149).


% 226
%III = DUREE DE VIE D'UN ETAT INSTABLE
\chapter{Durée de vie d'un état instable}

\section{Introduction}
\subsection{Position du problème}%1°)  :

La théorie que nous allons développer dans ce chapitre va nous permettre de préciser et d'étudier la notion de durée de vie d'un état quantique
préparé à un instant donné et évoluant irréversiblement, sous l'effet d’une
interaction donnée, vers les états d'un "continuum" d'énergie avec lequel il
est couplée.

C'est par exemple le cas de la désintégration d'une particule élémentaire :
\begin{center}
méson $\pi$ $\to$ méson $\mu$ + neutrino

$\pi^+$ $\to$  $\mu^+$ + $\nu$
\end{center}
ou encore celui de l'évolution irréversible d'un état atomique excité sous
l'effet du couplage avec le champ électromagnétique : l'émission spontanée.

On peut, en général, poser le problème de la durée de vie de la façon
suivante : un système physique (qui peut, par exemple, être composé de deux
parties) possède un hamiltonien H$_0$ + V (H$_0$ décrivant éventuellement l'énergie des
deux parties en présence et V 1‘interaction qui existe entre elles).

Le hamiltonien H$_0$, possède un spectre continu, c'est-à-dire un spectre
dans lequel l'énergie varie continuement à partir d'une valeur $E_0$, les états
propres correspondants a', b', n'étant pas normés (Leur norme n'est pas finie. Ils sont cependant "nornés" au sens de la distribution de Dirac). Superposé à ce spectre continu,
% 227
le hamiltonien H$_0$ possède un spectre discret, c'est-à-dire qu'il existe

\begin{center} \begin{tikzpicture}
\draw [->] (0,0) -- (7,0);
\draw (0,-0.07) -- (0,0.07) node [above]{$E_0$};
\draw (2.5,-0.07) -- (2.5,0.07) node [above]{E$_\mt{a}$};
\draw (4.7,-0.07) -- (4.7,0.07) node [above]{E$_\mt{b}$};
\draw [dashed] (-3,0) -- (0,0);
\end{tikzpicture} \end{center}

des états propres de H$_0$, | a >, | b > etc.., de norme finie dont l'énergie,
prenant des valeurs discrètes coïncide avec certaines valeurs du spectre
continu.

On suppose qu'à l'instant t $=$ 0, on peut, par un procédé quelconque,
préparer le système dans l'un de ces états, | b > par exemple,

| b >, état propre de H$_0$, n'est pas un état propre de H. Sous
l'effet de l'interaction V, le système va évoluer, l'état | b > se vidant au
profit des états du continuum d'énergie avec lesquels il se trouve couplé.

Les deux questions qui se posent alors sont :

Comment l'état instable | b > se vide-t-i1 ?

Comment les autres états se remplissent-ils ?

C'est la réponse à ces deux questions qui va constituer l'étude de la
durée de vie de l'état instable | b >.

Nous pouvons déjà leur fournir une réponse qualitative par un raisonnement
élémentaire.

L'état instable | b > est couplé par V à un continuum d'énergie. Il
existe donc, à l'instant initial, une probabilité de transition par unité de
temps vers le continuum que l'on peut calculer à l'aide de la règle d'or de
Fermi : il suffit de sommer sur les états finaux le carré de l'élément de matrice d'interaction V en ayant pris soin de multiplier par une distribution de
Dirac assurant la conservation de l'énergie. L'existence de cette probabilité
de transition par unité de temps implique une décroissence de la probabilité
% 228
de présence dans l'état initial qui acquiert ainsi une durée de vie finie
$\tau$ $=$ $\hbar/\Gamma$.

Cette durée de vie finie de l'état instable initial conduit à une
indétermination dans l'énergie de cet état, donc à une dispersion, de l'ordre
de $\Gamma$, en énergie des états finaux résultant de la disparition de l’état initial.

Toutes ces conclusions, établies par un calcul au premier ordre,
seront confirmées par le raisonnement rigoureux que nous allons faire dans
ce chapitre. Nous allons nous servir une fois de plus des techniques de la
résolvante afin de calculer les amplitudes de probabilité < b | U(t) | b >
pour que le système créé dans l'état b à l'instant 0 y soit encore à l'instant
t, et < a' | U(t) | b > pour que le système soit à l'instant t dans un état
a' du spectre continu de H$_0$ Nous devrons pour cela calculer les éléments de
matrice de la résolvante G$_\mt{bb}$ et G$_\mt{a'b}$ dont l'étude nous permettra ainsi de répondre aux deux questions que nous nous sommes posées. L'intérêt de la résolvante est qu'elle permet de mener jusqu'au bout des calculs rigoureux et d'en
tirer des conclusions rigoureuses sur le phénomène physique étudié, les approximations indispensables au calcul pratique n'étant faites qu'en tout dernier
lieu.

Nous devons enfin dire que la façon dont nous abordons ici le problème des états instables peut ne pas toujours être justifiée : nous avons en
effet admis implicitement que l'on pouvait préparer instantanément, à l'instant
t $=$ 0, un état | b > du système non perturbé H$_0$. Physiquement, cela veut dire que
le temps de préparation est très court devant la durée de vie de l'état instable.

Dans le cas de l'émission spontanée, par exemple, la préparation du
système peut se faire par excitation à partir de l'état fondamental de l'atome
à l'aide d'un train d'onde lumineux. Il faut donc que ce train d'onde ait une
dispersion en énergie $\Delta$ très grande devant la largeur $\Gamma$ de l'état excité étudié.

% 229
Une façon beaucoup plus générale d'aborder le problème, et qui
lève cette restriction est d'étudier non plus le spectre et les états propres
de H$_0$ qui n'ont pas une évolution simple, mais plutôt le spectre et les
états propres de H = H$_0$ + V.

Nous allons en effet montrer que le spectre de H est continu et
ne présente plus d'états discrets (sauf éventuellement l'état fondamental),
le spectre d'énergie partant d'une valeur $E$

\vspace{0.3cm}
\begin{center}
\begin{tikzpicture}
\draw [->] (0,0) -- (7,0);
\draw (0,0.07) -- (0,-0.07) node [below]{$E$};
\draw [dashed] (-3,0) -- (0,0);
\end{tikzpicture}
\end{center}

Les états propres de H sont donc des états stationnaires de collision et le problème se ramène à un problème de diffusion. Dans le cas de l'émission spontanée, il s'agit de la diffusion de photons par un atome.

Si le spectre de H ne possède plus d'état discret, au voisinage de
l'emplacement des états discrets du spectre de H$_0$, il reste en quelque sorte
“un souvenir" du spectre de H$_0$ et les éléments de matrice de collision et de
réaction S et R subissent des variations très importantes pour les énergies
correspondantes. Il en résulte une variation résonnante pour ces valeurs de la
section efficace de diffusion.

Le problème des états instables se trouve donc également lié de façon
étroite à celui de la diffusion résonnante et constitue ainsi en quelque sorte
un intermédiaire entre le problème des états liés et celui des états stationnaires de collision du spectre continu.

\subsection{Propriétés analytiques de la résolvante} %2°) :

Avant d'aborder le problème qui nous intéresse ici, nous allons étudier les propriétés analytiques de la résolvante.
% 230

Nous nous plaçons dans le cas, tout à fait général, d'un hamiltonien H possédant un spectre réel continu partant d'une valeur $E$ et un spectre
discret, d'énergie E$_\mt{i}$, dont une partie peut éventuellement être superposée au
spectre continue

\begin{center} \begin{tikzpicture}
\draw [->] (0,0) -- (7,0);
\draw [dashed] (-3,0) -- (0,0);
\draw (0,-0.07) -- (0,0.07) node [above]{$E$};
\draw (-1.7,0.07) -- (-1.7,-0.07) node [below]{E$_\mt{i}$};
\draw (2.5,0.07) -- (2.5,-0.07) node [below]{E$_\mt{j}$};
\draw (4.7,0.07) -- (4.7,-0.07) node [below]{E$_\mt{k}$};
\end{tikzpicture} \end{center}

En appelant $\gamma$ un ensemble de nombres quantiques, discrets ou continus, servant à distinguer les états de même énergie, on peut écrire la relation
de fermeture sous la forme très générale d'une somme sur tous les états discrets
et d'une intégrale sur le spectre continu :
\[
\tag{1} \sum_\mt{i} | \mt{E}_\mt{i} > < \mt{E}_\mt{i} | +
\int\int \mt{d}\gamma \mt{dE'} | \gamma\mt{E'} > < \mt{E'}\gamma | = 1
\]
Soit | u > un état normé. On se propose d'étudier les propriétés analytiques de
l'élément de matrice de la résolvante :
\begin{center}
G$_\mt{u}$(z) = < u | G(z) | u > $=$ < u | $\frac{1}{\mt{z}-\mt{H}}$ | u >
\end{center}
qui peut s'écrire, à l'aide de la relation de fermeture
\[
\mt{G}_\mt{u}(\mt{z}) = \sum_\mt{i}
\frac{| < \mt{u} | \mt{E}_\mt{i} > |^2}{\mt{z}-\mt{E}_\mt{i}} +
\int \mt{dE'}\frac{\int\mt{d}\gamma |< \mt{u} | \gamma\mt{E'} >|^2 }{\mt{z}-\mt{E'}}
\]
Plusieurs cas peuvent se présenter, suivant les valeurs de z :

1) z est strictement différent de l'une des valeurs propres de H

Il est alors clair que | z - E$_\mt{i}$ | et | z - E' | sont minorés par un
nombre $\Delta$ positif représentent la distance de z à la valeur propre la plus
% 231
voisine :
\begin{center}
 | z - E$_\mt{i}$ |, | z - E' | $\geq$ $\Delta$
\end{center}

Tous les numérateurs intervenant dans G$_\mt{u}$(z) étant positifs, on
peut majorer | G$_\mt{u}$(z) | :
\[
|\mt{G}_\mt{u}(\mt{z})| \leq \frac{1}{\Delta} \left[ \sum_\mt{i}
< \mt{u} | \mt{E}_\mt{i} > < \mt{E}_\mt{i} | \mt{u} > +
\int \int < \mt{u} | \gamma\mt{E'} > < \gamma\mt{E'} | \mt{u} > \mt{d}\gamma \, \mt{dE'} \right]
\]

et d'après la relation de fermeture (1) :
\begin{center}
| G$_\mt{u}$(z) | < $\frac{\mt{<u|u>}}{\Delta}$ $=$
$\frac{1}{\Delta}$ \ \ \ \ \ \ (| u > est norné)
\end{center}

I1 en résulte que G$_\mt{u}$(z) est borné, tant que | z $-$ E$_\mt{i}$ | | z $-$ E' | $\geq \Delta$.

On montre également très facilement que la dérivée G'$_\mt{u}$(z) reste bornée dans
les mêmes conditions. Il en résulte que G$_\mt{u}$(z) est analytique dans toute région
du plan complexe ne contenant pas le spectre de H.

2) z tend vers une des valeurs nronres du srsctre discret E$_\mt{i}$ de H

Le terme prépondérant dans la somme (2) est alors $\frac{|< \mt{u} | \mt{E}_\mt{i} >|^2}{\mt{z}-\mt{E}_\mt{i}}$ qui
tend vers l'infini :

Les valeurs propres du spectre discret de H sont donc en général
(sauf si < u|E$_\mt{i}$ > $=$ 0) des pôles de G$_\mt{u}$(z) admettant | < u|E$_\mt{i}$ > |$^2$ pour résidus.
Tous ces résultats ont déjà été obtenus au chapitre précédent.

3) z tend vers une valeur du spectre continu en un point E qui n'est
pas confondu avec un état du smectre discret, soit dans le demi-plan supérieur,
soit dans le demi-plan inférieur.

\begin{center} \begin{tikzpicture}
\draw [->] (0,0) -- (7,0);
\draw [dashed] (-3,0) -- (0,0);
\draw (0,-0.07) -- (0,0.07) node [above]{$E$};
\draw (-1.7,0.07) -- (-1.7,-0.07) node [below]{E$_\mt{i}$};
\draw (2.1,0.07) -- (2.1,-0.07) node [below]{E$_\mt{j}$};
\draw (5.7,0.07) -- (5.7,-0.07) node [below]{E$_\mt{k}$};
\draw [->] (3.9,0.9) -- (3.9,0.1) node [above right]{E};
\draw [->] (3.9,-0.9) -- (3.9,-0.1);
\end{tikzpicture} \end{center}

%232 =

Nous devons calculer $\lim_{\,\epsilon \to \,0^+} \mt{G}_\mt{u} (\mt{E} \pm \mt{i}\epsilon)$

avec
\[
\tag{3-a} \mt{G}_\mt{u} (\mt{E} \pm \mt{i}\epsilon) =
\sum_i\frac{|<\mt{u}|\mt{E}_i>|^2}{\mt{E} \pm \mt{i}\epsilon-\mt{E}_i} +
\int\mt{dE'}\frac{\mt{f}_\mt{u}(\mt{E'})}{\mt{E}-\mt{E'}\pm\mt{i}\epsilon}
\]

en posant
\[
\tag{3-b} \mt{f}_\mt{u}(\mt{E'})=\int\mt{d}\gamma<\mt{u}|\gamma \mt{E'}><\gamma \mt{E'}|\mt{u}>
\]

Le premier terme du second membre de (3-a) tend simplement,
lorsque $\epsilon \to 0$, vers $\sum_i\frac{|<\mt{u}|\mt{E}_i>|^2}{\mt{E} \pm \mt{i}\epsilon-\mt{E}_i}$, ce qui est une grandeur réelle.
Quant au deuxième terme du second membre, il tend comme nous l'avons déjà
vu à plusieurs reprises, lorsque $\epsilon \to 0$, vers

\[
\tag{4} <\mc{P}\frac{1}{\mt{E}-\mt{E'}},\mt{f}_\mt{u}(\mt{E'})> \mp
\mt{i}\pi<\delta(\mt{E}-\mt{E'}),\mt{f}_\mt{u}(\mt{E'})>
\]
\[
=<\mc{P}\frac{1}{\mt{E}-\mt{E'}},\mt{f}_\mt{u}(\mt{E'})> \mp
\mt{i}\pi\mt{f}_\mt{u}(\mt{E})
\]
Les limites de $\mt{G}_\mt{u} (\mt{E} \pm \mt{i}\epsilon)$ lorsque $\epsilon \to 0_+$, existent donc mais ne
sont pas les mêmes, suivant que l'on tend vers l'axe réel dans le demi-plan
supérieur ou dans le demi-plan inférieur (à condition que f (E) $\ne$ 0) : on
dit que la fonction analytique $\mt{G}_\mt{u} (\mt{z})$ présente une coupure le long du spectre
continu de H. Le point où commence la coupure, $\epsilon$, s'appelle le point de branchement.

La notion de coupure intervient pour toutes les fonctions analytiques à déterminations multiples : par exemple la fonction Log z augmente de
2i$\pi$ à chaque tour autour de l'origine. Le demi-axe ($0\to\infty$) peut donc être
considéré comme une coupure, les valeurs prises de part et d'autre par la
fonction différent de 2i$\pi$.

% 233

Dans le cas présent, les valeurs prises par la fonction G$_\mt{u}$(z)
de part et d'autre de la coupure sont complexes conjuguées l'une de l'eutre.
I1 est facile de s'en assurer sur les expressions (3-a) et (4).

Lorsqu'une fonction analytique est définie dans une partie du
plan complexe, on peut, sous certaines conditions, la prolonger par continuité en une fonction analytique dans une partie plus grande du plan complexe :
c'est le principe du prolongement analytique.

Le fonction G$_\mt{u}$(z) est analytique dans le demi-plan supérieur.
On peut la prolonger au delà de la coupure en une fonction analytique dans
le demi-plan inférieur. La valeur prise par la fonction prolongée par continuité est différente de le valeur prise au même point par la détermination
initiale de la fonction, puisque la coupure introduit justement une discontinuité dans cette détermination : on dit qu'on a prolongé G$_\mt{u}$(z) dans le
deuxième feuillet de Riemann, le plan complexe initial, constituant le premier
feuillet de Riemann.

On aurait pu également prolonger la détermination du demi-plan
inférieur au delà de la coupure dans le demi-plan supérieur.

Pour nous résumer, la fonction G$_\mt{u}$(z) est analytique dans le plan
complexe privé des valeurs du spectre discret de H qui constituent des pôles,
et du spectre continu de H qui constitue une coupure. Il est possible de prolonger analytiquement G$_\mt{u}$(z) au delà de la coupure dans le deuxième feuillet de
Riemann, ce qui constitue une deuxième détermination de G$_\mt{u}$(z). Cette deuxième
détermination peut posséder des pôles et nous allons voir que l'étude de ces
pôles va se révéler extrëmement utile dans le problème de la durée de vie des
états instables.



\subsection{Exemple choisi. Notations}% 3°)

Nous allons illustrer notre théorie de la durée de vie sur l'exemple
très important de l'émission spontanée, Ce problème présente des difficultés
% 234
inhérentes au fait que nous traitons l'interaction d'un système matériel avec
un champ : certaines des grandeurs que nous allons être amenés à définir
seront infinies car elles correspondront à des intégrales divergentes. Ces
divergences ne pourront être écartées que dans le cadre d'une théorie de la renormalisation. Cependant, comme ces difficultés ne sont pas liées à la théorie
de la durée de vie elle-même, elles n'infirmeront en rien les résultats généraux obtenus et nous les laisserons de côté pour l'instant.

(On peut montrer que si l'on prend pour masse et charge de l'électron les masse et
charge expérimentales, c'est-à-dire les masses et charges "habillées" par les fluctuations électromagnétiques du vide, on élimine les divergences. C'est ce que nous
ferons dans la suite de ce chapitre.)

Le système envisagé est donc celui d'un atome couplé au champ
électromagnétique.

Les états d'énergie de l'atome sont des états discrets $|$ a $>$,
$|$ b $>$ ..., d'énergie E$_\mt{a}$, E$_\mt{b}$ < 0 et des états du spectre continu pour E > 0

Les états d'énergie du champ s'obtiennent par action des opérateurs de création sur l'état du vide $|$ 0 $>$. Nous prendrons le vide normé :
\[
\tag{5}<0|0>=1
\]
et nous appellerons a$_\lambda^+(\vec{\mt{k}})$ et a$_\lambda^-(\vec{\mt{k}})$ respectivement les opérateurs de création
et d‘annihilation d'un photon de vecteur d'onde $\vec{\mt{k}}$ et de polarisation $\vec{\mc{E}}_\lambda$, perpendiculaire à $\vec{\mt{k}}$.

Nous prendrons des modes continus, avec pour relations de commutation :
\[
\tag{6}[\mt{a}_\lambda^-(\vec{\mt{k}}),\mt{a}_\mu^+(\vec{\mt{k'}})]=
\delta\lambda\mu\ \delta(\vec{\mt{k}}-\vec{\mt{k'}})
\]

Nous obtenons donc pour états propres d'énergie du système combiné
“atome-photons" en l'absence de l'interaction électromagnétique entre les atomes
et le champ, les états à zéro photon
\begin{center}
$|$ a, 0 $>$, $|$ b, 0 $>$, etc...
\end{center}
les états à un photon
\begin{center}
$|$ a, $\vec{\mt{k}}_\lambda$ $>$, $|$ b, $\vec{\mt{k}}_\lambda$ $>$, etc...
\end{center}
les états à deux photons
\begin{center}
$|$ a, $\vec{\mt{k}}_\lambda$,  $\vec{\mt{k'}}_{\lambda'}>$, ... et ainsi de suite.
\end{center}
% 235
$|$ a $>$ étant l'état fondamental de l'atome, le spectre de H$_0$ est un spectre
continu qui part de l'état fondamental $|$ a, 0 $>$ sur lequel se surimposent
des états discrets $|$ a, 0 $>$, $|$ b, 0 $>$, $|$ c, 0 $>$ etc...

Le vide de photons étant normé (relation 5), les états $|$ a, 0 $>$
etc... sont en effet des états normés alors que les états à un, deux, etc...
n photons ne le sont pas : on a, en effet, par exemple
\begin{center}
$<$ b $\vec{\mt{k}}, \lambda$ $|$ a $\vec{\mt{k'}}, \lambda' >\ =$
$<$ b 0 $|$ a$^-_\lambda(\vec{\mt{k}})$ a$^+_{\lambda'}(\vec{\mt{k'}})$ $|$ a 0 $>\ =$

$<$ b 0 $|$ a$^-_\lambda(\vec{\mt{k}})$ a$^+_{\lambda'}(\vec{\mt{k'}})$ - 
a$^+_{\lambda'}(\vec{\mt{k'}})$ a$^-_\lambda(\vec{\mt{k}}) \ |$ a 0 $>$
\end{center}
(car le 2e opérateur n'a pas d'élément de matrice entre deux états à zéro
photon).
et finalement
\begin{center}
$<$ b $\vec{\mt{k}}, \lambda$ $|$ a $\vec{\mt{k'}}, \lambda' >\ =$
$<$ b 0 $|$ $\lbrack$ a$^-_\lambda(\vec{\mt{k}})$, a$^+_{\lambda'}(\vec{\mt{k'}})\rbrack$ $|$ a 0 $>$

$ =\ \delta_\mt{ab}\delta_{\lambda\lambda'}\delta(\vec{\mt{k}}-\vec{\mt{k'}})$
\end{center}
d'après (6)

Il en résulte que l'état $|$ a, $\vec{\mt{k}}\ \lambda$ $>$, contrairement à l'état
$|$ b 0 $>$, n'est pas normé. (Voir note de la page 226.)

Nous sommes donc ramenés à étudier, comme nous l'avions annoncé
dans l'introduction, l'évolution d'un état $|$ b 0 $>$, normé, superposé avec un
continuum d'états non normés, $|$ a $\vec{\mt{k}}\ \lambda$ $>$ etc..., avec lesquels il est couplé
par une interaction.
% 236

Appelons G$^0$(z) la résolvante de l’hamiltonien H$_0$ :

Nous avons G$^0_b$(z) = $<$b 0 $|\ \frac{1}{\mt{z}-\mt{H}_0}\ |$ b 0 $>\ =\ \frac{1}{\mt{z}-\mt{E}_\mt{b}}$

et G$^0_b$(z) présente un pôle au point E$_b$ correspondant à l'état instable
$|$ b 0 $>$ étudié.

Introduisons maintenant l'interaction entre l'atome et le champ :
le hamiltonien du système devient H $=$ H $+\mc{H}_\mt{I}$

Nous allons voir que le pôle au point z = E$_b$, qui existait dans
la résolvante G$^0_b$(z) disparaît dans G$_b$(z).

Cependant G$_b$(z) va subir une variation rapide au voisinage de
z = E$_b$, ce qui constitue en quelque sorte un souvenir du pôle qui existait
en ce point sur G$^0_b$(z). C'est l'étude de cette variation qui va nous permettre de définir la notion de durée de vie de l'état instable $|$ b O $>$.

D'une façon plus précise, nous allons étudier dans une première
partie (\S B) un modèle simple dans lequel on supposera que l'interaction k,
ne connecte que l'état instable $|$ b O $>$ et l'état fondamental en présence
d'un seul photon k, , $|$ a, $\vec{\mt{k}}\ \lambda\ >$

Ce modèle,que nous pourrons traiter rigoureusement jusqu'au bout,
nous permettra de dégager les principaux résultats de la théorie : notion de
durée de vie et de déplacements radiatifs de l’état instable. Nous étudierons
ensuite, toujours dans ce modèle simple, les produits de désintégration de
l'état instable, ce qui nous permettra notamment de prévoir les formes de raie
d'émission.

Nous rattacherons ensuite le problème à l'étude de la diffusion
résonnante des photons par un atome et nous montrerons que chaque état instable
correspond à une résonance dans la section efficace de diffusion. Nous envisagerons enfin la préparation de l'état instable, ce qui nous permettra de préciser
dans quelles conditions il est possible de donner un sens à la notion de durée
de vie.
% 237

Nous passerons ensuite dans une seconde partie (\S C) à l'étude
du cas général où l'interaction $\mc{H}_\mt{I}$ a d'autres éléments de matrice que
$<$ a, $\vec{\mt{k}}\ \lambda\ |\ \mc{H}_\mt{I}\ |$ b 0 $>$, Nous verrons comment l'introduction de tous les autres éléments de matrice conduit à "habiller" les états atomiques et l'interaction
électromagnétique et nous corrigerons ainsi les résultats du \S B (déplacement
de l'état fondamental, influence de la désintégration de b vers des étets
autres que l'état fondemental).

Enfin, dans une dernière partie (\S D), nous appliquerons les
techniques développées dans les parties précédentes à l'étude de la diffusion
résonnante au voisinage d'un croisement de niveaux d'énergie (effet Hanle {\bf —}
effet Franken).

\subsection{Etude d'un modèle simple}% B..

\subsection{Hypothèse simplificatrice}% 1°) :

Faisons l'hypothèse que les seuls éléments non nuls de $\mc{H}_\mt{I}$ sont
les éléments $<$ a, $\vec{\mt{k}}\ \lambda\ |\ \mc{H}_\mt{I}\ |$ b 0 $>$ qui joignent l'état excité dans le vide
$|$ b 0 $>$ à l'état fondamental en présence d'un seul photon $|$ a, $\vec{\mt{k}}\ \lambda\ >$.

L'étude de l'évolution de l'état | b 0 > est done un problème circonscrit à l'espace $\mc{E}_\mt{0}$ sous tendu par les vecteurs | b 0 > et $|$ a, $\vec{\mt{k}}\ \lambda\ >$, vecteurs propres de H$_\mt{0}$ avec les énergies respectives E$_\mt{a}$ et E$_\mt{b}$ + $\hbar$ck

\subsection{Étude de G$_\mt{b}$(z) $=\ <$ b 0 $|$ G(z) $|$ b 0 $>$}% 2°)

L'état $|$ b 0 $>$ étant un état normé, tous les résultats sur la résolvante et sur ses éléments de matrice que nous avons établis dans le chapitre pré
cédent et au \S À 2°) de ce chapitre sont valables.

\subsubsection{Calcul de G$_\mt{b}$(z)}% a)

On peut, en reprenant la formule 16 p.197, écrire G$_\mt{b}$(z) $=\ \frac{1}{\mt{z}-\mt{E}_\mt{b}-\mt{T}_\mt{b}\mt{(z)}}$
% 238 —

en posant, d'après la formule 29 p.202 :
\[
\tag{8} \mt{T}_\mt{b}\mt{(z)} = < \mt{b 0 } |\ \mc{H}_\mt{I}\ |\mt{ b 0}> +
< \mt{b 0 } |\ \mc{H}_\mt{I}\ \mt{Q}_\mt{b}\ 
\frac{1}{\mt{z}-\mt{H}_0-\mt{Q}_\mt{b}\ \mc{H}_\mt{I}\ \mt{Q}_\mt{b}}
\mt{Q}_\mt{b}\ \mc{H}_\mt{I}\ |\mt{ b 0}>
\]

Q$b$ étant le projecteur 1 $-\ |$ b 0 $>\ <$ b 0 $|$.

Mais, d'après les hypothèses simplificatrices du 1°)
\[
   \left \{
   \begin{array}{r c l}
< \mt{b 0 } |\ \mc{H}_\mt{I}\ |\mt{ b 0}>  & = & 0 \\
\mt{Q}_\mt{b}\ \mc{H}_\mt{I}\ \mt{Q}_\mt{b} & = & 0 \\
\mt{Q}_\mt{b}\ \mc{H}_\mt{I}\ |\mt{ b 0}> & = & \mc{H}_\mt{I}\ |\mt{ b 0}>
   \end{array}
   \right .
\]
Si bien que finalement, (8) peut s'écrire :
\[
\tag{9} \mt{T}_\mt{b}\mt{(z)} =
< \mt{b 0 } |\ \mc{H}_\mt{I}\ \frac{1}{\mt{z}-\mt{H}_0}\ \mc{H}_\mt{I}\ |\mt{ b 0}>
\]
\[
=\sum_\lambda\int\mt{d}\vec{\mt{k}}\ 
\frac{|<\mt{b }0\ |\ \mc{H}_\mt{I}\ |\mt{ a }\vec{\mt{k}}\ \lambda> |^2}
{\mt{z}-\mt{E}_\mt{a}-\hbar\mt{ck}}
\]

où la sommation est étendue aux états de polarisation et aux vecteurs d'onde
des photons a $\vec{\mt{k}}\ \lambda$.

Afin de préciser les propriétés analytiques de G$_\mt{b}$(z) nous allons
d'abord étudier T$_\mt{b}$(z).

\subsubsection{Etude de T$_\mt{b}$(z)}% b)

- En conjuguant la relation (9), nous obtenons la relation
importante
\[
\tag{10} \mt{T}_\mt{b}(\mt{z})\mt{*}=\mt{T}_\mt{b}(\mt{z*})
\]

- D'autre part, nous voyons immédiatement que l'expression
(9) de T$_\mt{b}$(z) est analogue à l'expression (2) que nous avons donnée de G$_\mt{b}$(z) au \S
 A 2°) Il en résulte de la même façon que T$_\mt{b}$(z) est analytique dans 1e plan
complexe coupé de la demi droite $]$E$_\mt{a}, +\infty ]$
% 239

Remarque : D'après la relation (7), il est évident que G$_\mt{b}$(z) présente également
une coupure admettant le même point de branchement E$_\mt{a}$ que T$_\mt{b}$(z) : appelons
en effet E'$_\mt{a}$, le point de branchement de la coupure de G$_\mt{b}$(z); par définition
au-delà de E'$_\mt{a}$ il y a une discontinuité entre G$_\mt{b}$(E+i$\epsilon$) et G$_\mt{b}$(E-i$\epsilon$) et
au-delà du point de branchement E$_\mt{a}$ il y a une discontinuité entre
T$_\mt{b}$(E+i$\epsilon$) et T$_\mt{b}$(E-i$\epsilon$). D'après (7), la discontinuité de T$_\mt{b}$(z) est une
condition nécessaire et suffisante de celle de G$_\mt{b}$(z). On a donc
E'$_\mt{a}$=E$_\mt{a}$. (* D'après les résultats généraux de \S A 2°), la coupure de G$_\mt{b}$(z) s'étend sur le
spectre continu de H et le point de branchement correspond à l'énergie fondnmentale du système: il correspond ici avec l'énergie fondamental du système non
perturbé E$_\mt{a}$. Si nous voulons tenir compte du déplacement d'énergie de l'état fondamental, 11 faut traiter un modèle plus complet dans lequel l'état $|$ a 0 $>$ se
trouve 1ié à d'autres états par la perturbation. C'est ce que nous faisons au \S C)

- Afin de préciser les propriétés de la coupure de T$_\mt{b}$(z),
nous allons étudier la limite lorsque z tend vers l'axe réel de T$_\mt{b}$(z) et plus précisément calculer
\[
\lim_{\epsilon\to 0+}\mt{T}_\mt{b}(\mt{E}\pm\mt{i}\epsilon)=\mt{T}_\mt{b}^\pm(\mt{E})
\]

Pour cela, nous allons réécrire (9) sous une forme plus condensée
en séparant les intégrations sur l'angle solide et sur le module de $\vec{\mt{k}}$ en posant
\[
\tag{11}\mt{f(k)}=\mt{k}^2\sum_\lambda\int\mt{d}\Omega\ |<\mt{b 0 }|\ \mc{H}_\mt{I}\ |\mt{ a }\vec{\mt{k}}\ \lambda> |^2
\]
% 240
Il vient alors
\[
\tag{12}\mt{T}_\mt{b}(\mt{z})=\int\frac{\mt{f(k)}}
{\mt{z}-\mt{E}_\mt{a}-\hbar\mt{ck}}\ \mt{dk}
\]
et on obtient alors sans difficulté
\[
\tag{13}\mt{T}_\mt{b}^\pm(\mt{E})=\Delta(\mt{E})\mp\frac{\mt{i}\Gamma(\mt{E})}{2}
\]
en posant
\[
\tag{14-a}\Delta(\mt{E})=<\mc{P}\frac{1}
{\mt{E}-\mt{E}_\mt{a}-\hbar\mt{ck}}\ ,\mt{ f(k)}>
\]
\[
\tag{14-b}\Gamma(\mt{E})=2\pi<\delta
(\mt{E}-\mt{E}_\mt{a}-\hbar\mt{ck})\ ,\mt{ f(k)}>
\]
Nous voyons, d'après (14-b), que $\Gamma$(E) est nul pour E $<$ E$_\mt{a}$ car
alors l'argument de la distribution $\delta$ est négatif. f (k) étant strictement
positif d'après (11) dès que k > 0, $\Gamma$(E) est strictement positif dès que
E $>$ E$_\mt{a}$ Nous retrouvons ainsi le fait que la fonction T$_\mt{b}$(z) présente une
coupure s'étendant de E$_\mt{a}$ à l'infini, les valeurs prises par la fonction de
part et d'autre de la coupure étant complexes conjuguées l'une de l'autre,
ce qui est d'ailleurs évident d'après la relation (10).

Nous sommes maintenant en mesure de préciser complètement les
propriétés analytiques de G$_\mt{b}$(z).

\subsubsection{Propriétés analytiques de G$_\mt{b}$(z)}% c) 
Nous savons déjà que G$_\mt{b}$(z) présente une coupure coïncidant
avec la demi-droite $]\mt{E}_\mt{a},+\infty[$.

Nous savons également d'après A 2°) que les pôles éventuels de
G$_\mt{b}$(z) ne peuvent être que les énergies propres discrètes de l'Hamiltonien H,
ou plus précisément de P$_0$HP$_0$ étant le projecteur sur l'espace $\mc{E}_0$
dans lequel notre problème est circonscrit. Ces pôles éventuels sont donc réels
(à cause de l'hermiticité de H) et nécessairement situés sur la coupure (car
le point de branchement E$_\mt{a}$ correspond à l'état fondamental de H).

% 241
Mais s'il y avait un pôle au point E > E$_\mt{a}$, on aurait d'après (7)
\begin{center}
E $=$ E$_\mt{b}\ +$ T$_\mt{b}^\pm$(E) $=$ E$_\mt{b}\ +\ \Delta$(E) $\mp\frac{\mt{i}\Gamma\mt{(E)}}{2}$
\end{center}
qui est impossible car, pour E $>$ E$_\mt{a}$, $\Gamma$(E) est strictement positif.
G$_\mt{b}$(z) n'ayant pas de pôle sur la coupure n'en a nulle part et le spectre
de P$_0$HP$_0$ est entièrement continus. (* Une façon directe de voir que G$_\mt{b}$(z) n'a aucun pôle dans le plan complexe est de
reprendre l'expression (12) de T$_\mt{b}$(z) en explicitant les parties réelles et imaginaires de z : 
T$_\mt{b}$(z) $=$ T$_\mt{b}$(x + iy) $=$
$\int\frac{\mt{dk f(k)}}{(\mt{x-E}_\mt{a}-\hbar\mt{ck})^2+\mt{y}^2}$
$[\mt{x-E}_\mt{a}-\hbar\mt{ck-iy}]$. 
On en déduit immédiatement : Im$[$T$_\mt{b}$(z)$]=
-\mt{y}\int\frac{\mt{dk f(k)}}{(\mt{x-E}_\mt{a}-\hbar\mt{ck})^2+\mt{y}^2}$
qui est de signe opposé à y = Im(z).
Il en résulte que l'équation (G$_b$(x+iy))$^{-1}=$x+iy-E$_\mt{b}$-T$_\mt{b}$(x + iy)=0 ne peut avoir
de solution pour y $\neq$ 0. Nous avons vu qu'elle n'a pas de solution pour y $\to$ 0.
G$_b$(z) n'a donc pas de pôles dans le plan complexe.)

En conclusion, G$_\mt{b}$(z), contrairement à G$^0_\mt{b}$(z), n'admet aucun pôle
dans le plan complexe et est donc analytique dans le plan privé de la coupure
$]$E$_\mt{a}$, $+\infty ]$.

Cependant, au voisinage du point E = E$_\mt{b}$, pôle de G$^0_\mt{b}$(z), G$_\mt{b}$(z)
subit une variation importante.

I1 résulte en effet de l'expression (12) que T$_\mt{b}$(z) est une expression du second ordre en $\mc{H}_\mt{I}$, donc petite. Elle variera donc peu avec z
lorsque z variera autour de E$_\mt{b}$ dans le demi-plan supérieur et compte tenu de
(7), on peut écrire au voisinage de E$_\mt{b}$ dans le demi-plan supérieur :
%242
\[
\tag{15}\mt{G}_\mt{b}\mt{(z)}\sim\frac{1}{\mt{z}-\mt{E}_\mt{b}-\Delta(\mt{E}_\mt{b})
+\frac{\mt{i}\Gamma(\mt{E}_\mt{b})}{2}}
\]

On a de même dans le voisinage de E$_\mt{b}$, dans le demi-plan inférieur
\[
\tag{16}\mt{G}_\mt{b}\mt{(z)}\sim\frac{1}{\mt{z}-\mt{E}_\mt{b}-\Delta(\mt{E}_\mt{b})
-\frac{\mt{i}\Gamma(\mt{E}_\mt{b})}{2}}
\]

Comme nous l'avons déjà vu, ces expressions ne possèdent pas de
pôle et il est facile de s'en rendre compte directement.

Prolongeons maintenant analytiquement, dans le demi-plan inférieur G$_\mt{b}$(z) en partant du demi-plan supérieur. Le prolongement analytique se
faisant par continuité, T$_\mt{b}$(z) restera pratiquement égal à T$^+_\mt{b}$(E$_\mt{b}$) et nous
obtiendrons dans le deuxième feuillet de Riemann une détermination $\overline{\mt{G}_\mt{b}}$(z)
dont nous pouvons écrire la valeur approchée :
\[
\overline{\mt{G}_\mt{b}\mt{(z)}}=\frac{1}{\mt{z}-\mt{E}_\mt{b}-\Delta(\mt{E}_\mt{b})
+\frac{\mt{i}\Gamma(\mt{E}_\mt{b})}{2}}
\]

Le signe de Im(z) ayant changé à la traversée de la coupure, on
prévoit ainsi l'existence dans le deuxième feuillet de Riemann d'un pôle de
$\overline{\mt{G}_\mt{b}}$(z), z$_0$ dont la valeur est sensiblement
\[
\tag{17}\mt{z}_\mt{0}\sim\mt{E}_\mt{b}+\Delta(\mt{E}_\mt{b})-\frac{\mt{i}\Gamma(\mt{E}_\mt{b})}{2}
\]

\begin{center} \begin{tikzpicture}
\draw [->] (0,0) -- (7,0);
\draw [dashed] (-3,0) -- (0,0);
\draw (0,-0.07) -- (0,0.07) node [above]{E$_\mt{a}$};
\draw (2.1,0.07) -- (2.1,-0.07) node [above]{E$_\mt{b}$};
\draw [<->,dashed] (2.15,-0.15) --(3.85,-0.15);
\draw [<->,dashed] (3.9,0) --(3.9,-1.2);
\draw (3.95,-1.35)  node [right]{z$_0$};
\draw (3.8,-1.25) --(4,-1.45);
\draw (3.8,-1.45) --(4,-1.25);
\draw (4.5,-0.7) node {$\frac{\Gamma(\mt{E}_\mt{b})}{2}$};
\draw (2.9,-0.5) node {$\Delta(\mt{E}_\mt{b})$};
\end{tikzpicture} \end{center}

% 243
L'effet du couplage $\mc{H}_\mt{I}$ a donc été de déplacer le pôle de G$^0_\mt{b}$(z)
qui était en E$_\mt{b}$, dans le deuxième feuillet de Riemann. La présence de ce pôle
va affecter G$^+_\mt{b}$(E) qui va varier rapidement au voisinage du point E$_\mt{b}+\Delta$(E$_\mt{b}$).

Remarque : On aurait pu tout aussi bien prolonger G$^0_\mt{b}$(z) en partant du demi-plan
inférieur. On aurait alors obtenu un pôle à la position complexe conjuguée
de z$_0$.
 
Le raisonnement que nous venons de faire est resté qualitatif.
L'essentiel est qu'il prévoit la présence du pôle z$_0$.

Nous allons voir dans le paragraphe suivant que la partie imaginaire de ce pôle est liée à le durée de vie de l'état instable $|$ b 0 $>$ alors que
sa partie réelle nous fournit le déplacement en énergie de ce même niveau

\subsection{Evolution de L'état instable}%3°)
\subsubsection{Calcul de < b 0 | U (t) | b 0 >} %a)
Connaissant G$_\mt{b}$(z), on calcule $<$ b 0 $|$ U (t) $|$ b 0 $>$, amplitude de probabilité pour que l'atome créé dans l'état b à l'instant t = 0, y
soit encore à l'instant t par l'intégration
\[
\tag{18}<\mt{b }0\ |\ \mt{U(t)}\ |\ 0\mt{ b}> = \frac{1}{2\pi\mt{i}}
\int_\mt{C} \mt{G}_\mt{b}(\mt{z})\mt{e}^{-\mt{izt}/\hbar}\mt{dz}
\]

Nous calculerons l'intégrale (17) par les résidus en fermant convenablement le contour d'intégration C qui représente l'axe orienté ($+\infty$ +i$\epsilon$, $-\infty$ +i$\epsilon$)

Pour les instants t > 0, il faut fermer le contour d'intégration dans
le demi-plan inférieur de façon à ce que la contribution de e$^{-\mt{izt}/\hbar}$
tende vers
zéro. Mais il est indispensable que G$_\mt{b}$(z) soit continue le long du contour d'intégration. Il faut donc, au point $+\infty$ +i$\epsilon$, passer dans le deuxième feuillet de
Riemann et adopter la deuxième détermination G$_\mt{b}$(z) de G$_\mt{b}$(z). Enfin, de façon à
raccorder par continuité les deux arcs de cercle qui partent de $+\infty$ +i$\epsilon$ et $-\infty$ +i$\epsilon$
, il est nécessaire de contourner à l'aide d'un lacet le point de branchement E$_\mt{a}$ On réalise ainsi un contour $\mc{C}$ représenté sur la figure ci-dessous
qui comprend :

%244
1°) La droite C

2°) Un quart de cercle dans le demi-plan inférieur du premier
feuillet de Riemann (en trait plein)

3°) Un quart de cercle dans le demi-plan inférieur du deuxième
feuillet de Riemann (en trait pointillé)

4°) Un lacet C' qui permet de raccorder les deux quarts de
cercles et de passer du premier au second feuillet de Riemann en tournant
autour du point de branchement

\begin{center} \begin{tikzpicture}
\draw [->] (-4.25,0) -- (4.25,0);
\draw (0,0.07) -- (0,-0.07) node [below]{E$_\mt{a}$};
\draw (2,0.07) -- (2,-0.07) node [below]{E$_\mt{b}$};
\draw [dashed] (2.9,0) --(2.9,-0.7);
\draw (2.95,-0.7)  node [right]{z$_0$};
\draw (2.8,-0.8) --(3,-0.6);
\draw (2.8,-0.6) --(3,-0.8);
\draw (4,0.25) -- (-4,0.25) arc (180:270:3.75) -- (-0.25,0)  arc (180:0:0.25);
\draw [dashed] (0.25,0) -- (0.25,-3.5) arc (270:360:3.75);
\draw [thick] (-1.9,0.45) -- (-2.15,0.25) -- (-1.9,0.05);
\draw [thick] (-0.45,-2.25) -- (-0.25,-2) -- (-0.05,-2.25);
\draw [thick] (0.45,-1.75) -- (0.25,-2) -- (0.05,-1.75);
\end{tikzpicture} \end{center}

Le long de ce contour $\mc{C}$, G$_\mt{b}$(z) est continu et on peut appliquer
le théorème des résidus. Or les seuls pôles de G$_\mt{b}$(z) ne peuvent, d'après
\S B 2°),se trouver que dans le deuxième feuillet de Riemann. Nous avons montré
l'existence d'un pôle z$_0$ situé sensiblement à la distance $\frac{\Gamma(\mt{E}_\mt{b})}{2}$ de l'axe réel.
Nous supposerons dans la suite que ce pôle est bien le seul. On peut alors compléter le contour  à l'aide d'un petit cercle orienté autour de z$_0$ et
écrire, à la limite où le rayon des quarts de cercle tend vers l'infini et
en explicitant l'intégrale du lacet :
% 245
\[
\tag{19} \frac{1}{2\mt{i}\pi}\int_\mt{C}\mt{G}_\mt{b}\mt{(z)}\mt{e}^{-\frac{\mt{izt}}{\hbar}}\mt{dz} =
\frac{1}{2\mt{i}\pi}\int_{\mt{E}_\mt{a}-\mt{i}\infty}^{\mt{E}_\mt{a}}\overline{\mt{G}_\mt{b}}\mt{(z)}\mt{e}^{-\frac{\mt{izt}}{\hbar}}\mt{dz} +
\frac{1}{2\mt{i}\pi}\int_{\mt{E}_\mt{a}}^{\mt{E}_\mt{a}-\mt{i}\infty}\mt{G}_\mt{b}\mt{(z)}\mt{e}^{-\frac{\mt{izt}}{\hbar}}\mt{dz}
\]
\[
+\mt{Résidu en z}_0 \left \{ \overline{\mt{G}_\mt{b}}\mt{(z)}\mt{e}^{-\frac{\mt{izt}}{\hbar}} \right \}
\]
%Intégrale le long du lacet
le contour d'intégration correspondant étant représenté par la figure ci-dessous
\begin{center} \begin{tikzpicture}
\draw [->] (-4.25,0) -- (4.25,0);
\draw (0,-0.07) -- (0,0.07) node [above right]{E$_\mt{a}$};
\draw (2,-0.07) -- (2,0.07) node [above]{E$_\mt{b}$};
\draw [dashed] (2.9,-0.7) circle(0.5);
\draw (2.95,-0.7)  node [below]{z$_0$};
\draw (2.8,-0.8) --(3,-0.6);
\draw (2.8,-0.6) --(3,-0.8);
\draw (-0.25,-3.5) -- (-0.25,0)  arc (180:0:0.25);
\draw [dashed] (0.25,0) -- (0.25,-3.5);
\draw [thick] (3.2,-0.5) -- (3.1,-0.25) -- (3.35,-0.20);
\draw [thick] (-0.45,-1.75) -- (-0.25,-2) -- (-0.05,-1.75);
\draw [thick] (0.45,-2.25) -- (0.25,-2) -- (0.05,-2.25);
\end{tikzpicture} \end{center}
\subsubsection{Calcul approché} %b) :

Nous justifierons plus bas que la contribution à L'expression (19) du lacet C' est en général négligeable. Nous n'en tiendrons pas
compte pour commencer et nous calculerons La contribution prépondérante du
résidu :

Soit $\overline{\mt{T}_\mt{b}}$(z) la détermination de $\mt{T}_\mt{b}$(z) dans le deuxième feuillet
de Riemann, Nous avons
% 246 —
\[
\overline{\mt{G}_\mt{b}}\mt{(z)}=\frac{1}{\mt{z}-\mt{E}_\mt{b}\overline{\mt{T}_\mt{b}}\mt{(z)}}
\]
et
\[
\tag{20}\mt{Résidu en z}_0 \left \{ \overline{\mt{G}_\mt{b}}\mt{(z)}\mt{e}^{-\frac{\mt{izt}}{\hbar}} \right \} =
\frac{1}{1-\left .\frac{\mt{d}\overline{\mt{T}_\mt{b}}\mt{(z)}}{\mt{dz}}\right |_{\mt{z}_0}}
\mt{e}^{-\frac{\mt{iz}_0\mt{t}}{\hbar}}
\]
L'expression (20) est rigoureuse. $\mt{T}_\mt{b}$(z) est peu différent de $\mt{T}_\mt{b}^\dagger$(E$_\mt{b}$).
C'est une quantité du deuxième ordre en $\mc{H}_\mt{I}$ dont les variations sont très
faibles et $\left .\frac{\mt{d}\overline{\mt{T}_\mt{b}}\mt{(z)}}{\mt{dz}}\right |_{\mt{z}_0}$ est absolument négligeable devant 1, si bien que
l'on peut écrire, si on néglige la contribution du lacet
\[
\tag{21}<\mt{b 0 }|\mt{ U(t) }|\mt{ b 0}> =
\frac{1}{2\mt{i}\pi}\int_\mt{C}\mt{G}_\mt{b}\mt{(z) e}^{-\frac{\mt{izt}}{\hbar}}
= \mt{e}^{-\frac{\mt{itz}_0}{\hbar}}
= \mt{e}^{-\mt{i}\frac{\mc{R}e(\mt{z}_0)\mt{t}}{\hbar}} \mt{e}^{-\mt{i}\frac{\mc{I}m(\mt{z}_0)\mt{t}}{\hbar}}
\]

Nous prévoyons ainsi, rigoureusement, et non plus par un calcul
au premier ordre du type "règle d'or de Fermi", que dans l'hypothèse de notre
modèle simple, l'état | b 0 > disparaît avec une constante de temps égale à
$\frac{\hbar}{2\mc{I}m(\mt{z}_0)}$
 et voit son énergie déplacée de la quantité $\mc{R}e(\mt{z}_0)-\mt{E}_\mt{b}$.
 
Nous pouvons maintenant, à une excellente approximation, remplacer
z$_0$ Par son expression (17) et écrire
\[
\tag{22}<\mt{b 0 }|\mt{ U(t) }|\mt{ b 0}> =
\mt{e}^{-\mt{it}\frac{\mt{E}_\mt{b}+\Delta(\mt{E}_\mt{b})}{\hbar}}
\mt{e}^{-\frac{\Gamma(\mt{E}_\mt{b})\mt{t}}{2\hbar}}
\]
% 247

$\Delta(\mt{E}_\mt{b})$ et $\Gamma(\mt{E}_\mt{b})$ étant donnés par les expressions (14-a) et (14-b) :
\[
\tag{23-a}\Delta(\mt{E}_\mt{b})=<\mc{P}\frac{1}
{\mt{E}_\mt{b}-\mt{E}_\mt{a}-\hbar\mt{ck}}\ ,
\mt{ k}^2\sum_\lambda\int\mt{d}\Omega\ |<\mt{ b 0 }|\ \mc{H}_\mt{I}\ |\mt{ a }\vec{\mt{k}}\ \lambda >|^2>
\]
\[
\tag{23-b}\Gamma(\mt{E}_\mt{b})=2\pi<\delta
(\mt{E}_\mt{b}-\mt{E}_\mt{a}-\hbar\mt{ck})\ ,
\mt{ k}^2\sum_\lambda\int\mt{d}\Omega\ |<\mt{ b 0 }|\ \mc{H}_\mt{I}\ |\mt{ a }\vec{\mt{k}}\ \lambda >|^2>
\]

Remarques : — la formule (23-b) n'est autre que celle que nous aurions
obtenue pour la durée de vie de l'état | b 0 > en appliquent la règle d'or
de Fermi. Le traitement que nous donnons ici nous permet cependant de prévoir le décroissance exponentielle rigoureuse à tous les ordres et l'existence d'un déplacement radiatif du niveau instable.

— l'intégrale (23-a) permettant de calculer $\Delta(\mt{E}_\mt{b})$ est divergente (terme croissant en k à l'infini). Cette divergence est due au fait
que nous traitons un système en interaction avec un champ (voir remarque de
la page 234 ) et disparaît si nous prenons pour la masse et la charge de
l'électron les masse et charge expérimentales.

\subsubsection{Ordre de grandeur de l'erreur faite en négligeant le lacet} %c)
Revenons maintenant à l'intégrale le long du lacet intervenant dans (19). Elle s'écrit :
\[
\tag{24}\mt{A}=\frac{1}{2\mt{i}\pi}
\int_{\mt{E}_\mt{a}}^{\mt{E}_\mt{a}-\mt{i}\infty}
\mt{e}^{-\frac{\mt{izt}}{\hbar}}\mt{dz}
\left [
\frac{1}{\mt{z}-\mt{E}_\mt{b}-\mt{T}_\mt{b}(\mt{z})}-
\frac{1}{\mt{z}-\mt{E}_\mt{b}-\overline{\mt{T}_\mt{b}}(\mt{z})}
\right ]
\]
\[
\ \ =\frac{1}{2\mt{i}\pi}\int_{\mt{E}_\mt{a}}^{\mt{E}_\mt{a}-\mt{i}\infty}
\mt{e}^{-\frac{\mt{izt}}{\hbar}}\ \frac{\overline{\mt{T}_\mt{b}}(\mt{z})-\mt{T}_\mt{b}(\mt{z})}
{(\mt{z}-\mt{E}_\mt{b}-\mt{T}_\mt{b}(\mt{z}))(\mt{z}-\mt{E}_\mt{b}-\overline{\mt{T}_\mt{b}}(\mt{z}))}\ \mt{dz}
\]
% 248 —
les quantités T$_\mt{b}$(z) et $\overline{\mt{T}_\mt{b}}$(z) sont des quantités petites du second ordre en
$\mc{H}_\mt{I}$, de l'ordre de $\Gamma_\mt{b}$ (nous posons  $\Gamma(\mt{E}_\mt{b})=\Gamma_\mt{b}$). D'autre part,
e$^{-\frac{\mt{izt}}{\hbar}}$
décroît très vite lorsque $|$ $\mc{I}m$ (z) $|$ augmente le long du lacet et on peut
majorer de façon très sévère $\mt{e}^{-\frac{\mt{izt}}{\hbar}}$ par 1. On a alors une majoration de A :
\[
\mt{A}<<\frac{1}{2\mt{i}\pi}\int_{\mt{E}_\mt{a}}^{\mt{E}_\mt{a}-\mt{i}\infty}
\frac{\Gamma_\mt{b}}{(\mt{z}-\mt{E}_\mt{b})^2}\ \mt{dz}=
\frac{1}{2\mt{i}\pi}\ \frac{\Gamma_\mt{b}}{\mt{E}_\mt{a}-\mt{E}_\mt{b}}
\]

Deux cas peuvent alors se présenter :

$\alpha$) Le temps t n'est pas trop long devant $\hbar/\Gamma_{b}$ : la contribution du résidu à
l'amplitude < b 0 $|$ U(t) $|$ b 0 > est de l'ordre de 1. L'erreur commise en négligeant
le lacet est alors négligeable dès que $\frac{\Gamma_\mt{b}}{\mt{E}_\mt{a}-\mt{E}_\mt{b}}$ << 1, c'est-à-dire dès
que la largeur naturelle du niveau b, $\Gamma_{b}$, est petite devant l'énergie de la
transition optique, ce qui est le cas la plupart du temps.

$\beta$) Le temps t est très grand : la contribution du résidu, en e$^\frac{\Gamma_{b}\mt{t}}{2}$, devient
très petite et la condition $\frac{\Gamma_\mt{b}}{\mt{E}_\mt{a}-\mt{E}_\mt{b}}$ << 1 n'est plus suffisante pour assurer que la correction apportée par le lacet à la loi de la décroissance exponentielle est négligeable.

De façon plus précise, lorsque t est très grand, seules les portions du contour d'intégration pour lesquelles $\mc{I}m$(z) est voisin de zéro interviennent de façon notable, c'est-à-dire d'après la figure de la page 245,
la partie du lacet voisine du point E$_\mt{a}$.

Alors que pour le pôle z$_0$, $|$ $\mc{I}m$ (z) $|$ est égal à $\Gamma_mt{b}$, on peut
trouver sur le lacet des points tels que $|$ $\mc{I}m$ (z) $|$ soit aussi petit que l'on
veut. Pour t très grand, le facteur e$^{-\frac{\mt{izt}}{\hbar}}$ atténue la contribution du pôle
beaucoup plus que celle du lacet. Il est donc intéressant dans ce cas d'étudier
le comportement de la contribution du lacet.

%249
La partie la plus importante de cette contribution correspondra
à y = $\mc{I}m$ (z) très petit. Les deux déterninations de T$_\mt{b}$(z) diffèrent alors pratiquement uniquement par leur partie imsginaire et on peut écrire
\begin{center}
T$_\mt{b}$(z) $-\ \overline{\mt{T}_\mt{b}}$(z) $\sim$ 2 $\mc{I}m$ ( E$_\mt{a}$ $+$ iy )
\end{center}

Or, d'après (12) :
\begin{center}
T$_\mt{b}$( E$_\mt{a}$ $+$ iy ) = $\int_0^\infty\frac{\mt{f(k)dk}}{\mt{iy} - \hbar\mt{ck}}$
\hspace{1cm} et \hspace{1cm}
2$\mc{I}m$ T$_\mt{b}$( E$_\mt{a}$ $+$ iy ) = -2y $\int_0^\infty\frac{\mt{f(k)dk}}{\mt{y}^2 + \hbar^2\mt{c}^2\mt{k}^2}$
\end{center}

Or d'après (11), f(k) varie comme k$^2$,
\begin{center}
$\int_0^\infty\frac{\mt{f(k)dk}}{\mt{y}^2 + \hbar^2\mt{c}^2\mt{k}^2}$
\hspace{1cm} se comporte comme \hspace{1cm}
$\int_0^\infty\frac{\mt{k}^2\mt{dk}}{\mt{y}^2 + \hbar^2\mt{c}^2\mt{k}^2}$
\end{center}



Soit, en faisant le changement de variable u = k/y,
comme y$\int_0^\infty\frac{\mt{du}}{1+ \hbar^2\mt{c}^2\mt{u}^2}$

On en conclut donc que $\mc{I}m$ T$_\mt{b}$( E$_\mt{a}$ $+$ iy ) varie comme -y$^2$.

D'autre part, comme pour t très grand, la contribution la plus
importante à A dans (24) provient de z voisin de E$_\mt{a}$ on peut dans les dénomineteurs remplacer z par E$_\mt{a}$. On a alors
\[
\mt{A}\simeq\frac{1}{2\mt{i}\pi}\frac{1}{(\mt{E}_\mt{a}-\mt{E}_\mt{b})^2}\int_0^\infty\mt{e}^{-\frac{\mt{yt}}{\hbar}}
\ 2\mc{I}m\ (\ \mt{T}_\mt{b}(\mt{E}_\mt{a}+\mt{iy})\ )\ \mt{dy}
\]

quantité qui varie donc comme

% 250 =

En faisant enfin le changement de varieble v = ty, on a
 dt  .
 dv et on montre ainsi que, pour t très grand,

la contribution du lacet varie en 1/t”.

Elle diminue donc beaucoup moins vite que la contribution du

résidu qui varie en e D

On peut résumer les résultats précédents, en fonction du temps,
sur lea figure ci-dessous. On a porté en ordonnée le module de la contribution
à <b O[U (t[b O > du résidu exponentielle en trait plein) et du lacet

(en trait pointillé).


* On trouverait une contribution du lacet en 1/t3/2 au lieu de 1/43 si on étudiait
la durée de vie d'un système sous l'effet de désintégration produisent des parti
cules ayant une masse différente de zéro, et non des photons de masse nulle, comme
c'est le cas dans l'émission spontances

% 251
On voit que cette dernière vart, pour t  0, d'une quantité

de l'ordre de 

5 mm et décroit pour t très grand en 1/43, Four un certain

temps t. très rrand, elle devient plus importante que celle de 1
 de

xponentielle.

Mais alors U, (t) cst pratiquement nul ct la désintécration est pour ainsi dire

terminée.

En d'autres termes, la loi de décroissance exponentielle de
l'état instable n'est pas thécrinnement vérifiée pour des tenps très longs.
pas
Elle l'est cependant tant que l'on est pratiquement capuble d'observer le vro
cessus de la désintégration.

Conclusion : La contribution à l'évolution de l'était instable du pôle eu point

Ze est de loin la plus importante. Elle nous à permis de prévoir une duréc de

ice en générel

vie et un déplacement rediatif de l'énergie de cet état. On n

la contribution du lecut introduit par la méthode des résidus, ce qui revient

à remplacer G, (2) par


fonction analytique qui n'a plus de coupure mais qui présente un pôle au voisi: (T

nage de CP au point E

5

en posant 

\subsection{}%4°) Produits de dé

Nous alions maintenant étudier les produits de la désintégration de

tégration de l'état instable :



l'état | b O >. Pour cela, il nous faut calculer l'omplitude de probabilité

<a  [U (t) | b O > pour que le système dans l'état | b 0 > à l'instant t =
se trouve dans l'état | e k À > représentant l'atome dans l'état fondemental a
en présence d'un photon KA, à l'instent t. Nous allons d'abord calculer la quan
tité <a| G (x) | bo >.

% 252

: +
\subsubsection{}%a) Etude de

Nous partons de la relation

Etant donnée l'hypothèse simplificatrice, seuls les Éléments de matrice
<a | | b O > sont non nuls et on a, en adoptant l'expression (25)

pour G, (2)



\subsubsection{}%b) Etude de <a K À | U(t) | b 0 >
: On calcule l'expression

par la méthode des résidus.
Le quantité  possède deux pôles, l'un au point

, l'autre au point  l'intégration par les résidus
donne :

Pour des temps t grands devant H/Tis la première exponentielle
de (27), de module 1, est prépondérante et la probabilité de trouver un photon
 émis est alors

% 253

La formule (28) nous donne les propriétés de la raie d'émission spontanée à partir du niveau b 3

- L'élément de matrice <  k À | A | b O > qui donne l'intensité de la raie, décrit la "force" de la transition et contient toute l‘information sur le diagramme d'émission (direction et polarisation des photons
émis).

- Le dénominateur d'énergie indique que la raie est lorentzienne,
centrée pour une énergie du photon égale à la différence de l'énergie de l'état
excité corrigée par le déplecement radietif À, et de l'état fondéemental, de
largeur Ts inverse de la durée de vie de l'état excité. Tous ces résultats sont
en accord avec le principe de conservation de l'énergie et le relation â‘incertitude.

\subsection{}% Etude de la diffusion résonnante :5°)



Nous allons maintenant rattacher l'étude de l'évolution de l'état

s

instable à celle de la diffusion résonnante des photons par un atome.
\subsubsection{}%a) Position du problème :

Nous allons calculer la section efficace différentielle de diffusion d'un photon par un atome, l'interaction électromagnétique entre l'atome
et les photons ne possédant, suivant l'hypothèse simplificatrice que nous avons
faite, que les éléments de matrice du type < a | b O > non nuls. Dans
ces conditions, la diffusion EA, État ne peut s'effectuer que par le passage de
l'atome de l'état a à l'état b par absorption du photon  et la retombée de

l'état b à l'état a avec émission d'un photon k°A°', suivant ie schéma ci-dessous :

% 254

Il est donc naturel de s'attendre à des phénomènes résonnants
lorsque l'énergie du photon incident sera de l'ordre de la différence d'énergie E  - E +

Nous allons tout d'abord calculer l'élément de matrice de la
résolvante <a K' À! | G(z) [aka >= Gérar ane NOUS pourrons alors évaluer l'amplitude de probabilité < a k'  | U(t) | a > pour que l'atome

dans l'état fondamental a en présence d'un photon  à l'instant t ait diffu

sé à l'instant t/2 un photon rate
Nous avons vu (étude de la matrice S, page 149 et suivantes)

que, pour t + , cette amplitude de probabilité n'admet pas de limite au

sens usuel, mais que si on passe en "représentation d'interaction"

<ak'a U() | a > admet comme limite au sens des distributions sur

les fonctions de K, l'élément de matrice 

Ayant calculé ainsi ras, fa» nous en déduirons l'élément de

la matrice de réaction VERS à  la section efficace de diffusion,
» .

\subsection{} Calcul de %b)

Partons des deux relations équivalentes

qui, combinées, donnent

On obtient alors

% 255
Soit, en remplaçant G, (2) per son expression approchée (25) :

\subsection{} Calcul de <  > et de la matrice S%c)

Comme nous l'avons vu aux pages 149 et suivantes, il i'aut,
afin de calculer la matrice 5, évaluer la limite au sens des distributions
pour  de  (t/2, -t/2), le signe  dénotant la représentacion d'intere
action.

Or, nous avens, d'après lu relation (82), page 1kh :

Nous devons done commencer par calculer les éléments de matrice de
Nous savons d'autre part que 

Nous sommes donc ramenés au calcul de l'élément de matrice
 que nous évaluerons par les résidus à partir de la

relation

La contribution du premier terme du second. membre de (30) donne immédiatement

Comme nous allons Être amenés à prendre le limite pour 

de l'expression obtenue, nous pouvons dans la contribution du second terme de
r
dont le module décreft en



(30) négliger le résidu du pôle 2, = E, + Ai

% 256
 On ne tiendre compte finalement que des résidus des pôles

On obtient alors

La relation (32-a) s'écrit, en remarquant que l'on peut, dans le premier

terme, en raison de la distribution , remplacer

% 257

Passons maintenant en représentatica d'interaction, (31) s'écrit :

Le passage en représentation d'interaction revient donc à supprimer Le terme

de phase exp. qui est en facteur dans l'expression (32«b)
et on obtient finalement :


% 258

Séparons dans le crochet du second membre de (33) les parties
réelleset imaginaires des exponentielles. Le terme associé aux parties réelles,

À, s'écrit :


C'est une fonction de k et k' qui reste bornée quel que soit te
Lorsque t augmente indéfiniment, sa pé éricde, hn/ct, tend vers zéro et la fonction oscille de plus en plus vite. L'intégrale de son produit par une autre
fonction .sommable tend donc vers zéro lorsque t + et À tend xers zéro cu
sens des distributions sur les fonctions de k etk' :
 lim
Le terme associé aux parties imaginaires, B, s'écrit :

Or, nous savons a'après p. 152 et 153) que la limite au sens des distributions

sin ce (k-k') 
lorsque t + « de tr) est égale à 

Le présence de la distribution 6 permettant de poser k = k' dans

(3k-b), on a finalement

% 259

On a donc, d'après (33), (3h-a) et (3k-ce) :



(35) 

(35) nous donne pour l'élément de la matrice  une relation strictement
équivalente à la relation (28), page 107 :
On en déduit immédiatement l'élément de matrice de réaction :


Remarque : la relation (36) pouvait s'établir beaucoup plus rapidement en par
tent de la relation qui définit la matrice de réaction ‘:

et en prenant pour > la forme explicite :

On obtient alors

Or  est autre que G. Nous avons donc
(38)
Si on prend pour * la forme approchée (2%, (38) redonne (36).

(39)
% 260 —

\subsubsection{} Calcul de la section efficace de diffusion%d)

 permet de calculer la probabilité de transition

par mité de temps de l'état KA à l'état K'A' (cf formules (68) pe 136 ou

La probabilité de transition par unité de temps et d'angle

solide W () s'obtient en sommant sur le module du vecteur d'onde k

(cf formule (69) p. 137)

Le flux du photon incident est et la section efficace

de diffusion s'écrit, compte tenu de (36) et (40)

Nous voyons ainsi que les éléments de matrice S (formule 35),
BR (formule 36) et 1a section efficace différentielle de diffusion (formule 41)

subissent une résonence lorsque l'énergie du photon incident est égale à la
différence des énergies de l'état excité corripée par le déplacement rediatif
A, et de l'état fondamental. La courbe. de résonance de la section efficace est

lorentzienne, de larseur l, érale à la larseur naturelle de l'état excité.
% 261

Nous avons vu également que l'énergie du photon diffusé est
égale à celle du photon incident (à cause de la fonction 6) ce qui correspond au principe de conservation de l'énergiee

Le diagramme d'impulsion et de polarisation du rayonnement

diffusé est fourni par le produit des éléments de matrice d'interaction

\subsection{} Préparation de l'état instable :%6°)
Nous avons, grêüce au calcul de G  commencé
par étudier l'émission spontanée à partir d'un niveau instable b, responsabie

de la durée de vie de cet état. Puis, grâce au calcul de G Nous avons

étudié le diffusion des photons par un atome à deux niveaux  et Do Afin que
notre étude soit complète, il nous faut maintenant étudier l'absorption d'un
photon par l'atome passant de l'état fondemental à J'état excité b. Cette
étude fait intervenir le dernier élément de matrice non calculé de ‘la résolvante, G, a° Elle est importante car elle nous décrit la préparation de
l'état excité et nous montre dans quelles conditions il est possible de 5éparer la préparation de l'état instable de sa désintégration et done de donner
un sens à le notion de durée de vie de l'état excité

Nous envoyons sur l'atome un ‘paquet d'onde de photons" de
polarisation À et d'impulsion centrée autour d'une valeur k avec une disper
sion 4k que l'on peut noter

f(k) est une fonction centrée en k, de largeur Ak qui représente le profil de
le raie excitatrice. Nous envisageons ici uniquement une dispersion en module
du vecteur d'onde, et non en direction.

A l'instant t = O, l'atome est dans l'état fondamental a en

présence du paquet d'onde
% 262 —

L'amplitude de probabilité pour qu'il soit à l'instant t dans

l'état | b O > est

Or nous avons

et  calcule comme  (cf p.252 , formule 26)

- Faisons l'hypothèse que l'élément de matrice 

varie peu le long du profil de la raie excitatrice et calculons

 Compte tenu de (L2), (43) et (kh) à l'aide de la méthode
des résidus. Il vient finalement

Afin de rendre l'expression (k5) plus maniable, nous allons dé
finir la transformée de Fourier de la fonction f (k) par la relation

car nous pouvons toujours supposer que f(k) est identiquement nulle pour k < 0.

(47)

(48)

(49)

(50)

% 263
On a de même :

 étant la fonction échelon : 

comme il est facile de s'en rendre compte en intégrant per lu méthode des
résidus. On peut alors considérer les deux intégrales intervenant dans (45)
comme des transformées de Fourier prises aux instants O et t d'un produit de
fonction de k et en se servant du fait que la transformée de Fourier d'un
produit est égale au produit de convolution des transformées de Fourier, on

montre aisément la relation :

qui pour t = O s'écrit :

A l'aide de (48) et (49), (h5) s'écrit ;

ou encore

(r) étant donné par la relation (L6),

% 264

Lorsque t = 0, les deux bornes de l'intégrale de (50) sont
égeles à zéro : on retrouve le fait évident que l'amplitude de probabilité
d'excitation est nulle à l'instant t = O.

Afin d'étudier comment cette amplitude de probabilité varie
avec le temps, nous allons distinguer deux cas fondamentalement différents :
\subsection{} Excitation "Narrow-line" :%a)
La largeur de la raie excitatrice Ak est très
petite devant TL,  on excite à l'aide d'une raie très fine, On peut alors
considérer que f(k) est la distribution 6 (k-k)s On a alors

ce qui reporté dans (50) nous donne pour amplitude de probabilité :

(51)

Après un régime transitoire durant un temps de l'ordre de AT,

la deuxième exponentielle du crochet de (51) devient négligeable devant la
première dont le module est égal à 1 et on peut alors écrire la probabilité

P, de trouver l'atome dans l'état excité b :

Le varietion de Pen fonction du temps est représentée par la figure ci-dessous :

% 265
Après un régime transitoire durant un temps de l'ordre de
HT,» la probabilité de trouver l'atome dans l'état excité prend une valeur
constante d'autant plus grande que la fréquence du photon est plus proche de

 On ne peut pas parler de durée de vie



la fréquence atomique
de l'état excité.
Ceci se comprend très aisément : la raie d'excitation étant
très fine, le temps de passage du photon devant l'atome (donné par 1a largeur
de la transformée de Fourier  (tr) ) est très grand : en termes classiques,
l'atome voit passer une onde plane qui, après un régime transitoire durant le
temps UT, met le dipôle électrique atomique en vibration forcée. La probabilité de trouver l'atome dans l'état excité est alors une constante qui devient
très importante lorsque l'excitation du dipôle est résonnante, c'est-ü-dire
lorsque 1a fréquence d'excitation est égale à la fréquence propre atomique.
\subsection{} Excitation "Broad-line" :%b)
Supposons maintenant, au contraire, que la largeur de la raie
excitatrice Ak est très grande devant la largeur naturelle :

Corrélativement, la transformée de Fourier âu profil de la raie
d'excitation, (rt), aura une largeur très faible, T, de l'ordre de  << E..

T représente physiquement le temps de passage du train d'onde sur l'atome.
On peut toujours écrire  (tr) en introduisant la fréquence
moyenne k de la raie d'excitation

où g a le largeur T de.

(50) s'écrit alors

(53)

% 266
Dès que t devient égal à un certain instant T tel que
 est sensiblement nul et l'intégrale de l'expression (53)



Th 1
peut être remplacée par l'intégrale définie

, laquelle, puisque T, << Te

est sensiblement égale à
dr. Cette intégrale est indépendante
du temps et le seul facteur dépendant du temps dans reste

alors le facteur e dont le module décroît en
D'autre part l'intégrant de l'intégrale définie I est le

produit de g (tr), de largeur T par un facteur de phase de fréquence
 « 11 sera donc une fonction de k maximum pour 
et qui s'annulera lorsque le nombre d'oscillations du facteur oscillant sur la
largeur T de g (tr) sera très grand, c'est-à-dire dès que

 de largeur

Afin de résumer les résultats précédents, les variations de

sera donc une fonction de k centrée en

sont schématisées sur la figure ci-dessous :

% 267
On voit que pendant un temps de l'ordre de T à partir de t =
(temps de passage du train d'onde sur l'atome) la probnbilité augmente, ce
qui correspond à l'excitation de l'atome par le train d'onde, Puis, la probabilité passe par un maximum et décroît exponentiellement avec une constante de
temps H/T,s ce qui correspond à la phase de désintégration de l'état excité
sous l'effet de l'émission spontanée.

Le maximum de la probabilité d'excitation est lui-même une fonction de la fréquence moyenne k du train d'onde excitateur, maximale lorsque la

largeur sensible

fréquence k est égale à la fréquence atomique
ment égale à Ak, largeur de la raie excitatrices

Tous ces résultats sont en accord avec le principe de conserva
tion de l'énergie et la relation d'incertitude.
En résumé, l'excitation “broad-line" est une excitation ‘en

pulse" qui permet de séparer les pheses de préparation et de désintégration de
l'état excité. Ce n'est qu'avec une telle préparation qu'il est possible de onner un sens à la notion de durée de vie de l'état excité.

Remarque : Il est possible de calculer entièrement le modèle d'excitation "broad-line"
en prenant pour profil de la raie excitatrice une raie de Lorentz centrée en k
de largeur 4k. On a alors

et on peut calculer rigoureusement l'expression (53) :

On trouve, en négligeant Th devant Ak dans le dénomineteur résonnant :

% 268

Le dénominateur d'énergie nous montre que le maximum de probabilité d'exciteti ARS nt un profil de Lorentz centré en k = 
largeur ,

Le premier terme du crochet est un terme de "préparation" qui
disparaît après un temps de l'ordre de T =  Le second est le terme de
désintégration de l'état excité.

Tous ces résultats illustrent les considérations générales faites à propos de l'excitation "broad-line".

\section{} Etude du cas général.%C.
\subsection{} Introduction :%1°)

Dans le traitement que nous avons présenté au  B, l'hypothèse
que seuls les éléments de matrice < a | | bO > sont différents de zéro
nous a permis de nous limiter au sous-espace É, sous-tendu par | b O > et
| a > et le calcul des éléments de matrice de G(z) revenait à inverser dans

ce sous=espace la matrice z = H :

% 269
En fait , possède d'autres éléments de matrice que
.
D'une part, | bO > peut être 1ié à des états |  >, (ce étant,
un état atomique autre que a); les processus correspondants seront importants
si l'énergie de l'état c se trouve comprise entre celles des états b et a, et
il y aura possibilité d'émission spontanée de b vers Ce

D'autre part, l'état | ak > pourra être lié à des états à deux
photons du type |  >, ce couplage étant le cause d'un déplacement
d'énergie de l'état fondamental par émission et réabsorption virtuelles de
photons

Pour tenir compte de tous ces couplages, nous alïons utiliser
une méthode analogue à. celle qui nous a permis de généreliser la formule de
Rabi dans le chapitre sur les états discrets (cf pe 210 Je

Nous allons considérer la restriction G de G au sous-espace é,
sous-tendu par | bO > et | ak > et montrer que ê est l'inverse d'un opérateur
de ce sous-espace,  L'opérateur  contient les effects du couplage
des vecteurs d'état de é, avec tous les autres vecteurs. Ce couplage a pour
effet d' "habiller" les états | bO > et | aka >, et donc modifie leurs énergies.
Il modifie également le couplage entre les états de ER en "habiliant" 
ments de matrice d'interaction.

Nous commençons par définir, de façon analogue au chapitre sur
les états discrets (p. 210 ), des opérateurs F et R relatifs au sous-espace 
ce qui nous permet de donner une forme condensée de cz) (c-),

Puis nous étudions à l'aide d'une représentation diagrammatique
la restriction à de R à l'espace , (c-3),

Enfin, nous reprenons point par point, en indiquent les modifications apportées, l'étude que nous avons faite au  B de l'évolution de l'État instable ( C-k), de la raie d'émission spontanée vers l'état fondamental ( C-5)

et de la diffusion résonnante ( C-6).

% 270

\subsection{} Définition des opérateurs F et R. Calcul de G(z) :%2°)
\subsubsection{} Définitions Rappelons d'abord la définition des projecteurs%a)

Nous procédons ensuite exactement comme à la page 210 :

Nous posons
,
Nous définissons ensuite F(z) par la relation

commutant avec P, on peut écrire
(55-b)
De (55-b), on déduit

(56)

Nous définissons alors R(z) P, per la relation

et nous appelons R(z) la projection de R(z) dans le sous-espace 6. :

\subsubsection{} Calcul de F(z) et R(z)%b)
Nous calculons F(z) P, comme à la page 212 et nous en

déduisons d'après la relation (kh) de cette page :

(60)

(61)

(62)

(63)

% 271

(53) se développe en série de puissance de .

On a de même (relation 45, p. 212) :

où l'on peut déduire le développement de Yigner-Brillouin de R(z) :

\subsection{} Caleul de G(z)%c)
On montre comme à la page 211, relation :

(63) signifie que dans le sous-espace  opérateur G(z) est l'inverse de

Dans le modèle simple du  B, G(z) était l'inverse de
Nous sommes donc ramenés à un problème analogue à celui du  B, circonscrit
au sous-espace à condition de remplacer 8. par R(z)
\subsection{} Étude de H(z) :%3)
\subsubsection{} Représentation diagremmatique des éléments de matrice
de R(z)%a)
Le développement de R(z) en série des puissances de ,
est représenté par la formule (62) Nous voyons que pour calculer à un ordre n

déterminé l'élément de matrice de R(z) entre deux états de be il faut passer
par n-1 états intermédiaires situés en dehors de É, et diviser par les dénominateurs d'énergie correspondants.

Nous pouvons représenter chaque terme du développement d'un élément
de matrice de K(z) par des diagrammes qui permettront de "visualiser" le processus

physique correspondant à ce terme.

%272
Donnons-en quelques exemples :

nu
Soit à représenter l'élément de matrice < a(z)bO >

Le terme du ler ordre est, d'après (62)

KA
: ésentons nn
< aka [dB [bo > que nous représentons par le diagramme

Le terme du 3e ordre comprend par exemple l'expression :

-(Ce n'est pas là la seule contribution au terme du 3e ordre.
On doit, par exemple, également envisager les expressions dans lesquelles
interviennent des états intermédiaires à deux photons.)
Cependant, nous représentons l'expression (6k-a) par le diagramme

Ces deux exemples nous montrent clairement comment les diagrammes sont construits et nous permettent de préciser les règles suivantes :
- Chaque diagramme correspond à une expression intervenant dens le terme à un
ordre déterminé du développement d'un élément de matrice de É(z) suivant

l'expression (62).

Le diagramme se lit dans le même sens que l'élément de matrice correspondant,

Les états atomiques sont représentés par des traits horizontaux et les photons
par des "tortillons".
Les extrémités du diagramme sont les mêmes que celles de l'élément de matrice.

Chaque élément de matrice de e. correspondant à une interaction est repré
senté par un ronde

- Les états intermédiaires sont les mêmes que ceux de l'expression représentée,

Pour obtenir l'expression correspondante à chaque diagremme, il faut diviser

c'est-à-dire qu'ils doivent être tous extérieurs à

le produit des éléments de matrice par autant de dénominateurs d'énergie qu'il
y a d'états intermédiaires dans le diagramme (chaque dénominateur d'énergie
s'écrivant z = E, E étant l'énergie de l'état intermédieire envisagé) et
gommer sur tous les étuts intermédiaires possibles (c'est-ä-dire sur tous les
états intermédiaires extérieurs à 8 et représentés sur le diagramme),

Pour obtenir les éléments de matrice de R(z ), i1 faut sommer
tous les diagrammes possibles à un ordre déterminé et à tous les ordres:

Cette représentation diagranmatique permet d'écrire ce façon
condensée les éléments de matrice de Az) et surtout d'expliciter les proces=
sus physiques correspondant à un élément de matrice et à un ordre déterminés.
Ainsi, pour le diagramme (6h-b), l'atome dans 1'état b émet un photon
et passe dans l'état ca; puis il réabsorbe ce photon et passe dans l'état
db; enfin il émet un photon kÀA en passant dans l'état a.

L'hypothèse simplificatrice du  B revenait à remplacer X(z )
par son expression au prenier ordre, P 4 Poe

Nous allons ici nous attacher à étudier la correction au 2e
ordre aux résultats simples obtenus. Ceci nous permettra de dégager des résultats physiques simples et intéressants. Le calcul aux orûres supérieurs, quoique
plus long, ne présente pas de difficultés et s'effectuerait de la même manière o

Nous allons commencer par expliciter le développement de R(z) au

2e ordre inclus en ..
%274

\subsubsection{} Développement de R(2) au 2e ordre inclus en ,%b)
Hz) présente Fe types différents d'éléments de

matrice : 

Au deuxième ordre inclus : Ka
(z) est représenté per
b est représenté par (1)

quelconques

I1 est facile de voir qu'au second ordre inclus, seuls les six

diagrammes précédents sont possibles. Nous pouvons faire plusieurs remarques :
% 275 =

Nu
a) Seuls les diagrammes représentant RE D et Bi etx sont au premier ordre :

c'étaient ‘eux qui intervenaient dans le traitement du  B.:

8) Le diagramme (k) n'est possible que si les photons extrêmes sont identiques,
Le photon. KA joue alors le rôle de "spectateur" et n'intervient pes dons les
processus d'interaction. Les diagremmes (5) et (6) sunt possibles ausle que
soient les photons KA et k'A'.

y) Les photons de même K et À, en taut que bosons, sont des particules indis…
cernables. Or il n'a pas été tenu compte de l'indiscernabilité pour éteblir

les diagrammes précédents. IL est évident notamment que les diagrammes

correspondant au même processus physique sont équivalents et ne doivent être
comptés qu'une fois. Nous verrons au c) comment résoudre cette difficulté.
\subsection{} Séparation de R(z) en deux parties%c)

- Définition de R'(z) et R'"(z)

Posons par définition

Nous avons ainsi séparé R(z) = R(z) + R(z) en deux parties,

dont l'une, R(z), est diagonale.

(66)

(67)
% 276

Etude de R(z)

Nous allons montrer que les éléments de matrice

sont reliés très simplement à des quantités intervenant dans le calcul de la
self énergie de l'état fondamental.

Etudions pour cela la modification apportée par ie couplage
électromagnétique à l'état fondamental | a0 > et à son énergie.

Nous sommes amenés, comme nous l'avons fait dans le chapitre sur
les états discrets, à définir les projecteurs P, = | 20 > < a0 | et Q=l1-P,
sur l'état normé | a0 > et dans l'espace complémentaire, puis à définir

1 "Opérateur Shift" À, dont le développement de Wigner Brillouin est

. Pitoce
ce qui, à l'ordre le plus bas, donne

On montre iverente comme nous l'avons fait au  B que

et que  est une fonction analytique dans tout le plan complexe privé

d'une coupure partant de. étant l'énergie atomique la plus basse
au-dessus de l'état fondamental, E, étant inférieur à est analy



tique au point E a t par (68), on voit tout de suite que .

Il est même réget tif, car on voit sur l'expression (67) que tous Les dénominateurs d'énergies  sont négatifs.

% 277
(69)

(70)

(71)

On peut alors calculer l'élément de matrice de la résolvante

Les pôles de G (2) vérifient l'équation implicite

Il existe notamment une valeur propre ë, voisine de E, qui vérifie l'équation

dans laquelle on peut, à une bonne approximation, remplacer

par
On obtient alors

ce qui montre que sous l'effet de l'interaction électromagnétique, l'état
fondamental se trouve "habillé" par des photons virtuels et que son énergie

est abaissée d'une quantité réelle, | 5 | : c'est le 'Lamb-shift" de l'état

fondamental.

Revenons meintenant au calcul au second ordre de R'(z) et calculons plus précisément 

(73)

% 278
D'après les conventions que nous avons faites sur les diagrammes

Deux cas différents peuvent alors se présenter :
a) K'A'  KA : les deux photons et  sont discernables et on a alors

L'intégrant de l'expression (66) est alors égal à celui de

l'expression (61), à cela près qu'au dénominateur,z est remplacé par .
8)  les deux photons K et  sont indiscernables :

On sait alors que

Les opérateurs de création et d’annihilation de bosons qui sont

contenus dans l'opérateur a font intervenir des coefficients  2. On a

donc

l'indiscernabilité des photons introduit donc le facteur 2 dans le diagramme

Mais il faut également ne tenir compte qu'une fois des deux

diagrammes indiscemmebles

% 279
qui ont la même contribution. Il revient manifestement au même. de ne pas
tenir compte du facteur TS et de faire intervenir séparément les deux diagranmes équivalents ci-dessus.
On peut généraliser ce résultat à un ordre quelconque et ajouter

à nos conventions sur les diagrammes la règle suivante :
- On tient compte séparément de tous les diagrammes équivalents dont la géométrie est différente.
- On ne tient pas compte de la statistique des bosons, c'est-à-dire que pour
le calcul des diagrammes, on admet que

n fois n+1 fois

Cet artifice de calcul revient à ne pas symétriser les fonctions
d'ondes de bosons indiscernables et à remplacer le principe de symétrisation
par un principe d'addition des amplitudes de probabilités associées à des
chemins "classiques" indiscernabies quantiquement. Nous evons vu un résultat
équivalent pour des fermions, lorsque nous avons étudié à l'eide de la matrice

8, les sections efficaces de diffusion d'électrons : il fallait alors retren

cher les amplitudes de probabilités associées aux chemins "classiques" indiscernablese

Avec cette convention supplémentaire, les intégrants des expressions
(6T) et (73) deviennent les mêmes (à cela près qu'au dénominateur, z est remplacé par , que  et  soient identiques ou non. On a donc

% 280
Mais il ne faut pas oublier qu'il faut alors, dans R"(z), faire
intervenir le diagramme

Ainsi, si R'(z) est uniquement diagonal, R"(z) possède égalees

ment une partie diagonale : pour nous résumer, R'(z) fait intervenir les
diagrammes diagonaux (1) et (k) de la page 27, R"(z) les diagrammes (2),
(3), (5) et (6)

Ceci étant, nous pouvons poser

Nous avons alors la relation bien connue

Nous pouvons maintenant reprendre les différentes études que
nous avons faites sur le modèle simple du  B :

\subsection{}% Evolution de l'état instable :4°)

\subsubsection{} Calcul de Gb 2)%a)

En itérant (77), on obtient :

ce qui nous donne :

% 281
(78)

(79)

(80)

(81)



Pour calculer explicitement Gp(2) en ne faisant intervenir
que les corrections à l'ordre deux, il suffit de ne tenir compte que des

deux premiers termes du second membre de (78). Le dernier terme, en effet,

dt
fait intervenir  et pour faire apparaître à nouveau G il
faut développer au premier ordre, ce qui fait intervenir des

termes d'ordre supérieur à deux dans l'équation fournissant G DZ e

Finalement, en explicitent au 2e ordre inclus :

On obtient :
 avec

R',p(2) étant donné par la formule (79).

solo

(82)

(83-a)

(83-b)

% 282
On montre aisément, comme au paragraphe B, que (2) est une
fonction analytique qui présente une coupure sur l'axe réel” et qui possède
un pôle z) situé dans le deuxième feuillet de Riemann. La partie réelle de
ce pôle donne le déplacement radiatif de l'état instable | bO > et se partie imaginaire, la durée de vie finie de cet état, |

Tout comme au  B, on peut, pour étudier l'évolution de l'état
instable, remplacer dans (80) la quantité petite  par sa

valeur au point  On obtient alors

Evaluons séparément les quantités  et .

On peut, dans l'expression (81), négliger la quantité du
deuxième ordre en  et écrire

L'expression (83-a) est alors identique à l'expression

que nous avons été amenés à calculer au  B, si bien que l'on a

représente donc la contribution à la durée de vie et au déplacement du niveau b du couplare avec l’état fondamental a.

* Le point de branchement de la coupure correspond au point de branchement de
la coupure de  lequel est Alors que dans le modèle simple
du  B, le point de branchement de G (z) correspondait à l'énergie non perturbée
E, de l'état fondemental, il correspond ici à l'énergie vraie, corrigée par le
Lamb-shift de l'état fondamentale

% 283

On a, d'après (79)

Chacun des termes de la somme sur cfa de l'expression (8h) est analogue,

au remplacement de a par c près, à l'expression (83-a).

représente donc la contribution à la durée de vie et au dénla

cement du niveau b, du couplage «vec tous les états  autres que l'état fondamental.
De façon plus précise, nous avons

(85)
(86-a)
(86-b)

et enfin

(87)

En réunissent

(88)

(82), (8k-b) et (87), nous obtenons :

(89)

(90-a)

(90-b)

% 284

\subsubsection{} Calcul de U,,(t)%b)

On obtient immédiatement Up(t) par intégration par

les résidus à partir de l'expression (88) de Gp(2) 3

Nous voyons ainsi que sous l'effet de l'interaction électromagnétique, le niveau b acquiert une largeur naturelle V qui est la somme de
deux termes :
i'un To est aû à l'émission spontanée vers l'état fondemental a et est four
ni par le modèle simple étudié au  B;
l'autre T est aû à l'émission spontanée vers les niveaux c autres que e
Seuls y contribuent d'ailleurs les niveaux c tels que l'be est différent de
zéro, c'est-à-dire d'après (86-b) les niveaux pour lesquels l'argument de la
distribution   peut s'annuler et qui sont les niveaux

 l'émission spontanée n'est possible que vers les niveaux c d'énergie
ce

inférieure à celle de be

Le niveau b subit de plus un déplacement radiatif qui est aû au
couplage avec tous les états (a dû au couplage avec l'état fondamental a et
a! aû au couplage avec tous Les états

\subsection{} Etude de La désintégration de b vers l'état fondamental :%5°)

ro Tout comme au  B, nous devons pertir du calcul de la quantité
% 285
(91)

(92)

(93)

Cas

a) Calcui de Ca (z)
La relation (77) nous permet d'écrire immédiatement

Le second terme du crochet de (91) est un terme d'ordre supé
à 
rieur. En effet  est un terme du seconû orûre et si on développe

on obtiendra un terme du troisième ordre que nous négligeons dans cette étude.

L(2) suivant (91) de façon à rendre l'expression de  explicite,

En explicitant R" (z) et en remplaçant CE 2) par l'expression (88), on

obtient alors :

Cette quantité présente un pôle au voisinage du point
La quantité, du second ordre, est négligeable loin du pôle et peut
être remplacée près de celui-ci par sa valeur au point. On peut
donc partout remplacer à une bonne approximation  par
 “Lamb=shift" de l'état fondamental et remplacer (92) par


(94)

% 286

\subsection{} Etude de la probabilité de transition vers l'état%b)

L'expression (93) est analogue à l'expression (26)
de la page 252. On en déduit que la probabilité de trouver un photon KA
émis, l'atome étant dans l'état fondamental a, à un instant t grand devant
 est, par simple analogie avec la formule (28) :

La raie d'émission spontanée vers l'état fondamental est lorentzienne avec
un maximum correspondant à une énergie du photon égale à la différence des
énergies de l'état excité et de l'état fondamental, chacune d'entre elles
étant corrigée par la self énergie globale des niveaux (rappelons que dans
le modèle simple du  B, seul intervenait le déplacement radiatif a, du
niveau b sous l'effet du couplage avec l'état fondamental).

La largeur de la raie est égale à la largeur naturelle totale
du niveau b, Le (et non pas seulement à la largeur T due au processus d'émission vers l'état fondamental). Sur une transition particulière b + a, on voit
donc la largeur due à toutes Les transitions possibles b + c, ce qui est en .
accord avec la quatrième relation d'incertitude : l'état b évolue sous l'effet
de tous Les processus d'émission spontanée possibles vers les niveaux
d'énergie inférieure à b. Son énergie n'est pas déterminée à mieux que Y,.
près et il est normal que la mesure de cette énergie sur une transition
déterminée (par exemple b + a) donne une dispersion d'énergie 

Quant à l'intensité et au diagrenme d'émission de la raie, ils
sont déterminés par le carré de l'élément de matrice  et

sont les mêmes que pour le modèle simple du
% 287

(95)

(96)

\subsection{} Etude de la diffusion résonnante :%6°)

Nous allons maintenant reprendre le problème de la diffusion
résonnante déjà étudié avec le modèle simple du .

Nous calculons d'abord Gene axe) puis

et nous étudions la limite t. Nous verrons alors

dans quelle mesure il est possible de définir une matrice S qui nous per

mette de calculer la section efficace de collision et de mettre en évidence
les différences avec Le modèle élémentaire du  B,
\subsubsection{}%
Nous partons de la relation analogue à la relation (29):

qui nous donne : 

+ termes d'ordre supérieur

On  tout de suite, dans (96), remplacé Gp (2) per son expression (88).
Les termes d'ordre supérieur que nous avons négligés sont ceux qui feraient
apperaître dans le développement explicite de Ge, ,, + (z) des termes

d'ordre supérieur à 2 en 76.
% 288

ont un pôle près

de  On peut, sans commettre une erreur importante, remplacer,dans
les arguments des fonctions , et , z par , ce qui fait notamment
apparaître  Lamb-shift de l'état fondamental. On peut alors

réécrire (96) :

Eta
ment près de E, par  de à par 6 et de LE par  les 1er et 3e termes

(z) se présente ainsi comme la somme de trois termes. Au remplace

sont identiques à ceux de l'expression (30) du  B. Le deuxième terme,

nouveau,

\subsection{} Calcul de  et de la matrice S%b)
Comme nous l'avons fait au  B, nous pouvons calculer

à partir de l'expression (97) de

qui déeroît exponentiellement avec le temps, une expression analogue à l'expression (32-b)

On trouve alors, en négligeant le résidu au pôle z = ,

% 289

(98)

L'accolade comprend trois facteurs. Les deux premiers sont, aux changements
déjà signalés près, rigoureusement analogues à ceux de l'expression (32=b);
le troisième est nouveaue .

Afin de calculer la matrice 5, il feudrait maintenant passer en
représentation d'interaction par rapport à He et calculer

.
Nous avons vu, (p. 257 ) que le passage en représentation d'in
teraction revient à multiplier l'élément de matrice (98) par l'exponentielle

Nous avons donc



(99)
l'expression entre accolades étant la même que celle de la relation (98). Il est
facile de voir, comme nous l'avons fait p. 259 , que l'expression entre accola
des tend, au sens des distributions, lorsque t +  , vers

% 290

Mais, en raison du facteur oscillant exp — T ét de l'expression (99), nous voyons que  n'admet pas de limit,
au sens des distributions lorsque t >  Ceci semble en contradiction vec
les résultats de la théorie formelle des collisions. En fait, la contre Lo
tion n'est qu'appurente, En effet, nous avons pris comme hypothèse de  part
dens l'étude de la théorie des collisions le fait que 1l'in’eraction ent 'e
les deux systèmes décroît suffisamment vite à l'infini, ce qui correspc à à
l'hypothèse physique qu'à un instant suffisamment éloigné dans le passé ou
dans le futur, les deux systèmes n'interagissent pas et ar? les états ?
les énergies non perturbés sont bien adaptés à l'étude du problème physique
et sont, par suite de ‘bons états" et de "bonnes énergies" «

Dans le cas où l'un des deux systèmes est un champ, cette ayrothèse n'est manifestement pas vérifiée. Nous savons que mme si le pho’on
diffusé est loin de l’atome, l'atome interagit toujours avec les"fluctretions
du vide" et que son énergie se trouve ééplacée d'une quan* ité 6, qui c:nstitue
le "Lamb=shift" de 1'état fondamental «

Les "bons états" et les “bonnes énergies" du problème physique
de diffusion envisagé ne sont donc pas les états propres rt les énergies pro
pres de 1‘Hamiitonien non perturbé Jo mais ces états propres et ces éner=
gies propres corri.nés par l'effet des Mrluctuations du vide". À 1'ordre le
plus bas, la correction intervient sur les énergies, mais ne modifie p2s les
états propres, si bien que 1lfou peut prendre pour état représentent le 5ÿ5=
tème physique, longtemps event ou après la collision, les états | aka > ,
d'énergie , et | aka > d'énergie 

La "représentation" d'interaction qu'il faut choisir .ne doit
donc pas être prise par rapport à 1°Hamiltonien non perturbé Ho mais par

rapport à l'Hamiltonien non perturbé H, corrigé par le Lemb=shift de état

% 291 

fondamental. Nous la @istinguerons de la précédente par la notation Ts

On e alors :

 admet donc pour limite lorsque t +
 donné par la formule (100).

des photons par l'atome, Il permet d'effectuer le calcul de la matrice de

représente l'élément de la matrice S pour le problème de diffusion

réaction et de la section efficace des diffusions.
Nous allons maintenant en faire une étude plus détaillée.

\subsection{} Etude de la matrice S%c)
Les deux premiers termes de l'expression (100) qui

s'écrivent
< etat |

sont en exacte analogie avec l'expression (35) de la page 259 relstive au
modèle simple. Le deuxième terme résonne au voisinage de Mck = E,-E
Nous verrons plus loin que le troisième terme, nouveau par rap=
port à la formule (35), est très petit au voisinage de cette fréquence, Les
principales propriétés de le diffusion résonnante au voisinage de

découlent done des propriétés de s pri:

(102)
% 292

Nous voyons donc que la section efficace de diffusion subit
une résonance lorsque l'énergie du photon incident est égale à la différence des énergies de l'état excité corrigé rar le déplacement radiatif total
ôp et de l'état fondamental corrigé par le Lamb=shift (rappelons que dans
le modèle simple du  B, seul intervenait le déplacement radiatif A du
niveau b sous l'effet du couplage avec l'état fondamental).

La résonance est lorentzienne et sa largeur est égale à la largeur naturelle totale du niveau Y (et non pas seulement à la largeur à due
au processus d'émission vers l'état fondamental).

 Le diagramme d'impulsion et de polarisation du rayonnement diffusé, fourni par le produit des éléments de matrice d'interaction
<  > est le même que dans le cas du modèle simple
du  B.
Etudions maintenant la correction apportée par le troisième ter
me de l'expression (100) :

De

D'après la définition que nous avons donnée de R", nous avons
% 293

Le premier terme de l'expression (102) (cela est très clair
sur le représentation diagremmetique) correspond à la diffusion de L'état
| ex > à l'état | a > per j'intermédiaire d'un état  différent de be
C'est au voisinage de Ack = E, E, petit terme correctif qui représente
les ailes desrésonances de la section efficace de diffusion associées aux
états instables c différents de be

Le deuxième terme de l'expregsion (102) (voir sa représentation
diagrammetiqug) correspond au processus physique dans lequel l'atome émet
le second photon f'a! avant d'avoir absorbé Le premiere L'énergie n'est maniféBtement pas conservée gansl'étet intermédiaire et un tel terme sera né
cessairement très petite C'est ce que montre son expression explicite dans
laquelle on voit que le dénomineteur E, - Ea YHck' ne peut jamais être nul.
Ce terme que l'on pourrait considérer comme représentant l'aile d'une rÉéso=
nance correspondant à une fréquence négative du photon incident est ce qu'on
eppelle le terme “antirésonnant" de la formule de dispersione

En résumé, l'étude de 1'interaction électromagnétique que nous
venons de faire au second ordre nous  permis de corriger et de préciser les
résultats du modèle simple au premier ordre étudié au  B..

Nous avons vu qu'en ce qui concerne les intensités et Les diagrammes de rayonnement, l'étude au second ordre confirme Les résultats du premier
ordre.


Il est évident que l'espace Bo dans lequel nous avons circonscrit notre problème est adapté à l'étude des résonances  + b et que dans un traitement de
perturbation les autres résonances a + cb ne peuvent intervenir que comme des
corrections. Si on voulait étudier la résonance a + cb, il faudrait adapter
le problème au sous-espace  sous-tendu par les vecteurs

% 294
En ce qui concerne les déplacements radiatifs, nous avons pu
introduire le Lamb=-shift de l'état fondamental et nous avons montré que le
déplacement des états excités est dû non seulement au couplage avec l'état
fondamental, mais au couplage avec tous les autres états.

En ce qui concerne la durée de vie, nous avons vu que la largeur naturelle d'un état instable est due non seulement à l'émission spontenée vers l'état fondamental, mais à l'émission spontanée vers tous les états
d'énergie plus basse. |

Nous avons enfin montré comment on peut, sur une résonance de
diffusion, tenir compte des corrections apportées par les ailes des autres

résonances et.les termes antirésonnantse

\section{} Application : Diffusion résonnante au voisinage d'un ‘croisement de%D.

niveaux : Effet Hanle; effet Frankeno

Une application intéressante de la théorie précédente est l'étude
de la diffusion résonnante au voisinage d'un croisement’ de deux niveaux d'énergie etomiques.

\subsection{} Description de L'expérience :%1°)

Considérons un atome dont les niveaux d'énergie sont fonction.
linéaire d'un paramètre, par exemple du champ magnétique Be Appelons e l'état
fondamental que nous supposons diamagnétique et b et b deux niveaux qui se

croisent pour une valeur B, de B (voir figure).

Energie

(103)
% 295

Soient , et  deux polarisations orthogonales qui permettent d'exciter sélectivement les reies b,-8 et b,-e : en d'autres termes,

l'interaction électromagnétique Gr n'e päs à'élément de matrice entre

 et entre

Par exemple si  et  sont des états propres de la composante le long de

et ,

B du moment cinétique de valeurs propres respectives +1 et +1,  0

1
ne sont autres que les polarisations 0, et 0. le long de 
Si maintenant le polarisetion du photon n'est ni  ni  

mais une superposition linéaire  de A et de  :

on dit que L'on a ne polarisation cohérente : les éléments ès metrice dé
 entre  et d'une part, entre  et 
d'autre part, sont tous les deux différents de zéro : il est possible à
l'aide de la polarisation Ê d'exciter simultanément les deux niveaux b;

et b, qui se croisent.

Supposons maintenant que ion irradie cet atome avec un fais
ceau lumneux de polarisation cohérente E, de répartition spectrale u(k)
(1a lergeur de u(k), 4k, est supposée grande devant les largeurs naturelles
des niveaux b. et Ds ). Etudions pour un ‘champ B voisin de B, le lumière diffu
1
sée dans une Lertaine direction, avec une polarisation cohérente E”

% 296
On constate que l'intensité de cette diffusion passe, lorsqu!
on balaie le champ B, par une résonance centrée en B 5 dont la largeur dépend
de le somme des lergeurs naturelles des niveaux b, et bd, qui se croisente

L'intensité(ainsi que le forme) de la résonance dépend naturellement de La position de la fréquence moyenne de la raie excitatrice u(k) par
rapport à la fréquence de la transition atomique  — bis Dos mais elle dépend
aussi des direction et polarisation de la lumière incidente (directions et
polarisation d'excitation) et des directions et polarisation de la lumière
diffusée (directions et polarisation de détection).

On peut observer le phénomène précédent soit dans le cas où les
deux niveaux b; et b sont deux sous-niveaux Zeeman d'un même niveau atomique
excité qui se coupent en champ B, nul (on l'appelle alors l'effet Hanle),
soit dans le cas où les deux niveaux b et b, sont deux sous-niveaux appartenent à des niveaux atomiques différents qui peuvent se couper dens un champ
différent de zéro (effet Franken).

Les résonances que nous venons de décriré ont des applications
très importantes. La mesure de leur position permet de déterminer les valeurs
des champs correspondant à des croisements de niveaux d'énergie et d'en déduire par suite les valeurs de structures fines, hyperfines... La mesure de leur
largeur fournit très simplement la largeur naturelle d'un niveau excité.

Afin d'en donner une interprétation théorique, nous allons tout
d'abord calculer la section efficace pour un photon monochromatique de polarisation cohérente ( 2), puis nous caleulerons le signal détecté physiquement

en pondérant la section efficace par le profil de la raie excitatrice u(k)
% 297

\subsection{} Calcul de le section efficace de diffusion (dans le ces d'une excitation monochromatique) :%2°)



Nous allons commencer par calculer l'emplituée de diffusion
pour des. photons monochromatiques

Nous devons envisager ici deux états atomiques excités, La sie
tuation est cependant identique à celle. que nous avons étudiée aux 
à condition de remplacer le sous-espace . par le sous-espace 
tendu par les vecteurs et (m pouvant prendre les valeurs
, Nous nous placerons dans l'hypothèse simplificatrice où Les seuls
Éléments de matrice de de: non nuls sont les éléments 
Toutes les relations démontrées au  B se transposent alors à condition de
remplacer partout le projecteur  par Le projecteur
.

Nous nous plaçons dans l'hypothèse âu modèle simple du  B
parce qu'il est plus facile à traiter et contient tous les résultats essen—
tiels. La seule modification notable qu'apporte le traitement complet est qu'il
faut remplacer partout E, par  les déplacements radiatifs bn par les
déplacements complets bn et les largeurs naturelles Th par les largeurs naturelles totales Ybm° 11 suffira d'effectuer. formellement ce remplacement dans
tous les calculs.

En généralisant le formule (35), page 259, on obtient :
% 298 

D'où L'élément de matrice de réaction :

(105-a)


(105-b) A, =

(106)

avec
La section efficace est  d'après (k1), égale à

Elle est donc proportionnelle à De façon précise :

La formule (106) peut s'interpréter physiquement de façon très simple :
deux chemins, représentés par les diagrammes ci-dessous sont possibles pour
le processus de diffusion (et ces deux chemins seulement dans l'hypothèse

du modèle simple choisie),

"chemin" 1 "chemin" 2

Le "chemin" 1 représente une absorption du photon À  avec pas
sage de l'atome dans l'état excité b,» puis une émission du photon ' e”
Le"chemin" 2 représente un Processus analogue, mais avec passage de l'atome
dans l'état excité be
% 299

A chaque “chemin” est associée une amplitude de probabilité

(A, pour le "chemin" 1, A, pour le “chemin" 2), -L'emplitude de probabilité

associée au processus global est la somme Aj+hse La probabilité du proces2
sus global est le carré du module de cette somme. En plus des termes "carrés",

il apparaît un terme "rectangle", 2 ; qui décrit l'interférence
entre les deux chemins possibles. C'est ce terme interférence qui va Être
- responsable de la variation de ls section efficace au voisinage du point de
croisement. ‘
Remarque importante : A. et À, ne sont tous les deux différents de zéro simultaRemarque Importante 15/2 SIM
nément que parce que  et ’ sont tous les deux des superpositions 1inéais

res de , et ; (voir relation 105-b)
 . Le
Si Cet  n'étaient pas tous les deux des polarisations con
hérentes, un seul des chenins 1 et 2 resterait “ouvert et le phénomène d'inAgren es .
terférence disparaîtrait.
\subsection{} Calcul du signal détecté :%3°)
Nous avons calculé la section efficace de diffusion pour un pho. +
ton monochromatique de vecteur d'onde k.
En fait, la soures n'est pas monochromatique : elle émet des photons dont la répartition en fréquence est donnée par la fonction u(k) de lar
geur , grande devant  autour d'une valeur k

< Ne pas confondre ce modèle dans lequel la source émet des photons incohérents

entre eux avec une répartition en intensité u(k) et le modèle choisi pour traiter
1a préparation de l'état instable au  B-6°) dens lequel l'atome interagissait au
cours d'un processus élémentaire avec un photon constitué par une superposition

cohérente, pondérée par la fonction f(K), de photons moncchromatiques. Alors qu'il
fallait pondérer par f(x) les emnlitudes de probabilité (cf formule L2), nous de
vons ici pondérer la section efficece, donc 1a probubilité de diffusion par u(k)e


(110-a)

(110-b)

(111-a)

(111-b)

(112)

(113)

% 301


et de même

Comme  est grand devant  les expressions (110a) et
(110-b) varient peu avec k, et k, lorsqu'on varie k, ou k, autour du point
de croisement des niveaux b, et b d'une quantité de l'ordre de quelques
Helpe

Nous nous placerons dans la suite dans ces conditions et nous
poserons, en appelant k, la valeur ccemmune de k, et de k,, au croisement de

e

niveaux :

Nous voyons que les deux termes carrés sont sensiblement constants au voisinage du croisement de niveaux. Il ne nous reste plus qu'à étudier le terme rectangle :

On peut écrire

et compte tenu de (112) :
% 302

TL Étent petit devant largeur de u(k), la fonction
se comporte comme
vis à vis de l'intégration dans la formule (113). On a donc :

On a de même

. Les deux parties principales qui interviennent dans les expressions (11h-a) et (1llheb) sont pratiquement égales si , hypothèse

que nous avons admise en supposant que k, et k, ne varient que de quelques

ko, autour du point de croisement Les deux parties principales s'annulent |

donc dans l'expression (113) et on obtient en posant 

et finalement, compte tenu de (107), (1llea), (111-b) et (115), on a :
% 303

Nous voyons sur la formule (116), que l'intensité diffusée
dëns une direction donnée est résonnente au point de croisement k=k,
la résonance apparaissant uniquement sur le terme d'interférence entre les

deux chemins de diffusion possible. Le signal de résonance disparaît lorsque
 somme des largeurs naturelles des deux niveaux qui se
croisent. (ce qui. justifie. l'approximation. que nous. avons faite pour étudier
le phénomène. en posant k, =k, = %k dans u(k) ).

L'intensité de la résonance dépend de la fréquence d'excitation
moyenne k et.elle est d'autant plus grande que k, est plus proche de k, valeur
pour laquelle u(x) est maximum. Elle dépend également des polarisations et des
directions d'excitation et de détection par l'intermédiaire de B,B.*, Enfin, .

la forme de la raie de résonance, donnée par l'étude de

, est un mélange de courbes d'absorper + ike (k,-k,)

tion et de dispersion, la proportion des deux formes dépendant encore de

B,B, donc des polarisations et des directions d'excitation et de détection,


%
%
%====================== INCLUSION DE LA BIBLIOGRAPHIE ======================
%
%récupérer les citation avec "/footnotemark" : 
\nocite{*}
%
% choix du style de la biblio
\bibliographystyle{plain}
%
% inclusion de la biblio
\cleardoublepage
\addcontentsline{toc}{chapter}{Bibliographie}
\bibliography{bibliographie.bib}
%
%====================== FIN DU DOCUMENT ======================
%
\end{document}
%%%%%%%%%%%%%%%%%%%%%%%%%%%%%%%%%%%%%%%%%%%%%%%%%%%%%%%%%%%%%%%%%%%%%%%%%%%%%%%%%
