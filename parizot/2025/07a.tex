%2025 - 7a
\chapter{Dérivée covariante, connexion}
\section{07a Chapitre 5 : Dérivée covariante, connexion}

Motivation : dérivation de champs de vecteurs, de covecteurs ou de tenseurs dans une direction donnée

(M, $\mc{O}$, $\mc{A}$)


\begin{minipage}[c]{.5\linewidth}
\[
\begin{array}{ l c l l }
 \gamma : & \mb{R} & \to & M \\ 
  & \lambda & \mapsto & \gamma(\lambda) \\
\end{array}
\]
\begin{center}
Champ de vecteur V

dérivée de V

dans la direction X
\end{center}
\end{minipage}
\hfill
\begin{minipage}[c]{.5\linewidth}
\[
\begin{array}{ l l c l l }
 \gamma : &  {v}_{\gamma,p} : & \mc{C}^\infty(M) & \to & \mb{R} \\ 
  &  & f & \mapsto & {v}_{\gamma,p}(f)=(f\circ\gamma)'|_{\lambda_0} \\
X(f)  &  & df(X) & = & X(f) \\
\end{array}
\]
X(f) étant la dérivée de f dans la direction X
\[
\begin{array}{ l c l l }
 f : & M & \to & \mb{R} \\ 
  & p & \mapsto & f(p) \\
\end{array}
\]
\end{minipage}
\vspace{.3cm}

Contraintes à respecter : linéarité, règle(s) de Leibniz

Liberté restante : coefficient d'une connexion

Dérivée covariante de champs de tenseurs de rang quelconque

Transformation des coefficients de la connexion lors d'un changemment de carte
