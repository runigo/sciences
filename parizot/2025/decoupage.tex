1a - Chapitre 1 : Introduction à la Relativité Générale
\begin{itemize}[leftmargin=1cm, label=\ding{32}, itemsep=1pt]
\item {Qu'est-ce que la relativité générale (RG) ?}
\item {Théorie de l'espace-temps, théorie de la gravitation...}
\item {Espace, temps, corps physiques, référentiels}
\item {Lien entre la relativité générale et relativité restreinte ?}
\item {Gravitation newtonienne restreinte}
\item {Analogie avec l'électromagnétisme : jusqu'ou ?}
\end{itemize}

1b - Chapitre 1 : Introduction à la Relativité Générale (suite)
\begin{itemize}[leftmargin=1cm, label=\ding{32}, itemsep=1pt]
\item {Notion d'espace affine => à dépasser !}
\item {Qu'est-ce qu'un vecteur ? Vecteur d'espace, "entre deux points" ; vecteur "en un point"...}
\item {Gravitation : une "force" pas comme les autres !}
\item {Principe d'équivalence}
\item {Retour sur la notion de référentiel}
\item {Espace-temps : entité géométrique}
\item {Reformulation de la loi d'inertie}
\end{itemize}

2a - Chapitre 1 : Introduction à la Relativité Générale (fin)
\begin{itemize}[leftmargin=1cm, label=\ding{32}, itemsep=1pt]
\item {Retour sur la notion de référentiel}
\item {"Référentiel tournant" en relativité restreinte}
\item {Leibniz, Newton, Mach, Einstein...}
\item {Conséquences du principe d'équivalence ; effets gravitationnels sur la lumière :}
\item {-"déflection de la lumière par les  masses", vérification expérimentale (Eddington, 1919)}
\item {-"redschift gravitationnel", vérification expérimentale (Pound et Rebka, 1959//1960)}
\end{itemize}

2b - Chapitre 2 : Variétés différentielles
\begin{itemize}[leftmargin=1cm, label=\ding{32}, itemsep=1pt]
\item {Évolution des systèmes physiques : continuité et différentiabilité}
\item {Cas des trajectoires des corps physiques}
\item {Liberté de choix d'un système de coordonnées}
\item {Nécessité d'une notion {\bf intrinsèque} de continuité et de différentiabilité}
\item {Espaces topologiques :}
\item {-notion de topologie (ensembles "ouverts"), de voisinage, de limite, de continuité}
\end{itemize}

3a Chapitre 2 : Variétés différentielles (suite)
\begin{itemize}[leftmargin=1cm, label=\ding{32}, itemsep=1pt]
\item {Rappel : topologie, voisinage, limite, continuité}
\item {Notion d'homéomorphisme}
\item {Courbe sur un espace topologique}
\item {Variété topologique de dimension n : cartes, atlas}
\item {Variété différentielle}
\item {- cartes C\_k compatibles, atlas C\_k}
\item {- variété lisse}
\item {- structure différentielle}
\end{itemize}

3b - Chapitre 2 : Variétés différentielles (fin)
\begin{itemize}[leftmargin=1cm, label=\ding{32}, itemsep=1pt]
\item {Fonctions différentiables, C\_k, lisses (C\_infini):}
\item {-de R dans M (courbes lisses)}
\item {-entre deux variétés différentielles}
\item {Difféomorphisme}
\item {Unicité ou non des structures différentielles (à difféomorphisme près)}
\end{itemize}

3c - Chapitre 3 : Espace vectoriel tangent
\begin{itemize}[leftmargin=1cm, label=\ding{32}, itemsep=1pt]
\item {Discussion introductive :}
\item {- comment transposer la notion de tangente à une courbe différentiable, dans un contexte où l'espace n'a pas de structure vectorielle associée}
\item {- notion de "direction", de "vitesse de déplacement"}
\item {Définition fondamentale de la vitesse en un point d'une courbe différentiable}
\end{itemize}

4a - Chapitre 3 : Espace vectoriel tangent (suite)
\begin{itemize}[leftmargin=1cm, label=\ding{32}, itemsep=1pt]
\item {Vitesse d'une courbe en un point}
\item {"Reparamétrage" d'une courbe }
\item {Espace vectoriel tangent, TpM : addition de deux vitesses et multiplication par un scalaire}
\item {Courbes coordonnées (d'une carte donnée) et vitesses associées}
\item {Base de TpM associée à une carte donnée}
\item {Notation importante : "dérivée partielle d'une fonction", représentée dans une carte}
\end{itemize}

4b - Chapitre 3 : Espace vectoriel tangent (suite)
\begin{itemize}[leftmargin=1cm, label=\ding{32}, itemsep=1pt]
\item {Base de l'espace vectoriel tangent au point p (TpM), induite par une carte (U, x)}
\item {Composantes de la vitesse d'une courbe}
\item {Lien avec la notion habituelle de vitesse}
\end{itemize}

4c - Annexe : Algèbre multilinéaire
\begin{itemize}[leftmargin=1cm, label=\ding{32}, itemsep=1pt]
\item {Rappels sur les espaces vectoriels}
\item {Base et dimension}
\item {Composantes contravariantes}
\item {Fonctions linéaires}
\item {Espace vectoriel dual, V*}
\item {Base duale d'une base donnée}
\end{itemize}

5a - Annexe : Algèbre multilinéaire
\begin{itemize}[leftmargin=1cm, label=\ding{32}, itemsep=1pt]
\item {Espace vectoriel dual, V*, et notion de base duale}
\item {Composantes d'un vecteur ou d'une forme linéaire, relativement à une base donnée}
\item {Changement de base}
\item {Co-variance et contra-variance des composantes}
\item {Produit scalaire, forme bilinéaire non dégénérée}
\item {Isomorphisme canonique ?}
\item {Tenseurs de rang (r,s)}
\item {Composantes des tenseurs relativement à une base}
\end{itemize}

5b - Annexe : Algèbre multilinéaire (fin)
\begin{itemize}[leftmargin=1cm, label=\ding{32}, itemsep=1pt]
\item {Produit tensoriel de deux tenseurs}
\item {Vecteurs vus comme tenseurs de rang (1,0) :}
\item {identification de V et V**}
\item {Applications linéaires vues comme tenseurs de rang (1,1)}
\item {Attention à la représentation matricielle !}
\end{itemize}

5c - Chapitre 3 (retour) : Espace vectoriel tangent
\begin{itemize}[leftmargin=1cm, label=\ding{32}, itemsep=1pt]
\item {Rappel : définition de TpM}
\item {Base naturelle de TpM associée à une carte (U,x)}
\item {Changement de carte : changement de base et changement de coordonnées associés}
\item {Consistence des notations et familiarité des relations entre dérivées partielles}
\end{itemize}

6a - Chapitre 3 (fin) : Espace vectoriel tangent
\begin{itemize}[leftmargin=1cm, label=\ding{32}, itemsep=1pt]
\item {Espace vectoriel co-tangent, T*pM}
\item {Gradient d'une fonction lisse sur la variété M}
\item {Base de l'espace co-tangent associée à une carte}
\item {Base duale des dx$^i$}
\item {Composantes covariantes des covecteurs}
\item {Changement de carte, changement de composantes}
\item {Tenseurs de rang (r,s) au point p (tenseurs sur TpM)}
\end{itemize}

6b - Chapitre 4 : Espace fibré tangent, champs de vecteurs, de covecteurs, de tenseurs
\begin{itemize}[leftmargin=1cm, label=\ding{32}, itemsep=1pt]
\item {Notion d'espace fibré : espace total, base et projection}
\item {Section d'un espace fibré tangent, TM}
\item {Cartes de TM induite par les cartes de M}
\item {Structure différentielle du fibré tangent}
\end{itemize}

6c - Chapitre 4 (suite et fin): Espace fibré tangent, champs de vecteurs, de covecteurs, de tenseurs
\begin{itemize}[leftmargin=1cm, label=\ding{32}, itemsep=1pt]
\item {Définition d'un champ de vecteurs : section lisse de TM}
\item {G(TM) : C-infini(M)-module des champs de vecteurs lisses}
\item {Existence de bases de champs de vecteurs non garantie}
\item {Fibré cotangent T*M, cartes induites par les cartes de M}
\item {Champs de covecteurs}
\item {Champs tensoriels de rang (r,s)}
\end{itemize}

7a - Chapitre 5 : Dérivée covariante, connexion
\begin{itemize}[leftmargin=1cm, label=\ding{32}, itemsep=1pt]
\item {Motivation : dérivation de champs de vecteurs, de covecteurs ou de tenseurs dans une direction donnée}
\item {Contraintes à respecter : linéarité, règle(s) de Leibniz}
\item {Liberté restante : coefficient d'une connexion}
\item {Dérivée covariante de champs de tenseurs de rang quelconque}
\item {Transformation des coefficients de la connexion lors d'un changemment de carte}
\end{itemize}

7b - Chapitre 5 : Dérivée covariante, connexion
\begin{itemize}[leftmargin=1cm, label=\ding{32}, itemsep=1pt]
\item {Dérivée covariante d'un champ de covecteurs, ou d'un champ de tenseurs de rang quelconque}
\item {Application de la "règle de Leibniz"}
\item {Notion de transport parallèle (à compléter)}
\item {Notion de "courbe rectiligne" (à compléter)}
\end{itemize}

8a - Chapitre 5 : Dérivée covariante, connexion (suite)
\begin{itemize}[leftmargin=1cm, label=\ding{32}, itemsep=1pt]
\item {Dérivée covariante}
\item {Exemples de connexions}
\item {Transport parallèle}
\item {Courbe autoparallèle et autoparallèlement transportée}
\item {Équation géodésique (dans une carte donnée)}
\item {Annexe : reparamétrage}
\end{itemize}

8b - Chapitre 5 : Dérivée covariante, connexion (fin)
\begin{itemize}[leftmargin=1cm, label=\ding{32}, itemsep=1pt]
\item {Contenu tensoriel de la dérivée covariante}
\item {Commutateur d'un champs de vecteurs, crochet de Lie}
\item {Tenseur de torsion, de rang (1,2)}
\item {Tenseur de courbure de Riemann, de rang (1,3)}
\item {Transport parallèle le long d'une courbe fermée}
\end{itemize}

9a - Rappels et bilan

- ce qu'on peut définir au sein d'une variété différentielle

- ce qu'apporte l'ajout d'une dérivée covariante

Chapitre 6 : Métrique et géodésique
\begin{itemize}[leftmargin=1cm, label=\ding{32}, itemsep=1pt]
\item {champ de métrique : forme bilinéaire, symétrique, non dégénérée}
\item {applications "Bra", "Ket" : descente et montée d'indices}
\item {Norme d'un vecteur de TpM}
\item {Longueur d'une courbe, reparamétrage}
\item {Géodésique (longueur stationnaire)}
\item {Connexion de Levi-Civita}
\end{itemize}

9b - Chapitre 6 : Métrique et géodésiques (suite)
\begin{itemize}[leftmargin=1cm, label=\ding{32}, itemsep=1pt]
\item {"Compatibilité" entre métrique et dérivée covariante}
\item {Connexion de Levi-Civita}
\item {Tenseur de courbure de Ricci}
\item {Courbure scalaire de Ricci}
\item {Tenseur d'Einstein (et équations d'Einstein)}
\end{itemize}

Remarque :

gravitation newtonienne et espace-temps courbe

10a - Chapitre 7 : Espace-temps physique et équations d'Einstein
\begin{itemize}[leftmargin=1cm, label=\ding{32}, itemsep=1pt]
\item {Structure du cadre spatiotemporel : variété différentielle de dimension 4 munie d'une métrique Lorentzienne}
\item {Contenu matériel : "points matériels", champs...}
\end{itemize}

10b - Chapitre 7 : Espace-temps physique et équations d'Einstein (suite)
\begin{itemize}[leftmargin=1cm, label=\ding{32}, itemsep=1pt]
\item {Structure du cadre spatiotemporel}
\item {Contenu matériel}
\item {Notion d'observateur}
\item {Notion d'horloge ; temps propre}
\item {Grandeurs physiques relatives à un observateur}
\item {Dynamique ; évolution / déploiement dans l'espace-temps}
\item {Principe de moindre action ; lagrangien}
\item {Intégration sur une variété, jacobien, "forme volume"}
\end{itemize}

11a - Chapitre 7 : Espace-temps physique et équations d'Einstein (suite)
\begin{itemize}[leftmargin=1cm, label=\ding{32}, itemsep=1pt]
\item {Dynamique {\bf DANS} l'espace-temps :}
\item {- particules ponctuelles, champ électromagnétique}
\item {- Lagrangien, version 4D}
\item {- principe de moindre action, équations "du mouvement", équations de champ...}
\item {Dynamique {\bf DE} l'espace-temps :}
\item {- équations d'Einstein, approche heuristique}
\item {- tenseur énergie-impulsion ("stress-energy")}
\end{itemize}

11b - Chapitre 7 : Espace-temps physique et équations d'Einstein (fin)
\begin{itemize}[leftmargin=1cm, label=\ding{32}, itemsep=1pt]
\item {Dynamique DE l'espace-temps :}
\item {- équations d'Einstein}
\item {- cadre unificateur : principe de moindre action, lagrangien de l'espace-temps, action de Hilbert-Einstein}
\item {- tenseur d'Einstein issu d'un calcul variationnel}
\item {- "constante cosmologique"}
\item {- théorème de Cartan}
\item {- redéfinition du tenseur énergie-impulsion}
\item {- particule ponctuelle, limite continue, fluide parfait}
\end{itemize}

12a - Chapitre 8 : Symétries, flots, dérivée de Lie, champ de Killing...
\begin{itemize}[leftmargin=1cm, label=\ding{32}, itemsep=1pt]
\item {Pull back et push forward}
\item {- fonction, vecteur, covecteur, métrique, etc.}
\item {Cas particulier : métrique induite}
\item {Flot d'un champ de vecteurs}
\item {Groupe à un paramètre de difféomorphismes sur M}
\item {Isométrie}
\item {Champ de Killing}
\end{itemize}

12b - Chapitre 8 : Symétries, flots, dérivée de Lie, champ de Killing (fin)
\begin{itemize}[leftmargin=1cm, label=\ding{32}, itemsep=1pt]
\item {dérivée de Lie "le long d'un champ de vecteur"}
\item {Cas d'une fonction, d'un vecteur, d'un covecteur, d'un tenseur général}
\item {Application à une métrique :isométrie}
\item {Équation de Killing}
\item {Quantité conservée}
\end{itemize}

13a - Chapitre 9 : Exemples et application
\begin{itemize}[leftmargin=1cm, label=\ding{32}, itemsep=1pt]
\item {Rappel sur les symétries}
\item {Espace-temps de symétrie maximale : Minkowski}
\item {Vecteur de Killing de l'espace-temps de Minkowski}
\item {Limite Newtonienne}
\item {Métriques à symétrie sphérique}
\item {Dans le vide : métrique de Schwarzschild}
\item {Évocation du théorème de Birkhoff}
\end{itemize}

13b - Chapitre 9 : Exemples et application (suite)
\begin{itemize}[leftmargin=1cm, label=\ding{32}, itemsep=1pt]
\item {Métrique de Schwarzschild}
\item {- platitude asymptotique}
\item {- signification du rayon de Schwarzschild : "masse" de la distribution de matière}
\item {Géodésiques, traversée de l'horizon}
\item {Observateurs stationnaires}
\item {Désynchronisation des horloges}
\item {Correction de Relativité générale du système GPS}
\end{itemize}

13c - Chapitre 9 : Exemples et application (suite)
\begin{itemize}[leftmargin=1cm, label=\ding{32}, itemsep=1pt]
\item {Métrique de Schwarzschild :}
\item {- champs de Killing}
\item {- redshift gravitationnel}
\end{itemize}

14a - Correction de Examen - I
\begin{itemize}[leftmargin=1cm, label=\ding{32}, itemsep=1pt]
\item {}
\end{itemize}

14a - Correction de Examen - II
\begin{itemize}[leftmargin=1cm, label=\ding{32}, itemsep=1pt]
\item {}
\end{itemize}

15 - Complément : mécanique céleste relativiste
\begin{itemize}[leftmargin=1cm, label=\ding{32}, itemsep=1pt]
\item {}
\end{itemize}

16 - Complément : mécanique céleste relativiste
\begin{itemize}[leftmargin=1cm, label=\ding{32}, itemsep=1pt]
\item {}
\end{itemize}

17 - Complément : déflexion de la lumière
\begin{itemize}[leftmargin=1cm, label=\ding{32}, itemsep=1pt]
\item {}
\end{itemize}
