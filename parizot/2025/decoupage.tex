2025 - 1a - Chapitre 1 : Introduction à la Relativité Générale
\begin{itemize}[leftmargin=1cm, label=\ding{32}, itemsep=1pt]
\item {Qu'est-ce que la relativité générale (RG) ?}
\item {Théorie de l'espace-temps, théorie de la gravitation...}
\item {Espace, temps, corps physiques, référentiels}
\item {Lien entre la relativité générale et relativité restreinte ?}
\item {Gravitation newtonienne restreinte}
\item {Analogie avec l'électromagnétisme : jusqu'ou ?}
\end{itemize}

2025 - 1b - Chapitre 1 : Introduction à la Relativité Générale (suite)
\begin{itemize}[leftmargin=1cm, label=\ding{32}, itemsep=1pt]
\item {Notion d'espace affine => à dépasser !}
\item {Qu'est-ce qu'un vecteur ? Vecteur d'espace, "entre deux points" ; vecteur "en un point"...}
\item {Gravitation : une "force" pas comme les autres !}
\item {Principe d'équivalence}
\item {Retour sur la notion de référentiel}
\item {Espace-temps : entité géométrique}
\item {Reformulation de la loi d'inertie}
\end{itemize}

2025 - 2a - Chapitre 1 : Introduction à la Relativité Générale (fin)
\begin{itemize}[leftmargin=1cm, label=\ding{32}, itemsep=1pt]
\item {Retour sur la notion de référentiel}
\item {"Référentiel tournant" en relativité restreinte}
\item {Leibniz, Newton, Mach, Einstein...}
\item {Conséquences du principe d'équivalence ; effets gravitationnels sur la lumière :}
\item {-"déflection de la lumière par les  masses", vérification expérimentale (Eddington, 1919)}
\item {-"redschift gravitationnel", vérification expérimentale (Pound et Rebka, 1959//1960)}
\end{itemize}

2025 - 2b - Chapitre 2 : Variétés différentielles
\begin{itemize}[leftmargin=1cm, label=\ding{32}, itemsep=1pt]
\item {Évolution des systèmes physiques : continuité et différentiabilité}
\item {Cas des trajectoires des corps physiques}
\item {Liberté de choix d'un système de coordonnées}
\item {Nécessité d'une notion {\bf intrinsèque} de continuité et de différentiabilité}
\item {Espaces topologiques :}
\item {-notion de topologie (ensembles "ouverts"), de voisinage, de limite, de continuité}
\end{itemize}

2025 - 3a Chapitre 2 : Variétés différentielles (suite)
\begin{itemize}[leftmargin=1cm, label=\ding{32}, itemsep=1pt]
\item {Rappel : topologie, voisinage, limite, continuité}
\item {Notion d'homéomorphisme}
\item {Courbe sur un espace topologique}
\item {Variété topologique de dimension n : cartes, atlas}
\item {Variété différentielle}
\item {- cartes C\_k compatibles, atlas C\_k}
\item {- variété lisse}
\item {- structure différentielle}
\end{itemize}

2025 - 3b - Chapitre 2 : Variétés différentielles (fin)
\begin{itemize}[leftmargin=1cm, label=\ding{32}, itemsep=1pt]
\item {Fonctions différentiables, C\_k, lisses (C\_infini):}
\item {-de R dans M (courbes lisses)}
\item {-entre deux variétés différentielles}
\item {Difféomorphisme}
\item {Unicité ou non des structures différentielles (à difféomorphisme près)}
\end{itemize}

2025 - 3c - Chapitre 3 : Espace vectoriel tangent
\begin{itemize}[leftmargin=1cm, label=\ding{32}, itemsep=1pt]
\item {Discussion introductive :}
\item {- comment transposer la notion de tangente à une courbe différentiable, dans un contexte où l'espace n'a pas de structure vectorielle associée}
\item {- notion de "direction", de "vitesse de déplacement"}
\item {Définition fondamentale de la vitesse en un point d'une courbe différentiable}
\end{itemize}

2025 - 4a - Chapitre 3 : Espace vectoriel tangent (suite)
\begin{itemize}[leftmargin=1cm, label=\ding{32}, itemsep=1pt]
\item {Vitesse d'une courbe en un point}
\item {"Reparamétrage" d'une courbe }
\item {Espace vectoriel tangent, TpM : addition de deux vitesses et multiplication par un scalaire}
\item {Courbes coordonnées (d'une carte donnée) et vitesses associées}
\item {Base de TpM associée à une carte donnée}
\item {Notation importante : "dérivée partielle d'une fonction", représentée dans une carte}
\end{itemize}

2025 - 4b - Chapitre 3 : Espace vectoriel tangent (suite)
\begin{itemize}[leftmargin=1cm, label=\ding{32}, itemsep=1pt]
\item {Base de l'espace vectoriel tangent au point p (TpM), induite par une carte (U, x)}
\item {Composantes de la vitesse d'une courbe}
\item {Lien avec la notion habituelle de vitesse}
\end{itemize}

2025 - 4c - Annexe : Algèbre multilinéaire
\begin{itemize}[leftmargin=1cm, label=\ding{32}, itemsep=1pt]
\item {Rappels sur les espaces vectoriels}
\item {Base et dimension}
\item {Composantes contravariantes}
\item {Fonctions linéaires}
\item {Espace vectoriel dual, V*}
\item {Base duale d'une base donnée}
\end{itemize}

2025 - 5a - Annexe : Algèbre multilinéaire
\begin{itemize}[leftmargin=1cm, label=\ding{32}, itemsep=1pt]
\item {Espace vectoriel dual, V*, et notion de base duale}
\item {Composantes d'un vecteur ou d'une forme linéaire, relativement à une base donnée}
\item {Changement de base}
\item {Co-variance et contra-variance des composantes}
\item {Produit scalaire, forme bilinéaire non dégénérée}
\item {Isomorphisme canonique ?}
\item {Tenseurs de rang (r,s)}
\item {Composantes des tenseurs relativement à une base}
\end{itemize}

2025 - 5b - Annexe : Algèbre multilinéaire (fin)
\begin{itemize}[leftmargin=1cm, label=\ding{32}, itemsep=1pt]
\item {Produit tensoriel de deux tenseurs}
\item {Vecteurs vus comme tenseurs de rang (1,0) :}
\item {identification de V et V**}
\item {Applications linéaires vues comme tenseurs de rang (1,1)}
\item {Attention à la représentation matricielle !}
\end{itemize}

2025 - 5c - Chapitre 3 (retour) : Espace vectoriel tangent
\begin{itemize}[leftmargin=1cm, label=\ding{32}, itemsep=1pt]
\item {Rappel : définition de TpM}
\item {Base naturelle de TpM associée à une carte (U,x)}
\item {Changement de carte : changement de base et changement de coordonnées associés}
\item {Consistence des notations et familiarité des relations entre dérivées partielles}
\end{itemize}

2025 - 6a - Chapitre 3 (fin) : Espace vectoriel tangent
\begin{itemize}[leftmargin=1cm, label=\ding{32}, itemsep=1pt]
\item {Espace vectoriel co-tangent, T*pM}
\item {Gradient d'une fonction lisse sur la variété M}
\item {Base de l'espace co-tangent associée à une carte}
\item {Base duale des dx$^i$}
\item {Composantes covariantes des covecteurs}
\item {Changement de carte, changement de composantes}
\item {Tenseurs de rang (r,s) au point p (tenseurs sur TpM)}
\end{itemize}

\begin{itemize}[leftmargin=1cm, label=\ding{32}, itemsep=1pt]
\item {}
\item {}
\item {}
\item {}
\item {}
\item {}
\item {}
\item {}
\item {}
\item {}
\item {}
\end{itemize}

\begin{itemize}[leftmargin=1cm, label=\ding{32}, itemsep=1pt]
\item {}
\item {}
\item {}
\item {}
\item {}
\item {}
\item {}
\item {}
\item {}
\end{itemize}

\begin{itemize}[leftmargin=1cm, label=\ding{32}, itemsep=1pt]
\item {}
\item {}
\item {}
\item {}
\item {}
\item {}
\item {}
\item {}
\item {}
\end{itemize}

\begin{itemize}[leftmargin=1cm, label=\ding{32}, itemsep=1pt]
\item {}
\item {}
\item {}
\item {}
\item {}
\item {}
\item {}
\item {}
\item {}
\end{itemize}

