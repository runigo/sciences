%2025 - 5a
\section{05a : Annexe : Algèbre multilinéaire}
Nous avons introduit $T_pM$, l'espace vectoriel tangent au point p à la variété
lisse $(M, \mc{O}, \mc{A}_\infty)$
\[
T_pM = \{ \mr{V}_{\gamma,p}\ \ lisse\}
\]
Les vitesses $\mr{V}_{\gamma,p}$ étant définie comme des appplications :
\[
\begin{array}{ l c l l }
 \mr{V}_{\gamma,p} : & \mc{C}^\infty(M) & \to & \mb{R} \\ 
  & f & \mapsto & {v}_{\gamma,p}(f)=(f\circ\gamma)'|_\lambda \\
\end{array}
\qquad \gamma(\lambda)=p
\]
Et si on munie cet ensemble d'une addition et d'une multiplication par un
scalaire, on obtient bien un espace vectorielle.
\[ 
\begin{array}{ l c l }
+ &  &  \\
\cdot & \to & \mb{R} \\
\end{array}
\]

\subsection {Espace vectoriel dual, V*, et notion de base duale}
\subsubsection{Rappels}
Soit l'espace vectoriel (V, $+$,$\cdot$) de dimension fini n.
$\{\ \mv{e}_{\ (i)}\}_{i=1,...,n}$ une base de V.
\[
(V, +,\cdot) \qquad dim V < \infty
\]
on a alors,
\[
\{\,\mv{e}_{\ (i)}\}_{i=1...dim V}\qquad \forall\ \mv{v}\ \in V,
\exists!\ (v^1,...,v^n)\ tq\ \ \mv{v}\ =\sum_{i=1}^n \ v^{\,i\,}\ \mv{e}_{\ (i)}
\]
Naturellement, on peut changer de base
\[
\mv{e}_{\ (i)}\quad \to\qquad \mv{e'}_{\ (i)}=\sum_{j=1}^n \ P_i^{\,j}
\ \ \mv{e}_{\ (j)}
\]
$P_i^j$ est la matrice de passage. Chaque vecteur de la nouvelle
base est une combinaison linéaire des vecteur de l'ancienne base.
On a alors la nouvelle écriture de $\ \mv{v}\ $:
\[
\mv{v}\ =\sum_{i=1}^n \ v'^{\,i\,}\ \mv{e'}_{(i)} \qquad avec \qquad
\boxed{\ v^{\,i} = \sum_{j=1}^n P_j^{\,i}\ v'^j\ }
\]
$P_i^j$ permet de passer de v' à v
alors que $P_i^j$ permet de passer de $\ \mv{e}_{\ (i)}\ $ à $\ \mv{e'}_{\ (i)}\ $
($\ \mv{e}_{\ (i)}\ $ est covariant, $v^{\,i\,}$ est contravariant : ) ; un indice en bas correspond à une grandeur covariante, un indice en haut correspond à une grandeur contravariante.

Remarque
\[
[P^{\,i}_j]^{-1}\ \xcancel{=}\ [P^{\,j}_i] \qquad et
\qquad V = P V'\ \to\ V'=P^{-1}V
\]

Application linéaire

\[
(V, +,\cdot)\ \tilde{\to}\ (W, +,\cdot)
\]

\[
(Hom(V, W), +,\cdot)\ =\ espace\ vectoriel \ dim V \times dim W
\]
Dual de V
\[
dual : V^* = Hom(V,\mb{R})\ =\ ,\ \boxed{dim\ V^*\,=\ dim\ V}
\]
base duale : le choix d'une base de V définie de façon univoque une base duale. 
\[
\{\ \mv{e}_{\ (i)}\}_{i=1,...,n}\ \ base\ de\ V\ \ \to\ \ \exists!\ base\ duale\ \ 
\{\ \vd{\varepsilon}^{\ (i)}\}_{i=1,...,n} \qquad tq\qquad
\vd{\varepsilon}^{\ (i)}(\ \mv{e}_{\ (i)} ) = \delta^i_j
\]
On insiste sur le passage de V à $V*$, la distinction entre ces deux espaces: lettre latine $\to$ lettre
grec, flèche en haut $\to$ flèche en bas, flèche vers la droite $\to$ flèche
vers la gauche. Ceci est à rapprocher au passage du ket vers le bra en mécanique
quantique : le ket appartient à l'espace de Hilbert, le bra est une forme
linéaire sur l'espace de Hilbert.

 FIN DU RAPPEL
\subsection{Composantes d'un vecteur ou d'une forme linéaire, relativement
à une base donnée}

\[
\mv{v}\ \in\ V\ \ \to\ \ \mv{v}\ =\sum_i \ v^{\,i\,}\ \mv{e}_{\ (i)}
\qquad \llap{``} \equiv v^{\,i\,}\ \mv{e}_{\ (i)}\ \rlap{''}
\]
\[
\vd{\varepsilon}^{\ (k)}(\ \mv{v}\ ) =
\vd{\varepsilon}^{\ (k)}\left(\sum_i \ v^{\,i\,}\ \mv{e}_{\ (i)}\right) = 
\sum_iv^{\,i\,} \ \vd{\varepsilon}^{\ (k)}\ \ \mv{e}_{\ (i)} = v^{\,k\,}
\]
L'application du k-ième vecteur de la base duale permet "d'extraire la k-ième composante d'un vecteur dans une certaine base.

Dans notre cours, nous n'allons avoir à faire qu'à des applications linéaire. Ainsi dans l'expression on peut permuter le signe somme avec l'application $\varepsilon$. Cela justifie la convention de sommation sur les indices répétés :
\[
\vd{\varepsilon}^{\ (k)}(\ \mv{v}\ ) =
\vd{\varepsilon}^{\ (k)}\,v^{\,i\,}\ \mv{e}_{\ (i)} =  v^{\,k\,}
\]
\[
\forall\ \ \vd{\omega}\quad \exists! (\omega_1,...,\omega_2)\in\mb{R},
\ \ \vd{\omega}\ = \sum_{i=1}^n\omega_i\ \ \vd{\varepsilon}^{\ (i)}
\]
\[
\vd{\omega}\ \ (\ \mv{e}_{\ (k)} )=
\sum_{i=1}^n\omega_i\ \ \vd{\varepsilon}^{\ (i)}(\ \mv{e}_{\ (k)} ) 
\qquad \boxed{\ \ \vd{\omega}\ \ (\ \mv{e}_{\ (k)} )=\omega_k}
\]

\[
\vd{\omega}\ = \sum_i\omega_{\,i}\ \ \vd{\varepsilon}^{\ (\,i)}
\hspace{3cm} \mv{v}\ = \sum_jv^{\,j}\ \ \mv{e}_{\ (\,j)}
\]
\[
\vd{\omega}\ (\ \mv{v}\ ) = \sum_i\omega_{\,i}\ \ \vd{\varepsilon}^{\ (\,i)}
\left( \sum_jv^{\,j}\ \ \mv{e}_{\ (\,j)}\right)
\]
En utilisant la convention de sommation des indices répétés
\[
= \omega_{\,i}\ \ \vd{\varepsilon}^{\ (\,i)}
\left( v^{\,j}\ \ \mv{e}_{\ (\,j)}\right)
= \omega_{\,i}\ v^{\,j}\ \ \vd{\varepsilon}^{\ (\,i)}
\left( \ \mv{e}_{\ (\,j)}\right)
= \omega_{\,i}\ v^{\,j}\ \ \delta_{i,j}
\]
soit
\[
\boxed{\ \ \vd{\omega}\ (\ \mv{v}\ ) = \sum_i\omega_{\,i}\ v^{\,i}\ }
\]
\subsection{Changement de base}
\[
\{\ \mv{e}_{\ (i)}\}\ \to \ \{\ \mv{e'}_{\ (\,i)}\}
\qquad \mv{e'}_{\ (i)}=\sum_{j=1}^n \ P_i^{\,j}\ \ \mv{e}_{\ (j)}
\]
\[
\vd{\varepsilon}^{\ (i)}(v'^{\,k}\ \mv{e'}_{\ (k)})
=v'^{\,k}\ \vd{\varepsilon}^{\ (i)}(\ P_k^{\,j}\ \ \mv{e}_{\ (j)})
=v'^{\,k}\ P_k^{\,j}\ \delta^i_j = v'^{\,k}\ P_k^{\,i}
\]
\[
\forall\ \mv{v}\ \quad\vd{\varepsilon}^{\ (i)}(\ \mv{v}\ )
=P_k^{\,i}\ \ \vd{\varepsilon '\,}^{\ (k)}(\ \mv{v}\ )
\]
\[
\Rightarrow \qquad \boxed{\ \ \vd{\varepsilon}^{\ (i)}=
P_k^{\,i}\ \ \vd{\varepsilon '\,}^{\ (k)}(\ \mv{v}\ )\ }
\]
\[
\boxed{\ \ \vd{\varepsilon}^{\ (i)}\ \ \mv{e}_{\ (j)}=
\delta^i_j\ }
\]
\subsection{Co-variance et contra-variance des composantes}
\[
\boxed{\ \ \mv{e'}_{\ (i)}=\sum_j P^j_i\ \ \mv{e}_{\ (j)}}
\]
\[
\vd{\omega}\ =\sum_k\omega'_k\ \ \vd{\varepsilon '\,}^{\ (k)}
\]
\[
\vd{\omega}\ \ (\ \,\mv{e'}_{\ k})=\ \vd{\omega}\ \ (P_k^{\,i}\ \,\mv{e}_{\ i})=P_k^{\,i}\,\ \ \vd{\omega}\ \ (\ \,\mv{e}_{\ i})
\]
\[
\boxed{\ \omega'_{\ k}=\sum_i\ P_k^{\,i}\ \omega_{\ i}}\quad\to
\quad \omega_{\ i}\ sont\ dits\ covariants
\]

\subsection{Produit scalaire, forme bilinéaire non dégénérée}
\[
\forall\ \mv{v}\ \in V\qquad\forall\ \vd{\varepsilon} \ \in\ V^*
\qquad\vd{\varepsilon}\ \ (\ \mv{v}\ )\ \in \mb{R}
\]
\[
\vd{\varepsilon}\ \ (\ \mv{v}\ )\ =\sum_i\omega_iv^i\ =\sum_i\omega'_iv'\,^i
\]
\subsubsection{Isomorphisme canonique ?}
dim V = n, dim V*=n, Pas d'isomorphisme canonique
\[
\mv{v}\ \ \in\ V\quad\longrightarrow\qquad\vd{\varepsilon}\ \ \in\ V^*
\]
\[
\{\ \mv{e}_{\ (i)}\}\qquad\qquad\{\ \ \vd{\rho}^{\ (j)}\}
\]
\[
\mv{v}\ \ =\sum v^i\ \mv{e}_{\ (i)}\ \to\ \ \vd{v}=\sum v^i\ \ \vd{\rho}^{\ (j)}
\]
MAIS S'il existe un "produit scalaire", forme bilinéaire non dégénérée sur V
Alors il existe un isomorphosme canonique
\[
\begin{array}{ c c c }
V\times V & \longrightarrow & \mb{R} \\
(\ \ \mv{u}\ ,\ \mv{v}\ \ ) & \longmapsto & s\,(\,\ \mv{u}\ ,\ \mv{v}\,\ ) \\
\end{array}
\]
\[
s\,(\,\ \mv{u}\ ,\ \mv{v}\,\ )
= s\,(u^i\,\ \mv{e}_{\ i}\ ,v^{\,j}\,\ \mv{e}_{\ j}\,\ )
= u^i\,v^{\,j}\,s\,(\,\ \mv{e}_{\ i}\ ,\,\ \mv{e}_{\ j}\ )
\]
\[
\boxed{\ s\,(\,\ \mv{u}\ ,\ \mv{v}\,\ )=s_{\,i\,j}u^{\,i}\,v^{\,j}\ }
\]
\[
Si\quad\forall\ \mv{v}\ ,\quad s\,(\,\ \mv{u}\ ,\ \mv{v}\,\ )=0,\ alors
\quad \mv{u}\,\ =\,\ \mv{0}
\]
À tout $\ \mv{v}\ $, on peut associer la forme linéaire $\ \vd{\omega}\ \in V^*$
définie par $\ \vd{\omega}\ \equiv s\,(\ \mv{v}\ ,\cdot)$
\[
|\,\psi>\ \to(<\psi\,|\,)\ (\,|\,\phi>)=<\psi\,|\,\phi>=h(\,|\,\psi>,|\,\phi>)
\]

Soit $\ \mv{v}\ \ =\sum v^i\ \mv{e}_{\ (i)}\ \in V\quad$ Quelles sont les composantes de $\boxed{s(\ \mv{v}\ ,\cdot)}$ dans la base duale $\{\ \vd{\varepsilon}^{\ (i)}\}$ de $V^*$

Si $\ \vd{\omega}\ \ =\sum \omega\,_i\ \vd{\varepsilon}^{\ (i)}
\qquad\ \vd{\omega}\ (\ \mv{u}\ )=\sum \omega_iu^i$

\[
s(\ \mv{v}\ ,\cdot)(\ \mv{u}\ )\ =s\,(\,\ \mv{v}\ ,\ \mv{u}\,\ )
=\sum_i\omega_{\,i}u^i
\]

$\vd{\varepsilon}\ \ \in\ V^*$

$\to \ \mv{v}\ $ associé canoniquement à $ \ \mv{\omega}\ $

est l'unique vecteur tel que $\ \mv{\omega}\ =s(\ \mv{v}\ ,\cdot))$

$\ \mv{\omega}\ (\ \mv{e}\ _{i})=\omega_i$

$u^i=\sum_k t^{ik}\omega_k$

$\omega_i=\sum_i\sum_ks_{ij}t^{ik}\omega_k$

\subsection{Tenseurs de rang (r,s)}
$(V,+,\cdot)\to(V,+,\cdot)$

forme multilinéaire
\[
T_{rang(rs)} : \underbrace{V^*\times...\times V^*}_{\text{r}}
\times\underbrace{V\times...\times V}_{\text{s}}
\quad\xrightarrow{\sim}\quad\mb{R}
\]

Exemple : tenseur de rang (2,1)

\[
\begin{array}{ c c c }
 V^*\times V^*\times V & \xrightarrow{\sim} & \mb{R} \\
  (\omega, \rho, u) & \longmapsto & T(\omega, \rho, u) \\
\end{array}
\]
\[
T(\omega_1+\lambda\omega_2, \rho, u)=T(\omega_1, \rho, u)+\lambda T(\omega_2, \rho, u)
\]
\[
T(\omega, \rho, u_1+\lambda u_2)=T(\omega, \rho, u_1)+\lambda T(\omega, \rho, u_2)
\]
\subsection{Composantes des tenseurs relativement à une base}
\[
\{\ \mv{e}_{\ (i)}\}\quad et\quad\{\ \ \vd{\varepsilon}^{\ (j)}\}\qquad base\ duale\ de\ V\ et\ V^*
\]
Si on connait
\[
T\big(\,\underbrace{\ \vd{\varepsilon}^{\ (i)},...,\ \vd{\varepsilon}^{\ (j)}}_{\text{r}}\ ,
\ \underbrace{\ \mv{e}_{\ (k)},...,\ \mv{e}_{\ (l)}}_{\text{s}}\big)
\]
alors on connait
\[
T\big(\ \,\vd{\omega}^{\ (i)},...,\ \ \vd{\omega}^{\ (j)},\ \mv{u}_{\ (k)},...,\ \mv{u}_{\ (l)}\big)
\]
En effet, par multilinéarité
\[
T\big(\,\omega_{\,i_1}\,\ \vd{\varepsilon}^{\ (i_1)},...,\ \omega_{\,i_r}\ \vd{\varepsilon}^{\ (i_r)}
\ ,v^{j_1}\ \mv{e}_{\ (j_1)},...,v^{j_s}\ \mv{e}_{\ (j_s)}\,\big)
\]
\[
=\omega_{\,i_1}...\omega_{\,i_r}v^{\,j_1}...v^{\,j_s}\ T\ \big(\ \,\vd{\varepsilon}^{\ (i_1)},...,\ \vd{\varepsilon}^{\ (i_r)}
\ ,\ \mv{e}_{\ (\,j_1)},...,\ \mv{e}_{\ (\,j_s)}\,\big)
\]
Exemple : forme bilinéaire $s:V\times V\ \xrightarrow{\sim}\ \mb{R}$ : tenseur de rang(0,2)
\[
s\,(\ \mv{u}\ ,\ \mv{v}\ )=u^iv^{\,j}\,s\,(\ \mv{e}_{\,(i)}\ ,\ \mv{e}_{\,(j)})=s_{ij}\,u^iv^{\,j}
\]
\[
s\,(\ \mv{u}\ ,\ \mv{v}\ )=u'^{\,i}v'^{\,j}\,s\,(\ \mv{e'}_{\,(i)}\ ,\ \mv{e'}_{\,(j)})
\]
\[
s'_{ij}\,=\,s\,(\ \mv{e'}_{\,(i)}\ ,\ \mv{e'}_{\,(j)})
\]
\[
=\,s\,(\sum_kP^k_i\,\ \mv{e}_{\,(k)}\ ,\sum_lP^l_j\,\ \mv{e}_{\,(l)})
\]
\[
=\sum_k\sum_lP^k_i\,P^l_j\,s\,(\ \mv{e}_{\,(k)}\ ,\ \mv{e}_{\,(l)})
\]
\[
s'_{ij}\,=\sum_k\sum_lP^k_i\,P^l_j\,s_{kl}
\]
%\[\]   \ \ \boxed{}
