%2025 - 5a
\section{05a : Annexe : Algèbre multilinéaire}
Nous avons introduit $T_pM$, l'espace vectoriel tangent au point p à la variété
lisse $(M, \mc{O}, \mc{A}_\infty)$
\[
T_pM = \{ \mr{V}_{\gamma,p}\ \ lisse\}
\]
Les vitesses $\mr{V}_{\gamma,p}$ étant définie comme des appplications :
\[
\begin{array}{ l c l l }
 \mr{V}_{\gamma,p} : & \mc{C}^\infty(M) & \to & \mb{R} \\ 
  & f & \mapsto & {v}_{\gamma,p}(f)=(f\circ\gamma)'|_\lambda \\
\end{array}
\qquad \gamma(\lambda)=p
\]
Et si on munie cet ensemble d'une addition et d'une multiplication par un
scalaire, on obtient bien un espace vectorielle.
\[ 
\begin{array}{ l c l }
+ &  &  \\
\cdot & \to & \mb{R} \\
\end{array}
\]

\subsection {Espace vectoriel dual, V*, et notion de base duale}
Soit l'espace vectoriel (V, $+$,$\cdot$) de dimension fini n.
$\{\ \mv{e}_{\ (i)}\}_{i=1,...,n}$ une base de V.
\[
(V, +,\cdot) \qquad dim V < \infty
\]
on a alors,
\[
\{\,\mv{e}_{\ (i)}\}_{i=1...dim V}\qquad \forall\ \mv{v}\ \in V,
\exists!\ (v^1,...,v^n)\ tq\ \ \mv{v}\ =\sum_{i=1}^n \ v^{\,i\,}\ \mv{e}_{\ (i)}
\]
Naturellement, on peut changer de base
\[
\mv{e}_{\ (i)}\quad \to\qquad \mv{e'}_{\ (i)}=\sum_{j=1}^n \ P_i^{\,j}
\ \ \mv{e}_{\ (j)}
\]
$P_i^j$ est la matrice de passage. Chaque vecteur de la nouvelle
base est une combinaison linéaire des vecteur de l'ancienne base.
On a alors la nouvelle écriture de $\ \mv{v}\ $:
\[
\mv{v}\ =\sum_{i=1}^n \ v'^{\,i\,}\ \mv{e'}_{(i)} \qquad avec \qquad
\boxed{\ v^{\,i} = \sum_{j=1}^n P_j^{\,i}\ v'^j\ }
\]
$P_i^j$ permet de passer de v' à v
alors que $P_i^j$ permet de passer de $\ \mv{e}_{\ (i)}\ $ à $\ \mv{e'}_{\ (i)}\ $
($\ \mv{e}_{\ (i)}\ $ est covariant, $v^{\,i\,}$ est contravariant : ) ; un indice en bas correspond à une grandeur covariante, un indice en haut correspond à une grandeur contravariante.
\[
[P^{\,i}_j]^{-1}\ \xcancel{=}\ [P^{\,j}_i] \qquad et
\qquad V = P V'\ \to\ V'=P^{-1}V
\]

Application linéaire

\[
(V, +,\cdot)\ \tilde{\to}\ (W, +,\cdot)
\]

\[
(Hom(V, W), +,\cdot)\ =\ espace\ vectoriel \ dim V \times dim W
\]

\[
dual : V^* = Hom(V,\mb{R})\ =\ ,\ \boxed{dim\ V^*\,=\ dim\ V}
\]
base duale
\[
\{\ \mv{e}_{\ (i)}\}_{i=1,...,n}\ \ base\ de\ V\ \ \to\ \ \exists!\ base\ duale\ \ 
\{\ \vd{\varepsilon}^{\ (i)}\}_{i=1,...,n} \qquad tq\qquad
\vd{\varepsilon}^{\ (i)}(\ \mv{e}_{\ (i)} ) = \delta^i_j
\]
\[
\mv{v}\ \in\ V\ \ \to\ \ \mv{v}\ =\sum_i \ v^{\,i\,}\ \mv{e}_{\ (i)}
\qquad \llap{``} \equiv v^{\,i\,}\ \mv{e}_{\ (i)}\ \rlap{''}
\]
\[
\vd{\varepsilon}^{\ (k)}(\ \mv{v}\ ) =
\vd{\varepsilon}^{\ (k)}\left(\sum_i \ v^{\,i\,}\ \mv{e}_{\ (i)}\right) = 
\sum_iv^{\,i\,} \ \vd{\varepsilon}^{\ (k)}\ \ \mv{e}_{\ (i)} = v^{\,k\,}
\]
\subsection{Composantes d'un vecteur ou d'une forme linéaire, relativement
à une base donnée}
\[
\forall\ \ \vd{\omega}\quad \exists! (\omega_1,...,\omega_2)\in\mb{R},
\ \ \vd{\omega}\ = \sum_{i=1}^n\omega_i\ \ \vd{\varepsilon}^{\ (i)}
\]
\[
\vd{\omega}\ \ (\ \mv{e}_{\ (k)} )=
\sum_{i=1}^n\omega_i\ \ \vd{\varepsilon}^{\ (i)}(\ \mv{e}_{\ (k)} ) 
\qquad \boxed{\ \ \vd{\omega}\ \ (\ \mv{e}_{\ (k)} )=\omega_k}
\]

\[
\vd{\omega}\ = \sum_i\omega_{\,i}\ \ \vd{\varepsilon}^{\ (\,i)}
\hspace{3cm} \mv{v}\ = \sum_jv^{\,j}\ \ \mv{e}_{\ (\,j)}
\]
\[
\vd{\omega}\ (\ \mv{v}\ ) = \sum_i\omega_{\,i}\ \ \vd{\varepsilon}^{\ (\,i)}
\left( \sum_jv^{\,j}\ \ \mv{e}_{\ (\,j)}\right)
\]
En utilisant la convention de sommation des indices répétés
\[
= \omega_{\,i}\ \ \vd{\varepsilon}^{\ (\,i)}
\left( v^{\,j}\ \ \mv{e}_{\ (\,j)}\right)
= \omega_{\,i}\ v^{\,j}\ \ \vd{\varepsilon}^{\ (\,i)}
\left( \ \mv{e}_{\ (\,j)}\right)
= \omega_{\,i}\ v^{\,j}\ \ \delta_{i,j}
\]
soit
\[
\boxed{\ \ \vd{\omega}\ (\ \mv{v}\ ) = \sum_i\omega_{\,i}\ v^{\,i}\ }
\]
%\[\]   \ \ \boxed{}
\subsection{Changement de base}
\[
\{\ \mv{e}_{\ (i)}\}\ \to \ \{\ \mv{e'}_{\ (\,i)}\}
\qquad \mv{e'}_{\ (i)}=\sum_{j=1}^n \ P_i^{\,j}\ \ \mv{e}_{\ (j)}
\]
\[
\vd{\varepsilon}^{\ (\,i)}(v'^k\ \mv{e'}_{\ (\,k)})
=v'^k\ \vd{\varepsilon}^{\ (\,i)}(\ P_k^{\,j}\ \ \mv{e}_{\ (j)})
=v'^k\ P_k^{\,j}\ \delta^i_j = v'^k\ P_k^{\,i}
\]
Co-variance et contra-variance des composantes
Produit scalaire, forme bilinéaire non dégénérée
Isomorphisme canonique ?
Tenseurs de rang (r,s)
Composantes des tenseurs relativement à une base
