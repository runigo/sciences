%2025 - 4c
\section{04c Annexe : Algèbre multilinéaire}

Rappels sur les espaces vectoriels

(V, $+$,$\cdot$) est un espace vectoriel. Par définition, $+$ et $\cdot$ respecte les axiomes (CANI ADDU) :

L'opération $+$, est Commutative, Associative, il y a un vecteur Nul et tout vecteur possède un Inverse.

L'opération $\cdot$ est :

 - Associative
$\mu\cdot(\lambda\cdot\,\mv{v}\ )=(\mu.\lambda)\cdot\,\mv{v}$
{ \it il faut distinguer ici la multiplication dans $\mb{R}$ "." de la multiplication dans V "$\cdot$"}

 - Distributive :
$(\lambda + \mu)\cdot\,\mv{v} = \lambda\cdot\,\mv{v} + \mu\cdot\,\mv{v}$


 - Distributive :
 $\lambda\cdot(\,\mv{v}+\,\mv{w}\ ) = \lambda\cdot\mv{v}\, + \lambda\cdot\,\mv{w}$

 - Unité : $1\cdot\,\mv{v} = \mv{v}$

En dimension fini, il existe une base $\{\,\mv{e}_{\ (1)},\ \mv{e}_{\ (2)},\,...\,,\ \mv{e}_{\ (n)} \}$ tel que quelquesoit $\ \mv{v}\ $ il existe un unique $(v^1,...,v^n)$ tel que $\ \mv{v}\ =\sum \ v^{\,i\,}\ \mv{e}_{\ (i)}$. Autrement dit,

\[
\exists \{\,\mv{e}_{\ (i)}\}_{i=1,...,n}\ tq\ \forall\ \mv{v}\ \in V, \exists!\ (v^1,...,v^n)\ tq\ \ \mv{v}\ =\sum_{i=1}^n \ v^{\,i\,}\ \mv{e}_{\ (i)}
\]

\[
\mv{e}_{\ (i)} \to\ \mv{e'}_{\ (i)}=\sum_{j=1}^n \ P_i^{\,j}\ \ \mv{e'}_{\ (j)}
\]
$P_i^j$ est la matrice de passage. On a alors :

\[
\mv{v}\ =\sum_{i=1}^n \ v'^{\,i\,}\ \mv{e'}_{(i)} \qquad avec \qquad v^{\,i} = \sum_{j=1}^i P_i^{\,j} v'^j
\]

\[
[P^{\,i}_j]^{-1} \neq [P^{\,j}_i] \qquad et \qquad V = P V'\ \to\ V'=P'V
\]

Base et dimension

Composantes contravariantes

Fonctions linéaires

Espace vectoriel dual, V*

Base duale d'une base donnée


