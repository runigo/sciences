%2025 - 9a

{\bf Rappels et bilan}
- ce qu'on peut définir au sein d'une variété différentielle
- ce qu'apporte l'ajout d'une dérivée covariante

\[
\begin{array}{ l c l l l c l l l}
(M, \mc{O} \mc{A}_\infty) & \to & \mr{V}_{\gamma,p} & \to & T_pM & \to & TM & \to & \Gamma(TM) \\
 & &  &  & T^*_pM & \to & T^*M & \to & \Gamma(T^*M) \\
\end{array}
\]

\[
\begin{array}{ l c l l l c l l l}
 T_pM & \to & (\frac{\partial}{\partial x^i}) \\
 T^*_pM & \to & (dx^i) & \to & df \\
\end{array}
\]

\[
\mc{L}_X Y = [X,Y](f) =
X(\underbrace{Y(f)}_{Y^i\frac{\partial f}{\partial x^i}})
- Y(X(f))
\]

\[
\begin{array}{ l c l l l c l l l}
 \nabla & \  & \nabla_X f & = & Xf () \\
 \Gamma^k_{(x)ij} & \  & \nabla_X Y & = & X^i\underbrace{\nabla_{\frac{\partial}{\partial x^i}}Y}{} & \\
 & & & & \big(\nabla_{\frac{\partial}{\partial x^i}}Y\big)^j &
 = & \frac{\partial Y^j}{\partial x^i}+\Gamma_k \\
\end{array}
\]

{\bf Chapitre 6 : Métrique et géodésique}
champ de métrique : forme bilinéaire, symétrique, non dégénérée
applications "Bra", "Ket" : descente et montée d'indices
Norme d'un vecteur de TpM
Longueur d'une courbe, reparamétrage
Géodésique (longueur stationnaire)
Connexion de Levi-Civita
