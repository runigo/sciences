\documentclass[12pt, a4paper]{report}
%\documentclass[11pt, a4paper]{article}

%====================== PACKAGES ======================
\usepackage[french]{babel}

\frenchbsetup{StandardLists=true}
\usepackage{enumitem}
\usepackage{pifont}

\usepackage[utf8x]{inputenc}
%\usepackage[latin1]{inputenc}

% Pour gérer les positionnement d'images
\usepackage{float}
\usepackage{amsmath}
\usepackage{amssymb}
  \usepackage{amsthm}

  \renewcommand{\qedsymbol}{$\blacksquare$}
\DeclareMathOperator{\dt}{dt}
\usepackage{graphicx}
%\usepackage{tabularx}
\usepackage[colorinlistoftodos]{todonotes}
\usepackage{url}
% Pour les flèches sous les lettres
%\usepackage{accents}
\usepackage{mathtools}

% Pour les lettre \mathscr
\usepackage{mathrsfs}

% Pour les informations sur un document compilé en PDF et les liens externes / internes
\usepackage[pdfborder=0]{hyperref}
\hypersetup{
	colorlinks = true
	}

% Pour la mise en page des tableaux
\usepackage{array}
\usepackage{tabularx}
\usepackage{multirow}
\usepackage{multicol}
\setlength{\columnsep}{50pt}

% Utiliser une police droite dans les formules
\usepackage{mathastext}
\usepackage{cancel} % Pour barrer des caractères
% Pour utiliser \floatbarrier
%\usepackage{placeins}
%\usepackage{floatrow}

% Espacement entre les lignes
\usepackage{setspace}

% Modifier la mise en page de l'abstract
\usepackage{abstract}

% Police et mise en page (marges) du document
\usepackage[T1]{fontenc}
\usepackage[top=2cm, bottom=2cm, left=2cm, right=2cm]{geometry}

% Pour les galerie d'images
\usepackage{subfig}

\usepackage{pdfpages}

\usepackage{tikz}
\usetikzlibrary{trees}
\usetikzlibrary{decorations.pathmorphing}
\usetikzlibrary{decorations.markings}
\usetikzlibrary{decorations.pathreplacing,calligraphy}
\usetikzlibrary {arrows.meta}
%\usetikzlibrary{decorations}
\usetikzlibrary{angles, quotes}
\usepackage{verbatim}

\usepackage{appendix}

\usepackage{comment}

\usepackage{xcolor}

%\PreviewEnvironment{tikzpicture}
%\setlength\PreviewBorder{0pt}%

%====================== INFORMATION ET REGLES ======================

%rajouter les numérotation pour les \paragraphe et \subparagraphe
\setcounter{secnumdepth}{4}
\setcounter{tocdepth}{4}

\hypersetup{							% Information sur le document
pdfauthor = {Stephan Runigo},			% Auteurs
pdftitle = {Documentation},			% Titre du document
pdfsubject = {Documentation},		% Sujet
pdfkeywords = {Document},	% Mots-clefs
pdfstartview={FitH}}	% ajuste la page à la largeur de l'écran
%pdfcreator = {MikTeX},% Logiciel qui a crée le document
%pdfproducer = {} % Société avec produit le logiciel

%======================== DEFINITION COMMANDES ========================
\newcommand{\ul}[1]{\underline{#1}}
\newcommand{\mc}[1]{\mathcal{#1}}
\newcommand{\mb}[1]{\mathbb{#1}}
\newcommand{\mr}[1]{\mathscr{#1}}
\newcommand{\pt}[1]{\dot{\text{#1}}}
\newcommand{\mv}[1]{\mathclap{\overset{\rightarrow}{\bf{#1}}}}
\newcommand{\vd}[1]{\mathclap{\underset{\leftarrow}{\boldsymbol{#1}}}}
% Stereographic and cylindrical map projections
% Author: Tomasz M. Trzeciak
% Source: LaTeX-Community.org 
%         <http://www.latex-community.org/viewtopic.php?f=4&t=2111>
\newcommand\pgfmathsinandcos[3] {
  \pgfmathsetmacro#1{sin(#3)}
  \pgfmathsetmacro#2{cos(#3)} }
\newcommand\LongitudePlane[3][current plane] { 
  \pgfmathsinandcos\sinEl\cosEl{#2} % elevation
  \pgfmathsinandcos\sint\cost{#3} % azimuth
  \tikzset{#1/.style={cm={\cost,\sint*\sinEl,0,\cosEl,(0,0)}}} }
\newcommand\LatitudePlane[3][current plane] { %
  \pgfmathsinandcos\sinEl\cosEl{#2} % elevation
  \pgfmathsinandcos\sint\cost{#3} % latitude
  \pgfmathsetmacro\yshift{\cosEl*\sint}
  \tikzset{#1/.style={cm={\cost,0,0,\cost*\sinEl,(0,\yshift)}}} }
\newcommand\DrawLongitudeCircle[2][1]{
  \LongitudePlane{\angEl}{#2}
  \tikzset{current plane/.prefix style={scale=#1}}
   % angle of "visibility"
  \pgfmathsetmacro\angVis{atan(sin(#2)*cos(\angEl)/sin(\angEl))}
 % \draw[current plane,dashed] (\angVis-180:1) arc (\angVis-180:\angVis:1);
  \draw[current plane] (\angVis:1) arc (\angVis:\angVis+180:1); }

\newcommand\DrawLatitudeCircle[2][1]{
  \LatitudePlane{\angEl}{#2}
  \tikzset{current plane/.prefix style={scale=#1}}
  \pgfmathsetmacro\sinVis{sin(#2)/cos(#2)*sin(\angEl)/cos(\angEl)}
  % angle of "visibility"
  \pgfmathsetmacro\angVis{asin(min(1,max(\sinVis,-1)))}
 % \draw[current plane,dashed] (180-\angVis:1) arc (180-\angVis:\angVis:1);
  \draw[current plane] (\angVis:1) arc (\angVis:-\angVis-180:1); }


%======================== DEBUT DU DOCUMENT ========================
%
\begin{document}
% Écart entre le cadre et les formules
\setlength{\fboxsep}{7pt}
%régler l'espacement entre les lignes
\newcommand{\HRule}{\rule{\linewidth}{0.5mm}}
%
% Titre, résumé, ... %
Université Paris Cité // Année 2023-2024

Master FPA - M1 - Parcours Physique fondamentale

Relativité générale 2025

Séance 1a
(13 janvier 2025)

par Étienne Parizot



%
% Table des matières
\tableofcontents \thispagestyle{empty} \setcounter{page}{0}
%
%espacement entre les lignes des tableaux
\renewcommand{\arraystretch}{1.5}
%
%====================== INCLUSION DES CHAPITRES ======================
%
~
\thispagestyle{empty}
%recommencer la numérotation des pages à "1"
\setcounter{page}{0}
%\newpage
%


%1a - Chapitre 1 : Introduction à la Relativité Générale
\begin{itemize}[leftmargin=1cm, label=\ding{32}, itemsep=1pt]
\item {Qu'est-ce que la relativité générale (RG) ?}
\item {Théorie de l'espace-temps, théorie de la gravitation...}
\item {Espace, temps, corps physiques, référentiels}
\item {Lien entre la relativité générale et relativité restreinte ?}
\item {Gravitation newtonienne restreinte}
\item {Analogie avec l'électromagnétisme : jusqu'ou ?}
\end{itemize}

1b - Chapitre 1 : Introduction à la Relativité Générale (suite)
\begin{itemize}[leftmargin=1cm, label=\ding{32}, itemsep=1pt]
\item {Notion d'espace affine => à dépasser !}
\item {Qu'est-ce qu'un vecteur ? Vecteur d'espace, "entre deux points" ; vecteur "en un point"...}
\item {Gravitation : une "force" pas comme les autres !}
\item {Principe d'équivalence}
\item {Retour sur la notion de référentiel}
\item {Espace-temps : entité géométrique}
\item {Reformulation de la loi d'inertie}
\end{itemize}

2a - Chapitre 1 : Introduction à la Relativité Générale (fin)
\begin{itemize}[leftmargin=1cm, label=\ding{32}, itemsep=1pt]
\item {Retour sur la notion de référentiel}
\item {"Référentiel tournant" en relativité restreinte}
\item {Leibniz, Newton, Mach, Einstein...}
\item {Conséquences du principe d'équivalence ; effets gravitationnels sur la lumière :}
\item {-"déflection de la lumière par les  masses", vérification expérimentale (Eddington, 1919)}
\item {-"redschift gravitationnel", vérification expérimentale (Pound et Rebka, 1959//1960)}
\end{itemize}

2b - Chapitre 2 : Variétés différentielles
\begin{itemize}[leftmargin=1cm, label=\ding{32}, itemsep=1pt]
\item {Évolution des systèmes physiques : continuité et différentiabilité}
\item {Cas des trajectoires des corps physiques}
\item {Liberté de choix d'un système de coordonnées}
\item {Nécessité d'une notion {\bf intrinsèque} de continuité et de différentiabilité}
\item {Espaces topologiques :}
\item {-notion de topologie (ensembles "ouverts"), de voisinage, de limite, de continuité}
\end{itemize}

3a Chapitre 2 : Variétés différentielles (suite)
\begin{itemize}[leftmargin=1cm, label=\ding{32}, itemsep=1pt]
\item {Rappel : topologie, voisinage, limite, continuité}
\item {Notion d'homéomorphisme}
\item {Courbe sur un espace topologique}
\item {Variété topologique de dimension n : cartes, atlas}
\item {Variété différentielle}
\item {- cartes C\_k compatibles, atlas C\_k}
\item {- variété lisse}
\item {- structure différentielle}
\end{itemize}

3b - Chapitre 2 : Variétés différentielles (fin)
\begin{itemize}[leftmargin=1cm, label=\ding{32}, itemsep=1pt]
\item {Fonctions différentiables, C\_k, lisses (C\_infini):}
\item {-de R dans M (courbes lisses)}
\item {-entre deux variétés différentielles}
\item {Difféomorphisme}
\item {Unicité ou non des structures différentielles (à difféomorphisme près)}
\end{itemize}

3c - Chapitre 3 : Espace vectoriel tangent
\begin{itemize}[leftmargin=1cm, label=\ding{32}, itemsep=1pt]
\item {Discussion introductive :}
\item {- comment transposer la notion de tangente à une courbe différentiable, dans un contexte où l'espace n'a pas de structure vectorielle associée}
\item {- notion de "direction", de "vitesse de déplacement"}
\item {Définition fondamentale de la vitesse en un point d'une courbe différentiable}
\end{itemize}

4a - Chapitre 3 : Espace vectoriel tangent (suite)
\begin{itemize}[leftmargin=1cm, label=\ding{32}, itemsep=1pt]
\item {Vitesse d'une courbe en un point}
\item {"Reparamétrage" d'une courbe }
\item {Espace vectoriel tangent, TpM : addition de deux vitesses et multiplication par un scalaire}
\item {Courbes coordonnées (d'une carte donnée) et vitesses associées}
\item {Base de TpM associée à une carte donnée}
\item {Notation importante : "dérivée partielle d'une fonction", représentée dans une carte}
\end{itemize}

4b - Chapitre 3 : Espace vectoriel tangent (suite)
\begin{itemize}[leftmargin=1cm, label=\ding{32}, itemsep=1pt]
\item {Base de l'espace vectoriel tangent au point p (TpM), induite par une carte (U, x)}
\item {Composantes de la vitesse d'une courbe}
\item {Lien avec la notion habituelle de vitesse}
\end{itemize}

4c - Annexe : Algèbre multilinéaire
\begin{itemize}[leftmargin=1cm, label=\ding{32}, itemsep=1pt]
\item {Rappels sur les espaces vectoriels}
\item {Base et dimension}
\item {Composantes contravariantes}
\item {Fonctions linéaires}
\item {Espace vectoriel dual, V*}
\item {Base duale d'une base donnée}
\end{itemize}

5a - Annexe : Algèbre multilinéaire
\begin{itemize}[leftmargin=1cm, label=\ding{32}, itemsep=1pt]
\item {Espace vectoriel dual, V*, et notion de base duale}
\item {Composantes d'un vecteur ou d'une forme linéaire, relativement à une base donnée}
\item {Changement de base}
\item {Co-variance et contra-variance des composantes}
\item {Produit scalaire, forme bilinéaire non dégénérée}
\item {Isomorphisme canonique ?}
\item {Tenseurs de rang (r,s)}
\item {Composantes des tenseurs relativement à une base}
\end{itemize}

5b - Annexe : Algèbre multilinéaire (fin)
\begin{itemize}[leftmargin=1cm, label=\ding{32}, itemsep=1pt]
\item {Produit tensoriel de deux tenseurs}
\item {Vecteurs vus comme tenseurs de rang (1,0) :}
\item {identification de V et V**}
\item {Applications linéaires vues comme tenseurs de rang (1,1)}
\item {Attention à la représentation matricielle !}
\end{itemize}

5c - Chapitre 3 (retour) : Espace vectoriel tangent
\begin{itemize}[leftmargin=1cm, label=\ding{32}, itemsep=1pt]
\item {Rappel : définition de TpM}
\item {Base naturelle de TpM associée à une carte (U,x)}
\item {Changement de carte : changement de base et changement de coordonnées associés}
\item {Consistence des notations et familiarité des relations entre dérivées partielles}
\end{itemize}

6a - Chapitre 3 (fin) : Espace vectoriel tangent
\begin{itemize}[leftmargin=1cm, label=\ding{32}, itemsep=1pt]
\item {Espace vectoriel co-tangent, T*pM}
\item {Gradient d'une fonction lisse sur la variété M}
\item {Base de l'espace co-tangent associée à une carte}
\item {Base duale des dx$^i$}
\item {Composantes covariantes des covecteurs}
\item {Changement de carte, changement de composantes}
\item {Tenseurs de rang (r,s) au point p (tenseurs sur TpM)}
\end{itemize}

6b - Chapitre 4 : Espace fibré tangent, champs de vecteurs, de covecteurs, de tenseurs
\begin{itemize}[leftmargin=1cm, label=\ding{32}, itemsep=1pt]
\item {Notion d'espace fibré : espace total, base et projection}
\item {Section d'un espace fibré tangent, TM}
\item {Cartes de TM induite par les cartes de M}
\item {Structure différentielle du fibré tangent}
\end{itemize}

6c - Chapitre 4 (suite et fin): Espace fibré tangent, champs de vecteurs, de covecteurs, de tenseurs
\begin{itemize}[leftmargin=1cm, label=\ding{32}, itemsep=1pt]
\item {Définition d'un champ de vecteurs : section lisse de TM}
\item {G(TM) : C-infini(M)-module des champs de vecteurs lisses}
\item {Existence de bases de champs de vecteurs non garantie}
\item {Fibré cotangent T*M, cartes induites par les cartes de M}
\item {Champs de covecteurs}
\item {Champs tensoriels de rang (r,s)}
\end{itemize}

7a - Chapitre 5 : Dérivée covariante, connexion
\begin{itemize}[leftmargin=1cm, label=\ding{32}, itemsep=1pt]
\item {Motivation : dérivation de champs de vecteurs, de covecteurs ou de tenseurs dans une direction donnée}
\item {Contraintes à respecter : linéarité, règle(s) de Leibniz}
\item {Liberté restante : coefficient d'une connexion}
\item {Dérivée covariante de champs de tenseurs de rang quelconque}
\item {Transformation des coefficients de la connexion lors d'un changemment de carte}
\end{itemize}

7b - Chapitre 5 : Dérivée covariante, connexion
\begin{itemize}[leftmargin=1cm, label=\ding{32}, itemsep=1pt]
\item {Dérivée covariante d'un champ de covecteurs, ou d'un champ de tenseurs de rang quelconque}
\item {Application de la "règle de Leibniz"}
\item {Notion de transport parallèle (à compléter)}
\item {Notion de "courbe rectiligne" (à compléter)}
\end{itemize}

8a - Chapitre 5 : Dérivée covariante, connexion (suite)
\begin{itemize}[leftmargin=1cm, label=\ding{32}, itemsep=1pt]
\item {Dérivée covariante}
\item {Exemples de connexions}
\item {Transport parallèle}
\item {Courbe autoparallèle et autoparallèlement transportée}
\item {Équation géodésique (dans une carte donnée)}
\item {Annexe : reparamétrage}
\end{itemize}

8b - Chapitre 5 : Dérivée covariante, connexion (fin)
\begin{itemize}[leftmargin=1cm, label=\ding{32}, itemsep=1pt]
\item {Contenu tensoriel de la dérivée covariante}
\item {Commutateur d'un champs de vecteurs, crochet de Lie}
\item {Tenseur de torsion, de rang (1,2)}
\item {Tenseur de courbure de Riemann, de rang (1,3)}
\item {Transport parallèle le long d'une courbe fermée}
\end{itemize}

9a - Rappels et bilan

- ce qu'on peut définir au sein d'une variété différentielle

- ce qu'apporte l'ajout d'une dérivée covariante

Chapitre 6 : Métrique et géodésique
\begin{itemize}[leftmargin=1cm, label=\ding{32}, itemsep=1pt]
\item {champ de métrique : forme bilinéaire, symétrique, non dégénérée}
\item {applications "Bra", "Ket" : descente et montée d'indices}
\item {Norme d'un vecteur de TpM}
\item {Longueur d'une courbe, reparamétrage}
\item {Géodésique (longueur stationnaire)}
\item {Connexion de Levi-Civita}
\end{itemize}

9b - Chapitre 6 : Métrique et géodésiques (suite)
\begin{itemize}[leftmargin=1cm, label=\ding{32}, itemsep=1pt]
\item {"Compatibilité" entre métrique et dérivée covariante}
\item {Connexion de Levi-Civita}
\item {Tenseur de courbure de Ricci}
\item {Courbure scalaire de Ricci}
\item {Tenseur d'Einstein (et équations d'Einstein)}
\end{itemize}

Remarque :

gravitation newtonienne et espace-temps courbe

10a - Chapitre 7 : Espace-temps physique et équations d'Einstein
\begin{itemize}[leftmargin=1cm, label=\ding{32}, itemsep=1pt]
\item {Structure du cadre spatiotemporel : variété différentielle de dimension 4 munie d'une métrique Lorentzienne}
\item {Contenu matériel : "points matériels", champs...}
\end{itemize}

10b - Chapitre 7 : Espace-temps physique et équations d'Einstein (suite)
\begin{itemize}[leftmargin=1cm, label=\ding{32}, itemsep=1pt]
\item {Structure du cadre spatiotemporel}
\item {Contenu matériel}
\item {Notion d'observateur}
\item {Notion d'horloge ; temps propre}
\item {Grandeurs physiques relatives à un observateur}
\item {Dynamique ; évolution / déploiement dans l'espace-temps}
\item {Principe de moindre action ; lagrangien}
\item {Intégration sur une variété, jacobien, "forme volume"}
\end{itemize}

11a - Chapitre 7 : Espace-temps physique et équations d'Einstein (suite)
\begin{itemize}[leftmargin=1cm, label=\ding{32}, itemsep=1pt]
\item {Dynamique {\bf DANS} l'espace-temps :}
\item {- particules ponctuelles, champ électromagnétique}
\item {- Lagrangien, version 4D}
\item {- principe de moindre action, équations "du mouvement", équations de champ...}
\item {Dynamique {\bf DE} l'espace-temps :}
\item {- équations d'Einstein, approche heuristique}
\item {- tenseur énergie-impulsion ("stress-energy")}
\end{itemize}

11b - Chapitre 7 : Espace-temps physique et équations d'Einstein (fin)
\begin{itemize}[leftmargin=1cm, label=\ding{32}, itemsep=1pt]
\item {Dynamique DE l'espace-temps :}
\item {- équations d'Einstein}
\item {- cadre unificateur : principe de moindre action, lagrangien de l'espace-temps, action de Hilbert-Einstein}
\item {- tenseur d'Einstein issu d'un calcul variationnel}
\item {- "constante cosmologique"}
\item {- théorème de Cartan}
\item {- redéfinition du tenseur énergie-impulsion}
\item {- particule ponctuelle, limite continue, fluide parfait}
\end{itemize}

12a - Chapitre 8 : Symétries, flots, dérivée de Lie, champ de Killing...
\begin{itemize}[leftmargin=1cm, label=\ding{32}, itemsep=1pt]
\item {Pull back et push forward}
\item {- fonction, vecteur, covecteur, métrique, etc.}
\item {Cas particulier : métrique induite}
\item {Flot d'un champ de vecteurs}
\item {Groupe à un paramètre de difféomorphismes sur M}
\item {Isométrie}
\item {Champ de Killing}
\end{itemize}

12b - Chapitre 8 : Symétries, flots, dérivée de Lie, champ de Killing (fin)
\begin{itemize}[leftmargin=1cm, label=\ding{32}, itemsep=1pt]
\item {dérivée de Lie "le long d'un champ de vecteur"}
\item {Cas d'une fonction, d'un vecteur, d'un covecteur, d'un tenseur général}
\item {Application à une métrique :isométrie}
\item {Équation de Killing}
\item {Quantité conservée}
\end{itemize}

13a - Chapitre 9 : Exemples et application
\begin{itemize}[leftmargin=1cm, label=\ding{32}, itemsep=1pt]
\item {Rappel sur les symétries}
\item {Espace-temps de symétrie maximale : Minkowski}
\item {Vecteur de Killing de l'espace-temps de Minkowski}
\item {Limite Newtonienne}
\item {Métriques à symétrie sphérique}
\item {Dans le vide : métrique de Schwarzschild}
\item {Évocation du théorème de Birkhoff}
\end{itemize}

13b - Chapitre 9 : Exemples et application (suite)
\begin{itemize}[leftmargin=1cm, label=\ding{32}, itemsep=1pt]
\item {Métrique de Schwarzschild}
\item {- platitude asymptotique}
\item {- signification du rayon de Schwarzschild : "masse" de la distribution de matière}
\item {Géodésiques, traversée de l'horizon}
\item {Observateurs stationnaires}
\item {Désynchronisation des horloges}
\item {Correction de Relativité générale du système GPS}
\end{itemize}

13c - Chapitre 9 : Exemples et application (suite)
\begin{itemize}[leftmargin=1cm, label=\ding{32}, itemsep=1pt]
\item {Métrique de Schwarzschild :}
\item {- champs de Killing}
\item {- redshift gravitationnel}
\end{itemize}

14a - Correction de Examen - I
\begin{itemize}[leftmargin=1cm, label=\ding{32}, itemsep=1pt]
\item {}
\end{itemize}

14a - Correction de Examen - II
\begin{itemize}[leftmargin=1cm, label=\ding{32}, itemsep=1pt]
\item {}
\end{itemize}

15 - Complément : mécanique céleste relativiste
\begin{itemize}[leftmargin=1cm, label=\ding{32}, itemsep=1pt]
\item {}
\end{itemize}

16 - Complément : mécanique céleste relativiste
\begin{itemize}[leftmargin=1cm, label=\ding{32}, itemsep=1pt]
\item {}
\end{itemize}

17 - Complément : déflexion de la lumière
\begin{itemize}[leftmargin=1cm, label=\ding{32}, itemsep=1pt]
\item {}
\end{itemize}


%%2025 - 1a

Chapitre 1 : Introduction à la Relativité Générale

Qu'est-ce que la relativité générale (RG) ?
Théorie de l'espace-temps, théorie de la gravitation...
Espace, temps, corps physiques, référentiels
Lien entre la relativité générale et relativité restreinte ?
Gravitation newtonienne restreinte
Analogie avec l'électromagnétisme : jusqu'ou ?

%%2025 - 1b

Chapitre 1 : Introduction à la Relativité Générale (suite)

Notion d'espace affine => à dépasser !
Qu'est-ce qu'un vecteur ? Vecteur d'espace, "entre deux points" ; vecteur "en un point"...
Gravitation : une "force" pas comme les autres !
Principe d'équivalence
Retour sur la notion de référentiel
Espace-temps : entité géométrique
Reformulation de la loi d'inertie

%%2025 - 2a

Chapitre 1 : Introduction à la Relativité Générale (fin)

Retour sur la notion de référentiel
"Référentiel tournant" en relativité restreinte
Leibniz, Newton, Mach, Einstein...
Conséquences du principe d'équivalence ; effets gravitationnels sur la lumière :
-"déflection de la lumière par les  masses", vérification expérimentale (Eddington, 1919)
-"redschift gravitationnel", vérification expérimentale (Pound et Rebka, 1959//1960)

%%2025 - 2b
\chapter{Variétés différentielles}
{\bf 02b Chapitre 2 : Variétés différentielles}

Évolution des systèmes physiques : continuité et différentiabilité

Cas des trajectoires des corps physiques

Liberté de choix d'un système de coordonnées

Nécessité d'une notion {\bf intrinsèque} de continuité et de différentiabilité

Espaces topologiques :

-notion de topologie (ensembles "ouverts"), de voisinage, de limite, de continuité



%%2025 - 3a

{\bf Chapitre 2 : Variétés différentielles (suite)}

Rappel : topologie, voisinage, limite, continuité
Notion d'homéomorphisme
Courbe sur un espace topologique
Variété topologique de dimension n : cartes, atlas
Variété différentielle
- cartes C\_k compatibles, atlas C\_k
- variété lisse
- structure différentielle

%%2025 - 3b
\section{03b Chapitre 2 : Variétés différentielles (fin)}

Fonctions différentiables, C\_k, lisses (C\_infini):

-de R dans M (courbes lisses)

-entre deux variétés différentielles

Difféomorphisme

Unicité ou non des structures différentielles (à difféomorphisme près)



%%2025 - 3c

{\bf Chapitre 3 : Espace vectoriel tangent}
Discussion introductive :
- comment transposer la notion de tangente à une courbe différentiable, dans un contexte où l'espace n'a pas de structure vectorielle associée
- notion de "direction", de "vitesse de déplacement"
Définition fondamentale de la vitesse en un point d'une courbe différentiable

%2025 - 4a
{}%
{\bf Chapitre 3 : Espace vectoriel tangent (suite)}

%{Vitesse d'une courbe en un point}%(rappel)
\subsection{rappels}
Une variété topologique de dimension n un ensemble de point munie d'une topologie et d'un atlas : (M, $\mc{O}$, $\mc{A}$).
Un atlas est un ensemble de cartes, une carte est une application homéomorphe d'un ouvert autour d'un point dans  $\mb{R}^\mt{n}$ :
\[
\mc{A} = \{ (\mc{U}, x) : \mc{U} \to \mb{R}^\mt{n}
\}
\]
Si l'atlas est restreint aux cartes qui peuvent être reliées par des transformations infiniment dérivables, on a une variété différentielle : (M, $\mc{O}$, $\mc{A}_\infty$).

 = localement (autour de chaque point), la variété est isomorphe à $\mb{R}^\mt{n}$ : $\mc{A}$ est un ensemble de cartes, les cartes étant des applications homéomorphe d'un ouvert autour de chaque point dans $\mb{R}^\mt{n}$ qui permet de donner des coordonnées aux points

\begin{minipage}[c]{.5\linewidth}
\[
\begin{array}{ l r l l }
 \gamma : & \mb{R}^\mt{n} & \to & M \\ 
  & \lambda & \mapsto & \gamma(\lambda) \\
\end{array}
\]
\end{minipage}
\hfill
\begin{minipage}[c]{.5\linewidth}
\[
\begin{array}{ l r l l }
 {v}_{\gamma,p} : & \mc{C}^\infty(\mt{M}) & \to & \mb{R} \\ 
  & \mt{f} & \mapsto & {v}_{\gamma,p}(\mt{f})=(f\circ\gamma)'|_{\lambda_0} \\
\end{array}
\]
\end{minipage}


\[
\gamma(\lambda_0)=p\in M
\]


%{"Reparamétrage" d'une courbe}%
%{Espace vectoriel tangent, TpM : addition de deux vitesses et multiplication par un scalaire}%
%{Courbes coordonnées (d'une carte donnée) et vitesses associées}%
%{Base de TpM associée à une carte donnée}%
%{Notation importante : "dérivée partielle d'une fonction", représentée dans une carte}%

%%2025 - 4b
\section{04b : Espace vectoriel tangent (suite)}

Base de l'espace vectoriel tangent au point p (TpM), induite par une carte (U, x)

Composantes de la vitesse d'une courbe

Lien avec la notion habituelle de vitesse



%%2025 - 4c
\section{04c Annexe : Algèbre multilinéaire}

Rappels sur les espaces vectoriels

(V, $+$,$\cdot$) est un espace vectoriel. Par définition, $+$ et $\cdot$ respecte les axiomes (CANI ADDU) :

L'opération $+$, est Commutative, Associative, il y a un vecteur Nul et tout vecteur possède un Inverse.

L'opération $\cdot$ est :

 - Associative
$\mu\cdot(\lambda\cdot\,\mv{v}\ )=(\mu.\lambda)\cdot\,\mv{v}$
{ \it il faut distinguer ici la multiplication dans $\mb{R}$ "." de la multiplication dans V "$\cdot$"}

 - Distributive :
$(\lambda + \mu)\cdot\,\mv{v} = \lambda\cdot\,\mv{v} + \mu\cdot\,\mv{v}$


 - Distributive :
 $\lambda\cdot(\,\mv{v}+\,\mv{w}\ ) = \lambda\cdot\mv{v}\, + \lambda\cdot\,\mv{w}$

 - Unité : $1\cdot\,\mv{v} = \mv{v}$

En dimension fini, il existe une base $\{\,\mv{e}_{\ (1)},\ \mv{e}_{\ (2)},\,...\,,\ \mv{e}_{\ (n)} \}$ tel que quelquesoit $\ \mv{v}\ $ il existe un unique $(v^1,...,v^n)$ tel que $\ \mv{v}\ =\sum \ v^{\,i\,}\ \mv{e}_{\ (i)}$. Autrement dit,

\[
\exists \{\,\mv{e}_{\ (i)}\}_{i=1,...,n}\ tq\ \forall\ \mv{v}\ \in V, \exists!\ (v^1,...,v^n)\ tq\ \ \mv{v}\ =\sum_{i=1}^n \ v^{\,i\,}\ \mv{e}_{\ (i)}
\]

\[
\mv{e}_{\ (i)} \to\ \mv{e'}_{\ (i)}=\sum_{j=1}^n \ P_i^{\,j}\ \ \mv{e'}_{\ (j)}
\]
$P_i^j$ est la matrice de passage. On a alors :

\[
\mv{v}\ =\sum_{i=1}^n \ v'^{\,i\,}\ \mv{e'}_{(i)} \qquad avec \qquad v^{\,i} = \sum_{j=1}^i P_i^{\,j} v'^j
\]

\[
[P^{\,i}_j]^{-1} \neq [P^{\,j}_i] \qquad et \qquad V = P V'\ \to\ V'=P'V
\]

Base et dimension

Composantes contravariantes

Fonctions linéaires

Espace vectoriel dual, V*

Base duale d'une base donnée



%%2025 - 5a

{\bf Annexe : Algèbre multilinéaire}
Espace vectoriel dual, V*, et notion de base duale
Composantes d'un vecteur ou d'une forme linéaire, relativement à une base donnée
Changement de base
Co-variance et contra-variance des composantes
Produit scalaire, forme bilinéaire non dégénérée
Isomorphisme canonique ?
Tenseurs de rang (r,s)
Composantes des tenseurs relativement à une base

%%2025 - 5b
\section{05B Annexe : Algèbre multilinéaire (fin)}

Produit tensoriel de deux tenseurs
\[
T_1 \quad de\ rang \quad (r_1, s_1)
\qquad T_2 \quad de\ rang \quad (r_2, s_2)
\]
\[
T_1 \otimes T_2 : V_1^*
\]
Vecteurs vus comme tenseurs de rang (1,0) : On peut voir un vecteur comme un tenseur de rang (1,0)
\[\]
 - identification de V et V**

Applications linéaires vues comme tenseurs de rang (1,1) : On peut voir une application linéaire comme un tenseur de rang (1,1)

Attention à la représentation matricielle !

%%2025 - 5c
\section{05c Chapitre 3 (retour) : Espace vectoriel tangent}

Rappel : définition de TpM

Base naturelle de TpM associée à une carte (U,x)

Changement de carte : changement de base et changement de coordonnées associés

Consistence des notations et familiarité des relations entre dérivées partielles



%%2025 - 6a
\section{Chapitre 3 (fin) : Espace vectoriel tangent}


Espace vectoriel co-tangent, T*pM

Gradient d'une fonction lisse sur la variété M

Base de l'espace co-tangent associée à une carte

Base duale des dx$^i$

Composantes covariantes des covecteurs

Changement de carte, changement de composantes

Tenseurs de rang (r,s) au point p (tenseurs sur TpM)



%%2025 - 6b

{\bf Chapitre 4 : Espace fibré tangent, champs de vecteurs, de covecteurs, de tenseurs}
Notion d'espace fibré : espace total, base et projection
Section d'un espace fibré tangent, TM
Cartes de TM induite par les cartes de M
Structure différentielle du fibré tangent

%%2025 - 6c
\section{06c Chapitre 4 (suite et fin): Espace fibré tangent, champs de vecteurs, de covecteurs, de tenseurs}

Définition d'un champ de vecteurs : section lisse de TM

G(TM) : C-infini(M)-module des champs de vecteurs lisses

Existence de bases de champs de vecteurs non garantie

Fibré cotangent T*M, cartes induites par les cartes de M

Champs de covecteurs

Champs tensoriels de rang (r,s)



%%2025 - 7a
\chapter{Dérivée covariante, connexion}
{\bf  07a Chapitre 5 : Dérivée covariante, connexion}

Motivation : dérivation de champs de vecteurs, de covecteurs ou de tenseurs dans une direction donnée

Contraintes à respecter : linéarité, règle(s) de Leibniz

Liberté restante : coefficient d'une connexion

Dérivée covariante de champs de tenseurs de rang quelconque

Transformation des coefficients de la connexion lors d'un changemment de carte

%%2025 - 7b

{\bf Chapitre 5 : Dérivée covariante, connexion}
Dérivée covariante d'un champ de covecteurs, ou d'un champ de tenseurs de rang quelconque
Application de la "règle de Leibniz"
Notion de transport parallèle (à compléter)
Notion de "courbe rectiligne" (à compléter)

%%2025 - 8a

{\bf Chapitre 5 : Dérivée covariante, connexion (suite)}
Dérivée covariante
Exemples de connexions
Transport parallèle
Courbe autoparallèle et autoparallèlement transportée
Équation géodésique (dans une carte donnée)
Annexe : reparamétrage

%%2025 - 8b

{\bf Chapitre 5 : Dérivée covariante, connexion (fin)}
Contenu tensoriel de la dérivée covariante
Commutateur d'un champs de vecteurs, crochet de Lie
Tenseur de torsion, de rang (1,2)
Tenseur de courbure de Riemann, de rang (1,3)
Transport parallèle le long d'une courbe fermée

%%2025 - 9a

{\bf Rappels et bilan}
- ce qu'on peut définir au sein d'une variété différentielle
- ce qu'apporte l'ajout d'une dérivée covariante

{\bf Chapitre 6 : Métrique et géodésique}
champ de métrique : forme bilinéaire, symétrique, non dégénérée
applications "Bra", "Ket" : descente et montée d'indices
Norme d'un vecteur de TpM
Longueur d'une courbe, reparamétrage
Géodésique (longueur stationnaire)
Connexion de Levi-Civita

%%2025 - 9b

{\bf Chapitre 6 : Métrique et géodésiques (suite)}
"Compatibilité" entre métrique et dérivée covariante
Connexion de Levi-Civita
Tenseur de courbure de Ricci
Courbure scalaire de Ricci
Tenseur d'Einstein (et équations d'Einstein)

Remarque :
gravitation newtonienne et espace-temps courbe

%%2025 - 10a

{\bf Chapitre 7 : Espace-temps physique et équations d'Einstein}
Structure du cadre spatiotemporel : variété différentielle de dimension 4 munie d'une métrique Lorentzienne
Contenu matériel : "points matériels", champs...

%%2025 - 10b

{\bf Chapitre 7 : Espace-temps physique et équations d'Einstein (suite)}
Structure du cadre spatiotemporel
Contenu matériel
Notion d'observateur
Notion d'horloge ; temps propre
Grandeurs physiques relatives à un observateur
Dynamique ; évolution / déploiement dans l'espace-temps
Principe de moindre action ; lagrangien
Intégration sur une variété, jacobien, "forme volume"

%%2025 - 11a

{\bf Chapitre 7 : Espace-temps physique et équations d'Einstein (suite)}
Dynamique {\bf DANS} l'espace-temps :
- particules ponctuelles, champ électromagnétique
- Lagrangien, version 4D
- principe de moindre action, équations "du mouvement", équations de champ...
Dynamique {\bf DE} l'espace-temps :
- équations d'Einstein, approche heuristique
- tenseur énergie-impulsion ("stress-energy")

%%2025 - 11b

{\bf Chapitre 7 : Espace-temps physique et équations d'Einstein (fin)}
Dynamique DE l'espace-temps :
- équations d'Einstein
- cadre unificateur : principe de moindre action, lagrangien de l'espace-temps, action de Hilbert-Einstein
- tenseur d'Einstein issu d'un calcul variationnel
- "constante cosmologique"
- théorème de Cartan
- redéfinition du tenseur énergie-impulsion
- particule ponctuelle, limite continue, fluide parfait

%%2025 - 12a

{\bf Chapitre 8 : Symétries, flots, dérivée de Lie, champ de Killing...}
Pull back et push forward
- fonction, vecteur, covecteur, métrique, etc.
Cas particulier : métrique induite
Flot d'un champ de vecteurs
Groupe à un paramètre de difféomorphismes sur M
Isométrie
Champ de Killing

%%2025 - 12b

{\bf Chapitre 8 : Symétries, flots, dérivée de Lie, champ de Killing (fin)}
dérivée de Lie "le long d'un champ de vecteur"
Cas d'une fonction, d'un vecteur, d'un covecteur, d'un tenseur général
Application à une métrique : isométrie
Équation de Killing
Quantité conservée

%%2025 - 13a

{\bf Chapitre 9 : Exemples et application}
Rappel sur les symétries
Espace-temps de symétrie maximale : Minkowski
Vecteur de Killing de l'espace-temps de Minkowski
Limite Newtonienne
Métriques à symétrie sphérique
Dans le vide : métrique de Schwarzschild
Évocation du théorème de Birkhoff

%%2025 - 13b

{\bf Chapitre 9 : Exemples et application (suite)}
Métrique de Schwarzschild
- platitude asymptotique
- signification du rayon de Schwarzschild : "masse" de la distribution de matière
Géodésiques, traversée de l'horizon
Observateurs stationnaires
Désynchronisation des horloges
Correction de Relativité générale du système GPS

%\input{./2025/14a.tex}
%%2025 - 14a

{\bf Correction de Examen - II}

%%2025 - 15

{\bf Complément : mécanique céleste relativiste}

%%2025 - 16

{\bf Complément : avance du périhélie de Mercure}

%%2025 - 17

{\bf Complément : déflexion de la lumière}


%
%
%====================== INCLUSION DE LA BIBLIOGRAPHIE ======================
%
%récupérer les citation avec "/footnotemark" : 
\nocite{*}
%
% choix du style de la biblio
\bibliographystyle{plain}
%
% inclusion de la biblio
\cleardoublepage
\addcontentsline{toc}{chapter}{Bibliographie}
\bibliography{bibliographie.bib}
%
%====================== FIN DU DOCUMENT ======================
%
\end{document}
%%%%%%%%%%%%%%%%%%%%%%%%%%%%%%%%%%%%%%%%%%%%%%%%%%%%%%%%%%%%%%%%%%%%%%%%%%%%%%%%%
