\documentclass[11pt, a4paper]{report}
%\documentclass[11pt, a4paper]{article}

%====================== PACKAGES ======================
\usepackage[french]{babel}

\frenchbsetup{StandardLists=true}
\usepackage{enumitem}
\usepackage{pifont}

\usepackage[utf8x]{inputenc}
%\usepackage[latin1]{inputenc}

%pour gérer les positionnement d'images
\usepackage{float}
\usepackage{amsmath}
\usepackage{amsfonts}
\DeclareMathOperator{\dt}{dt}
\usepackage{graphicx}
%\usepackage{tabularx}
\usepackage[colorinlistoftodos]{todonotes}
\usepackage{url}

%pour les informations sur un document compilé en PDF et les liens externes / internes
\usepackage[pdfborder=0]{hyperref}
\hypersetup{
	colorlinks = true
	}

%pour la mise en page des tableaux
\usepackage{array}
\usepackage{tabularx}
\usepackage{multirow}
\usepackage{multicol}
\setlength{\columnsep}{50pt}

%pour utiliser \floatbarrier
%\usepackage{placeins}
%\usepackage{floatrow}

%espacement entre les lignes
\usepackage{setspace}

%modifier la mise en page de l'abstract
\usepackage{abstract}

%police et mise en page (marges) du document
\usepackage[T1]{fontenc}
\usepackage[top=2cm, bottom=2cm, left=2cm, right=2cm]{geometry}

%Pour les galerie d'images
\usepackage{subfig}

\usepackage{pdfpages}

\usepackage{tikz}
\usetikzlibrary{trees}
\usetikzlibrary{decorations.pathmorphing}
\usetikzlibrary{decorations.markings}
\usetikzlibrary{decorations.pathreplacing,calligraphy}
%\usetikzlibrary{decorations}
\usetikzlibrary{angles, quotes}
\usepackage{verbatim}

\usepackage{appendix}

\usepackage{comment}

\usepackage{xcolor}

%\PreviewEnvironment{tikzpicture}
%\setlength\PreviewBorder{0pt}%

%====================== INFORMATION ET REGLES ======================

%rajouter les numérotation pour les \paragraphe et \subparagraphe
\setcounter{secnumdepth}{4}
\setcounter{tocdepth}{4}

\hypersetup{							% Information sur le document
pdfauthor = {Stephan Runigo},			% Auteurs
pdftitle = {Documentation},			% Titre du document
pdfsubject = {Documentation},		% Sujet
pdfkeywords = {Document},	% Mots-clefs
pdfstartview={FitH}}	% ajuste la page à la largeur de l'écran
%pdfcreator = {MikTeX},% Logiciel qui a crée le document
%pdfproducer = {} % Société avec produit le logiciel

%======================== DEFINITION COMMANDES ========================
\newcommand{\mc}[1]{\mathcal{#1}}
\newcommand{\so}{\vspace{.323cm}}
\newcommand{\ib}[1]{\item {\bf #1}}
\newcommand{\bi}[1]{\textbf{\textit {#1}}}
%======================== DEBUT DU DOCUMENT ========================
%
\begin{document}
%
%régler l'espacement entre les lignes
\newcommand{\HRule}{\rule{\linewidth}{0.5mm}}
%
% Titre, résumé, ... %
%
%
\begin{titlepage}
%
~\\[1cm]

\begin{center}
%\includegraphics[scale=0.5]{./presentation/chambreABulle}
\end{center}

\textsc{\Large }\\[0.5cm]

% Title \\[0.4cm]
\HRule

\begin{center}
{\huge \bfseries Relativité restreinte et\\
électromagnétisme\\[0.4cm] }
\end{center}

\HRule \\[1.5cm]

\begin{center}
%\includegraphics[scale=0.3]{./presentation/ptoleme}
\end{center}

% Author and supervisor
\begin{minipage}{0.4\textwidth}
\begin{flushleft} \large
%\emph{Auteur:}\\
%Stephan \textsc{Runigo}
\end{flushleft}
\end{minipage}
\begin{minipage}{0.4\textwidth}
\begin{flushright} \large
\emph{Latex-iation:}\\
Stephan \textsc{Runigo}
\end{flushright}
\end{minipage}

\vfill
\begin{minipage}{0.4\textwidth}
\begin{flushleft} \large
Exercices de l'ENS extraits de https://www.phys.ens.fr/
\end{flushleft}
\end{minipage}
\begin{minipage}{0.4\textwidth}
\begin{flushright} \large
\end{flushright}
\end{minipage}

\vfill

% Bottom of the page
{\large \today}

\end{titlepage}

\newpage
\begin{center}
\Large
Résumé
\normalsize
\end{center}
\vspace{3cm}
\begin{itemize}[leftmargin=1cm, label=\ding{32}, itemsep=21pt]
\item {\bf Objet : } Mécanique Quantique.
\item {\bf Contenu : } Notes de cours.
\item {\bf Niveau requis : } Maitrise de sciences physiques.
\end{itemize}

\vspace{3cm} \large

Notes de cours rédigées en 1966 par Serge Haroche
\vspace{3cm}

http://enseignement.phys.ens.fr/spip.php?article110
\vspace{3cm}


%

%
% Table des matières
\tableofcontents
\thispagestyle{empty}
\setcounter{page}{0}
%
%espacement entre les lignes des tableaux
\renewcommand{\arraystretch}{1.5}
%
%====================== INCLUSION DES CHAPITRES ======================
%
~
\thispagestyle{empty}
%recommencer la numérotation des pages à "1"
\setcounter{page}{0}
\newpage
%\chapter{Ensemble et application}
%

%%%%%%%%%%%%%%%%%%%%%%%%%
\section{Ensemble}
%%%%%%%%%%%%%%%%%%%%%%%%%
%$\mathcal{}$
\subsection{définitions}
Un \textbf{\textit {ensemble}} est une {\it collection d'objets}. Les objets appartenant à un ensemble sont appelés \textbf{\textit {élément}}.

L'élément $a$ appartient à l'ensemble $\mathcal{A}$ s'écrit :

\[
 a \in \mathcal{A}
\]

Un ensemble peut être représenté par une enveloppe autour de ses éléments représentés par des points :



\begin{center}
\begin{tikzpicture}

	\def\a{3} \def\b{1.75} % Taille de l'ellipse
\draw[black!60,very thick] (0,0) circle[x radius = \a cm, y radius = \b cm];

	%\node[draw] (P) at (0,0) {Paris};

	\def\i{1.85} \def\j{-0.25}% Position relative des points

	\coordinate (x) at ({\a*\i},0);
	\coordinate[label=above:x] (x);
	\node at (4,-1) {$\times$};
	
	\tikzset{mypoints/.style={fill=white,draw=black,thick}}
\end{tikzpicture}
\end{center}



\begin{center}
\begin{tikzpicture}

	\def\a{3} \def\b{1.75} % Taille de l'ellipse
\draw[black!60,very thick]
		(0,0) circle[x radius = \a cm, y radius = \b cm];

	\def\i{1.85} \def\j{-0.25}% Position relative des points

	\coordinate (x) at ({\a*\i},0);
	\draw [line width=2pt] (-6.51,1.52)-- (-6.19,1.8);
	\coordinate[label=above:x] (x);
	\coordinate (y) at (-{\a*\i},0);
	\coordinate [label=below left:y] (y);
	
	\coordinate (z) at (0,-\b*\i);
	\coordinate [label=below left:z] (z);
	
	\coordinate (w) at (0,\b*\i);
	\coordinate [label=below left:w] (w);
	\node at (4,-1) {$\times$};
	
	\tikzset{mypoints/.style={fill=white,draw=black,thick}}
	\def\ptsize{2.0pt}
	\foreach \p in {x,y,z,w}\fill[mypoints] (\p) circle (\ptsize);
\end{tikzpicture}
\end{center}

    \begin{tikzpicture}[scale=1.2]
	\tikzset{mypoints/.style={fill=white,draw=black,thick}}
	\def\ptsize{2.0pt}
	\def\a{3} \def\b{1.75}
	% warning: construction fails if xp<0 or yp<=0
	\def\xp{4.0} \def\yp{3.5} 
	\def\i{0.85} \def\j{-0.25}%determines rays from P
	\coordinate[label=above:P] (P) at (\xp,\yp);
	\coordinate (M) at ({\a*\i},0);
	\coordinate (N) at ({\a*\j},0);
	\coordinate (AA) at (0,-\b);
	\coordinate (BB) at (1,-\b);
	\coordinate (CC) at (-\a,0);
	\coordinate (DD) at (-\a,1);
	\coordinate (Q) at (intersection of P--N and AA--BB);
	\coordinate (R) at (intersection of P--M and AA--BB);
	%\draw[name path=ellipse,red,very thick]
		%(0,0) circle[x radius = \a cm, y radius = \b cm];
	%\path[name path=linePQ,blue] (P)--(Q);
	%\path[name path=linePR,green] (P)--(R);
	%\path [name intersections={of = ellipse and linePQ}];
	%\coordinate[label=above:A] (A)  at (intersection-1);
	%\coordinate[label=below left:B] (B) at (intersection-2);
	%\path [name intersections={of = ellipse and linePR}];
	%\coordinate[label=above right:C] (C)  at (intersection-1);
	%\coordinate[label=above left:D] (D) at (intersection-2);
	%\draw (B)--(P)--(D) (A)--(D) (C)--(B);
	%\coordinate [label=below left:E] (E) at (intersection of A--D and B--C);
	%\coordinate[label=above right:F] (F) at (intersection of A--C and B--D);
	%\coordinate (G) at (intersection of E--F and CC--DD);
	%\draw [name path=lineFG,blue,thick] (F)--(G);
	%\draw (A)--(F)--(B);
	%\path [name intersections={of = ellipse and lineFG}];
	%\coordinate[label=above:X] (X) at (intersection-1);
	%\coordinate[label=above right:Y] (Y) at (intersection-2);
	%\coordinate (XX) at ($(P)!1.5!(X)$);
	%\coordinate (YY) at ($(P)!1.5!(Y)$);
	%\draw[very thick,green!50!black!50] (XX)--(P)--(YY);
	%\foreach \p in {A,B,C,D,E,F,P,X,Y}\fill[mypoints] (\p) circle (\ptsize);
  \end{tikzpicture}
%%%%%%%%%%%%%%%%%%%%%%%%%%%%%%%%%%%%%%%%%%%%%%%%%%%%%%%%%%%%%%%%%%%%%%%%

%

%%%%%%%%%%%%%%%%%%%%%%%%%
\section{Ensembles de nombres}
%%%%%%%%%%%%%%%%%%%%%%%%%

\subsection{Entiers naturels}
\[
\mathbb{N} = \{ 0, 1, 2, 3, ... \}
\]
\subsection{Entiers relatifs}
\[
\mathbb{Z} = \{ ..., -2, -1, 0, 1, 2, 3, ... \}
\]
\subsection{Nombres rationnels}
\[
\mathbb{Q} = \{ 0, 1, 2, 3, ... \}
\]
\subsection{Nombres réels}
\[
\mathbb{R} = \{ 0, 1, 2, 3, ... \}
\]
\subsection{Nombres complexes}
\[
\mathbb{C} = \{ 0, 1, 2, 3, ... \}
\]
Une application ($f$) met en relation des éléments ($a$) d'un ensemble ($\mathcal{A}$, dit de départ) avec des éléments ($b$) d'un autre ensemble ($\mathcal{B}$, dit d'arrivé). :
\begin{align*}
f :\ \ \ \ \ \ \ \ \ \mathcal{A} \ \  & \rightarrow \ \ \ \mathcal{B} \\
a \ \ & \mapsto \ \ b = f(a)
\end{align*}

Une loi de composition est une application qui associe deux éléments (éventuellement du même ensemble) à un troisième élément. 
\begin{align*}
f :\ \ \ \ \ \ \ \ \ \mathcal{A} \times \mathcal{B} \ \  & \rightarrow \ \ \ \mathcal{C} \\
(a,b) \ \ & \mapsto \ \ c = f(a,b)
\end{align*}

Une loi de composition est dite interne si $\mathcal{A} = \mathcal{B} = \mathcal{C}$, externe sinon.

%%%%%%%%%%%%%%%%%%%%%%%%%%%%%%%%%%%%%%%%%%%%%%%%%%%%%%%%%%%%%%%%%%%%%%%%%%%%%%%%%%%%%
%%%%%%%%%%%%%%%%%%%%%%%%%%%%%%%%%%%%%%%%%%%%%%%%%%%%%%%%%%%%%%%%%%%%%%%%

%

%%%%%%%%%%%%%%%%%%%%%%%%%
\section{Application}
%%%%%%%%%%%%%%%%%%%%%%%%%
Une application ($f$) met en relation des éléments ($a$) d'un ensemble ($\mathcal{A}$, dit de départ) avec des éléments ($b$) d'un autre ensemble ($\mathcal{B}$, dit d'arrivé). :
\begin{align*}
f :\ \ \ \ \ \ \ \ \ \mathcal{A} \ \  & \rightarrow \ \ \ \mathcal{B} \\
a \ \ & \mapsto \ \ b = f(a)
\end{align*}

Une loi de composition est une application qui associe deux éléments (éventuellement du même ensemble) à un troisième élément. 
\begin{align*}
f :\ \ \ \ \ \ \ \ \ \mathcal{A} \times \mathcal{B} \ \  & \rightarrow \ \ \ \mathcal{C} \\
(a,b) \ \ & \mapsto \ \ c = f(a,b)
\end{align*}

Une loi de composition est dite interne si $\mathcal{A} = \mathcal{B} = \mathcal{C}$, externe sinon.

%%%%%%%%%%%%%%%%%%%%%%%%%%%%%%%%%%%%%%%%%%%%%%%%%%%%%%%%%%%%%%%%%%%%%%%%%%%%%%%%%%%%%
%%%%%%%%%%%%%%%%%%%%%%%%%%%%%%%%%%%%%%%%%%%%%%%%%%%%%%%%%%%%%%%%%%%%%%%%

%

%\input{./espace/espace.tex}
%
\chapter{Algebre linéaire}
%

%%%%%%%%%%%%%%%%%%%%%
\section{Définitions}
%%%%%%%%%%%%%%%%%%%%%
\subsection{Matrices}
Une matrice est un tableau de nombres. Ces nombres sont des éléments d'un ensemble, parfois $\mathbb{Z}$, souvent $\mathbb{R}$, plus généralement $\mathbb{C}$.

L'ensemble des matrices à  $N$ lignes et $M$ colonnes ($N \times M$) à coeficients réèl est
$\mathbb{R}^N \times \mathbb{R}^M$.

Une matrice carée contient autant de lignes que de colonnes.
L'ensemble des matrices carré ($N \times N$) à coeficients réèl est
$\mathbb{R}^N \times \mathbb{R}^N$.

\subsection{Inverse d'une matrice}
%Une matrice colonne (matrice a n ligne et 1 colonne) à coefficients réèls (resp. complexe) est apte à représenter un vecteur.  
%Un vecteur est une matrice colonne. L'ensemble des vecteurs sur un corps forment un espace vectoriel ($\mathbb{R}^n$ et $\mathbb{C}^n$ sont des espaces vectoriels).

Une matrice $\mc{A}∈\mc{M}_n(\mathbb{C})$ est inversible s'il existe une matrice
$\mc{A}^{-1}$ tel que :
\[
\mc{A}.\mc{A}^{-1} = \mc{I}_n
\]

Dès lors, les matrices colonnes constituant $\mc{A}$ représentent $n$ vecteurs libres. Ces vecteurs forment une base de l'espace vectoriel $\mathbb{C}^n$.

\subsection{Matrice diagonale}

Une matrice $\mc{B}$ de $\mc{M}_n(\mathbb{C})$ est diagonale si
\[
b_{ij}=0 \ \ \text{ si } \ \ i\neq j
\]
Les $b_{ii}$ sont les valeurs propre de $\mc{B}$

\so
Une matrice $\mc{C}$ de $\mc{M}_n(\mathbb{C})$ est diagonalisable s'il existe une matrice $\mc{A}$ tel que
\[
\mc{A}.\mc{C}.\mc{A}^{-1} \text{ est diagonale}
\]

\subsection{Matrice stochastique}
Une matrice $\mathcal{A}∈\mathcal{M}_n(\mathbb{R})$ est une matrice stochastique si la somme des coefficients sur chaque colonne de $\mathcal{A}$ est égale à 1.
%%%%%%%%%%%%%%%%%%%%%%%%%%%%%%%%%%%%%%%%%%%%%%%%%%%%%%%%%%%%%%%%%%%%%%%%

%

%%%%%%%%%%%%%%%%%%%%%
%\section{Exercices sur les matrices}
%%%%%%%%%%%%%%%%%%%%%
\newpage
\subsection{Produit de matrice}
Exercice n°7.
Effectuer les produits suivants lorsque c’est possible. Lorsque c’est impossible, dire pourquoi.

\begin{minipage}[c]{.45\linewidth}
a)
$\left( \begin{array}{ c c }
 2 & 5 \\ 3 & 6 \\ 4 & 7
\end{array} \right)
×
\left( \begin{array}{ c c }
 2 & 5 \\ 4 & 6
\end{array} \right)$
\end{minipage}
\hfill
\begin{minipage}[c]{.45\linewidth}
b)
$\left( \begin{array}{ c c }
 2 & 5 \\ 4 & 6
\end{array} \right)
×
\left( \begin{array}{ c c }
 2 & 5 \\ 3 & 6 \\ 4 & 7
\end{array} \right)$
\end{minipage}

\begin{minipage}[c]{.45\linewidth}
c)
$\left( \begin{array}{ c c c }
 -1 & 4 & 5
\end{array} \right)
×
\left( \begin{array}{ c c c }
 0 & -1 & 6 \\ 2 & 4 & -2 \\ 3 & 5 & 3
\end{array} \right)$
\end{minipage}
\hfill
\begin{minipage}[c]{.45\linewidth}
d)
$\left( \begin{array}{ c c c }
 2 & 5 & 0 \\ 3 & 6 & 3 \\ 4 & 1 & 2
\end{array} \right)
×
\left( \begin{array}{ c c }
 1 & -1 \\ 2 & 0 \\ 3 & 5
\end{array} \right)$
\end{minipage}

\begin{minipage}[c]{.45\linewidth}
e)
$\left( \begin{array}{ c c }
 1 & -1 \\ 2 & 0 \\ 3 & 5
\end{array} \right)
×
\left( \begin{array}{ c c }
 2 & 5 \\ 3 & 6 \\ 4 & 1
\end{array} \right)$
\end{minipage}
\hfill
\begin{minipage}[c]{.45\linewidth}
f)
$\left( \begin{array}{ c c c }
 1 & 0 & 5 \\ 2 & -1 & 6 \\ 3 & 4 & 7 \\
\end{array} \right)
×
\left( \begin{array}{ c c c }
 2 & 7 & 8 \\ 0 & 2 & 3 \\ 4 & 5 & 6 \\
\end{array} \right)$
\end{minipage}

\subsection{Matrice 2×2}
Exercice n°8. Calculer, puis comparer les produits A×B et B×A

\begin{center}
a)
$ A = 
\left( \begin{array}{ c c }
 -1 & 8 \\ 2 & 11
\end{array} \right)$
et
$B =
\left( \begin{array}{ c c }
4 & 2 \\ -5 & 18
\end{array} \right)$
\end{center}

\begin{minipage}[c]{.45\linewidth}
b)
$ A = 
\left( \begin{array}{ c c }
 4 & 8 \\ 1 & 2
\end{array} \right)$
et
$ B = 
\left( \begin{array}{ c c }
 3 & 9 \\ 1 & 1
\end{array} \right)$
\end{minipage}
\hfill
\begin{minipage}[c]{.45\linewidth}
c)
$ A = 
\left( \begin{array}{ c c }
 2 & 1 \\ 1 & 1
\end{array} \right)$
et
$ B = 
\left( \begin{array}{ c c }
 5 & 2 \\ 2 & 3
\end{array} \right)$
\end{minipage}

\vspace{.323cm}
Exercice n°9.
Dans chacun des cas, calculer les produits A×B et B×A . Quelle particularité présente-t-il ?

a)
$ A = 
\left( \begin{array}{ c c }
 6 & -12 \\ -3 & 6
\end{array} \right)$
et
$ B = 
\left( \begin{array}{ c c }
 12 & 6 \\ 6 & 3
\end{array} \right)$
b)
$ A = 
\left( \begin{array}{ c c }
 2 & 4 \\ -1 & -2
\end{array} \right)$
et
$ B = 
\left( \begin{array}{ c c }
 0 & 2 \\ 0 & -1
\end{array} \right)$

\vspace{.323cm}
Exercice n°11.

Calculez et comparez $A^2 + 2 AB + B^2$ et $( A + B )^2$ avec : 
\begin{center}
$ A = 
\left( \begin{array}{ c c }
4 & 8 \\ 3 & 9
\end{array} \right)$
et
$ B =
\left( \begin{array}{ c c }
1 & 2 \\ 1 & 1
\end{array} \right)$
\end{center}

\vspace{.323cm}
{\footnotesize Source : https://maurimath.net/documents/ecly/matricesexoscorriges.pdf}


\newpage
\section{Opérations sur les matrices}
\subsection{Matrices stochastiques}

Exercice 9 - Matrices stochastiques en petite taille

On dit qu'une matrice $\mathcal{A}∈\mathcal{M}_n(\mathbb{R})$ est une matrice stochastique si la somme des coefficients sur chaque colonne de $\mathcal{A}$ est égale à 1. Démontrer que le produit de deux matrices stochastiques est une matrice stochastique si $n=2$. Reprendre la question si $n=3$.

\vspace{.323cm}
Exercice 10 - Matrices stochastiques

Soit $\mathcal{A},\mathcal{B}∈\mathcal{M}_n(\mathbb{R})$ deux matrices telles que la somme des coefficients sur chaque colonne de A et sur chaque colonne de B vaut 1 (on dit qu'une telle matrice est une matrice stochastique). Montrer que la somme des coefficients sur chaque colonne de AB vaut 1.

\vspace{.323cm}
{\scriptsize Source : 

\texttt{https://www.bibmath.net/ressources/index.php?action=affiche\&quoi=bde/algebrelineaire/matrices\&type=fexo}}

\subsection{Inverses de matrices}

Exercice 4 - Inverser une matrice sans calculs !
\vspace{.323cm}

\begin{minipage}[c]{.45\linewidth}
1. Soit $ A=\left(
\begin{array}{ccc}
-1&1&1\\1&-1&1\\1&1&-1
\end{array}\right)$.
\end{minipage}
\hfill
\begin{minipage}[c]{.45\linewidth}
Montrer que $A^2=2I_3-A$, en déduire que $A$ est inversible et calculer $A^{-1}$.
\end{minipage}

\vspace{.323cm}
\begin{minipage}[c]{.45\linewidth}
2. Soit $ A=\begin{pmatrix} 1 & 0 & 2 \cr
0 & -1 & 1 \cr
1 & -2 & 0 \cr \end{pmatrix} .$
\end{minipage}
\hfill
\begin{minipage}[c]{.45\linewidth}
Calculer $A^3-A .$ En déduire
que $ A $ est inversible puis déterminer $ A^{-1} .$ 
\end{minipage}

\vspace{.323cm}
\begin{minipage}[c]{.45\linewidth}
3. Soit $A=\begin{pmatrix} 
0&1&-1\\-1&2&-1\\1&-1&2
\end{pmatrix}$.
\end{minipage}
\hfill
\begin{minipage}[c]{.45\linewidth}
Calculer $A^2-3A+2I_3$. En déduire que $A$ est inversible, et calculer $A^{-1}$.
\end{minipage}

\vspace{.323cm}
{\scriptsize Source : 

\texttt{https://www.bibmath.net/ressources/index.php?action=affiche\&quoi=bde/algebrelineaire/matricesaa\&type=fexo}}



%%%%%%%%%%%%%%%%%%%%%%%%%%%%%%%%%%%%%%%%%%%%%%%%%%%%%%%%%%%%%%%%%%%%%%%%

%
\newpage
%%%%%%%%%%%%%%%%%%%%%
\section{Vecteur}
%%%%%%%%%%%%%%%%%%%%%
%\vspace{.323cm}
Exercice 1 - Combinaisons linéaires ?

\vspace{.323cm}
Les vecteurs $u$ suivants sont-ils combinaison linéaire des vecteurs $u_i$ ?

\vspace{.323cm}
$ u = 
\left( \begin{array}{ c }
 1 \\ 2
\end{array} \right)$
,
$ u_1 = 
\left( \begin{array}{ c }
 1 \\ −2
\end{array} \right)$
,
$ u_2 = 
\left( \begin{array}{ c }
 2 \\ 3
\end{array} \right)$

\vspace{.323cm}
$ u = 
\left( \begin{array}{ c }
 1 \\ 2
\end{array} \right)$
,
$ u_1 = 
\left( \begin{array}{ c }
 1 \\ −2
\end{array} \right)$
,
$ u_2 = 
\left( \begin{array}{ c }
 2 \\ 3
\end{array} \right)$
,
$ u_3 = 
\left( \begin{array}{ c }
 −4 \\ 5
\end{array} \right)$

\vspace{.323cm}
$ u = 
\left( \begin{array}{ c }
 2 \\ 5 \\ 3
\end{array} \right)$
,
$ u_1 = 
\left( \begin{array}{ c }
 1 \\ 3 \\ 2
\end{array} \right)$
,
$ u_2 = 
\left( \begin{array}{ c }
 1 \\ −1 \\ 4
\end{array} \right)$

\vspace{.323cm}
$ u = 
\left( \begin{array}{ c }
 3 \\ 1 \\ m
\end{array} \right)$
,
$ u_1 = 
\left( \begin{array}{ c }
 1 \\ 3 \\ 2
\end{array} \right)$
,
$ u_2 = 
\left( \begin{array}{ c }
 1 \\ −1 \\ 4
\end{array} \right)$

 (discuter suivant la valeur de m). 


\vspace{.323cm}
Exercice 3 - Combinaisons linéaires?

\vspace{.323cm}
 Dans l'espace vectoriel $\mathbb{R}[X]$, le polynôme $P(X)=16X^3−7X^2+21X−4$ est-il combinaison linéaire de $P_1(X)=8X^3−5X^2+1$ et $P_2(X)=X^2+7X−2$ ?

\vspace{.323cm}
Dans l'espace vectoriel $\mathcal{F}(\mathbb{R},\mathbb{R})$
des fonctions de $\mathbb{R}$ dans $\mathbb{R}$, la fonction $x ↦ \sin(2x)$ est-elle combinaison linéaire des fonctions $\sin$ et $\cos$?
\vspace{.323cm}

{\scriptsize Source : 

\texttt{https://www.bibmath.net/ressources/index.php?action=affiche\&quoi=bde/algebrelineaire/evfamilleslibres\&type=fexo}}

%%%%%%%%%%%%%%%%%%%%%%%%%%%%%%%%%%%%%%%%%%%%%%%%%%%%%%%%%%%%%%%%%%%%%%%%

%
\newpage

%%%%%%%%%%%%%%%%%%%%%
\section{Système et états}
%%%%%%%%%%%%%%%%%%%%%
\subsection{Définitions}
	\begin{itemize}[leftmargin=1cm, label=\ding{32}, itemsep=1pt]

\ib{Système} — \fsb{S. concr.} {\bf 1.} Ensemble organisé dont les parties
ou éléments sont interdépendants ou obéissent à une loi unique : « Le
système solaire »; « Le système nerveux »; « Tout fait psychique est un
système » (Paulhan). — \fsb{S. abstr.} {\bf 2.} Combinaison d'idées ou de
procédés coordonnés et ramenés à un petit nombre de principes : « Le système
héliocentrique »; « Les systèmes philosophiques »; « Le système métrique ».
Qqfs. {\it péj.}, spéc. dans l'expression « esprit de système » : voir {\it
Systématique}$^2$, et cf. {\it Textes choisis}, II, p. 38.

\ib{Évolution} — \si{Méta.} {\bf 1.} {\it Lato.} Suite de
transformations régie par une loi$^5$,
et gén. conçue comme graduelle et
continue : « La formation des mondes
expliquée par voie de développement lent et graduel ou, selon
l'expression, moderne d'{\it évolution} »
(Fouillée). — {\bf 2.} {\it Str.} ({\it spéc.} chez
Spencer). Transformation universelle définie surtout par la différenciation$^3$ et l'intégration$^3$ ([v. ces
mots) progressives : « L’évolution
est une intégration de matière, pendant laquelle celle-ci passe d’une
homogénéité indéfinie, incohérente,
à une hétérogénéité définie, cohérente. » (Spencer). — {\bf 3.}
{\it Évolution créatrice} (Bergson) : celle qui, au
lieu de « reconstituer l’évolution
avec des fragments de l’évolué »
[comme chez Spencer), consiste en
un élan* créateur : « L'évolution
est une création sans cesse renouvelée » ({\it E. C.}, II).

	\end{itemize}
%%%%%%%%%%%%%%%%%%%%%%%%%%%%%%%%%%%%%%%%%%%%%%%%%%%%%%%%%%%%%%%%%%%%%%%%

%
\newpage
%%%%%%%%%%%%%%%%%%%%%
\section{Markov}
%%%%%%%%%%%%%%%%%%%%%
%
\subsection{Exemple à deux états}
\subsubsection{Système et vecteur d'état}
On considère un système pouvant se trouver dans deux états. P.e. une pièce de monnaie posée sur une table. Elle présente soit son coté face soit son coté pile.

On représente ces deux états par deux vecteurs : 

\begin{minipage}[c]{.45\linewidth}
\[
\text{ pile :} \ \ 
\left( \begin{array}{ c }
 1 \\ 0
\end{array} \right)
\]
\end{minipage}
\hfill
\begin{minipage}[c]{.45\linewidth}
\[
\text{ face :} \ \ 
\left( \begin{array}{ c }
 0 \\ 1
\end{array} \right)
\]
\end{minipage}

\subsubsection{Opérateur}

L'opération consistant à retourner la pièce fait passer le système d'un état à l'autre. Dans l'espace des états, cela consiste à appliquer un opérateur sur le vecteur d'état. Cette opérateur est représenté par une matrice.

\so Exercice 1 : écrire cette matrice. Quelle est son inverse ?

\so Exercice 2 : La pièce se trouvant dans l'état $\bf u_0$, on la retourne à chaque seconde, quelle est son état après $n$ secondes ?

\subsubsection{Processus aléatoire}
On peut lancer la pièce plutôt que de la retourner. On considère alors le procédé suivant :
\begin{itemize}[leftmargin=1cm, label=\ding{32}, itemsep=1pt]
\item si la pièce est sur pile, on la retourne,
\item si la pièce est sur face, on la lance.
\end{itemize}

\so Exercice 3 : écrire la matrice représentant l'opérateur correspondant à ce processus.

\so Exercice 4 : La pièce se trouvant dans l'état $\bf u_0$, on la soumet à ce processus à chaque seconde, que peut-on dire de son état après $n$ secondes ?

\subsection{Exemple à trois états}
On défini un système à trois états : pile, face, dans la main. On considère le procédé
\begin{itemize}[leftmargin=1cm, label=\ding{32}, itemsep=1pt]
\item si la pièce est sur pile, on la prend dans la main,
\item si la pièce est dans la main, on la retourne,
\item si la pièce est sur face, on la lance.
\end{itemize}

\so Exercice 5 : schématiser le processus, écrire les vecteurs d'états et l'opérateur correspondant au processus.

\so Exercice 6 : La pièce se trouvant dans l'état $\bf u_0$, on la soumet à ce processus à chaque seconde, que peut-on dire de son état après $n$ secondes ?
%%%%%%%%%%%%%%%%%%%%%%%%%%%%%%%%%%%%%%%%%%%%%%%%%%%%%%%%%%%%%%%%%%%%%%%%

%

%\newpage
%%%%%%%%%%%%%%%%%%%%%
\section{Markov}
%%%%%%%%%%%%%%%%%%%%%
%
\subsection{Exemple à deux états}
\subsubsection{Système et vecteur d'état}
On considère un système pouvant se trouver dans deux états. P.e. une pièce de monnaie posée sur une table. Elle présente soit son coté face soit son coté pile.

On représente ces deux états par deux vecteurs : 

\begin{minipage}[c]{.45\linewidth}
\[
\text{ pile :} \ \ 
\left( \begin{array}{ c }
 1 \\ 0
\end{array} \right)
\]
\end{minipage}
\hfill
\begin{minipage}[c]{.45\linewidth}
\[
\text{ face :} \ \ 
\left( \begin{array}{ c }
 0 \\ 1
\end{array} \right)
\]
\end{minipage}

\subsubsection{Opérateur}

L'opération consistant à retourner la pièce fait passer le système d'un état à l'autre. Dans l'espace des états, cela consiste à appliquer un opérateur sur le vecteur d'état. Cette opérateur est représenté par une matrice.

\so Exercice 1 : écrire cette matrice. Quelle est son inverse ?

\so Exercice 2 : La pièce se trouvant dans l'état $\bf u_0$, on la retourne à chaque seconde, quelle est son état après $n$ secondes ?

\subsubsection{Processus aléatoire}
On peut lancer la pièce plutôt que de la retourner. On considère alors le procédé suivant :
\begin{itemize}[leftmargin=1cm, label=\ding{32}, itemsep=1pt]
\item si la pièce est sur pile, on la retourne,
\item si la pièce est sur face, on la lance.
\end{itemize}

\so Exercice 3 : écrire la matrice représentant l'opérateur correspondant à ce processus.

\so Exercice 4 : La pièce se trouvant dans l'état $\bf u_0$, on la soumet à ce processus à chaque seconde, que peut-on dire de son état après $n$ secondes ?

\subsection{Exemple à trois états}
On défini un système à trois états : pile, face, dans la main. On considère le procédé
\begin{itemize}[leftmargin=1cm, label=\ding{32}, itemsep=1pt]
\item si la pièce est sur pile, on la prend dans la main,
\item si la pièce est dans la main, on la retourne,
\item si la pièce est sur face, on la lance.
\end{itemize}

\so Exercice 5 : schématiser le processus, écrire les vecteurs d'états et l'opérateur correspondant au processus.

\so Exercice 6 : La pièce se trouvant dans l'état $\bf u_0$, on la soumet à ce processus à chaque seconde, que peut-on dire de son état après $n$ secondes ?
%%%%%%%%%%%%%%%%%%%%%%%%%%%%%%%%%%%%%%%%%%%%%%%%%%%%%%%%%%%%%%%%%%%%%%%%

%
%====================== INCLUSION DE LA BIBLIOGRAPHIE ======================
%
%récupérer les citation avec "/footnotemark" : 
\nocite{*}
%
% choix du style de la biblio
\bibliographystyle{plain}
%
% inclusion de la biblio
\cleardoublepage
\addcontentsline{toc}{chapter}{Bibliographie}
\bibliography{bibliographie.bib}
%
%====================== FIN DU DOCUMENT ======================
%
\end{document}
%%%%%%%%%%%%%%%%%%%%%%%%%%%%%%%%%%%%%%%%%%%%%%%%%%%%%%%%%%%%%%%%%%%%%%%%%%%%%%%%%
