\documentclass[12pt, a4paper]{report}
%\documentclass[11pt, a4paper]{article}

%====================== PACKAGES ======================
\usepackage[french]{babel}

\frenchbsetup{StandardLists=true}
\usepackage{enumitem}
\usepackage{pifont}

\usepackage[utf8x]{inputenc}
%\usepackage[latin1]{inputenc}

%pour gérer les positionnement d'images
\usepackage{float}
\usepackage{amsmath}
\usepackage{amssymb}
\DeclareMathOperator{\dt}{dt}
\usepackage{graphicx}
%\usepackage{tabularx}
\usepackage[colorinlistoftodos]{todonotes}
\usepackage{url}

%pour les informations sur un document compilé en PDF et les liens externes / internes
\usepackage[pdfborder=0]{hyperref}
\hypersetup{
	colorlinks = true
	}

%pour la mise en page des tableaux
\usepackage{array}
\usepackage{tabularx}
\usepackage{multirow}
\usepackage{multicol}
\setlength{\columnsep}{50pt}

%pour utiliser \floatbarrier
%\usepackage{placeins}
%\usepackage{floatrow}

%espacement entre les lignes
\usepackage{setspace}

%modifier la mise en page de l'abstract
\usepackage{abstract}

%police et mise en page (marges) du document
\usepackage[T1]{fontenc}
\usepackage[top=2cm, bottom=2cm, left=2cm, right=2cm]{geometry}

%Pour les galerie d'images
\usepackage{subfig}

\usepackage{pdfpages}

\usepackage{tikz}
\usetikzlibrary{trees}
\usetikzlibrary{decorations.pathmorphing}
\usetikzlibrary{decorations.markings}
\usetikzlibrary{decorations.pathreplacing,calligraphy}
\usetikzlibrary {arrows.meta}
%\usetikzlibrary{decorations}
\usetikzlibrary{angles, quotes}
\usepackage{verbatim}

\usepackage{appendix}

\usepackage{comment}

\usepackage{xcolor}

%\PreviewEnvironment{tikzpicture}
%\setlength\PreviewBorder{0pt}%

%====================== INFORMATION ET REGLES ======================

%rajouter les numérotation pour les \paragraphe et \subparagraphe
\setcounter{secnumdepth}{4}
\setcounter{tocdepth}{4}

\hypersetup{							% Information sur le document
pdfauthor = {Stephan Runigo},			% Auteurs
pdftitle = {Documentation},			% Titre du document
pdfsubject = {Documentation},		% Sujet
pdfkeywords = {Document},	% Mots-clefs
pdfstartview={FitH}}	% ajuste la page à la largeur de l'écran
%pdfcreator = {MikTeX},% Logiciel qui a crée le document
%pdfproducer = {} % Société avec produit le logiciel

%======================== DEFINITION COMMANDES ========================
\newcommand{\mt}[1]{\text{#1}}
\newcommand{\ul}[1]{\underline{#1}}
\newcommand{\mc}[1]{\mathcal{#1}}
\newcommand{\pt}[1]{\dot{\text{#1}}}
%======================== DEBUT DU DOCUMENT ========================
%
\begin{document}
%
%régler l'espacement entre les lignes
\newcommand{\HRule}{\rule{\linewidth}{0.5mm}}
%
% Titre, résumé, ... %%
%
\begin{titlepage}
%
~\\[1cm]

\begin{center}
%\includegraphics[scale=0.5]{./presentation/chambreABulle}
\end{center}

\textsc{\Large }\\[0.5cm]

% Title \\[0.4cm]
\HRule

\begin{center}
{\huge \bfseries Relativité restreinte et\\
électromagnétisme\\[0.4cm] }
\end{center}

\HRule \\[1.5cm]

\begin{center}
%\includegraphics[scale=0.3]{./presentation/ptoleme}
\end{center}

% Author and supervisor
\begin{minipage}{0.4\textwidth}
\begin{flushleft} \large
%\emph{Auteur:}\\
%Stephan \textsc{Runigo}
\end{flushleft}
\end{minipage}
\begin{minipage}{0.4\textwidth}
\begin{flushright} \large
\emph{Latex-iation:}\\
Stephan \textsc{Runigo}
\end{flushright}
\end{minipage}

\vfill
\begin{minipage}{0.4\textwidth}
\begin{flushleft} \large
Exercices de l'ENS extraits de https://www.phys.ens.fr/
\end{flushleft}
\end{minipage}
\begin{minipage}{0.4\textwidth}
\begin{flushright} \large
\end{flushright}
\end{minipage}

\vfill

% Bottom of the page
{\large \today}

\end{titlepage}

\newpage
\begin{center}
\Large
Résumé
\normalsize
\end{center}
\vspace{3cm}
\begin{itemize}[leftmargin=1cm, label=\ding{32}, itemsep=21pt]
\item {\bf Objet : } Mécanique Quantique.
\item {\bf Contenu : } Notes de cours.
\item {\bf Niveau requis : } Maitrise de sciences physiques.
\end{itemize}

\vspace{3cm} \large

Notes de cours rédigées en 1966 par Serge Haroche
\vspace{3cm}

http://enseignement.phys.ens.fr/spip.php?article110
\vspace{3cm}


%

%
% Table des matières
\tableofcontents
\thispagestyle{empty}
\setcounter{page}{0}
%
%espacement entre les lignes des tableaux
\renewcommand{\arraystretch}{1.5}
%
%====================== INCLUSION DES CHAPITRES ======================
%
~
\thispagestyle{empty}
%recommencer la numérotation des pages à "1"
\setcounter{page}{0}
\newpage
%
\chapter{Ensemble et application}
%

%%%%%%%%%%%%%%%%%%%%%%%%%
\section{Ensemble}
%%%%%%%%%%%%%%%%%%%%%%%%%
%$\mathcal{}$
\subsection{définitions}
Un \textbf{\textit {ensemble}} est une {\it collection d'objets}. Les objets appartenant à un ensemble sont appelés \textbf{\textit {élément}}.

L'élément $a$ appartient à l'ensemble $\mathcal{A}$ s'écrit :

\[
 a \in \mathcal{A}
\]

Un ensemble peut être représenté par une enveloppe autour de ses éléments représentés par des points :



\begin{center}
\begin{tikzpicture}

	\def\a{3} \def\b{1.75} % Taille de l'ellipse
\draw[black!60,very thick] (0,0) circle[x radius = \a cm, y radius = \b cm];

	%\node[draw] (P) at (0,0) {Paris};

	\def\i{1.85} \def\j{-0.25}% Position relative des points

	\coordinate (x) at ({\a*\i},0);
	\coordinate[label=above:x] (x);
	\node at (4,-1) {$\times$};
	
	\tikzset{mypoints/.style={fill=white,draw=black,thick}}
\end{tikzpicture}
\end{center}



\begin{center}
\begin{tikzpicture}

	\def\a{3} \def\b{1.75} % Taille de l'ellipse
\draw[black!60,very thick]
		(0,0) circle[x radius = \a cm, y radius = \b cm];

	\def\i{1.85} \def\j{-0.25}% Position relative des points

	\coordinate (x) at ({\a*\i},0);
	\draw [line width=2pt] (-6.51,1.52)-- (-6.19,1.8);
	\coordinate[label=above:x] (x);
	\coordinate (y) at (-{\a*\i},0);
	\coordinate [label=below left:y] (y);
	
	\coordinate (z) at (0,-\b*\i);
	\coordinate [label=below left:z] (z);
	
	\coordinate (w) at (0,\b*\i);
	\coordinate [label=below left:w] (w);
	\node at (4,-1) {$\times$};
	
	\tikzset{mypoints/.style={fill=white,draw=black,thick}}
	\def\ptsize{2.0pt}
	\foreach \p in {x,y,z,w}\fill[mypoints] (\p) circle (\ptsize);
\end{tikzpicture}
\end{center}

    \begin{tikzpicture}[scale=1.2]
	\tikzset{mypoints/.style={fill=white,draw=black,thick}}
	\def\ptsize{2.0pt}
	\def\a{3} \def\b{1.75}
	% warning: construction fails if xp<0 or yp<=0
	\def\xp{4.0} \def\yp{3.5} 
	\def\i{0.85} \def\j{-0.25}%determines rays from P
	\coordinate[label=above:P] (P) at (\xp,\yp);
	\coordinate (M) at ({\a*\i},0);
	\coordinate (N) at ({\a*\j},0);
	\coordinate (AA) at (0,-\b);
	\coordinate (BB) at (1,-\b);
	\coordinate (CC) at (-\a,0);
	\coordinate (DD) at (-\a,1);
	\coordinate (Q) at (intersection of P--N and AA--BB);
	\coordinate (R) at (intersection of P--M and AA--BB);
	%\draw[name path=ellipse,red,very thick]
		%(0,0) circle[x radius = \a cm, y radius = \b cm];
	%\path[name path=linePQ,blue] (P)--(Q);
	%\path[name path=linePR,green] (P)--(R);
	%\path [name intersections={of = ellipse and linePQ}];
	%\coordinate[label=above:A] (A)  at (intersection-1);
	%\coordinate[label=below left:B] (B) at (intersection-2);
	%\path [name intersections={of = ellipse and linePR}];
	%\coordinate[label=above right:C] (C)  at (intersection-1);
	%\coordinate[label=above left:D] (D) at (intersection-2);
	%\draw (B)--(P)--(D) (A)--(D) (C)--(B);
	%\coordinate [label=below left:E] (E) at (intersection of A--D and B--C);
	%\coordinate[label=above right:F] (F) at (intersection of A--C and B--D);
	%\coordinate (G) at (intersection of E--F and CC--DD);
	%\draw [name path=lineFG,blue,thick] (F)--(G);
	%\draw (A)--(F)--(B);
	%\path [name intersections={of = ellipse and lineFG}];
	%\coordinate[label=above:X] (X) at (intersection-1);
	%\coordinate[label=above right:Y] (Y) at (intersection-2);
	%\coordinate (XX) at ($(P)!1.5!(X)$);
	%\coordinate (YY) at ($(P)!1.5!(Y)$);
	%\draw[very thick,green!50!black!50] (XX)--(P)--(YY);
	%\foreach \p in {A,B,C,D,E,F,P,X,Y}\fill[mypoints] (\p) circle (\ptsize);
  \end{tikzpicture}
%%%%%%%%%%%%%%%%%%%%%%%%%%%%%%%%%%%%%%%%%%%%%%%%%%%%%%%%%%%%%%%%%%%%%%%%

%

%%%%%%%%%%%%%%%%%%%%%%%%%
\section{Ensembles de nombres}
%%%%%%%%%%%%%%%%%%%%%%%%%

\subsection{Entiers naturels}
\[
\mathbb{N} = \{ 0, 1, 2, 3, ... \}
\]
\subsection{Entiers relatifs}
\[
\mathbb{Z} = \{ ..., -2, -1, 0, 1, 2, 3, ... \}
\]
\subsection{Nombres rationnels}
\[
\mathbb{Q} = \{ 0, 1, 2, 3, ... \}
\]
\subsection{Nombres réels}
\[
\mathbb{R} = \{ 0, 1, 2, 3, ... \}
\]
\subsection{Nombres complexes}
\[
\mathbb{C} = \{ 0, 1, 2, 3, ... \}
\]
Une application ($f$) met en relation des éléments ($a$) d'un ensemble ($\mathcal{A}$, dit de départ) avec des éléments ($b$) d'un autre ensemble ($\mathcal{B}$, dit d'arrivé). :
\begin{align*}
f :\ \ \ \ \ \ \ \ \ \mathcal{A} \ \  & \rightarrow \ \ \ \mathcal{B} \\
a \ \ & \mapsto \ \ b = f(a)
\end{align*}

Une loi de composition est une application qui associe deux éléments (éventuellement du même ensemble) à un troisième élément. 
\begin{align*}
f :\ \ \ \ \ \ \ \ \ \mathcal{A} \times \mathcal{B} \ \  & \rightarrow \ \ \ \mathcal{C} \\
(a,b) \ \ & \mapsto \ \ c = f(a,b)
\end{align*}

Une loi de composition est dite interne si $\mathcal{A} = \mathcal{B} = \mathcal{C}$, externe sinon.

%%%%%%%%%%%%%%%%%%%%%%%%%%%%%%%%%%%%%%%%%%%%%%%%%%%%%%%%%%%%%%%%%%%%%%%%%%%%%%%%%%%%%
%%%%%%%%%%%%%%%%%%%%%%%%%%%%%%%%%%%%%%%%%%%%%%%%%%%%%%%%%%%%%%%%%%%%%%%%

%

%%%%%%%%%%%%%%%%%%%%%%%%%
\section{Application}
%%%%%%%%%%%%%%%%%%%%%%%%%
Une application ($f$) met en relation des éléments ($a$) d'un ensemble ($\mathcal{A}$, dit de départ) avec des éléments ($b$) d'un autre ensemble ($\mathcal{B}$, dit d'arrivé). :
\begin{align*}
f :\ \ \ \ \ \ \ \ \ \mathcal{A} \ \  & \rightarrow \ \ \ \mathcal{B} \\
a \ \ & \mapsto \ \ b = f(a)
\end{align*}

Une loi de composition est une application qui associe deux éléments (éventuellement du même ensemble) à un troisième élément. 
\begin{align*}
f :\ \ \ \ \ \ \ \ \ \mathcal{A} \times \mathcal{B} \ \  & \rightarrow \ \ \ \mathcal{C} \\
(a,b) \ \ & \mapsto \ \ c = f(a,b)
\end{align*}

Une loi de composition est dite interne si $\mathcal{A} = \mathcal{B} = \mathcal{C}$, externe sinon.

%%%%%%%%%%%%%%%%%%%%%%%%%%%%%%%%%%%%%%%%%%%%%%%%%%%%%%%%%%%%%%%%%%%%%%%%%%%%%%%%%%%%%
%%%%%%%%%%%%%%%%%%%%%%%%%%%%%%%%%%%%%%%%%%%%%%%%%%%%%%%%%%%%%%%%%%%%%%%%

%

%
%
%====================== INCLUSION DE LA BIBLIOGRAPHIE ======================
%
%récupérer les citation avec "/footnotemark" : 
\nocite{*}
%
% choix du style de la biblio
\bibliographystyle{plain}
%
% inclusion de la biblio
\cleardoublepage
\addcontentsline{toc}{chapter}{Bibliographie}
\bibliography{bibliographie.bib}
%
%====================== FIN DU DOCUMENT ======================
%
\end{document}
%%%%%%%%%%%%%%%%%%%%%%%%%%%%%%%%%%%%%%%%%%%%%%%%%%%%%%%%%%%%%%%%%%%%%%%%%%%%%%%%%
