\newpage

%%%%%%%%%%%%%%%%%%%%%
\section{Système et états}
%%%%%%%%%%%%%%%%%%%%%
\subsection{Définitions}
	\begin{itemize}[leftmargin=1cm, label=\ding{32}, itemsep=1pt]

\ib{Système} — \fsb{S. concr.} {\bf 1.} Ensemble organisé dont les parties
ou éléments sont interdépendants ou obéissent à une loi unique : « Le
système solaire »; « Le système nerveux »; « Tout fait psychique est un
système » (Paulhan). — \fsb{S. abstr.} {\bf 2.} Combinaison d'idées ou de
procédés coordonnés et ramenés à un petit nombre de principes : « Le système
héliocentrique »; « Les systèmes philosophiques »; « Le système métrique ».
Qqfs. {\it péj.}, spéc. dans l'expression « esprit de système » : voir {\it
Systématique}$^2$, et cf. {\it Textes choisis}, II, p. 38.

\ib{Évolution} — \si{Méta.} {\bf 1.} {\it Lato.} Suite de
transformations régie par une loi$^5$,
et gén. conçue comme graduelle et
continue : « La formation des mondes
expliquée par voie de développement lent et graduel ou, selon
l'expression, moderne d'{\it évolution} »
(Fouillée). — {\bf 2.} {\it Str.} ({\it spéc.} chez
Spencer). Transformation universelle définie surtout par la différenciation$^3$ et l'intégration$^3$ ([v. ces
mots) progressives : « L’évolution
est une intégration de matière, pendant laquelle celle-ci passe d’une
homogénéité indéfinie, incohérente,
à une hétérogénéité définie, cohérente. » (Spencer). — {\bf 3.}
{\it Évolution créatrice} (Bergson) : celle qui, au
lieu de « reconstituer l’évolution
avec des fragments de l’évolué »
[comme chez Spencer), consiste en
un élan* créateur : « L'évolution
est une création sans cesse renouvelée » ({\it E. C.}, II).

	\end{itemize}
%%%%%%%%%%%%%%%%%%%%%%%%%%%%%%%%%%%%%%%%%%%%%%%%%%%%%%%%%%%%%%%%%%%%%%%%
