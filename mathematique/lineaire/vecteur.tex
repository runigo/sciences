\newpage
%%%%%%%%%%%%%%%%%%%%%
\section{Vecteur}
%%%%%%%%%%%%%%%%%%%%%
%\vspace{.323cm}
Exercice 1 - Combinaisons linéaires ?

\vspace{.323cm}
Les vecteurs $u$ suivants sont-ils combinaison linéaire des vecteurs $u_i$ ?

\vspace{.323cm}
$ u = 
\left( \begin{array}{ c }
 1 \\ 2
\end{array} \right)$
,
$ u_1 = 
\left( \begin{array}{ c }
 1 \\ −2
\end{array} \right)$
,
$ u_2 = 
\left( \begin{array}{ c }
 2 \\ 3
\end{array} \right)$

\vspace{.323cm}
$ u = 
\left( \begin{array}{ c }
 1 \\ 2
\end{array} \right)$
,
$ u_1 = 
\left( \begin{array}{ c }
 1 \\ −2
\end{array} \right)$
,
$ u_2 = 
\left( \begin{array}{ c }
 2 \\ 3
\end{array} \right)$
,
$ u_3 = 
\left( \begin{array}{ c }
 −4 \\ 5
\end{array} \right)$

\vspace{.323cm}
$ u = 
\left( \begin{array}{ c }
 2 \\ 5 \\ 3
\end{array} \right)$
,
$ u_1 = 
\left( \begin{array}{ c }
 1 \\ 3 \\ 2
\end{array} \right)$
,
$ u_2 = 
\left( \begin{array}{ c }
 1 \\ −1 \\ 4
\end{array} \right)$

\vspace{.323cm}
$ u = 
\left( \begin{array}{ c }
 3 \\ 1 \\ m
\end{array} \right)$
,
$ u_1 = 
\left( \begin{array}{ c }
 1 \\ 3 \\ 2
\end{array} \right)$
,
$ u_2 = 
\left( \begin{array}{ c }
 1 \\ −1 \\ 4
\end{array} \right)$

 (discuter suivant la valeur de m). 


\vspace{.323cm}
Exercice 3 - Combinaisons linéaires?

\vspace{.323cm}
 Dans l'espace vectoriel $\mathbb{R}[X]$, le polynôme $P(X)=16X^3−7X^2+21X−4$ est-il combinaison linéaire de $P_1(X)=8X^3−5X^2+1$ et $P_2(X)=X^2+7X−2$ ?

\vspace{.323cm}
Dans l'espace vectoriel $\mathcal{F}(\mathbb{R},\mathbb{R})$
des fonctions de $\mathbb{R}$ dans $\mathbb{R}$, la fonction $x ↦ \sin(2x)$ est-elle combinaison linéaire des fonctions $\sin$ et $\cos$?
\vspace{.323cm}

{\scriptsize Source : 

\texttt{https://www.bibmath.net/ressources/index.php?action=affiche\&quoi=bde/algebrelineaire/evfamilleslibres\&type=fexo}}

%%%%%%%%%%%%%%%%%%%%%%%%%%%%%%%%%%%%%%%%%%%%%%%%%%%%%%%%%%%%%%%%%%%%%%%%
