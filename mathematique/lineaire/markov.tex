\newpage
%%%%%%%%%%%%%%%%%%%%%
\section{Markov}
%%%%%%%%%%%%%%%%%%%%%
%
\subsection{Exemple à deux états}
\subsubsection{Système et vecteur d'état}
On considère un système pouvant se trouver dans deux états. P.e. une pièce de monnaie posée sur une table. Elle présente soit son coté face soit son coté pile.

On représente ces deux états par deux vecteurs : 

\begin{minipage}[c]{.45\linewidth}
\[
\text{ pile :} \ \ 
\left( \begin{array}{ c }
 1 \\ 0
\end{array} \right)
\]
\end{minipage}
\hfill
\begin{minipage}[c]{.45\linewidth}
\[
\text{ face :} \ \ 
\left( \begin{array}{ c }
 0 \\ 1
\end{array} \right)
\]
\end{minipage}

\subsubsection{Opérateur}

L'opération consistant à retourner la pièce fait passer le système d'un état à l'autre. Dans l'espace des états, cela consiste à appliquer un opérateur sur le vecteur d'état. Cette opérateur est représenté par une matrice.

\so Exercice 1 : écrire cette matrice. Quelle est son inverse ?

\so Exercice 2 : La pièce se trouvant dans l'état $\bf u_0$, on la retourne à chaque seconde, quelle est son état après $n$ secondes ?

\subsubsection{Processus aléatoire}
On peut lancer la pièce plutôt que de la retourner. On considère alors le procédé suivant :
\begin{itemize}[leftmargin=1cm, label=\ding{32}, itemsep=1pt]
\item si la pièce est sur pile, on la retourne,
\item si la pièce est sur face, on la lance.
\end{itemize}

\so Exercice 3 : écrire la matrice représentant l'opérateur correspondant à ce processus.

\so Exercice 4 : La pièce se trouvant dans l'état $\bf u_0$, on la soumet à ce processus à chaque seconde, que peut-on dire de son état après $n$ secondes ?

\subsection{Exemple à trois états}
On défini un système à trois états : pile, face, dans la main. On considère le procédé
\begin{itemize}[leftmargin=1cm, label=\ding{32}, itemsep=1pt]
\item si la pièce est sur pile, on la prend dans la main,
\item si la pièce est dans la main, on la retourne,
\item si la pièce est sur face, on la lance.
\end{itemize}

\so Exercice 5 : schématiser le processus, écrire les vecteurs d'états et l'opérateur correspondant au processus.

\so Exercice 6 : La pièce se trouvant dans l'état $\bf u_0$, on la soumet à ce processus à chaque seconde, que peut-on dire de son état après $n$ secondes ?
%%%%%%%%%%%%%%%%%%%%%%%%%%%%%%%%%%%%%%%%%%%%%%%%%%%%%%%%%%%%%%%%%%%%%%%%
