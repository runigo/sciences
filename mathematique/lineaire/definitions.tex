
%%%%%%%%%%%%%%%%%%%%%
\section{Définitions}
%%%%%%%%%%%%%%%%%%%%%
\subsection{Matrices}
Une matrice est un tableau de nombres. Ces nombres sont des éléments d'un ensemble, parfois $\mathbb{Z}$, souvent $\mathbb{R}$, plus généralement $\mathbb{C}$.

L'ensemble des matrices à  $N$ lignes et $M$ colonnes ($N \times M$) à coeficients réèl est
$\mathbb{R}^N \times \mathbb{R}^M$.

Une matrice carée contient autant de lignes que de colonnes.
L'ensemble des matrices carré ($N \times N$) à coeficients réèl est
$\mathbb{R}^N \times \mathbb{R}^N$.

\subsection{Inverse d'une matrice}
%Une matrice colonne (matrice a n ligne et 1 colonne) à coefficients réèls (resp. complexe) est apte à représenter un vecteur.  
%Un vecteur est une matrice colonne. L'ensemble des vecteurs sur un corps forment un espace vectoriel ($\mathbb{R}^n$ et $\mathbb{C}^n$ sont des espaces vectoriels).

Une matrice $\mc{A}∈\mc{M}_n(\mathbb{C})$ est inversible s'il existe une matrice
$\mc{A}^{-1}$ tel que :
\[
\mc{A}.\mc{A}^{-1} = \mc{I}_n
\]

Dès lors, les matrices colonnes constituant $\mc{A}$ représentent $n$ vecteurs libres. Ces vecteurs forment une base de l'espace vectoriel $\mathbb{C}^n$.

\subsection{Matrice diagonale}

Une matrice $\mc{B}$ de $\mc{M}_n(\mathbb{C})$ est diagonale si
\[
b_{ij}=0 \ \ \text{ si } \ \ i\neq j
\]
Les $b_{ii}$ sont les valeurs propre de $\mc{B}$

\so
Une matrice $\mc{C}$ de $\mc{M}_n(\mathbb{C})$ est diagonalisable s'il existe une matrice $\mc{A}$ tel que
\[
\mc{A}.\mc{C}.\mc{A}^{-1} \text{ est diagonale}
\]

\subsection{Matrice stochastique}
Une matrice $\mathcal{A}∈\mathcal{M}_n(\mathbb{R})$ est une matrice stochastique si la somme des coefficients sur chaque colonne de $\mathcal{A}$ est égale à 1.
%%%%%%%%%%%%%%%%%%%%%%%%%%%%%%%%%%%%%%%%%%%%%%%%%%%%%%%%%%%%%%%%%%%%%%%%
