\newpage
\subsection{Matrices stochastiques}

Exercice 9 - Matrices stochastiques en petite taille

On dit qu'une matrice $\mathcal{A}∈\mathcal{M}_n(\mathbb{R})$ est une matrice stochastique si la somme des coefficients sur chaque colonne de $\mathcal{A}$ est égale à 1. Démontrer que le produit de deux matrices stochastiques est une matrice stochastique si $n=2$. Reprendre la question si $n=3$.

\vspace{.323cm}
Exercice 10 - Matrices stochastiques

Soit $\mathcal{A},\mathcal{B}∈\mathcal{M}_n(\mathbb{R})$ deux matrices telles que la somme des coefficients sur chaque colonne de A et sur chaque colonne de B vaut 1 (on dit qu'une telle matrice est une matrice stochastique). Montrer que la somme des coefficients sur chaque colonne de AB vaut 1.

\vspace{.323cm}
{\scriptsize Source : 

\texttt{https://www.bibmath.net/ressources/index.php?action=affiche\&quoi=bde/algebrelineaire/matrices\&type=fexo}}

\subsection{Matrice Inverse}

Exercice 4 - Inverser une matrice sans calculs !
\vspace{.323cm}

\begin{minipage}[c]{.45\linewidth}
1. Soit $ A=\left(
\begin{array}{ccc}
-1&1&1\\1&-1&1\\1&1&-1
\end{array}\right)$.
\end{minipage}
\hfill
\begin{minipage}[c]{.45\linewidth}
Montrer que $A^2=2I_3-A$, en déduire que $A$ est inversible et calculer $A^{-1}$.
\end{minipage}

\vspace{.323cm}
\begin{minipage}[c]{.45\linewidth}
2. Soit $ A=\begin{pmatrix} 1 & 0 & 2 \cr
0 & -1 & 1 \cr
1 & -2 & 0 \cr \end{pmatrix} .$
\end{minipage}
\hfill
\begin{minipage}[c]{.45\linewidth}
Calculer $A^3-A .$ En déduire
que $ A $ est inversible puis déterminer $ A^{-1} .$ 
\end{minipage}

\vspace{.323cm}
\begin{minipage}[c]{.45\linewidth}
3. Soit $A=\begin{pmatrix} 
0&1&-1\\-1&2&-1\\1&-1&2
\end{pmatrix}$.
\end{minipage}
\hfill
\begin{minipage}[c]{.45\linewidth}
Calculer $A^2-3A+2I_3$. En déduire que $A$ est inversible, et calculer $A^{-1}$.
\end{minipage}

\vspace{.323cm}
{\scriptsize Source : 

\texttt{https://www.bibmath.net/ressources/index.php?action=affiche\&quoi=bde/algebrelineaire/matricesaa\&type=fexo}}


