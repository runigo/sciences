\newpage
\section{Produit de matrice}
Exercice n°7.
Effectuer les produits suivants lorsque c’est possible. Lorsque c’est impossible, dire pourquoi.

\begin{minipage}[c]{.45\linewidth}
a)
$\left( \begin{array}{ c c }
 2 & 5 \\ 3 & 6 \\ 4 & 7
\end{array} \right)
×
\left( \begin{array}{ c c }
 2 & 5 \\ 4 & 6
\end{array} \right)$
\end{minipage}
\hfill
\begin{minipage}[c]{.45\linewidth}
b)
$\left( \begin{array}{ c c }
 2 & 5 \\ 4 & 6
\end{array} \right)
×
\left( \begin{array}{ c c }
 2 & 5 \\ 3 & 6 \\ 4 & 7
\end{array} \right)$
\end{minipage}

\begin{minipage}[c]{.45\linewidth}
c)
$\left( \begin{array}{ c c c }
 -1 & 4 & 5
\end{array} \right)
×
\left( \begin{array}{ c c c }
 0 & -1 & 6 \\ 2 & 4 & -2 \\ 3 & 5 & 3
\end{array} \right)$
\end{minipage}
\hfill
\begin{minipage}[c]{.45\linewidth}
d)
$\left( \begin{array}{ c c c }
 2 & 5 & 0 \\ 3 & 6 & 3 \\ 4 & 1 & 2
\end{array} \right)
×
\left( \begin{array}{ c c }
 1 & -1 \\ 2 & 0 \\ 3 & 5
\end{array} \right)$
\end{minipage}

\begin{minipage}[c]{.45\linewidth}
e)
$\left( \begin{array}{ c c }
 1 & -1 \\ 2 & 0 \\ 3 & 5
\end{array} \right)
×
\left( \begin{array}{ c c }
 2 & 5 \\ 3 & 6 \\ 4 & 1
\end{array} \right)$
\end{minipage}
\hfill
\begin{minipage}[c]{.45\linewidth}
f)
$\left( \begin{array}{ c c c }
 1 & 0 & 5 \\ 2 & -1 & 6 \\ 3 & 4 & 7 \\
\end{array} \right)
×
\left( \begin{array}{ c c c }
 2 & 7 & 8 \\ 0 & 2 & 3 \\ 4 & 5 & 6 \\
\end{array} \right)$
\end{minipage}

\subsection{Matrice 2×2}
Exercice n°8. Calculer, puis comparer les produits A×B et B×A

\begin{center}
a)
$ A = 
\left( \begin{array}{ c c }
 -1 & 8 \\ 2 & 11
\end{array} \right)$
et
$B =
\left( \begin{array}{ c c }
4 & 2 \\ -5 & 8
\end{array} \right)$
\end{center}

\begin{minipage}[c]{.45\linewidth}
b)
$ A = 
\left( \begin{array}{ c c }
 4 & 8 \\ 1 & 2
\end{array} \right)$
et
$ B = 
\left( \begin{array}{ c c }
 3 & 9 \\ 1 & 1
\end{array} \right)$
\end{minipage}
\hfill
\begin{minipage}[c]{.45\linewidth}
c)
$ A = 
\left( \begin{array}{ c c }
 2 & 1 \\ 1 & 1
\end{array} \right)$
et
$ B = 
\left( \begin{array}{ c c }
 5 & 2 \\ 2 & 3
\end{array} \right)$
\end{minipage}

\vspace{.323cm}
Exercice n°9.
Dans chacun des cas, calculer les produits A×B et B×A . Quelle particularité présente-t-il ?

a)
$ A = 
\left( \begin{array}{ c c }
 6 & -12 \\ -3 & 6
\end{array} \right)$
et
$ B = 
\left( \begin{array}{ c c }
 12 & 6 \\ 6 & 3
\end{array} \right)$
b)
$ A = 
\left( \begin{array}{ c c }
 2 & 4 \\ -1 & -2
\end{array} \right)$
et
$ B = 
\left( \begin{array}{ c c }
 0 & 2 \\ 0 & -1
\end{array} \right)$

\vspace{.323cm}
Exercice n°11.

Calculez et comparez $A^2 + 2 AB + B^2$ et $( A + B )^2$ avec : 
\begin{center}
$ A = 
\left( \begin{array}{ c c }
4 & 8 \\ 3 & 9
\end{array} \right)$
et
$ B =
\left( \begin{array}{ c c }
1 & 2 \\ 1 & 1
\end{array} \right)$
\end{center}

\vspace{.323cm}
{\scriptsize Source : 

\texttt{https://maurimath.net/documents/ecly/matricesexoscorriges.pdf}}

