\begin{itemize}[leftmargin=0.1cm, itemsep=40pt]
\item Théorie des ensembles
	\begin{itemize}[leftmargin=1cm, itemsep=5pt]
	\item Un ensemble est une colection d'élément.
	\item Les entiers forment un ensemble infini.
	\item Les réels forment un ensemble continu.
	\item L'ensemble des complexes est algébriquement clos.
	\end{itemize}



\item Résoudre les équations suivantes :

\hfill $2x = 6$ \hfill $8x - 136 = 0$  \hfill $2x + 4 = 3x + 1$
  \hfill $x^2 = 16$   \hfill $x^2 = 5x$  % $$  label=\ding{32}, label=\ding{32},

\item Expressions et équations
	\begin{itemize}[leftmargin=1cm, itemsep=5pt]
	\item Calculer la valeur des expressions suivante pour $x=0$
		
		\hfill $ A=5(x-2) $ \hfill $ B=2x(x-3) $ \hfill $ C=(x-5)(x+1) $ \hfill $ D=(2x-5)(x-4) $
	\item Développer, puis réduire, si possible, chaque expression
	\item Résoudre alors les équations suivantes :

		\hfill $ A=0 $ \hfill $ B=0 $ \hfill $ C=0 $ \hfill $ D=0 $
	\end{itemize}


\item Résoudre les équations suivantes :

\hfill $3x = 15$ \hfill $8x - 136 = 0$  \hfill $2x + 3 = 3x + 4$
  \hfill $x^2 = 49$   \hfill $x^2 = 3x$  % $$  label=\ding{32}, label=\ding{32},

\item Expressions et équations
	\begin{itemize}[leftmargin=1cm, itemsep=5pt]
	\item Calculer la valeur des expressions suivante pour $x=0$
		
		\hfill $ A=5(x+2) $ \hfill $ B=2x(x+3) $ \hfill $ C=(x-5)(x+1) $ \hfill $ D=(2x-5)(x+4) $
	\item Développer, puis réduire, si possible, chaque expression
	\item Résoudre alors les équations suivantes :

		\hfill $ A=-1 $ \hfill $ B=0 $ \hfill $ C=0 $ \hfill $ D=0 $
	\end{itemize}


\item Résoudre les problèmes suivants :
	\begin{itemize}[leftmargin=1cm, itemsep=1pt]
		\item 
	\end{itemize}

	\end{itemize}
