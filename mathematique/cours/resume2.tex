\begin{itemize}[leftmargin=0.1cm, itemsep=40pt]
\item Résoudre les équations suivantes :

\hfill $2x = 6$ \hfill $8x - 136 = 0$  \hfill $2x + 4 = 3x + 1$
  \hfill $x^2 = 16$   \hfill $x^2 = 5x$  % $$  label=\ding{32}, label=\ding{32},

\item Expressions et équations
	\begin{itemize}[leftmargin=1cm, itemsep=5pt]
	\item Calculer la valeur des expressions suivante pour $x=0$
		
		\hfill $ A=5(x-2) $ \hfill $ B=2x(x-3) $ \hfill $ C=(x-5)(x+1) $ \hfill $ D=(2x-5)(x-4) $
	\item Développer, puis réduire, si possible, chaque expression
	\item Résoudre alors les équations suivantes :

		\hfill $ A=0 $ \hfill $ B=0 $ \hfill $ C=0 $ \hfill $ D=0 $
	\end{itemize}


\item Résoudre les équations suivantes :

\hfill $3x = 15$ \hfill $8x - 136 = 0$  \hfill $2x + 3 = 3x + 4$
  \hfill $x^2 = 49$   \hfill $x^2 = 3x$  % $$  label=\ding{32}, label=\ding{32},

\item Expressions et équations
	\begin{itemize}[leftmargin=1cm, itemsep=5pt]
	\item Calculer la valeur des expressions suivante pour $x=0$
		
		\hfill $ A=5(x+2) $ \hfill $ B=2x(x+3) $ \hfill $ C=(x-5)(x+1) $ \hfill $ D=(2x-5)(x+4) $
	\item Développer, puis réduire, si possible, chaque expression
	\item Résoudre alors les équations suivantes :

		\hfill $ A=-1 $ \hfill $ B=0 $ \hfill $ C=0 $ \hfill $ D=0 $
	\end{itemize}


\item Résoudre les problèmes suivants :
	\begin{itemize}[leftmargin=1cm, itemsep=1pt]
		\item 
Trois amis passent commande dans un café : « Deux sodas et un café ».
Un moment plus tard, ils sont rejoints par un autre ami,
ils passent alors une nouvelle commande : « 3 cafés et un soda ».

Sachant qu’un soda coûte 0,50 euros de plus qu’un café et que la deuxième commande coûte 0,70 euros de plus que la première.

Déterminer le prix d’un café et et le prix d’un soda.

		\item 
Claire a 12 ans et est trois fois plus âgée que sa petite sœur.
Elle se demande dans combien d’années elle sera deux fois plus âgée.

		\item 
Sarah a cueilli 84 trèfles, certains ont 3 feuilles, les autres 4 feuilles.
Au total, il y a 258 feuilles.

Déterminer le nombre de trèfles à trois feuilles et le nombre de trèfles à quatre feuilles.


	\end{itemize}

	\end{itemize}
