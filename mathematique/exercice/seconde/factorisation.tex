Exercice 1. 
Factoriser en utilisant des identités remarquables.

$A=x^2-10x+25$

$B=9+6x+x^2$

$C=1-x^2$

$D=4x^2+12x+9$

$E=x^2-16$

$F=9x^2-4$

$G=9x^2-6x+1$

$H=25-4x^2$

\vspace{1cm}
Exercice 2. Factoriser les expressions suivantes en utilisant des identités remarquables.

$A=25x^2-10x+1$

$B=36x^2+84x+49$

$C=81x^2-16$

$D=4x^2+12x+9$

$E=64x^2-121$

$F=256x^2+384x+144$

\vspace{1cm}
Exercice 3. Lorsque cela est possible, factoriser les expressions suivantes en utilisant des identités remarquables.

$A=4x^2+20x+25$

$B=36x^2+12x-1$

$C=9x^2+4$

$D=100-49x^2$

$E=16x^2+32x+64$

$F=x^2+1-2x$

\newpage
Exercice 4. Factoriser

$A=(x-1)^2-(4x-2)^2$

$B=9x^2-(x+1)^2$

$C=(2x+3)^2-(1+x)^2$

$D=(3x+2)^2-(5x+1)^2$

$E=x^2+6x+9-(x+3)(x-2)$

$F=25-(2x+3)^2$

$G=3x^2-6x+3$

$H=(3x+3)-(x+1)(2x-1)$

\vspace{1cm}
Exercice 5. Factoriser en utilisant au préalable une identité remarquable.

$A=x^2-4+(x+2)(x+3)$

$B=x^2+6x+9-(x+3)(x-1)$

$C=(3x-2)(x+5)+9x^2-4$

$D=9x^2-1+(3x+1)(2x+3)$

$E=x^2-4x+4+(x+3)(x-2)$

\vspace{1cm}
Exercice 6. Factoriser les expressions suivantes en utilisant des identités remarquables.

$A=\dfrac{1}{4}-25x^2$

$B=\dfrac{x^2}{36}-\dfrac{25}{49}$

$C=\dfrac{4}{9}x^2+\dfrac{49}{36}+\dfrac{14}{9}x$

$D=\dfrac{81}{16}x^2-\dfrac{33}{2}x+\dfrac{121}{9}$

$E=\dfrac{25}{4}x^2-\dfrac{169}{144}$

