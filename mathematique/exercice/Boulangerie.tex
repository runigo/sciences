I CALCUL

Dans une boulangerie, j'achète trois baguettes à 1€10. Je donne à la vendeuse un billet de 5€, combien doit-elle me rendre ?


Dans une boulangerie, j'achète trois patisseries à 2€20. Je donne à la vendeuse un billet de 10€, combien doit-elle me rendre ?












II  ÉQUATION

Dans une boulangerie, j'achète trois baguettes. Je donne à la vendeuse un billet de 5€, elle me rend 2€45. Quel est le prix d'une baguette ?



Dans une autre boulangerie, je me laisse tenter par un pain au chocolat à 1€35 et je prend également trois baguettes. Je donne à la vendeuse un billet de 5€, elle me rend 20 centimes. Quel est le prix d'une baguette ?



Dans une troisiemme boulangerie, la baguette au graines vaut 45 centimes de plus que la baguette classique. J'achète alors deux baguettes classique et une baguettes au graines. Je donne à la vendeuse un billet de 5€, elle me rend 2€. Quel est le prix d'une baguette ?

Indice : la réponse n'est pas si compliquée.




Dans cette quatriemme boulangerie, la baguette au graines vaut 45 centimes de plus que la baguette classique. J'achète deux baguettes classique et une baguettes au graines. Je donne à la vendeuse un billet de 5€, elle me rend 1€45. Quel est le prix d'une baguette ?

Ici, il n'y a pas d'indice...
