
%%%%%%%%%%%%%%%%%%%%%%%%%
\section{Ensembles de nombres}
%%%%%%%%%%%%%%%%%%%%%%%%%

\subsection{Entiers naturels}
\[
\mathbb{N} = \{ 0, 1, 2, 3, ... \}
\]
\subsection{Entiers relatifs}
\[
\mathbb{Z} = \{ ..., -2, -1, 0, 1, 2, 3, ... \}
\]
\subsection{Nombres rationnels}
\[
\mathbb{Q} = \{ 0, 1, 2, 3, ... \}
\]
\subsection{Nombres réels}
\[
\mathbb{R} = \{ 0, 1, 2, 3, ... \}
\]
\subsection{Nombres complexes}
\[
\mathbb{C} = \{ 0, 1, 2, 3, ... \}
\]
Une application ($f$) met en relation des éléments ($a$) d'un ensemble ($\mathcal{A}$, dit de départ) avec des éléments ($b$) d'un autre ensemble ($\mathcal{B}$, dit d'arrivé). :
\begin{align*}
f :\ \ \ \ \ \ \ \ \ \mathcal{A} \ \  & \rightarrow \ \ \ \mathcal{B} \\
a \ \ & \mapsto \ \ b = f(a)
\end{align*}

Une loi de composition est une application qui associe deux éléments (éventuellement du même ensemble) à un troisième élément. 
\begin{align*}
f :\ \ \ \ \ \ \ \ \ \mathcal{A} \times \mathcal{B} \ \  & \rightarrow \ \ \ \mathcal{C} \\
(a,b) \ \ & \mapsto \ \ c = f(a,b)
\end{align*}

Une loi de composition est dite interne si $\mathcal{A} = \mathcal{B} = \mathcal{C}$, externe sinon.

%%%%%%%%%%%%%%%%%%%%%%%%%%%%%%%%%%%%%%%%%%%%%%%%%%%%%%%%%%%%%%%%%%%%%%%%%%%%%%%%%%%%%
%%%%%%%%%%%%%%%%%%%%%%%%%%%%%%%%%%%%%%%%%%%%%%%%%%%%%%%%%%%%%%%%%%%%%%%%
